\documentclass[12pt, a4paper]{article}
\usepackage[T2A]{fontenc}
\usepackage[utf8]{inputenc}
\usepackage[russian]{babel}
\usepackage{amsmath}
\usepackage{amssymb}
\usepackage{amsfonts}
\usepackage{graphicx}
\usepackage{tikz}
\usepackage{textcomp} % For \textcircled
\usepackage{graphicx} % For including images if needed for diagrams, though the diagram is simple 
\usepackage[hidelinks]{hyperref}
\newcommand{\backtotoc}{\small\hyperref[toc]{[← Вернуться к содержанию]}}
\newcommand{\R}{\mathbb{R}}
\newcommand{\N}{\mathbb{N}}
\newcommand{\dd}{\mathop{}\!\mathrm{d}} % for differentials
\newcommand{\grad}{\operatorname{grad}}
\newcommand{\dive}{\operatorname{div}}
% Для возможной вставки изображения диаграммы

\newtheorem{theorem}{Теорема}[section] % Нумерация теорем по разделам
\newtheorem{definition}[theorem]{Определение}
\newtheorem{lemma}[theorem]{Лемма}
\newtheorem{corollary}[theorem]{Следствие}
\newenvironment{proof}[1][\proofname]{\par\noindent\textit{#1.}\quad}{\hfill$\square$}
\renewcommand{\proofname}{Доказательство}



\newcommand{\mes}{\operatorname{mes}} % мера

\newcommand{\Dcal}{\mathcal{D}} % Пространство основных функций
\newcommand{\Dprime}{\mathcal{D}'} % Пространство обобщенных функций
\newcommand{\Sprime}{\mathcal{S}'} % Пространство обобщенных функций медленного роста
\newcommand{\Eprime}{\mathcal{E}'} % Пространство обобщенных функций с компактным носителем
\newcommand{\supp}{\operatorname{supp}}

% Поля (можно настроить по вкусу)
\usepackage{geometry}

\geometry{left=2cm, right=2cm, top=2cm, bottom=2cm}

\title{Приведение к каноническому виду уравнений второго порядка}
\author{} % Автор не указан
\date{}   % Дату можно убрать или поставить \today

\begin{document}

% \maketitle % Раскомментируйте, если нужен заголовок

\section*{Содержание}\label{toc}
\begin{itemize}
  \item \hyperref[sec:ticket1]{Билет №1 — 2025 - Приведение к каноническому виду уравнений второго порядка вокрестности точки в случае двух переменных.}
  
  \item \hyperref[sec:ticket2]{Билет №2 — 2025 -   Задача Коши для уравнения колебания струны. Представление решения. Принцип Дюамеля. Формула Даламбера. Теорема существования и единственности классического решения}

  \item \hyperref[sec:ticket4]{Билет №4 — 2025 - Смешанная задача для волнового уравнения в Rn. Закон сохранения энергии. Априорные оценки решения. Единственность классического решения + Корректность смешанной задачи для волнового уравнения в G из априорной оценки решения.}
  
  \item \hyperref[sec:ticket5]{Билет №5 — 2025 - Понятие о корректной постановке задачи математической физики (по Адамару). Пример Адама - ра некорректно поставленной задачи Коши (для уравнения Лапласа). Корректность смешанной
задачи для волнового уравнения в области из априорной оценки
решения}
  
  \item \hyperref[sec:ticket6]{Билет №6 — 2025 - Задача Коши для волнового уравнения в R3 и R2. Принцип Дюамеля. Формула Кирхгофа. Метод Спуска.
Формула Пуассона}
  
  \item \hyperref[sec:ticket7]{Билет №7 — 2025 - Метод Фурье решения начально-краевой задачи для уравнения колебания струны на конечном отрезке. Условия согласования начальных и граничных данных.}
  
  \item \hyperref[sec:ticket8_9]{Билеты №8 и №9 — 2025}
  \item \hyperref[sec:ticket10]{Билет №10 — 2025 -   Дельта - образная последовательность. Формула Пуассона для решения уравнения теплопроводности с непрерывной начальной функцией. }

  \item \hyperref[sec:ticket11]{Билет №11 — 2025}
  Пространства Шварца S и S′. Свертка функций, её
свойства
  \item \hyperref[sec:ticket12]{Билет №12 — 2025 -   Преобразование Фурье и его свойства. Фундаментальное решение оператора с постоянными коэффициентами помогает найти частное решение
уравнения.}

  \item \hyperref[sec:ticket13]{Билет №13 — 2025 -   Фундаментальное решение оператора Лапласа для n=3. Фундаментальное решение оператора теплопроводности (без доказательства) }

  \item \hyperref[sec:ticket14]{Билет №14 — 2025 -   Симметрический оператор в гильбертовом пространстве,
свойства его собственных чисел и собственных функций. Формулы Грина и симмет-
ричность оператора Лапласа в L2(G) с гранич ны- ми условиями трех видов. Теорема
Гильберта-Шмидта и полнота системы собственных функций оператора Лапласа с
граничными условиями (без доказательства).}

  \item \hyperref[sec:ticket15]{Билет №15 — 2025 -   Формула представления решения уравнения Пуассона. Потенциалы, их физический смысл и их свойства (без доказательства).}

  \item \hyperref[sec:ticket16]{Билет №16 — 2025 -   Функция Грина задачи Дирихле для уравнения Пуассона.
Интеграл Пуассона в R3 .}

  \item \hyperref[sec:ticket17]{Билет №17 — 2025 -   Теорема о среднем для гармонических функций. Принцип
максимума для гармонических функций в ограниченной области. Единственность
классического решения задачи Дирихле для уравнения Пуассона в ограниченной
области.}

  \item \hyperref[sec:ticket18]{Билет №18 — 2025 - Определение слабого решения задачи Дирихле для уравнения Пуассона. Теорема существования и единственности слабого решения краевой
задачи для уравнения Пуассона в ограниченной области. }
  
  \item \hyperref[sec:ticket19]{Билет №19 — 2025}
  \item \hyperref[sec:ticket20]{Билет №20 — 2025}
\end{itemize}

\newpage

\newpage

\section*{Билет №1 -- 2025}\label{sec:ticket1}
\backtotoc
\subsection*{Приведение к каноническому виду уравнений второго порядка в окрестности точки в случае
двух переменных. [2] -- 17-23 или [1] -- 47-52.}
\noindent\textbf{Приведение к каноническому виду уравнений второго порядка в окрестности точки в случае двух переменных.} [4, с. 17--23]

\paragraph{$n=2$:}
Рассмотрим приведение к каноническому виду, что в $U(x_0)$ при $n \ge 3$ сделать невозможно. \\
В двумерном случае это возможно в $U(x_0, y_0)$.

% --- Начало блока для диаграммы ---
\begin{figure}[h!]
\centering
% Здесь должна быть диаграмма. Пример с includegraphics:
% \includegraphics[width=0.8\textwidth]{your_diagram_filename.png}
% Или нарисовать с помощью TikZ.
% Ниже приведены текстовые элементы с диаграммы:
% Левая часть: область \Omega с точкой (x_0, y_0) и окрестностью U(x_0, y_0). Оси x, y.
% Стрелка "диффеоморфизм" к правой части.
% Правая часть: область V(\xi_0, \eta_0) с точкой (\xi_0, \eta_0). Оси \xi, \eta.
% \xi_0 = \xi(x_0, y_0)
% \eta_0 = \eta(x_0, y_0)
\caption{Иллюстрация диффеоморфизма для смены координат.}
\label{fig:diffeomorphism}
\end{figure}
% --- Конец блока для диаграммы ---

Рассмотрим общий вид квазилинейного уравнения второго порядка в $\mathbb{R}^2$:
\begin{equation} \label{eq:quasilinear}
a(x,y) u_{xx} + 2b(x,y) u_{xy} + c(x,y) u_{yy} + F(x, y, u, \nabla u) = 0 \tag{$*$}
\end{equation}
Матрица коэффициентов при старших производных:
\[
A(x,y) = \begin{pmatrix} a(x,y) & b(x,y) \\ b(x,y) & c(x,y) \end{pmatrix}
\]
Обозначим дискриминант: $d = \det(A(x,y)) = ac - b^2$.

Сделаем невырожденную замену координат:
\[
\begin{cases}
\xi = \xi(x,y) \\
\eta = \eta(x,y)
\end{cases}
\]
в некоторой окрестности $U(x_0, y_0)$ точки $(x_0, y_0)$.
Это -- диффеоморфизм класса $C^2$: $\xi \in C^2(U(x_0, y_0))$, $\eta \in C^2(U(x_0, y_0))$, такой что якобиан
\[
\begin{vmatrix} \xi_x & \xi_y \\ \eta_x & \eta_y \end{vmatrix} \neq 0 \quad \text{в } U(x_0, y_0).
\]
Считаем также, что $a, b, c, u$ принадлежат $C^2(U(x_0, y_0))$ или $C^2(\Omega)$.

Уравнение в новых переменных $(\xi, \eta)$:
\[
\hat{a}(\xi,\eta) \hat{u}_{\xi\xi} + 2\hat{b}(\xi,\eta) \hat{u}_{\xi\eta} + \hat{c}(\xi,\eta) \hat{u}_{\eta\eta} + \hat{F}(\xi, \eta, \hat{u}, \nabla_{\xi,\eta} \hat{u}) = 0
\]
Матрица коэффициентов нового уравнения:
\[
\hat{A} = \begin{pmatrix} \hat{a} & \hat{b} \\ \hat{b} & \hat{c} \end{pmatrix}
% = J A(x(\xi,\eta), y(\xi,\eta)) J^T, \quad \text{где } J = J(x(\xi,\eta), y(\xi,\eta))
\]
% Якобиан преобразования координат \xi(x,y), \eta(x,y)
Матрица Якоби $J = \begin{pmatrix} \xi_x & \xi_y \\ \eta_x & \eta_y \end{pmatrix}$.
% (Примечание: в оригинале J = || ... ||, что может означать матрицу или ее определитель.
% Формула J A J^T относится к преобразованию квадратичной формы,
% но коэффициенты PDE преобразуются по более сложным формулам, которые приведены ниже.)

Запишем вид элементов матрицы $\hat{A}$ (коэффициентов при старших производных в новых координатах):
Общая формула для преобразования коэффициентов $\alpha_{ij}$ тензора второго ранга при замене координат $x \to y$:
\[ \hat{\alpha}_{kl}(y) = \sum_{i,j=1}^n \alpha_{ij}(x(y)) \frac{\partial y_k}{\partial x_i} \frac{\partial y_l}{\partial x_j} \]
Применительно к нашим обозначениям ($y_1=\xi, y_2=\eta$, $x_1=x, x_2=y$):
\begin{align*}
\hat{a}(\xi,\eta) &= a(x,y) (\xi_x)^2 + 2b(x,y) \xi_x \xi_y + c(x,y) (\xi_y)^2 \\
\hat{c}(\xi,\eta) &= a(x,y) (\eta_x)^2 + 2b(x,y) \eta_x \eta_y + c(x,y) (\eta_y)^2 \\
\hat{b}(\xi,\eta) &= a(x,y) \xi_x \eta_x + b(x,y) (\xi_x \eta_y + \xi_y \eta_x) + c(x,y) \xi_y \eta_y
\end{align*}
где $a, b, c$ берутся в точке $(x(\xi,\eta), y(\xi,\eta))$.

Напомним, что мы предполагаем $a(x,y), b(x,y), c(x,y) \in C^2(\Omega)$. Найдём такое диффеоморфное преобразование, которое приведёт исходное уравнение \eqref{eq:quasilinear} в области $V(\xi_0, \eta_0)$ к каноническому виду.

\paragraph{Рассмотрим отдельно случаи:} (используем результат классификации в $\mathbb{R}^2$ через $\det(A)$)

\subsection*{I. Гиперболический случай, в т. $(x_0, y_0)$}
Пусть исходное уравнение в точке $(x_0, y_0)$ имеет гиперболический вид, тогда $\det(A(x_0, y_0)) < 0$, т.е. $d < 0$ в точке $(x_0, y_0)$.
В силу непрерывности коэффициентов $a, b, c$ существует окрестность точки $(x_0, y_0)$, в которой $d < 0$.

% ... (начало документа из предыдущего ответа) ...

% \subsection*{I. Гиперболический случай, в т. $(x_0, y_0)$} % Уже было объявлено
% Пусть исходное уравнение в точке $(x_0, y_0)$ имеет гиперболический вид, тогда $\det(A(x_0, y_0)) < 0$, т.е. $d < 0$ в точке $(x_0, y_0)$.
% В силу непрерывности коэффициентов $a, b, c$ существует окрестность точки $(x_0, y_0)$, в которой $d < 0$.

При этом исходная окрестность $U(x_0, y_0)$ может сузиться. Приведём исходное уравнение к 2-ой канонической форме (остаются только смешанные производные):
\[
\hat{b}(\xi,\eta) \hat{u}_{\xi\eta} + \hat{F}(\xi, \eta, \hat{u}, \nabla_{\xi,\eta} \hat{u}) = 0
\]
(Если $\hat{b}(\xi,\eta) \neq 0$, можно разделить на $\hat{b}$ или поглотить его в $\hat{F}$, чтобы получить $\hat{u}_{\xi\eta} + \dots = 0$).
Для этого нужно, чтобы коэффициенты при $\hat{u}_{\xi\xi}$ и $\hat{u}_{\eta\eta}$ были равны нулю:
\[
\text{Т.е. требуем в окр. } V(\xi_0, \eta_0): \quad
\begin{cases}
\hat{a}(\xi,\eta) = 0 \\
\hat{c}(\xi,\eta) = 0
\end{cases}
\quad \forall (\xi,\eta) \in V(\xi_0, \eta_0)
\]
Используя формулы для $\hat{a}$ и $\hat{c}$ из предыдущей части:
\[
\Leftrightarrow \quad
\begin{cases}
a(x,y)\xi_x^2 + 2b(x,y)\xi_x\xi_y + c(x,y)\xi_y^2 = 0 \\
a(x,y)\eta_x^2 + 2b(x,y)\eta_x\eta_y + c(x,y)\eta_y^2 = 0
\end{cases}
\quad \forall (x,y) \in U(x_0, y_0)
\]
Это одинаковые уравнения относительно функций $\xi(x,y)$ и $\eta(x,y)$. Обозначим искомую функцию через $\omega(x,y)$, т.е. $\omega \in \{\xi, \eta\}$.
Т.е. хотим решить уравнение (уравнение характеристик):
\begin{equation} \label{eq:char_pde}
a(x,y)\omega_x^2 + 2b(x,y)\omega_x\omega_y + c(x,y)\omega_y^2 = 0, \quad (x,y) \in U(x_0, y_0) \quad (1)
\end{equation}
При этом должно выполняться условие невырожденности преобразования (Якобиан не равен нулю):
\[
\mathcal{J}(x,y) = \begin{vmatrix} \xi_x(x,y) & \xi_y(x,y) \\ \eta_x(x,y) & \eta_y(x,y) \end{vmatrix} \neq 0 \quad \forall (x,y) \in U(x_0, y_0)
\]
Это влечёт, в частности, что $\nabla \xi(x,y) \neq 0$ и $\nabla \eta(x,y) \neq 0$, и градиенты $\nabla \xi$ и $\nabla \eta$ не коллинеарны.
Т.о. необходимо получить два решения $\xi(x,y), \eta(x,y)$ уравнения \eqref{eq:char_pde}, при условии, что $\omega(x,y) \in C^2(U(x_0,y_0))$ и $\nabla \omega(x,y) \neq 0 \quad \forall (x,y) \in U(x_0,y_0)$.
Уравнение \eqref{eq:char_pde} является уравнением характеристик, поскольку последнее задаётся как $\sum_{i,j=1}^n a_{ij} \frac{\partial \omega}{\partial x_i} \frac{\partial \omega}{\partial x_j} = 0$.
Функции $\xi, \eta$ называются \textit{характеристическими переменными}.

\paragraph{I.1)} Пусть $a(x_0, y_0) \neq 0$ (или $c(x_0, y_0) \neq 0$; аналогично, если $a=0$, но $c \neq 0$, можно поменять ролями $x$ и $y$ или решить относительно $\omega_y/\omega_x$).
В силу непрерывности $a(x,y)$, существует (возможно, меньшая) окрестность $U(x_0, y_0)$, где $a(x,y) \neq 0$.
Поделим обе части уравнения \eqref{eq:char_pde} на $a(x,y) \neq 0 \quad \forall (x,y) \in U(x_0, y_0)$:
\[
\left(\frac{\omega_x}{\omega_y}\right)^2 + 2\frac{b}{a}\left(\frac{\omega_x}{\omega_y}\right) + \frac{c}{a} = 0 \quad (\text{если } \omega_y \neq 0)
\]
Или, работая с исходным уравнением:
\[
\omega_x^2 + 2\frac{b}{a}\omega_x\omega_y + \frac{c}{a}\omega_y^2 = 0
\]
Дополним до полного квадрата:
\[
\left(\omega_x + \frac{b}{a}\omega_y\right)^2 - \left(\frac{b}{a}\right)^2\omega_y^2 + \frac{c}{a}\omega_y^2 = 0
\]
\[
\left(\omega_x + \frac{b}{a}\omega_y\right)^2 - \frac{b^2-ac}{a^2}\omega_y^2 = 0
\]
Поскольку рассматривается гиперболический тип, $d = ac-b^2 < 0 \implies b^2-ac > 0$.
Тогда уравнение раскладывается на множители (предполагая $a>0$ для простоты $\sqrt{a^2}=a$, иначе $|a|$):
\[
\left( \omega_x + \frac{b}{a}\omega_y - \frac{\sqrt{b^2-ac}}{a}\omega_y \right) \left( \omega_x + \frac{b}{a}\omega_y + \frac{\sqrt{b^2-ac}}{a}\omega_y \right) = 0
\]
\[
\left( \omega_x + \left(\frac{b - \sqrt{b^2-ac}}{a}\right)\omega_y \right) \left( \omega_x + \left(\frac{b + \sqrt{b^2-ac}}{a}\right)\omega_y \right) = 0
\]
Обозначим
\[ \Omega_1(x,y) = \frac{b - \sqrt{b^2-ac}}{a}, \quad \Omega_2(x,y) = \frac{b + \sqrt{b^2-ac}}{a} \]
Поскольку $b^2-ac > 0$, то $\Omega_1(x,y) \neq \Omega_2(x,y)$ в $U(x_0,y_0)$.
Если $a,b,c \in C^2(U(x_0,y_0))$, и $b^2-ac > 0$, то $\Omega_{1,2}(x,y) \in C^2(U(x_0,y_0))$.
Таким образом, уравнение \eqref{eq:char_pde} эквивалентно совокупности двух линейных однородных дифференциальных уравнений первого порядка:
\begin{align}
\omega_x(x,y) + \Omega_1(x,y)\omega_y(x,y) &= 0 \label{eq:char_ode1_pde_form} \\
\omega_x(x,y) + \Omega_2(x,y)\omega_y(x,y) &= 0 \label{eq:char_ode2_pde_form}
\end{align}
Или в более общем виде:
\begin{equation} \label{eq:char_pde_factor_generic}
[\omega_x(x,y) + \Omega(x,y)\omega_y(x,y) = 0] \quad (2)
\end{equation}
где $\Omega(x,y)$ -- это либо $\Omega_1(x,y)$, либо $\Omega_2(x,y)$.

\paragraph{Лемма.}
Пусть есть уравнение $a_1(x,y)\omega_x(x,y) + a_2(x,y)\omega_y(x,y) = 0 \quad (**)$, причём $a_1(x,y), a_2(x,y) \in C^1(U(x_0,y_0))$, и $a_1^2(x,y) + a_2^2(x,y) \neq 0$.
\newline Тогда:
\begin{enumerate}
    \item Любое решение уравнения $(**)$ $\omega(x,y) \in C^2(U(x_0,y_0))$ (если $a_1, a_2 \in C^1 \Rightarrow \omega \in C^2$).
    При этом $\nabla \omega(x,y) \neq 0 \quad \forall (x,y) \in U(x_0,y_0)$.
    \item $\omega(x,y)$ является общим интегралом (т.е. $\omega(x,y)=C$ описывает семейство интегральных кривых) обыкновенного дифференциального уравнения (ОДУ) для характеристик:
    \[ \frac{dx}{a_1(x,y)} = \frac{dy}{a_2(x,y)}, \quad \text{или, если } a_1 \neq 0, \quad \frac{dy}{dx} = \frac{a_2(x,y)}{a_1(x,y)} \]
\end{enumerate}
Верно и обратное.

\textit{Замечание к уравнению $(**)$: оно означает, что вектор $(a_1, a_2)$ ортогонален градиенту $\nabla \omega = (\omega_x, \omega_y)$. Следовательно, векторное поле $(a_1, a_2)$ является касательным к линиям уровня функции $\omega(x,y)$. Эти линии уровня и есть характеристики.}

% ... (Далее идет применение леммы для нахождения \xi и \eta) ...
% Маленькая диаграмма внизу может быть описана или нарисована с помощью TikZ
% \begin{figure}[h!]
% \centering
% % TikZ код для диаграммы характеристик
% \caption{Интегральные кривые (характеристики) уравнения $dy/dx = a_2(x,y)/a_1(x,y)$. Они являются линиями уровня $\omega(x,y)=\text{const}$.}
% \end{figure}
% ... (продолжение из предыдущего ответа) ...

\textit{Продолжение доказательства и применения Леммы:}
\newline
Пусть $\omega(x,y)$ -- решение уравнения $(**)$, т.е. $a_1(x,y)\omega_x(x,y) + a_2(x,y)\omega_y(x,y) = 0$.
Равенство $\omega(x,y)=C$ определяет (локально, если $\omega_y \neq 0$) неявную функцию $y=y(x,C)$, для которой, дифференцируя по $x$, получаем $\omega_x + \omega_y \frac{dy}{dx} = 0$, откуда $\frac{dy}{dx} = -\frac{\omega_x(x,y)}{\omega_y(x,y)}$.
Из уравнения $(**)$, если $\omega_y \neq 0$, имеем $a_1 \frac{\omega_x}{\omega_y} + a_2 = 0 \quad (\lambda)$.
Тогда $\frac{\omega_x}{\omega_y} = -\frac{a_2}{a_1}$.
Следовательно, $\frac{dy}{dx} = - \left(-\frac{a_2(x,y)}{a_1(x,y)}\right) = \frac{a_2(x,y)}{a_1(x,y)}$.
Это означает, что $y=y(x,C)$ является решением обыкновенного дифференциального уравнения (ОДУ)
\[ \frac{dx}{a_1(x,y)} = \frac{dy}{a_2(x,y)} \quad (\lambda\lambda) \]
Обратно, если $\omega(x,y)=C$ -- общий интеграл ОДУ $(\lambda\lambda)$, то вдоль этих кривых $dy/dx = a_2/a_1$.
Поскольку $dy/dx = -\omega_x/\omega_y$, то $-\omega_x/\omega_y = a_2/a_1 \implies a_1\omega_x + a_2\omega_y = 0$ (при условии $a_1 \neq 0, \omega_y \neq 0$).
Условие $\nabla \omega \neq 0$ в Лемме важно, т.к. иначе $dy/dx$ не определена.

Для уравнений \eqref{eq:char_ode1_pde_form} и \eqref{eq:char_ode2_pde_form}, которые мы запишем как
\begin{align}
\omega_x(x,y) + \Omega_+(x,y)\omega_y(x,y) &= 0 \label{eq:char_ode_plus} \\
\omega_x(x,y) + \Omega_-(x,y)\omega_y(x,y) &= 0 \label{eq:char_ode_minus}
\end{align}
(где $\Omega_+(x,y) = \Omega_1(x,y) = \frac{b - \sqrt{b^2-ac}}{a}$ и $\Omega_-(x,y) = \Omega_2(x,y) = \frac{b + \sqrt{b^2-ac}}{a}$ из предыдущего разложения, или наоборот, главное что они различны. Будем считать, что $\Omega_+$ и $\Omega_-$ --- это два различных корня уравнения для характеристических направлений), имеем $a_1=1$, $a_2=\Omega_{\pm}(x,y)$.
Если $a,b,c \in C^2(U(x_0,y_0))$, то $\Omega_{\pm} \in C^2(U(x_0,y_0))$ (в области, где $b^2-ac > 0$ и $a \neq 0$).
Тогда уравнения \eqref{eq:char_ode_plus} и \eqref{eq:char_ode_minus} имеют решения $\xi(x,y)$ и $\eta(x,y)$ соответственно, обладающие свойствами из Леммы:
$\xi(x,y), \eta(x,y) \in C^2(U(x_0,y_0))$, $\nabla\xi \neq 0$, $\nabla\eta \neq 0$.
\begin{numcases}{}
  \xi_x(x,y) + \Omega_+(x,y)\xi_y(x,y) = 0 \implies \xi_x = -\Omega_+(x,y)\xi_y(x,y) \label{eq:xi_pde_solved} \\
  \eta_x(x,y) + \Omega_-(x,y)\eta_y(x,y) = 0 \implies \eta_x = -\Omega_-(x,y)\eta_y(x,y) \label{eq:eta_pde_solved}
\end{numcases}
Остаётся вопрос: является ли полученное отображение $(x,y) \mapsto (\xi(x,y), \eta(x,y))$ взаимнооднозначным (диффеоморфизмом)?
Рассмотрим матрицу Якоби в точке $(x_0, y_0)$ (или в произвольной точке окрестности):
\begin{align*}
\mathcal{J}(x_0,y_0) &= \begin{vmatrix} \xi_x(x_0,y_0) & \xi_y(x_0,y_0) \\ \eta_x(x_0,y_0) & \eta_y(x_0,y_0) \end{vmatrix} \\
&= \begin{vmatrix} -\Omega_+(x_0,y_0)\xi_y(x_0,y_0) & \xi_y(x_0,y_0) \\ -\Omega_-(x_0,y_0)\eta_y(x_0,y_0) & \eta_y(x_0,y_0) \end{vmatrix} \\
&= -\Omega_+(x_0,y_0)\xi_y(x_0,y_0)\eta_y(x_0,y_0) - \xi_y(x_0,y_0)(-\Omega_-(x_0,y_0)\eta_y(x_0,y_0)) \\
&= (\Omega_-(x_0,y_0) - \Omega_+(x_0,y_0)) \xi_y(x_0,y_0) \eta_y(x_0,y_0)
\end{align*}
Поскольку мы в гиперболическом случае, $\Omega_-(x_0,y_0) \neq \Omega_+(x_0,y_0)$.
По Лемме, $\nabla\xi \neq 0$ и $\nabla\eta \neq 0$. Если $\xi_y(x_0,y_0)=0$, то из \eqref{eq:xi_pde_solved} $\xi_x(x_0,y_0)=0$, что противоречит $\nabla\xi \neq 0$. Следовательно, $\xi_y(x_0,y_0) \neq 0$. Аналогично $\eta_y(x_0,y_0) \neq 0$.
Таким образом, $\mathcal{J}(x_0,y_0) \neq 0$.
По теореме об обратном отображении, существует окрестность $U(x_0,y_0)$ (возможно, меньше исходной), в которой существует обратное преобразование $(\xi,\eta) \mapsto (x,y)$, т.е. преобразование является диффеоморфизмом класса $C^2$ (так как $\xi, \eta \in C^2$).
Итак, получено преобразование, которое при $a(x_0,y_0) \neq 0$ приводит уравнение \eqref{eq:quasilinear} (гиперболическое в $(x_0,y_0)$) к каноническому виду $\eqref{eq:char_pde}$ с $\hat{a}=0, \hat{c}=0$.

\paragraph{Предложение.} Уравнение для определения характеристических линий можно получить более просто.
Характеристики -- это кривые, вдоль которых уравнение \eqref{eq:char_pde} превращается в ОДУ.
Уравнение \eqref{eq:char_pde} можно получить из квадратичной формы (уравнения характеристических направлений):
\begin{equation} \label{eq:char_quadratic_form}
a(x,y)(dy)^2 - 2b(x,y)dxdy + c(x,y)(dx)^2 = 0 \quad (4)
\end{equation}
(Примечание: знак при $2b$ может отличаться в зависимости от того, как выводится это уравнение, но это не меняет корни $dy/dx$).
Действительно, разделив на $(dx)^2$ (если $dx \neq 0$):
\[ a(x,y)\left(\frac{dy}{dx}\right)^2 - 2b(x,y)\frac{dy}{dx} + c(x,y) = 0 \]
Корни этого квадратного уравнения относительно $dy/dx$:
\[ \frac{dy}{dx} = \frac{2b(x,y) \pm \sqrt{4b^2(x,y) - 4a(x,y)c(x,y)}}{2a(x,y)} = \frac{b(x,y) \pm \sqrt{b^2(x,y)-a(x,y)c(x,y)}}{a(x,y)} \]
Эти два значения для $dy/dx$ как раз и есть наши $\Omega_+(x,y)$ и $\Omega_-(x,y)$ (или $\Omega_1, \Omega_2$ из предыдущего шага, в зависимости от знака перед корнем).
Пусть $\frac{dy}{dx} = \Omega_+(x,y)$ и $\frac{dy}{dx} = \Omega_-(x,y)$.
Тогда $dy - \Omega_+(x,y)dx = 0$ и $dy - \Omega_-(x,y)dx = 0$.
Перемножая эти два уравнения, получаем:
\[ (dy - \Omega_+(x,y)dx)(dy - \Omega_-(x,y)dx) = 0 \]
\[ (dy)^2 - (\Omega_+(x,y) + \Omega_-(x,y))dxdy + \Omega_+(x,y)\Omega_-(x,y)(dx)^2 = 0 \]
Используя выражения для $\Omega_{\pm}$ через $a,b,c$:
$\Omega_+ + \Omega_- = \frac{b+\sqrt{b^2-ac}}{a} + \frac{b-\sqrt{b^2-ac}}{a} = \frac{2b}{a}$.
$\Omega_+\Omega_- = \left(\frac{b+\sqrt{b^2-ac}}{a}\right)\left(\frac{b-\sqrt{b^2-ac}}{a}\right) = \frac{b^2 - (b^2-ac)}{a^2} = \frac{ac}{a^2} = \frac{c}{a}$.
Подставляя:
\[ (dy)^2 - \frac{2b}{a}dxdy + \frac{c}{a}(dx)^2 = 0 \]
Умножая на $a(x,y)$ (которое мы считаем $\neq 0$), получаем в точности уравнение \eqref{eq:char_quadratic_form}.

Таким образом, характеристики $\xi(x,y)=C_1$ и $\eta(x,y)=C_2$ являются двумя независимыми семействами интегральных кривых ОДУ, полученных из \eqref{eq:char_quadratic_form}:
$dy - \Omega_+(x,y)dx = 0 \implies \xi(x,y) = \text{const}$
$dy - \Omega_-(x,y)dx = 0 \implies \eta(x,y) = \text{const}$

\paragraph{Теорема.} Для того, чтобы функция $\omega(x,y)$ была решением уравнения характеристик
\begin{equation} \label{eq:char_pde_again}
a(x,y)\omega_x^2 + 2b(x,y)\omega_x\omega_y + c(x,y)\omega_y^2 = 0 \quad (*)
\end{equation}
необходимо и достаточно, чтобы линии уровня $\omega(x,y)=C$ являлись интегральными кривыми (характеристиками), определяемыми одним из двух ОДУ первого порядка, полученных из
\begin{equation} \label{eq:char_quadratic_form_again}
a(x,y)(dy)^2 - 2b(x,y)dxdy + c(x,y)(dx)^2 = 0 \quad (**)
\end{equation}
\textit{Доказательство аналогично Лемме.}

% ... (продолжение, если есть) ...
% ... (продолжение из предыдущего ответа) ...

\paragraph{I.1) Гиперболический тип (продолжение) -- Явное преобразование}
Итак, в случае \textbf{I.1)} $a(x_0,y_0) \neq 0$ (или $c(x_0,y_0) \neq 0$) гиперболического типа ($d=ac-b^2 < 0$), можно привести уравнение \eqref{eq:quasilinear} к 2-ой канонической форме $\hat{b}(\xi,\eta)\hat{u}_{\xi\eta} + \hat{F}(\xi,\eta,\hat{u},\nabla_{\xi,\eta}\hat{u}) = 0$.
Покажем явно преобразование, приводящее к каноническому виду, используя характеристики $\xi=\xi(x,y)$ и $\eta=\eta(x,y)$.
Более стандартная каноническая форма для гиперболического уравнения --- это форма с волновым оператором. Для этого делается еще одна линейная замена:
Пусть $\xi_0, \eta_0$ — это характеристики.
Новая замена:
\[
\begin{cases}
\alpha = \xi + \eta \\
\beta = \xi - \eta
\end{cases}
\]
И новую функцию $\tilde{u}(\alpha,\beta) = \hat{u}(\xi(\alpha,\beta), \eta(\alpha,\beta))$.
Тогда
\begin{align*}
\frac{\partial \hat{u}}{\partial \xi} &= \frac{\partial \tilde{u}}{\partial \alpha}\frac{\partial \alpha}{\partial \xi} + \frac{\partial \tilde{u}}{\partial \beta}\frac{\partial \beta}{\partial \xi} = \tilde{u}_{\alpha} \cdot 1 + \tilde{u}_{\beta} \cdot 1 = \tilde{u}_{\alpha} + \tilde{u}_{\beta} \\
\frac{\partial \hat{u}}{\partial \eta} &= \frac{\partial \tilde{u}}{\partial \alpha}\frac{\partial \alpha}{\partial \eta} + \frac{\partial \tilde{u}}{\partial \beta}\frac{\partial \beta}{\partial \eta} = \tilde{u}_{\alpha} \cdot 1 + \tilde{u}_{\beta} \cdot (-1) = \tilde{u}_{\alpha} - \tilde{u}_{\beta}
\end{align*}
Следовательно, оператор смешанной производной:
\begin{align*}
\frac{\partial^2 \hat{u}}{\partial \xi \partial \eta} &= \frac{\partial}{\partial \xi} \left( \frac{\partial \hat{u}}{\partial \eta} \right) = \frac{\partial}{\partial \xi} (\tilde{u}_{\alpha} - \tilde{u}_{\beta}) \\
&= \left( \frac{\partial}{\partial \alpha}\frac{\partial \alpha}{\partial \xi} + \frac{\partial}{\partial \beta}\frac{\partial \beta}{\partial \xi} \right) (\tilde{u}_{\alpha} - \tilde{u}_{\beta}) \\
&= \left( \frac{\partial}{\partial \alpha} + \frac{\partial}{\partial \beta} \right) (\tilde{u}_{\alpha} - \tilde{u}_{\beta}) \\
&= \frac{\partial^2 \tilde{u}}{\partial \alpha^2} - \frac{\partial^2 \tilde{u}}{\partial \alpha \partial \beta} + \frac{\partial^2 \tilde{u}}{\partial \beta \partial \alpha} - \frac{\partial^2 \tilde{u}}{\partial \beta^2} \\
&= \tilde{u}_{\alpha\alpha} - \tilde{u}_{\beta\beta} \quad (\text{при условии } \tilde{u} \in C^2)
\end{align*}
Таким образом, уравнение $\hat{b}\hat{u}_{\xi\eta} + \hat{F} = 0$ (где $\hat{b}$ может быть поглощено в $\hat{F}$ или вынесено как коэффициент) преобразуется в
\begin{equation} \label{eq:hyperbolic_canonical_wave}
k(\alpha,\beta)(\tilde{u}_{\alpha\alpha} - \tilde{u}_{\beta\beta}) + \tilde{F}(\alpha,\beta,\tilde{u},\nabla_{\alpha,\beta}\tilde{u}) = 0
\end{equation}
Это и есть канонический вид уравнения гиперболического типа.

\paragraph{I.2) Гиперболический тип, случай $a(x,y) \equiv c(x,y) \equiv 0 \quad \forall (x,y) \in U(x_0,y_0)$.}
В данном случае уравнение \eqref{eq:quasilinear} имеет вид $2b(x,y)u_{xy} + F(x,y,u,\nabla u) = 0$.
Коэффициент $b(x,y) \neq 0 \quad \forall (x,y) \in U(x_0,y_0)$, поскольку иначе (если $b(x_0,y_0)=0$) уравнение в точке $(x_0,y_0)$ не было бы уравнением второго порядка (или сводилось бы к $F=0$, что не является УЧП второго порядка).
Дискриминант $d = ac - b^2 = 0 \cdot 0 - b^2 = -b^2 < 0$ (так как $b \neq 0$).
Уравнение уже имеет вид $u_{xy} + \frac{1}{2b(x,y)}F(x,y,u,\nabla u) = 0$, что является канонической формой (второй).
Сразу получим 2-ю каноническую форму.
Характеристики здесь тривиальны: $\xi=x, \eta=y$.

\paragraph{I.3) Гиперболический тип, случай $a(x_0,y_0) = c(x_0,y_0) = 0$, но $a(x,y)$ и $c(x,y)$ не равны тождественно нулю в $U(x_0,y_0)$.}
То есть, $a(x_0,y_0)=0, c(x_0,y_0)=0$, но $b(x_0,y_0) \neq 0$ (для гиперболичности $d(x_0,y_0) = -b^2(x_0,y_0) < 0$).
В $U(x_0,y_0)$ может существовать точка $(x^*,y^*)$, где $a(x^*,y^*)\neq 0$ или $c(x^*,y^*)\neq 0$.
Если такое $(x^*,y^*)$ существует, то в окрестности $(x^*,y^*)$ применим случай I.1.
Если же $a(x,y)\equiv 0$ и $c(x,y)\equiv 0$ во всей окрестности $U(x_0,y_0)$, то это случай I.2.
Рассмотрим преобразование, которое иногда предлагают для случая $a=c=0, b \neq 0$:
Введём замену
\[
\begin{cases}
\xi = x+y \\
\eta = x-y
\end{cases}
\]
Тогда $\xi_x=1, \xi_y=1, \eta_x=1, \eta_y=-1$.
Используем формулы для $\hat{a}, \hat{b}, \hat{c}$:
\begin{align*}
\hat{a}(\xi,\eta) &= a\xi_x^2 + 2b\xi_x\xi_y + c\xi_y^2 = a(1)^2 + 2b(1)(1) + c(1)^2 = a+2b+c \\
\hat{c}(\xi,\eta) &= a\eta_x^2 + 2b\eta_x\eta_y + c\eta_y^2 = a(1)^2 + 2b(1)(-1) + c(-1)^2 = a-2b+c \\
\hat{b}(\xi,\eta) &= a\xi_x\eta_x + b(\xi_x\eta_y + \xi_y\eta_x) + c\xi_y\eta_y \\
                 &= a(1)(1) + b((1)(-1) + (1)(1)) + c(1)(-1) = a - c
\end{align*}
Если были $a(x_0,y_0)=0, c(x_0,y_0)=0$, то в точке $(\xi_0,\eta_0)$:
$\hat{a}(\xi_0,\eta_0) = 2b(x_0,y_0) \neq 0$.
$\hat{c}(\xi_0,\eta_0) = -2b(x_0,y_0) \neq 0$.
$\hat{b}(\xi_0,\eta_0) = 0$.
Т.о. свели к случаю, где коэффициенты при производных по $\xi\xi$ и $\eta\eta$ не равны нулю, а коэффициент при смешанной производной равен нулю. Это уравнение вида $\hat{a}\tilde{u}_{\xi\xi} + \hat{c}\tilde{u}_{\eta\eta} + \dots = 0$.
Поскольку $\hat{a}\hat{c} = (2b)(-2b) = -4b^2 < 0$, это всё ещё гиперболический тип.
Такая форма также является канонической (первая каноническая форма для гиперболического типа).
Схема: $U(x;y)$ область, $(x_0;y_0)$ точка.

\begin{theorem}
Если уравнение \eqref{eq:quasilinear} имеет гиперболический вид в точке $(x_0,y_0)$, то найдётся такое преобразование координат, которое приведёт уравнение к каноническому виду \eqref{eq:hyperbolic_canonical_wave} (или $\hat{b}\hat{u}_{\xi\eta}+\dots=0$) в некоторой окрестности этой точки.
\end{theorem}

\subsection*{II. Параболический тип в т. $(x_0,y_0)$}
Пусть исходное уравнение \eqref{eq:quasilinear} в точке $(x_0,y_0)$ имеет параболический тип, тогда $\det(A(x_0,y_0))=0$, т.е. $ac-b^2=0$ в точке $(x_0,y_0)$.
Рассматриваем точку $(x_0,y_0)$ и её окрестность, в которой $ac-b^2=0 \quad \forall (x,y) \in U(x_0,y_0)$.
(Важное уточнение: классификация по типу обычно проводится в точке. Если $ac-b^2=0$ только в $(x_0,y_0)$, а в окрестности нет, то это особая точка).
Мы предполагаем, что $ac-b^2=0$ в некоторой окрестности $U(x_0,y_0)$.
Требуется, чтобы хотя бы один из коэффициентов $a,b,c$ был не равен нулю, иначе уравнение не второго порядка.
Если, например, $a(x_0,y_0) \neq 0$ (или $c(x_0,y_0) \neq 0$). Если и $a(x_0,y_0)=0$ и $c(x_0,y_0)=0$, то из $ac-b^2=0$ следует $b(x_0,y_0)=0$. Тогда все коэффициенты старших производных равны нулю в $(x_0,y_0)$, и это не уравнение второго порядка в этой точке, что недопустимо.
Значит, в силу непрерывности, в (возможно, меньшей) окрестности $U(x_0,y_0)$ будет $a(x,y) \neq 0$ (или $c(x,y) \neq 0$).
Аналогично случаю I.1, пробуем занулить коэффициент $\hat{a}(\xi,\eta)$ (или $\hat{c}(\xi,\eta)$).
Получаем уравнение характеристик:
\[ a(x,y)\omega_x^2 + 2b(x,y)\omega_x\omega_y + c(x,y)\omega_y^2 = 0 \]
Поскольку $ac-b^2=0$, это выражение является полным квадратом (если $a \neq 0$):
\[ a\left(\omega_x + \frac{b}{a}\omega_y\right)^2 = 0 \quad \text{или} \quad c\left(\frac{b}{c}\omega_x + \omega_y\right)^2 = 0 \quad (\text{если } c \neq 0) \]
Если $a \neq 0$, то $\omega_x + \frac{b}{a}\omega_y = 0$. Это одно уравнение на $\omega$.

% ... (продолжение для параболического типа) ...
% ... (продолжение из предыдущего ответа) ...

\subsection*{II. Параболический тип в т. $(x_0,y_0)$}
% Пусть исходное уравнение \eqref{eq:quasilinear} в точке $(x_0,y_0)$ имеет параболический тип, тогда $\det(A(x_0,y_0))=0$, т.е. $ac-b^2=0$ в точке $(x_0,y_0)$.
% Рассматриваем точку $(x_0,y_0)$ и её окрестность, в которой $ac-b^2=0 \quad \forall (x,y) \in U(x_0,y_0)$.
% ...
% Если $a \neq 0$, то $\omega_x + \frac{b}{a}\omega_y = 0$. Это одно уравнение на $\omega$.

Вспомним, что в гиперболическом случае уравнение характеристик $(\omega_x + \Omega_+\omega_y)(\omega_x + \Omega_-\omega_y)=0$ давало два различных семейства характеристик, так как $\Omega_{\pm}(x,y) = \frac{b \pm \sqrt{b^2-ac}}{a}$ и $b^2-ac > 0$.
В параболическом случае $ac-b^2=0$, поэтому $b^2-ac=0$, и тогда $\Omega_+(x,y) = \Omega_-(x,y) = \frac{b}{a}$ (если $a \neq 0$).
Таким образом, исследование характеристического уравнения $a\omega_x^2 + 2b\omega_x\omega_y + c\omega_y^2 = 0$ свелось к исследованию одного линейного уравнения
\[ \omega_x + \Omega(x,y)\omega_y = 0, \quad \text{где } \Omega(x,y) = \frac{b(x,y)}{a(x,y)} \quad (\text{если } a \neq 0) \]
(или, если $c \neq 0$, то $\frac{b}{c}\omega_x + \omega_y = 0$).
Т.е. можем найти только одно функционально независимое решение $\omega(x,y)$ для замены координат, которое будет одной из новых переменных, скажем $\eta(x,y) = \omega(x,y)$.
Из Леммы (с предыдущих страниц) известно, что $\exists \omega(x,y) \in C^2(U(x_0,y_0))$, $\nabla\omega(x,y) \neq 0 \quad \forall (x,y) \in U(x_0,y_0)$.
Отличие от гиперболического случая состоит в том, что в гиперболическом случае сразу могли найти две функционально независимые переменные $\xi, \eta$ как решения двух различных ОДУ.

Введём замену координат:
\[
\begin{cases}
\xi = \xi(x,y) \quad \text{-- произвольная функция} \\
\eta = \omega(x,y) \quad \text{-- решение характеристического уравнения}
\end{cases}
\]
где $\xi(x,y) \in C^2(U(x_0,y_0))$ и $\nabla\xi(x,y) \neq 0 \quad \forall (x,y) \in U(x_0,y_0)$.
Функцию $\xi(x,y)$ подбираем такой, чтобы вместе с $\eta(x,y)=\omega(x,y)$ осуществлялся диффеоморфизм класса $C^2$ из $U(x_0,y_0)$ в $V(\xi_0,\eta_0)$. Это означает, что $\nabla\xi$ и $\nabla\eta$ должны быть неколлинеарны, т.е. $\mathcal{J} = \xi_x\eta_y - \xi_y\eta_x \neq 0$.
Например, $\xi$ можно выбрать так, чтобы ее линии уровня $\xi(x,y)=\text{const}$ не были касательны к характеристикам $\eta(x,y)=\text{const}$.
Часто $\xi$ выбирают просто: $\xi=x$ или $\xi=y$, если это обеспечивает $\mathcal{J} \neq 0$.
Другой вариант: пусть $\vec{t} = (\eta_y, -\eta_x)$ --- касательный вектор к характеристике $\eta(x,y)=C$. Тогда $\xi$ можно выбрать так, чтобы $\nabla\xi \cdot \vec{t} \neq 0$. Например, $\xi = \vec{r} \cdot \vec{n}$, где $\vec{n}$ - некоторый постоянный вектор, не ортогональный $\vec{t}$.
Например, можно выбрать $\xi = (\vec{x}-\vec{x}_0) \cdot \vec{\tau}$, где $\vec{\tau}$ --- касательный вектор к поверхности $S: \omega(x,y)=c$ в точке $\vec{x}_0$, а $\nabla \omega$ --- нормаль.
Или $\xi(x,y)$ может быть выбрана как $(x-x_0) n_x + (y-y_0) n_y$, где $\vec{n}=(\eta_y(x_0,y_0), -\eta_x(x_0,y_0))$ --- вектор касательной к характеристике в точке $(x_0,y_0)$.
(Маленькая диаграмма: кривая $S: \omega(x,y)=C$, точка $(x_0,y_0)$ на ней, векторы $\nabla\omega$ (нормаль) и $\vec{\tau}$ (касательная)).

В общем случае, из-за выбора $\eta(x,y)=\omega(x,y)$ как решения $a\eta_x^2 + 2b\eta_x\eta_y + c\eta_y^2 = 0$, коэффициент $\hat{c}(\xi,\eta)$ в новых координатах будет равен нулю:
$\hat{c}(\xi,\eta) = a\eta_x^2 + 2b\eta_x\eta_y + c\eta_y^2 = 0 \quad \forall (\xi,\eta) \in V(\xi_0,\eta_0)$,
так как $\eta$ является решением характеристического уравнения.
То есть занулится только один коэффициент при старшей производной, связанный с $\eta$.

Покажем, что $\hat{b}(\xi,\eta) = 0 \quad \forall (\xi,\eta) \in V(\xi_0,\eta_0)$.
Рассмотрим, как меняется определитель матрицы старших коэффициентов при преобразовании:
$\det(\hat{A}) = \det(A) \cdot (\det(J))^2$, где $J = \begin{pmatrix} \xi_x & \xi_y \\ \eta_x & \eta_y \end{pmatrix}$.
Поскольку мы находимся в параболическом случае (и предполагаем, что тип сохраняется в окрестности), $\det(A) = ac-b^2 = 0 \quad \forall (x,y) \in U(x_0,y_0)$.
Следовательно, $\det(\hat{A}) = 0 \quad \forall (\xi,\eta) \in V(\xi_0,\eta_0)$.
Мы знаем, что $\det(\hat{A}) = \hat{a}\hat{c} - \hat{b}^2$.
Так как $\hat{c}(\xi,\eta)=0$, то $\det(\hat{A}) = \hat{a}(\xi,\eta) \cdot 0 - \hat{b}^2(\xi,\eta) = -\hat{b}^2(\xi,\eta)$.
Из $\det(\hat{A})=0$ следует $-\hat{b}^2(\xi,\eta) = 0$, откуда $\hat{b}(\xi,\eta) = 0 \quad \forall (\xi,\eta) \in V(\xi_0,\eta_0)$.

Таким образом, уравнение в новых координатах принимает вид:
\[ \hat{a}(\xi,\eta)\hat{u}_{\xi\xi} + \hat{F}(\xi,\eta,\hat{u},\nabla_{\xi,\eta}\hat{u}) = 0 \]
(где $\nabla_{\xi,\eta}\hat{u}$ будет содержать $\hat{u}_\xi$ и $\hat{u}_\eta$, но не $\hat{u}_{\xi\eta}$ и $\hat{u}_{\eta\eta}$).

Покажем, что $\hat{a}(\xi,\eta) \neq 0 \quad \forall (\xi,\eta) \in V(\xi_0,\eta_0)$.
Действительно, если бы в некоторой точке $(\xi^*,\eta^*) \in V(\xi_0,\eta_0)$ оказалось $\hat{a}(\xi^*,\eta^*)=0$, то вместе с $\hat{b}(\xi^*,\eta^*)=0$ и $\hat{c}(\xi^*,\eta^*)=0$, это означало бы, что уравнение в этой точке $(\xi^*,\eta^*)$ (и, соответственно, в соответствующей $(x^*,y^*)$) не является уравнением второго порядка, что противоречит нашему исходному предположению.
Мы строили диффеоморфизм между областями, значит, если бы после преобразования уравнение стало уравнением первого порядка в $V(\xi_0,\eta_0)$, то и обратное преобразование дало бы уравнение первого порядка в $U(x_0,y_0)$, что противоречит условию.
Следовательно, $\hat{a}(\xi,\eta) \neq 0 \quad \forall (\xi,\eta) \in V(\xi_0,\eta_0)$.
Разделив на $\hat{a}(\xi,\eta)$, получим канонический вид для параболического уравнения:
\begin{equation} \label{eq:parabolic_canonical}
\hat{u}_{\xi\xi} + \tilde{F}(\xi,\eta,\hat{u},\hat{u}_\xi,\hat{u}_\eta) = 0
\end{equation}
где $\tilde{F} = \hat{F}/\hat{a}$.

% ... (продолжение для эллиптического типа, если есть) ...
% ... (продолжение из предыдущего ответа, вывод формул для эллиптического типа) ...

% Мы получили:
% \begin{align*}
% \hat{a}(\xi,\eta) - \hat{c}(\xi,\eta) &= 0 \quad \implies \hat{a}(\xi,\eta) = \hat{c}(\xi,\eta) \\
% \hat{b}(\xi,\eta) &= 0
% \end{align*}
% Используя выражения для \hat{a}, \hat{b}, \hat{c}:
% \[
% \underbrace{[a(\xi_x^2-\eta_x^2) + 2b(\xi_x\xi_y-\eta_x\eta_y) + c(\xi_y^2-\eta_y^2)]}_{\text{Вещественная часть} = \hat{a}-\hat{c}}
% + i \cdot 2 \underbrace{[a\xi_x\eta_x + b(\xi_x\eta_y+\xi_y\eta_x) + c\xi_y\eta_y]}_{\text{Мнимая часть}/2 = \hat{b}} = 0
% \]
% Отсюда следует $\hat{a}(\xi,\eta) = \hat{c}(\xi,\eta)$ и $\hat{b}(\xi,\eta)=0$.

Исходное уравнение \eqref{eq:quasilinear} принимает вид:
\[
\hat{a}(\xi,\eta)\hat{u}_{\xi\xi} + \hat{a}(\xi,\eta)\hat{u}_{\eta\eta} + \hat{F}(\xi,\eta,\hat{u},\nabla_{\xi,\eta}\hat{u}) = 0.
\]
Аналогично параболическому случаю показывается, что $\hat{a}(\xi,\eta) \neq 0$ (а значит и $\hat{c}(\xi,\eta) \neq 0$) в $V(\xi_0,\eta_0)$, поскольку иначе получили бы уравнение 1-го порядка, а обратное преобразование переведёт его в области $U(x_0,y_0)$ в уравнение 1-го порядка -- противоречие.
Следовательно, разделив на $\hat{a}(\xi,\eta)$, получаем:
\begin{equation} \label{eq:elliptic_canonical_final}
\boxed{
\hat{u}_{\xi\xi} + \hat{u}_{\eta\eta} + \tilde{F}(\xi,\eta,\hat{u},\nabla_{\xi,\eta}\hat{u}) = 0
}
\end{equation}
-- канонический вид уравнения эллиптического типа.

\hrulefill
\section*{N3}
\textit{Как уже было сказано, ... (далее неразборчиво, вероятно, начало новой темы или примера)}

% ... (Если есть еще текст для N3)
%%%------------------------------------------------------------------------------------------------------------------------------------------------------------------------------------------------------------------------------------------------------------------------------------------------------------------------------------------------------------------------------------------------------



% Верхние символы, возможно, идентификаторы
% У М Ф
% N2
\newpage
\section*{Приведение к каноническому виду линейных уравнений второго порядка в точке и их классификация [4]-27-30 + Задача Коши + характ. поверхности.}

\subsection*{Примеры уравнений разных типов:}

\subsubsection*{1) Гиперболические уравнения:}
\begin{itemize}
    \item $u_{tt} = a^2 u_{xx} + f(t,x)$, $x \in \mathbb{R}^1$ -- одномерные малые колебания струны.
    \item $u_{tt} = a^2 (u_{xx} + u_{yy}) + f(t,x,y)$, $(x,y) \in \mathbb{R}^2$ -- малые колебания пластины.
    \item $u_{tt} = a^2 (u_{xx} + u_{yy} + u_{zz}) + f(t,x,y,z)$, $(x,y,z) \in \mathbb{R}^3$ -- распространение сигнала в среде.
\end{itemize}
(1) $u_{tt} = a^2 \Delta u + f(t,x)$, $x \in \mathbb{R}^n$ -- гиперболическое ур-ние, опис. колеб. процессы.

\subsubsection*{2) Параболические уравнения:}
(2) $u_t = a^2 \Delta u + f(t,x)$, $x \in \mathbb{R}^n$ -- распространение тепла в среде.

\subsubsection*{3) Эллиптические уравнения:}
(3) $-\Delta u = f(t,x)$, $x \in \mathbb{R}^n$ -- ур-ние Пуассона ($f(x)=0$ -- ур-ние Лапласа) -- описывает установившиеся колебательные процессы.

\subsection*{Обозначения:}
\begin{enumerate}
    \item $x = (x_1, x_2, \dots, x_n) \in \mathbb{R}^n$ -- точка $n$-мерного евклидова пространства.
\end{enumerate}

\noindent\textbf{Определение.} Пусть $\alpha_1, \dots, \alpha_n$ -- целые неотрицательные числа $\alpha_k \ge 0$. $\alpha = (\alpha_1, \dots, \alpha_n)$ -- \underline{мультииндекс}, $|\alpha| = \alpha_1 + \dots + \alpha_n$ -- \underline{модуль мультииндекса}.

\begin{enumerate}
    \setcounter{enumi}{1} % Продолжаем нумерацию
    \item $D_j = \frac{\partial}{\partial x_j}$ -- оператор частной производной, $D_j^{\alpha_j} = \frac{\partial^{\alpha_j}}{\partial x_j^{\alpha_j}}$.\\
    $D^\alpha = \frac{\partial^{\alpha_1}}{\partial x_1^{\alpha_1}} \frac{\partial^{\alpha_2}}{\partial x_2^{\alpha_2}} \dots \frac{\partial^{\alpha_n}}{\partial x_n^{\alpha_n}} = \frac{\partial^{|\alpha|}}{\partial x_1^{\alpha_1} \dots \partial x_n^{\alpha_n}}$.

    \item $\Omega \subseteq \mathbb{R}^n$ -- область. $C^k(\Omega)$ -- множество функций $n$ переменных: функция и её производные, определяемые мультииндексом $\alpha: |\alpha| \le k$ -- непрерывны. \\
    $\bar{\Omega}$ -- замыкание $\Omega$: $\bar{\Omega} = \Omega \cup \partial \Omega$. \\
    $C^k(\bar{\Omega})$ -- часть функций из $C^k(\Omega)$, такие что функция и её производные до $k$-го порядка могут быть по непрерывности продолжены вплоть до точек границы.
\end{enumerate}

\noindent Пусть $\Omega$ -- область в $\mathbb{R}^n$. Рассмотрим квазилинейное дифф. уравнение второго порядка для произвольного количества переменных.
\[
\underbrace{\sum_{i,j=1}^n a_{ij}(x) \frac{\partial^2 u}{\partial x_i \partial x_j}}_{\text{Старшая часть}} + \underbrace{F(x, u, \nabla u)}_{\text{Младшая часть}} = 0 \quad (*)
\]

\hfill \framebox{1} % Номер страницы в углу







Предполагаем, что $a_{ij}(x) \in C(\Omega)$, $F(x, u, \nabla u)$ -- достаточно гладкая, и $u \in C^2(\Omega) \implies$ вид уравнения зависит только от суммы $a_{ij}(x) + a_{ji}(x)$, т.к. $\frac{\partial^2 u}{\partial x_i \partial x_j} = \frac{\partial^2 u}{\partial x_j \partial x_i}$, т.е. можно считать, что $a_{ij}(x) = a_{ji}(x)$.

Теперь с линейными и квазилинейными ур-ниями можно связать матрицу:
\[
A = \begin{pmatrix}
a_{11}(x) & \dots & a_{1n}(x) \\
\vdots & \ddots & \vdots \\
a_{n1}(x) & \dots & a_{nn}(x)
\end{pmatrix} \quad \text{-- симметр. матр.}
\]

\textbf{Пример:}
Для $u \in C^2(\Omega)$:
$u_{x_1x_1} + 4u_{x_1x_2} + u_{x_2x_2} = 0 \Leftrightarrow u_{x_1x_1} + 2u_{x_1x_2} + 2u_{x_2x_1} + u_{x_2x_2} = 0$.
Тогда матрица $A = \begin{pmatrix} 1 & 2 \\ 2 & 1 \end{pmatrix}$.

\textbf{Замечание.} Все три рассматриваемых типа уравнений вложены в вид $(*)$.
Пусть $y=y(x)$ -- замена координат, $y^0=y(x^0)$.
Преобразование $y$ переносит некоторую окрестность $U(x^0)$ точки $x^0 \in \Omega$ в окрестность $V(y^0)$ точки $y^0 \in W$. (Здесь $\Omega$ и $W$ -- области определения $x$ и $y$ соответственно, $U(x^0) \subseteq \Omega$, $V(y^0) \subseteq W$).
Необходимо найти такое преобр. $y$, чтоб в $W$ ур-ние $(*)$ примет вид (1)-(3). (Гиперболическое, параболическое или эллиптическое).

\underline{Ограничения на $y$:}
\begin{enumerate}
    \item $y = y(x) \in C^2(U(x^0))$ (где $U(x^0)$ -- окрестность $x^0$).
    \item $\exists$ обратное отображение $x = x(y) \in C^2(V(y^0))$ (где $V(y^0)$ -- окрестность $y^0$).
\end{enumerate}
$\iff y$ -- диффеоморфизм класса $C^2$.
\[
y = y(x) = \begin{cases} y_1 = y_1(x_1, \dots, x_n) \\ \vdots \\ y_n = y_n(x_1, \dots, x_n) \end{cases} \in C^2(U(x^0)), \quad
x = x(y) = \begin{cases} x_1 = x_1(y_1, \dots, y_n) \\ \vdots \\ x_n = x_n(y_1, \dots, y_n) \end{cases} \in C^2(V(y^0)).
\]

\textbf{Опр.} $\hat{u}(y) \stackrel{\text{def}}{=} u(x(y)) = u(x_1(y_1, \dots, y_n), \dots, x_n(y_1, \dots, y_n))$ -- новая функция $u$ в обл. $W$.
$u(x) = \hat{u}(y(x)) = \hat{u}(y_1(x_1, \dots, x_n), \dots, y_n(x_1, \dots, x_n))$.
$\hat{u}$ -- вид $u$ в переменных $y$.

Рассмотрим, какому ур-нию будет удовлетворять $\hat{u}$, если $u$ удовлетворяет $(*)$.
Производные преобразуются по правилам:
\[
\frac{\partial u}{\partial x_i} = \sum_{k=1}^n \frac{\partial \hat{u}}{\partial y_k} \frac{\partial y_k}{\partial x_i}
\]
\[
\frac{\partial^2 u}{\partial x_i \partial x_j} = \sum_{k,l=1}^n \frac{\partial^2 \hat{u}}{\partial y_k \partial y_l} \frac{\partial y_k}{\partial x_i} \frac{\partial y_l}{\partial x_j} + \sum_{k=1}^n \frac{\partial \hat{u}}{\partial y_k} \frac{\partial^2 y_k}{\partial x_i \partial x_j}
\]

Подставляем в ур-ние $(*)$: $u$ -- решение $(*)$.
\begin{align*} \label{eq:star_transform_full}
\sum_{i,j=1}^n a_{ij}(x) \left[ \sum_{k,l=1}^n \frac{\partial^2 \hat{u}}{\partial y_k \partial y_l} \frac{\partial y_k}{\partial x_i} \frac{\partial y_l}{\partial x_j} + \sum_{k=1}^n \frac{\partial \hat{u}}{\partial y_k} \frac{\partial^2 y_k}{\partial x_i \partial x_j} \right] + F[x(y), \hat{u}, \nabla \hat{u}] = 0
\end{align*}
(слагаемые с первыми производными $\frac{\partial \hat{u}}{\partial y_k}$ перейдут в $\hat{F}$, т.к. $F$ содержит 1-ю производную $\hat{u}$)
\[
\implies \sum_{k,l=1}^n \underbrace{\left[ \sum_{i,j=1}^n a_{ij}(x(y)) \frac{\partial y_k}{\partial x_i} \frac{\partial y_l}{\partial x_j} \right]}_{= \hat{a}_{kl}(y)} \frac{\partial^2 \hat{u}}{\partial y_k \partial y_l} + \hat{F}[y, \hat{u}, \nabla \hat{u}] = 0
\]
\[
\implies (*): \quad \sum_{k,l=1}^n \hat{a}_{kl}(y) \frac{\partial^2 \hat{u}}{\partial y_k \partial y_l} + \hat{F}[y, \hat{u}, \nabla \hat{u}] = 0,
\]
где
\[
\hat{a}_{kl}(y) = \sum_{i,j=1}^n a_{ij}(x(y)) \frac{\partial y_k}{\partial x_i} \frac{\partial y_l}{\partial x_j} \quad \tag{$\Psi$}
\]
(Элементы матрицы $\hat{A}$)



Выберем и зафиксируем точку $x^0 \in \Omega$. Тогда коэфф. $a_{ij} = a_{ij}(x^0)$ -- фикс. числа.
\[
\implies A(x^0) = \begin{pmatrix}
a_{11}(x^0) & \dots & a_{1n}(x^0) \\
\vdots & \ddots & \vdots \\
a_{n1}(x^0) & \dots & a_{nn}(x^0)
\end{pmatrix}; \quad
\hat{A}(y^0) = \begin{pmatrix}
\hat{a}_{11}(y^0) & \dots & \hat{a}_{1n}(y^0) \\
\vdots & \ddots & \vdots \\
\hat{a}_{n1}(y^0) & \dots & \hat{a}_{nn}(y^0)
\end{pmatrix}
\]
Рассмотрим матр. Якоби преобразования $y(x)$ в т. $x^0$:
\[
\mathcal{J}(x^0) = \begin{pmatrix}
\frac{\partial y_1}{\partial x_1}(x^0) & \dots & \frac{\partial y_1}{\partial x_n}(x^0) \\
\vdots & \ddots & \vdots \\
\frac{\partial y_n}{\partial x_1}(x^0) & \dots & \frac{\partial y_n}{\partial x_n}(x^0)
\end{pmatrix}
\]
С учётом формулы $(\Psi)$ связи $\hat{a}_{kl}(y)$ и $a_{ij}(x(y))$:
\[
\hat{A}(y^0) = \mathcal{J}(x^0) \cdot A(x^0) \cdot \mathcal{J}^T(x^0)
\]
% Промежуточное равенство в тексте: $= (\mathcal{J}^T(x^0))^T \cdot A(x^0) \cdot \mathcal{J}^T(x^0) = S^T \cdot A(x^0) \cdot S$
% где $S = \mathcal{J}^T(x^0)$. Это верно, т.к. $(\mathcal{J}^T)^T = \mathcal{J}$.
\[
\implies \boxed{\hat{A}(y^0) = S^T \cdot A(x^0) \cdot S \quad (1)}, \quad \text{где } S = \mathcal{J}^T(x^0).
\]
Итак, нам необходимо найти такую замену переменных, чтобы коэфф. перед смешанными производными в $(*)$ были нулевыми, т.е. $\hat{A}$ стала диагональной ($\hat{a}_{kl}(y)=0$ при $k \ne l$).

В \underline{общем случае} (в окрестности $x^0$) такое невозможно, т.к. всего имеется $\frac{n^2-n}{2}$ уравнений ($\hat{a}_{kl}(y)=0, k \ne l$, $\hat{A}$ будет диаг. и явл. симм.), а управляющих функций в преобразовании: $n$.
При $n>3: \frac{n^2-n}{2} > n \implies$ задача \underline{переопределена};

\underline{В точке} -- возможно:
Рассмотрим $\mathbb{R}^n$, $h \in \mathbb{R}^n$, $\Phi(h)$ -- некоторая квадратичная форма.
$\{e_1, \dots, e_n\}$ -- базисные вектора, $\{\tilde{e}_1, \dots, \tilde{e}_n\}$ -- второй базис.
Координаты вектора $h$:
в базисе $\{e_1, \dots, e_n\}$: $\xi = \begin{pmatrix} \xi_1 \\ \vdots \\ \xi_n \end{pmatrix}$.
в базисе $\{\tilde{e}_1, \dots, \tilde{e}_n\}$: $\eta = \begin{pmatrix} \eta_1 \\ \vdots \\ \eta_n \end{pmatrix}$.

Матрица квадратичной формы:
в базисе $\{e_i\}$: $C = \begin{pmatrix} C_{11} & \dots & C_{1n} \\ \vdots & \ddots & \vdots \\ C_{n1} & \dots & C_{nn} \end{pmatrix}$.
в базисе $\{\tilde{e}_i\}$: $\tilde{C} = \begin{pmatrix} \tilde{C}_{11} & \dots & \tilde{C}_{1n} \\ \vdots & \ddots & \vdots \\ \tilde{C}_{n1} & \dots & \tilde{C}_{nn} \end{pmatrix}$.

Пусть $S$ -- матрица перехода от базиса $\{e_i\}$ к базису $\{\tilde{e}_j\}$ такая, что старые координаты $\xi$ выражаются через новые координаты $\eta$ как $\xi = S \eta$.
(В тексте указано $\tilde{\xi} = \bar{S} \cdot \eta$ рядом с "матр. перехода", далее используется $\xi^T C \xi = (S\eta)^T C (S\eta)$).
Тогда квадратичная форма:
\[
\Phi(h) = \xi^T C \xi = (S\eta)^T C (S\eta) = \eta^T S^T C S \eta = \eta^T \tilde{C} \eta, \quad \forall \eta \in \mathbb{R}^n
\]
\[
\implies \tilde{C} = S^T C S
\]
-- преобразование матр. кв. формы при переходе к новому базису похоже на преобразование (1) матрицы $A(x^0)$. Для любой квадратичной формы $\exists$ базис (или замена переменных), такой что новая кв. форма имеет канонический вид.

\hfill \framebox{2} % Номер страницы в углу


Матрица $\tilde{C}$ в каноническом виде (диагональная, с $+1$, $-1$ или $0$ на диагонали):
\[
\tilde{C} = \begin{pmatrix}
+1 & & & & & & \\
 & \ddots & & & & \text{\huge O} & \\
 & & +1 & & & & \\
 & & & -1 & & & \\
 & & & & \ddots & & \\
 & \text{\huge O} & & & & -1 & \\
 & & & & & & 0 \\
 & & & & & & & \ddots \\
 & & & & & & & & 0
\end{pmatrix}
\begin{array}{l}
\left. \vphantom{\begin{matrix} +1 \\ \vdots \\ +1 \end{matrix}} \right\} p \\
\left. \vphantom{\begin{matrix} -1 \\ \vdots \\ -1 \end{matrix}} \right\} q \\
\left. \vphantom{\begin{matrix} 0 \\ \vdots \\ 0 \end{matrix}} \right\} n-(p+q)
\end{array}
\]
\[
\implies \Phi(h) = \eta_1^2 + \dots + \eta_p^2 - \eta_{p+1}^2 - \dots - \eta_{p+q}^2.
\]
Такая квадратичная форма $\Phi$ - не единственна, но числа $p$ и $q$ всегда одни и те же вне зависимости от преобр., приводящего форму к канон. виду (закон инерции кв.форм), т.е. не зависит от матр. $S$.

\textbf{Алгоритм преобразования $(*)$ к каноническому виду:}
\begin{enumerate}
    \item Строим квадратичную форму $\Phi(h) = \xi^T \cdot A(x^0) \cdot \xi = \sum_{i,j=1}^n a_{ij}(x^0) \xi_i \xi_j$ по матрице $A(x^0)$.
    \item Приводим $\Phi(h)$ к каноническому виду, находя при этом матр. перехода $S$.
    \item Берём $\mathcal{J}(x^0) = S^T(x^0)$ -- матр. Якоби; для преобразования координат используется именно $\mathcal{J}(x^0)$.
    \item Одно из возможных преобразований: $y = y(x) = y^0 + S^T(x^0) \cdot (x-x^0)$ -- после такого преобразования ур-ние будет иметь \underline{канонический вид}, т.е.
    \[
    \frac{\partial^2 \hat{u}}{\partial y_1^2} + \dots + \frac{\partial^2 \hat{u}}{\partial y_p^2} - \frac{\partial^2 \hat{u}}{\partial y_{p+1}^2} - \dots - \frac{\partial^2 \hat{u}}{\partial y_{p+q}^2} + \hat{F}(y, u, \nabla u) = 0.
    \]
\end{enumerate}
Т.о. доказана теорема:

\textbf{Th.} Любое уравнение вида $(*)$: $\sum_{i,j=1}^n a_{ij}(x) \frac{\partial^2 u}{\partial x_i \partial x_j} + F(x,u,\nabla u) = 0$ может в любой наперёд заданной точке быть приведено к каноническому виду.

\textbf{Определение} [Классификация уравнений]
\begin{enumerate}
    \item Если $p=n$ (или $q=n$), то ур-ние $(*)$ имеет \textit{эллиптический тип} (кв.форма знакоопр.) и в новых переменных имеет вид: $\Delta_y \hat{u}(y) + \hat{F}(y, \hat{u}, \nabla_y \hat{u}) = 0$ ($-\Delta u(x) = f(x)$ -- ур-ние Лапласа).
    \item $p+q=n$, $p>0, q>0$ -- \textit{общий гиперболический тип}.
    \begin{itemize}
        \item $u_{x_1x_1} + u_{x_2x_2} - u_{x_3x_3} - u_{x_4x_4} = f(x_1, x_2, x_3, x_4)$ -- \textit{ультрагиперболический тип} ($p+q=n, p>1, q>1$ (ур-ние колебаний)).
        \item \textit{Нормально гиперболический тип} (или просто гиперболический): $p=1$ и $q=n-1$ (или $p=n-1$ и $q=1$).
    \end{itemize}
    \item $0 < p+q < n$ -- \textit{общий параболический вид} (есть хотя бы один нуль на диагонали матрицы старшей части ур-ния $(*)$).
    \begin{itemize}
        \item $u_{x_1x_1} + u_{x_2x_2} - u_{x_3x_3} = f(x_1,x_2,x_3)$ -- \textit{ультрапараболический тип}.
        \item \textit{Нормально параболический тип} (или просто параболический): $p=n-1$ и $q=0$ (или $p=0$ и $q=n-1$).
    \end{itemize}
\end{enumerate}
Итак, было получено, что при $n>3 (\mathbb{R}^n)$ в окрестности точки уравнение в общем случае нельзя привести к каноническому виду. Преобразовать можно в одной точке невырожденным преобр. $y=y(x)$.



\begin{itemize}
    \item При $n=2$ найти вид уравнения можно проще:
    $\hat{A}(y^0) = \mathcal{J}(x^0) \cdot A(x^0) \cdot \mathcal{J}^T(x^0) \implies |\hat{A}(y^0)| = |\mathcal{J}(x^0)| \cdot |A(x^0)| \cdot |\mathcal{J}^T(x^0)| = |A(x^0)| \cdot |\mathcal{J}(x^0)|^2$.
    $\implies \text{sign}|\hat{A}(y^0)| = \text{sign}|A(x^0)|$, т.к. $|\mathcal{J}(x^0)|^2 > 0$, т.к. $y=y(x)$ невырожд. преобр.
\end{itemize}

\begin{enumerate}
    \item \textit{Эллиптический случай}: $|\hat{A}(y^0)| = \begin{vmatrix} \pm 1 & 0 \\ 0 & \pm 1 \end{vmatrix} = +1 > 0 \iff \underline{|A(x^0)| > 0}$ -- эллиптический тип.
    \item \textit{Гиперболический случай}: $|\hat{A}(y^0)| = \begin{vmatrix} +1 & 0 \\ 0 & -1 \end{vmatrix} = -1 < 0 \iff \underline{|A(x^0)| < 0}$ -- гиперболический тип.
    \item \textit{Параболический случай}: $|\hat{A}(y^0)| = \begin{vmatrix} 1 & 0 \\ 0 & 0 \end{vmatrix} = 0 \iff \underline{|A(x^0)| = 0}$ -- параболический тип.
\end{enumerate}

\textbf{Пример.} Ур-ние Трикоми.
$x_2 \frac{\partial^2 u}{\partial x_1^2} + \frac{\partial^2 u}{\partial x_2^2} = 0$; $A(x^0) = \begin{pmatrix} x_2 & 0 \\ 0 & 1 \end{pmatrix} \implies |A(x^0)| = x_2$.


\begin{tikzpicture}
\draw[->] (0,0) -- (4,0) node[right] {$x_1$};
\draw[->] (0,-1.5) -- (0,1.5) node[above] {$x_2$};
\fill[pattern=north east lines] (0,0) rectangle (3,1);
\node at (1.5,0.5) {элл. тип};
\draw[dashed] (0,0) -- (3,0);
\node at (1.5,-0.3) {парабол. тип};
\fill[pattern=north west lines] (0,-1) rectangle (3,0);
\node at (1.5,-0.7) {гип. тип};
\end{tikzpicture}

Ни в какой окрестности т. с нулевой $x_2$ ур-ние не может быть приведено к каноническому виду, однако при $x_2 \ne 0$ ур-ние уже имеет канонический вид. (При $x_2=0$ оно параболическое).

\textbf{Замечание:} данный способ работает только для $n=2$; при $n=3$:
$A_1 = \begin{pmatrix} 1 & 0 & 0 \\ 0 & 1 & 0 \\ 0 & 0 & 1 \end{pmatrix}$ (элл. тип), $A_2 = \begin{pmatrix} -1 & 0 & 0 \\ 0 & -1 & 0 \\ 0 & 0 & 1 \end{pmatrix}$ (гип. тип),
однако $|A_1| = |A_2|$.

\textbf{Замечание.} Корректность определения классификации уравнений, т.е. независимость от матр. $S$ следует из закона инерции кв.форм.

\subsection*{Задача Коши для ур-ния с част. произв.}
Рассмотрим ур-ние $\sum_{i,j=1}^n a_{ij}(x) \frac{\partial^2 u}{\partial x_i \partial x_j} + F(x,u,\nabla u) = 0^{(*)}, x \in \mathbb{R}^n$, где $a_{ij}(x) \in C$, $F$ -- непр. по совокупности всех своих аргументов.
Пусть $S$ -- поверх. в $\mathbb{R}^n$, $\Omega$ -- обл. в $\mathbb{R}^n$, а в $\Omega \setminus \partial \Omega$ задано ур-ние $(*)$.
б) Пусть в $\bar{\Omega}$ задана поверхность $S: \omega(x) = \omega(x_1, \dots, x_n) = 0$, $\omega(x) \in C^2(\Omega)$, $\nabla \omega(x) \ne 0 \quad \forall x \in S$.
в) Пусть на $S$ задано гладкое \underline{некасательное} векторное поле $\vec{\nu}(x)$.
Т.к. $S$-гладкая, то $\exists \vec{n} \quad \forall x \in S \implies$ т.к. $\vec{\nu}$ некасательное, то $(\vec{\nu}(x); \vec{n}(x)) \ne 0 \quad \forall x \in S$.

\textbf{Задача Коши.} В некоторой окрестности $U(x^0)$ т. $x^0 \in S$ найти функцию $u(x)$, которая удовлетворяет ур-нию $(*)$ в $U(x^0)$ и двум условиям:
\[
\left. u(x) \right|_{S \cap U(x^0)} = u_0(x); \quad \left. \frac{\partial u(x)}{\partial \vec{\nu}} \right|_{S \cap U(x^0)} = u_1(x)
\]

\hfill \framebox{3} % Номер страницы в углу





где $\frac{\partial u}{\partial \vec{\nu}} = (\nabla u, \vec{\nu}) = \sum_{i=1}^n \nu_i(x) \frac{\partial u(x)}{\partial x_i}$.
\begin{itemize}
    \item Для ОДУ 2-го порядка, в частности:
    $\begin{cases} u''(x) + F(x,u,u')=0 \\ u(x^0)=u_0 \\ u'(x^0)=u_1 \end{cases}$
\end{itemize}

Под \underline{корректностью} постановки задачи понимается выполнение следующих трёх условий:
\begin{enumerate}
    \item Решение существует в заданном классе функций.
    \item Это решение является единственным.
    \item Решение непрерывно зависит от начальных и краевых данных.
\end{enumerate}

% Рисунки из текста
\begin{tikzpicture}[scale=0.7]
% n=3
\draw (0,0) -- (3,0) node[right] {$x_2$};
\draw (0,0) -- (0,2) node[above] {$x_3$};
\draw (0,0) -- (-1.5,-0.75) node[below] {$x_1$};
\draw (0.5,0.5) .. controls (1.5,1.5) and (3,1) .. (2.5,0.3) .. controls (2, -0.4) and (0.5,0) .. (0.5,0.5); % Область Omega
\node at (2.8,0.8) {$\Omega$};
\draw (1,0.7) .. controls (1.3,1) and (1.8,0.9) .. (1.7,0.6); % Поверхность S
\draw (1,0.7) .. controls (0.8,0.5) and (1.3,0.4) .. (1.7,0.6); % Поверхность S
\node at (2,0.7) {$S$};
\fill (1.4,0.7) circle (0.05) node[right] {$x^0$};
\node at (1,-0.5) {$n=3$};

% n=2
\begin{scope}[xshift=6cm]
\draw[->] (0,-1) -- (0,3) node[above] {$x_2$};
\draw[->] (-1,0) -- (3,0) node[right] {$x_1$};
\draw (0,0) ellipse (1.5cm and 1cm); % Область Omega
\node at (2.5,0.5) {$\Omega$};
\draw (-0.5,0.5) .. controls (0,0.8) and (0.5,0.6) .. (0.3,0.2); % Поверхность S
\node at (0.7,0.7) {$S$};
\fill (0,0.4) circle (0.05) node[right] {$x^0$};
\draw[->] (0,0.4) -- (0,1);
\draw[->] (-0.2,0.6) -- (-0.2,1.2);
\draw[->] (0.2,0.2) -- (0.2,0.8);
\node at (1.8,2) {задано не касат. вект. поле $\nu_i$};
\node at (2.5,-0.5) {задана знак. $u_0(x)$};
\node at (1,-1.5) {$n=2$};
\end{scope}
\end{tikzpicture}

\textbf{Пример.} Некорректность может быть связана с несуществованием или с неединств. реш.
$(*) \begin{cases} \frac{\partial^2 u}{\partial x \partial y} = 0 \\ u(x,0)=0 \\ \frac{\partial u}{\partial y}(x,0)=x \end{cases}$
% Рисунок из текста
\begin{tikzpicture}[scale=0.7]
\draw[->] (0,0) -- (4,0) node[right] {$x$};
\draw[->] (0,0) -- (0,2) node[above] {$y$};
\draw[thick] (0,0) -- (3.5,0);
\node at (1.75,-0.3) {$S$};
\foreach \x in {0.5,1,...,3}
  \draw[->, thick] (\x,0) -- (\x,0.5);
\node at (2,1) {$\Omega$};
\draw (0,0) .. controls (0.5,1.5) and (3,1.5) .. (3.5,0);
\end{tikzpicture}

$u(x,y) \in C^2(\Omega)$. $y=0: \frac{\partial}{\partial x} \left( \frac{\partial u}{\partial y} \right) = 0 \implies \frac{\partial u}{\partial y}$ не должна зависеть от $x$, а в 3-м К.У. задано: $\frac{\partial u}{\partial y}(x,0)=x$ на $S$. В частности $\implies$ \underline{Нет решений}.
\begin{itemize}
    \item Чтобы у задачи $(*)$ было решение: $\frac{\partial u}{\partial y}(x,0)=0$.
    $\begin{cases} \frac{\partial^2 u}{\partial x \partial y} = 0 \\ u(x,0)=0 \\ \frac{\partial u}{\partial y}(x,0)=0 \end{cases}$
    $u(x,y) \in C^2(\Omega)$; $u_1(x,y)=0$ -- подходит, но подходит и $u_2(x,y)=y^2$.
    $\implies \Sigma$ решение не единственное.
\end{itemize}
% Рисунок не связанный с текстом рядом
\begin{tikzpicture}[scale=0.5, transform shape]
\draw[->] (-2,0) -- (2,0) node[right] {$t$};
\draw[->] (0,-2) -- (0,2) node[above] {$x$};
\draw plot[domain=-1.5:1.5, samples=50] (\x, {\x*\x});
\draw plot[domain=-1.5:1.5, samples=50] (\x, {-\x*\x});
\node at (1.5,1.8) {некорр};
\node at (-1.5,1.8) {корр};
\node at (0.7,-1.8) {корр};
\node at (-0.7,-1.8) {некорр};
\node at (2.5, 0.5) {$\Omega$};
\end{tikzpicture}

Итак, рассматриваем ур-ние $\sum_{i,j=1}^n a_{ij}(x) \frac{\partial^2 u}{\partial x_i \partial x_j} + F(x,u,\nabla u) = 0$.
Рассмотрим подмножество таких уравнений: $x_n=0$, $\omega(x)=0$, $\omega(x) \in C^2$, $\nabla \omega(x) \ne 0$ -- гиперплоскость, $x \in S \implies x=(x_1, \dots, x_{n-1},0)$.
$x'$ -- мн-во $x: x_n=0$, которые принадл. $S$.
Пусть $a_{nn}(x)=0$ при $x \in S$. В качестве некасательного вект. поля $\vec{\nu}(x)$ возьмём $\vec{\nu}(x) = \vec{n}(x)$, $x \in S$, $\vec{n}(x)=(0, \dots, 0,1)$.
Поставим на $S$ условия Коши:
$u(x_1, \dots, x_{n-1},0) = u_0(x_1, \dots, x_{n-1}) = u_0(x')$.
$\frac{\partial u}{\partial n}(x_1, \dots, x_{n-1},0) = u_1(x_1, \dots, x_{n-1}) = u_1(x')$.
% Рисунок из текста
\begin{tikzpicture}[scale=0.7]
\draw[->] (0,0) -- (3.5,0) node[right] {$x_2$};
\draw[->] (0,0) -- (0,3) node[above] {$x_n$};
\draw[->] (0,0) -- (-2,-1) node[below] {$x_1$};
\fill[gray!30] (-1.5,-0.75) rectangle (3,0.75); % Поверхность S
\node at (0.75,0) {$S$};
\draw (-1.8,-0.9) .. controls (-1,2.5) and (3.5,2.5) .. (3.3,-0.2) .. controls (3.1,-1.5) and (-2,-1.2) .. (-1.8,-0.9); % Область Omega
\node at (1,2) {$\Omega$};
\draw[->, thick] (0.75,0.3) -- (0.75,1.5) node[right] {$\vec{n}(x)$};
\fill (0.75,0.3) circle (0.05);
\end{tikzpicture}





\underline{Tmb}: В точках поверхности $S$ задан \textit{градиент функции} $u$.
\[
\frac{\partial u}{\partial x_1}(x_1, \dots, x_{n-1},0) = \lim_{\Delta x_1 \to 0} \frac{u(x_1+\Delta x_1, x_2, \dots, x_{n-1},0) - u(x_1, x_2, \dots, x_{n-1},0)}{\Delta x_1}
\]
\[
= \lim_{\Delta x_1 \to 0} \frac{u_0(x_1+\Delta x_1, x_2, \dots, x_{n-1},0) - u_0(x_1, x_2, \dots, x_{n-1},0)}{\Delta x_1} \underset{\text{по опр. } u_0}{=} \frac{\partial u_0}{\partial x_1}(x');
\]
Аналогичное верно $\forall x_i, i = \overline{1, n-1} \implies \frac{\partial u}{\partial x_i}(x_1, \dots, x_{n-1},0) = \frac{\partial u_0}{\partial x_i} \underbrace{(x_1, \dots, x_{n-1})}_{x'}, i=\overline{1,n-1}; \quad (\lambda) $
Т.о., зная значения функции $u$ в т. поверх. $S$, можем определить все касательные производные, кроме производной по $x_n$, однако она равна $u_1$, действительно:
$\frac{\partial u}{\partial x_n}(x_1, \dots, x_{n-1},0) \underset{\text{по опр. } u_1}{=} u_1(x_1, \dots, x_{n-1}) \quad (\lambda\lambda)$
$\implies$ Знаем все частные производные функц. $u \implies$ знаем \textit{градиент функции $u$ на поверх. $S$}.

Из формулы $(\lambda)$: $\frac{\partial^2 u}{\partial x_i \partial x_j}(x_1, \dots, x_{n-1},0) = \frac{\partial^2 u_0}{\partial x_i \partial x_j}(x'), i,j=\overline{1,n-1}$;
Из $(\lambda\lambda)$: $\frac{\partial^2 u}{\partial x_n \partial x_j}(x_1, \dots, x_{n-1},0) = \frac{\partial u_1}{\partial x_j}(x'), j=\overline{1,n-1}$.
$\implies$ Знаем все вторые производные от $u$ на $S$, кроме $\frac{\partial^2 u}{\partial x_n^2} = ?$

Чтобы найти $\frac{\partial^2 u}{\partial x_n^2}$, перепишем $(*)$ в другом виде:
\[
\underbrace{ \sum_{i,j=1}^{n-1} a_{ij}(x) \frac{\partial^2 u}{\partial x_i \partial x_j} }_{\text{известно из }(\lambda)} + \underbrace{ \sum_{j=1}^{n-1} \left[ a_{nj}(x) \frac{\partial^2 u}{\partial x_n \partial x_j} + a_{jn}(x) \frac{\partial^2 u}{\partial x_j \partial x_n} \right] }_{\text{известно из }(\lambda\lambda)} + \underbrace{a_{nn}(x)}_{\substack{=0 \\ \text{по опр.} \\ a_{nn}(x)=0, \\ x \in S}} \frac{\partial^2 u}{\partial x_n^2} +
\]
\[
+ \underbrace{F(x,u,\nabla u)}_{\substack{\text{известно, т.к. } u, \nabla u \\ \text{- известны из Tmb}}} = 0, \quad x \in S \quad (**)
\]
(Если $u \in C^2(\Omega) \implies \frac{\partial^2 u}{\partial x_n^2}$ -- огранич.)
Т.о. $u_0$ и $u_1$ на $S$ обязаны удовлетворять конечному функциональному ур-нию, которое называется \textit{условием совместности} $(**)$.
Т.к. $a_{nn}(x)=0$, то $\frac{\partial^2 u}{\partial x_n^2}$ -- любое $\implies$ решение не единственное.

\textbf{Опр.} \textit{Характеристическая поверхность} -- поверхность, которая требует выполнения условия совместности.

\begin{itemize}
    \item Рассмотрим теперь \textit{общий случай} для ур-ния $(*)$. Найдём поверхности, являющиеся \textit{характеристическими} для $(*)$.
    Подберём такую замену координат, чтобы часть поверхности, на которой задана З.К., перешла бы в гиперплоскость.
\end{itemize}

% Рисунки из текста
\begin{tikzpicture}[scale=0.7]
% Левый рисунок
\draw (0,0) .. controls (0.5,1) and (2,1) .. (2.5,0) .. controls (2,-1) and (0.5,-1) .. (0,0); % Omega
\node at (2.8,0.5) {$\Omega$};
\draw (-0.5,0.5) .. controls (0,1) and (1,0.8) .. (0.8,0.2); % S
\node at (1.2,0.6) {$S'$};
\node at (0.3,0.3) {$u(x')$};
\node at (0.3,0) {$\omega(x)=0$};
\fill (0.2,0.5) circle (0.05) node[above right] {$x^0$};
\draw[->, thick] (0.2,0.5) -- (0.2,1.2) node[right] {$\vec{n}=\nabla \omega (x^0) \ne 0$};
\node at (1.5,-1.5) {$x^0 \in S: \omega(x)=0$};
\node at (1.5,-2) {$U(x^0) \subseteq \Omega$};
\node at (1.5,-2.8) {Замена: $y=y(x)$};

% Правый рисунок
\begin{scope}[xshift=6cm]
\draw[->] (0,-1) -- (0,2) node[above] {$y_n$};
\draw[->] (-2,0) -- (2,0);
\draw (0,0) circle (1cm);
\draw[thick, decorate, decoration={coil, segment length=2mm, amplitude=1mm}] (-1,0) -- (1,0);
\node at (0.7,0.3) {$y(x_0)$};
\node at (0.3,-0.4) {$y(U(x^0))$};
\fill (0,0) circle (0.05);
\node at (0.5,-1.5) {Кусок поверх. переходит в $\{y_n=0\}$};
\end{scope}
\end{tikzpicture}

\hfill \framebox{4} % Номер страницы в углу



Построим преобразование:
\begin{enumerate}
    \item Вектор $\nabla \omega(x^0) \ne 0 \parallel \vec{n}$;
    \item Дополним $\nabla \omega(x^0)$ до базиса: $\{\vec{l}^1, \dots, \vec{l}^{n-1}\}$, $\vec{l}^k=(l_1^k, \dots, l_n^k)$, $k=\overline{1,n-1}$;
    \item Ортогонализуем полученный базис, не трогая $\nabla \omega(x^0)$: $\{\hat{\vec{l}}^1, \hat{\vec{l}}^2, \dots, \hat{\vec{l}}^{n-1}, \nabla \omega(x^0)\}$ -- ортог. базис.
\end{enumerate}

Преобразование имеет вид:
\[
y=y(x) = \begin{cases}
y_1 = y_1(x_1, \dots, x_n) = (\hat{\vec{l}}^1, x-x^0) = \sum_{j=1}^n \hat{l}^1_j (x_j-x_j^0) \\
\vdots \\
y_{n-1} = y_{n-1}(x_1, \dots, x_n) = (\hat{\vec{l}}^{n-1}, x-x^0) = \sum_{j=1}^n \hat{l}^{n-1}_j (x_j-x_j^0) \\
y_n = y_n(x_1, \dots, x_n) = \omega(x_1, \dots, x_n)
\end{cases}
\]
Точку $x^0$ это преобр. приводит в т. $0$, т.е. $y(x^0)=0$, как и хотелось.
$U(x^0) \cap S$ преобр. переводит в гиперплоскость, поскольку $(U(x^0) \cap S) \subseteq \{\omega(x)=0\}$, а $y_n = \omega(x_1, \dots, x_n) = 0$.

Проверим, что $y(x)$ является диффеоморфизмом класса $C^2$: $y(x)$ -- гладкость $C^2$; проверим наличие обр. преобр.
Матрица Якоби $\mathcal{J}(x^0)$:
\[
\mathcal{J}(x^0) = \begin{pmatrix}
\frac{\partial y_1}{\partial x_1}(x^0) & \dots & \frac{\partial y_1}{\partial x_n}(x^0) \\
\vdots & \ddots & \vdots \\
\frac{\partial y_n}{\partial x_1}(x^0) & \dots & \frac{\partial y_n}{\partial x_n}(x^0)
\end{pmatrix}
= \begin{pmatrix}
\hat{l}^1_1 & \dots & \hat{l}^1_n \\
\vdots & \ddots & \vdots \\
\hat{l}^{n-1}_1 & \dots & \hat{l}^{n-1}_n \\
\frac{\partial \omega}{\partial x_1}(x^0) & \dots & \frac{\partial \omega}{\partial x_n}(x^0)
\end{pmatrix} \ne 0, \text{ т.к. явл. ортог. базисом.}
\]
$\implies y(x) \in C^2(U(x^0))$, $|\mathcal{J}(x^0)| \ne 0 \implies$ сущ. обр. отобр., оно единств. и явл. класса $C^2 \implies$
В новых переменных ур-ние примет вид: $\sum_{k,l=1}^n \hat{a}_{kl}(y) \frac{\partial^2 \hat{u}}{\partial y_k \partial y_l} + \hat{F}(y, \hat{u}, \nabla_y \hat{u}) = 0$.

Получили модельную задачу, за исключением условия $a_{nn}=0$, потребуем тогда, чтобы
\[
\hat{a}_{nn}(y) = \sum_{i,j=1}^n a_{ij}(x(y)) \frac{\partial y_n}{\partial x_i} \frac{\partial y_n}{\partial x_j} = \sum_{i,j=1}^n a_{ij}(x(y)) \frac{\partial \omega}{\partial x_i} \frac{\partial \omega}{\partial x_j} = 0
\]
\[
\implies \boxed{ \sum_{i,j=1}^n a_{ij}(x) \frac{\partial \omega}{\partial x_i}(x) \frac{\partial \omega}{\partial x_j}(x) = 0 \quad (\heartsuit) } \quad \text{-- условие, при котором поверх. } S: \omega(x)=0
\]
становится характеристической для ур-ния $(*)$.

\textbf{Опр.} Гладкая поверхность $\omega(x)=0$ называется \textit{характеристической} для ур-ния $(*)$, если в каждой точке она удовлетворяет $(\heartsuit)$.

\textbf{Пример.} Разные характеристические поверхности.
\begin{enumerate}
    \item $u_{tt} - a^2 \Delta_x u = f(t,x)$, $x \in \mathbb{R}^n$. Характеристическую поверхность ищем из ур-ния:
    $\left(\frac{\partial \omega}{\partial t}\right)^2 - a^2 \left( \left(\frac{\partial \omega}{\partial x_1}\right)^2 + \dots + \left(\frac{\partial \omega}{\partial x_n}\right)^2 \right) = 0$.
\end{enumerate}


% Рисунок из текста - конус
\begin{tikzpicture}[scale=0.7]
\draw[->] (0,-2.5) -- (0,2.5) node[above] {$t$};
\draw[->] (-2,0) -- (2,0) node[right] {$x_3$}; % Или x_1
\draw[->] (0,0) -- (1.732*1.5, -1.5*0.866) node[right] {$x_1$}; % Или x_2 (30 градусов)
\draw[->] (0,0) -- (-1.732*1.5, -1.5*0.866) node[left] {$x_2$}; % Или x_3 (210 градусов)

% Верхний конус
\draw (0,2) ellipse (1cm and 0.5cm);
\draw (-1,2) -- (0,0) -- (1,2);
\draw[dashed] (0,1) ellipse (0.5cm and 0.25cm);
\draw[dashed] (0,0.5) ellipse (0.25cm and 0.125cm);

% Нижний конус
\draw (0,-2) ellipse (1cm and 0.5cm);
\draw (-1,-2) -- (0,0) -- (1,-2);
\draw[dashed] (0,-1) ellipse (0.5cm and 0.25cm);
\draw[dashed] (0,-0.5) ellipse (0.25cm and 0.125cm);

\node at (2.5,2) {$n=3$};
\end{tikzpicture}
Решение: $\omega(t,x) = a^2 t^2 - (x_1^2 + \dots + x_n^2) = 0$ -- \textit{конусы} в $n+1$-мерном пр-ве.

\begin{enumerate}\setcounter{enumi}{1}
    \item $u_t - a^2 \Delta_x u = f(t,x)$, $x \in \mathbb{R}^n$. Ищем характ. поверхности из ур-ния:
    $\frac{\partial \omega}{\partial t} - a^2 \left[ \left(\frac{\partial \omega}{\partial x_1}\right)^2 + \dots + \left(\frac{\partial \omega}{\partial x_n}\right)^2 \right] = 0 \implies \frac{\partial \omega}{\partial x_1} = \dots = \frac{\partial \omega}{\partial x_n} = 0$,
    $\omega_t \ne 0 \implies \omega(t,x) = t+c \implies \underline{t=\text{const}}$ -- характ. поверх. ($S: \omega=0$)

    \item $-\Delta u = f(t,x)$, $x \in \mathbb{R}^n$.
    $-\left[ \left(\frac{\partial \omega}{\partial x_1}\right)^2 + \dots + \left(\frac{\partial \omega}{\partial x_n}\right)^2 \right] = 0 \implies \frac{\partial \omega}{\partial x_1} = \dots = \frac{\partial \omega}{\partial x_n} = 0 \implies \nabla \omega = 0$ -- противоречие ур-нию поверхн.
    $\implies$ \underline{Характ. поверх. отсутствуют} $\implies$ решение сущ. на любой поверх., но может быть не ед.
\end{enumerate}

\textbf{Tmb.} Характеристических поверхностей бесконечно много.
$\triangleleft$ Пусть $\omega(x)=0$ -- характ. поверх. Пусть $\tilde{\omega}(x) = \omega(x)+c, c \in \mathbb{R}$. Рассмотрим поверх. $\tilde{\omega}(x)=0$, $\tilde{\omega}(x) \in C^2(\Omega)$, $\nabla \tilde{\omega} \ne 0$, т.к. $\nabla \omega \ne 0 \implies \tilde{\omega}$ удовлетворяет характ. ур-нию. $\triangleright$
% Рисунок - смещение поверхности
\begin{tikzpicture}[scale=0.7]
\draw[->] (-1,0) -- (3,0) node[right] {$x$};
\draw[->] (0,-1) -- (0,2) node[above] {$y$};
\draw plot[domain=0:2.5, samples=50] (\x, {sqrt(\x)});
\node at (2,0.7) {$\omega(x)=0$};
\draw plot[domain=0:2.5, samples=50] (\x, {sqrt(\x)+0.5});
\node at (1.5,1.5) {$\omega(x,y)+c=0$};
\draw[<->] (0.5, {sqrt(0.5)}) -- (0.5, {sqrt(0.5)+0.5});
\node at (0.3, {sqrt(0.5)+0.25}) {$+c$};
\end{tikzpicture}

\textbf{Опр.} Точка $x^0 \in S$ назыв. \textit{характ. точкой} для ур-ния $(*)$, если в этой точке вып.:
$\sum_{i,j=1}^n a_{ij}(x^0) \frac{\partial \omega}{\partial x_i}(x^0) \frac{\partial \omega}{\partial x_j}(x^0) = 0$.

\textbf{Опр.} Функция $u(x)=u(x_1, \dots, x_n)$ называется \underline{вещественно-аналитической} в т. $x^0 \in \mathbb{R}^n$, если в некоторой окр. $U(x^0)$ $u(x)$ представима степенным рядом:
$u(x) = \sum_{\alpha: |\alpha| \ge 0} u_\alpha (x-x^0)^\alpha$, где $\alpha=(\alpha_1, \dots, \alpha_n)$, $u_\alpha$ -- ч.п.с $\alpha$, $(x-x^0)^\alpha = (x_1-x_1^0)^{\alpha_1} \dots (x_n-x_n^0)^{\alpha_n}$.

\textbf{Th.} [\textit{Коши-Ковалевской}]
Пусть в ур-ние $(*)$ выполнено:
\underline{а)} все $a_{ij}(x)$ -- вещ.-ан. в т. $x^0$;
\underline{б)} $F(x,u,\nabla u)$ -- вещ.-ан. в т. $\{x^0, u(x^0), \nabla u(x^0)\}$;
\underline{в)} $\omega(x)$ -- вещ.-ан. в т. $x^0$ ($S: \omega(x)=0$);
\underline{г)} $u_0(x), u_1(x)$ из З.К. на поверх. $S$ -- вещ.-ан. в т. $x^0$;
\underline{д)} $x^0$ -- нехаракт. т. для $S$ и $(*)$.
Тогда:
\begin{enumerate}
    \item $\exists U(x^0)$, в которой $\exists$ вещ.-ан. решение $u(x)$ задачи Коши.
    \item В классе вещ.-ан. функций это решение \underline{единственное}.
\end{enumerate}

\hfill \framebox{5}
\newpage

\section*{Билет №4 -- 2025}\label{sec:ticket4}
\backtotoc


\newtheorem{theorem}{Теорема}
\newtheorem{definition}{Определение}
\newtheorem{lemma}{Лемма}
\newenvironment{proof}[1][\proofname]{\par\noindent\textit{#1.}\quad}{\hfill$\square$}
\renewcommand{\proofname}{Доказательство}






\noindent\textbf{N5}
\paragraph*{Смешанная задача для волнового уравнения в $\R^n$. Закон сохранения энергии. Априорные оценки решения. Единственность классического решения + N6: Корректность смешанной задачи для волнового уравнения в $G$ из априорной оценки решения.}

Рассматривается задача:
$$
\left\{
\begin{aligned}
&u_{tt} - a^2 \Delta u = F(t,x), && x \in G, t \in (0;+\infty) \quad &(1) \\
&u|_{t=0} = u_0(x), \quad u_t|_{t=0} = u_1(x), && Z_\infty = (0;+\infty) \times G, \quad \partial G = S \quad &(2) \\
&\left(\alpha u(t,x) + \beta \frac{\partial u}{\partial \vec{n}}\right)\Big|_S = 0, && t \in (0;+\infty) \quad &(3)
\end{aligned}
\right.
$$
где $G$ -- ограниченная область, $F \in C(Z_\infty)$, $u_0 \in C^2(\bar{G})$, $u_1 \in C^1(\bar{G})$, $\alpha, \beta \in C^1(\bar{G})$, $S_0 \subset S$: та часть $S$, где $\alpha(x) > 0$ и $\beta(x) > 0$.

\begin{definition}[Классическое решение]
\label{def:classical_solution}
\underline{Классическим решением} смешанной задачи (1)-(3) называется функция $u(t,x) \in C^2(Z_\infty) \cap C^1(\bar{Z}_\infty)$, удовлетворяющая (1)-(3) в классическом смысле (т.е. поточечно).
\end{definition}

Необходимым условием существования классического решения является выполнение условия согласования:
\begin{equation*}
\left(\alpha u_0 + \beta \frac{\partial u_0}{\partial \vec{n}}\right)\Big|_S = 0.
\end{equation*}

\begin{definition}[Интеграл энергии]
\label{def:energy_integral}
\underline{Опр.} Пусть $u(t,x)$ -- классическое решение (1)-(3). Квадратом интеграла энергии называется величина:
\begin{equation*}
\mathcal{J}^2(t) = \frac{1}{2} \int_G [u_t^2(t,x) + a^2 |\nabla u(t,x)|^2] \dd x + \frac{a^2}{2} \int_{S_0} \frac{\alpha(x)}{\beta(x)} u^2(t,x) \dd S.
\end{equation*}
\end{definition}

\begin{theorem}[Th1]
\label{thm:energy_conservation}
Пусть $u(t,x)$ -- классическое решение (1)-(3), т.е. $u(t,x) \in C^2(Z_\infty) \cap C^1(\bar{Z}_\infty)$ и удовлетворяет (1)-(3). Тогда:
\begin{equation} \label{eq:4}
\mathcal{J}^2(t) = \mathcal{J}^2(0) + \int_0^t \int_G F(x,\tau) u_\tau(x,\tau) \dd\tau \dd x, \quad \underline{(4)}
\end{equation}
где:
\begin{equation*}
\mathcal{J}^2(0) = \frac{1}{2} \int_G [u_1^2(x) + a^2 |\nabla u_0(x)|^2] \dd x + \frac{a^2}{2} \int_{S_0} \frac{\alpha(x)}{\beta(x)} u_0^2(x) \dd S.
\end{equation*}
\end{theorem}

\begin{proof}
$\blacktriangleright$ Возьмём произвольные $\varepsilon > 0$ и $T>\varepsilon$. Рассмотрим область $G' \subset G$ с кусочно-гладкой границей $S'$.
Умножим уравнение (1) на $u_t$:
\begin{equation*}
u_{tt} u_t - a^2 (\Delta u) u_t = F(t,x) u_t.
\end{equation*}
Проинтегрируем это равенство по $G' \times (\varepsilon, T)$:
\begin{equation*}
\int_\varepsilon^T \int_{G'} u_{tt} u_t \dd x \dd t - \int_\varepsilon^T \int_{G'} a^2 (\Delta u) u_t \dd x \dd t = \int_\varepsilon^T \int_{G'} F(t,x) u_t \dd x \dd t.
\end{equation*}
Преобразуем интегралы в левой части.
Первый интеграл (обозначенный $I_1$ на полях рукописи):
\begin{align*}
\int_\varepsilon^T \int_{G'} u_{tt} u_t \dd x \dd t &= \int_{G'} \left( \int_\varepsilon^T \frac{1}{2} \frac{\partial}{\partial t}(u_t)^2 \dd t \right) \dd x \\
&= \frac{1}{2} \int_{G'} (u_t)^2 \Big|_\varepsilon^T \dd x = \frac{1}{2} \int_{G'} (u_t^2(T,x) - u_t^2(\varepsilon,x)) \dd x.
\end{align*}
Второй интеграл (связанный с $I_2$ на полях рукописи), используя $\Delta u = \dive(\grad u)$:
\begin{align*}
\int_\varepsilon^T \int_{G'} a^2 \dive(\grad u) u_t \dd x \dd t
\end{align*}
В рукописи приводится тождество: $\dive( (\grad u) u_t) = \dive(\grad u) u_t + (\grad u, \grad u_t)$.
Отсюда $\dive(\grad u) u_t = \dive( (\grad u) u_t) - (\grad u, \grad u_t)$.
Тогда интеграл равен:
\begin{align*}
&= a^2 \int_\varepsilon^T \int_{G'} \left[ \dive( (\grad u) u_t) - (\grad u \cdot \grad u_t) \right] \dd x \dd t \\
&\text{(По теореме Гаусса-Остроградского (Th. G.-O.) для первого слагаемого)} \\
&= a^2 \int_\varepsilon^T \left( \oint_{\partial G'} u_t (\grad u \cdot \vec{n}) \dd S_x \right) \dd t - a^2 \int_\varepsilon^T \int_{G'} (\grad u \cdot \grad u_t) \dd x \dd t \\
&\text{Здесь } (\grad u \cdot \vec{n}) = \frac{\partial u}{\partial \vec{n}} \text{ и } (\grad u \cdot \grad u_t) = \frac{1}{2}\frac{\partial}{\partial t}|\grad u|^2. \\
&= a^2 \int_\varepsilon^T \oint_{\partial G'} u_t \frac{\partial u}{\partial \vec{n}} \dd S_x \dd t - \frac{a^2}{2} \int_{G'} |\grad u|^2 \Big|_\varepsilon^T \dd x.
\end{align*}
(Замечание из рукописи: $(\nabla u, \vec{n}) = \frac{\partial u}{\partial \vec{n}}$)
\end{proof}
\hfill \textit{73}

\documentclass[12pt,a4paper]{article}
\usepackage[T2A]{fontenc}
\usepackage[utf8]{inputenc}
\usepackage[russian]{babel}
\usepackage{amsmath, amssymb, amsfonts}
\usepackage{geometry}
\geometry{a4paper, margin=1in}

\newtheorem{theorem}{Теорема}
\newtheorem{definition}{Определение}
\newtheorem{lemma}{Лемма}
\newtheorem{corollary}{Следствие}
\newenvironment{proof}[1][\proofname]{\par\noindent\textit{#1.}\quad}{\hfill$\square$}
\renewcommand{\proofname}{Доказательство (продолжение)}






\begin{proof}[Доказательство (продолжение)]
Напомним, что уравнение (1), умноженное на $u_t$, было проинтегрировано по $G' \times (\varepsilon, T)$:
$$ \int_\varepsilon^T \int_{G'} u_{tt} u_t \dd x \dd t - \int_\varepsilon^T \int_{G'} a^2 (\Delta u) u_t \dd x \dd t = \int_\varepsilon^T \int_{G'} F(t,x) u_t \dd x \dd t. $$
Преобразование первого слагаемого дало:
$$ \int_\varepsilon^T \int_{G'} u_{tt} u_t \dd x \dd t = \frac{1}{2} \int_{G'} (u_t^2(T,x) - u_t^2(\varepsilon,x)) \dd x. $$
Преобразование второго слагаемого:
\begin{align*}
- \int_\varepsilon^T \int_{G'} a^2 (\Delta u) u_t \dd x \dd t &= - a^2 \int_\varepsilon^T \left( \oint_{S'} u_t \frac{\partial u}{\partial \vec{n}} \dd S_x - \int_{G'} \nabla u \cdot \nabla u_t \dd x \right) \dd t \\
&= - a^2 \int_\varepsilon^T \oint_{S'} u_t \frac{\partial u}{\partial \vec{n}} \dd S_x \dd t + a^2 \int_\varepsilon^T \int_{G'} \frac{1}{2} \frac{\partial}{\partial t} |\nabla u|^2 \dd x \dd t \\
&= - a^2 \int_\varepsilon^T \oint_{S'} u_t \frac{\partial u}{\partial \vec{n}} \dd S_x \dd t + \frac{a^2}{2} \int_{G'} (|\nabla u(T,x)|^2 - |\nabla u(\varepsilon,x)|^2) \dd x.
\end{align*}
Объединяя эти части, получаем:
\begin{multline*}
\frac{1}{2} \int_{G'} [u_t^2(T,x) + a^2 |\nabla u(T,x)|^2] \dd x - \frac{1}{2} \int_{G'} [u_t^2(\varepsilon,x) + a^2 |\nabla u(\varepsilon,x)|^2] \dd x \\
- a^2 \int_\varepsilon^T \int_{S'} u_t \frac{\partial u}{\partial \vec{n}} \dd S \dd t = \int_\varepsilon^T \int_{G'} F(t,x) u_t \dd x \dd t.
\end{multline*}
Переходя к пределу при $\varepsilon \to 0$, $G' \to G$ и пользуясь тем, что $u \in C^1(\bar{Z}_T)$, $F \in C(Z_\infty)$:
\begin{multline*}
\frac{1}{2} \int_G [u_t^2(T,x) + a^2 |\nabla u(T,x)|^2] \dd x - \frac{1}{2} \int_G [u_t^2(0,x) + a^2 |\nabla u(0,x)|^2] \dd x \\
- a^2 \int_0^T \int_S u_t \frac{\partial u}{\partial \vec{n}} \dd S \dd t = \int_0^T \int_G F(t,x) u_t \dd x \dd t.
\end{multline*}
Из граничных условий (3): $\alpha u + \beta \frac{\partial u}{\partial \vec{n}} = 0$.
На $S_0 = \{x \in S : \alpha(x)>0, \beta(x)>0\}$, имеем $\frac{\partial u}{\partial \vec{n}} = -\frac{\alpha}{\beta} u$.
На $S \setminus S_0$: если $\beta=0$, то $\alpha u=0$. Если $\alpha \neq 0$, то $u=0 \Rightarrow u_t=0$. Интеграл по этой части $S$ равен нулю. Если $\alpha=0$, то $\beta \frac{\partial u}{\partial \vec{n}}=0$. Если $\beta \neq 0$, то $\frac{\partial u}{\partial \vec{n}}=0$. Интеграл по этой части $S$ также равен нулю.
Таким образом, поверхностный интеграл рассматривается только по $S_0$.
\begin{align*}
- a^2 \int_0^T \int_S u_t \frac{\partial u}{\partial \vec{n}} \dd S \dd t &= - a^2 \int_0^T \int_{S_0} u_t \left(-\frac{\alpha}{\beta} u\right) \dd S \dd t \\
&= a^2 \int_0^T \int_{S_0} \frac{\alpha}{\beta} u u_t \dd S \dd t = a^2 \int_0^T \int_{S_0} \frac{\alpha}{\beta} \frac{1}{2} \frac{\partial (u^2)}{\partial t} \dd S \dd t \\
&= \frac{a^2}{2} \int_{S_0} \frac{\alpha}{\beta} u^2(T,x) \dd S - \frac{a^2}{2} \int_{S_0} \frac{\alpha}{\beta} u^2(0,x) \dd S.
\end{align*}
Подставляя это в предыдущее равенство и используя определение $\mathcal{J}^2(t)$:
$$ \mathcal{J}^2(T) - \mathcal{J}^2(0) = \int_0^T \int_G F(x,t) u_t(x,t) \dd x \dd t. $$
Заменяя $T$ на $t$, получаем формулу (4):
$$ \mathcal{J}^2(t) = \mathcal{J}^2(0) + \int_0^t \int_G F(x,\tau) u_\tau(x,\tau) \dd\tau \dd x. \blacktriangleleft $$
\end{proof}

\begin{corollary}
При $F=0: \mathcal{J}^2(t) = \mathcal{J}^2(0)$ -- закон сохранения энергии.
\end{corollary}

\paragraph*{Априорные оценки}
Получим некоторые априорные оценки, используя норму $\| \cdot \| \equiv \| \cdot \|_{L^2(G)} = \left(\int_G |\cdot|^2 \dd x\right)^{1/2}$.
Запишем интеграл энергии (4) и продифференцируем его по $t$:
$$ \frac{d}{dt} \mathcal{J}^2(t) = \int_G F(x,t) u_t(x,t) \dd x. $$
Так как $\frac{d}{dt} \mathcal{J}^2(t) = 2 \mathcal{J}(t) \mathcal{J}'(t)$, то по неравенству Коши-Буняковского (К.-Б.):
$$ 2 \mathcal{J}(t) \mathcal{J}'(t) = \int_G F(t,x) u_t(t,x) \dd x \le \|F(t)\| \|u_t(t)\|. $$
Из определения $\mathcal{J}^2(t)$ (и предполагая $\alpha/\beta \ge 0$ на $S_0$, что следует из определения $S_0$):
$$ \|u_t(t)\|^2 = \int_G u_t^2(t,x) \dd x \le 2 \mathcal{J}^2(t) \implies \|u_t(t)\| \le \sqrt{2} \mathcal{J}(t). \quad \text{(I)} $$
$$ a^2 \int_G |\nabla u(t,x)|^2 \dd x \le 2 \mathcal{J}^2(t) \implies \|\nabla u(t)\|^2 \le \frac{2}{a^2} \mathcal{J}^2(t) \implies \|\nabla u(t)\| \le \frac{\sqrt{2}}{a} \mathcal{J}(t) \quad (\text{при } a \ne 0). \quad \text{(II)} $$
Подставляя (I) в неравенство для производной:
$$ 2 \mathcal{J}(t) \mathcal{J}'(t) \le \|F(t)\| \sqrt{2} \mathcal{J}(t). $$
Если $\mathcal{J}(t) > 0$, делим на $2\mathcal{J}(t)$:
$$ \mathcal{J}'(t) \le \frac{1}{\sqrt{2}} \|F(t)\|. $$
Интегрируя от $0$ до $t$:
$$ \mathcal{J}(t) - \mathcal{J}(0) \le \int_0^t \frac{1}{\sqrt{2}} \|F(\tau)\| \dd\tau $$
$$ \mathcal{J}(t) \le \mathcal{J}(0) + \frac{1}{\sqrt{2}} \int_0^t \|F(\tau)\| \dd\tau. \quad \text{(III)} $$
Подставляя (III) в (I) и (II), получаем априорные оценки:
$$ \|u_t(t)\| \le \sqrt{2} \mathcal{J}(t) \le \sqrt{2} \mathcal{J}(0) + \int_0^t \|F(\tau)\| \dd\tau. \quad \text{(IV)} $$
$$ \|\nabla u(t)\| \le \frac{\sqrt{2}}{a} \mathcal{J}(t) \le \frac{\sqrt{2}}{a} \mathcal{J}(0) + \frac{1}{a} \int_0^t \|F(\tau)\| \dd\tau. \quad \text{(V)} $$




\setcounter{section}{1} % Начинаем нумерацию теорем с этого раздела (фиктивно)

\begin{theorem}[Th2]
\label{thm:apriori_u}
Для классического решения справедлива априорная оценка для $\|u(t)\|$:
\begin{equation} \label{eq:apriori_u}
\|u(t)\| \le \|u_0\| + \sqrt{2} \mathcal{J}(0) t + \int_0^t (t-\tau) \|F(\tau)\| \dd\tau.
\end{equation}
\end{theorem}
\begin{proof}
$\blacktriangle$ Имеем $\|u(t)\|^2 = \int_G u^2(t,x) \dd x$. Дифференцируем по $t$:
$$ \frac{d}{dt} \|u(t)\|^2 = 2 \int_G u(t,x) u_t(t,x) \dd x \le 2 \|u(t)\| \|u_t(t)\|. $$
Если $\|u(t)\| > 0$, то $(\|u(t)\|)' = \frac{1}{2\|u(t)\|} \frac{d}{dt} \|u(t)\|^2 \le \|u_t(t)\|$.
Используя оценку (IV) для $\|u_t(t)\|$:
$$ (\|u(t)\|)' \le \|u_t(t)\| \le \sqrt{2} \mathcal{J}(0) + \int_0^t \|F(\tau')\| \dd\tau'. $$
Проинтегрируем это неравенство от $0$ до $t$:
$$ \int_0^t (\|u(\xi)\|)' \dd\xi \le \int_0^t \left(\sqrt{2} \mathcal{J}(0) + \int_0^\xi \|F(\tau')\| \dd\tau' \right) \dd\xi $$
$$ \|u(t)\| - \|u(0)\| \le \sqrt{2} \mathcal{J}(0) t + \int_0^t \left( \int_0^\xi \|F(\tau')\| \dd\tau' \right) \dd\xi. $$
Здесь $\|u(0)\| = \|u_0\| = \left(\int_G u_0^2(x) \dd x\right)^{1/2}$.
Двойной интеграл в правой части преобразуется сменой порядка интегрирования (см. рисунок в рукописи):
$$ \int_0^t \dd\xi \int_0^\xi \|F(\tau')\| \dd\tau' = \int_0^t \dd\tau' \int_{\tau'}^t \|F(\tau')\| \dd\xi = \int_0^t \|F(\tau')\| (t-\tau') \dd\tau'. $$
Обозначая переменную интегрирования $\tau'$ как $\tau$, получаем:
$$ \|u(t)\| \le \|u_0\| + \sqrt{2} \mathcal{J}(0) t + \int_0^t (t-\tau) \|F(\tau)\| \dd\tau. \blacktriangleleft $$
\end{proof}

\begin{theorem}[Th3. Единственность]
\label{thm:uniqueness}
Классическое решение задачи (1)-(3) единственно.
\end{theorem}
\begin{proof}
$\blacktriangle$ Пусть есть два решения задачи (1)-(3): $u$ и $\tilde{u}$. Рассмотрим разность $v = u - \tilde{u}$.
Функция $v(t,x)$ удовлетворяет задаче:
$$
\left\{
\begin{aligned}
&v_{tt} - a^2 \Delta v = 0, && x \in G, t \in (0;+\infty) \\
&v|_{t=0} = 0, \quad v_t|_{t=0} = 0 && \\
&\left(\alpha v + \beta \frac{\partial v}{\partial \vec{n}}\right)\Big|_S = 0 &&
\end{aligned}
\right.
$$
Для этой задачи $F \equiv 0$, $v_0 \equiv 0$, $v_1 \equiv 0$.
Из определения $\mathcal{J}_v(0)$ (интеграл энергии для $v$ в начальный момент):
$$ \mathcal{J}_v^2(0) = \frac{1}{2} \int_G [v_1^2 + a^2 |\nabla v_0|^2] \dd x + \frac{a^2}{2} \int_{S_0} \frac{\alpha}{\beta} v_0^2 \dd S = 0. $$
Следовательно, $\mathcal{J}_v(0)=0$.
Из закона сохранения энергии (Следствие к Th1, так как $F \equiv 0$): $\mathcal{J}_v^2(t) = \mathcal{J}_v^2(0) = 0$ для всех $t \ge 0$.
Поскольку $\mathcal{J}_v^2(t) = \frac{1}{2} \int_G [v_t^2 + a^2 |\nabla v|^2] \dd x + \frac{a^2}{2} \int_{S_0} \frac{\alpha}{\beta} v^2 \dd S = 0$,
и все слагаемые под интегралами неотрицательны (предполагаем $\alpha/\beta \ge 0$ на $S_0$), то $v_t(t,x) \equiv 0$ и $\nabla v(t,x) \equiv 0$ в $G$, и $v(t,x) \equiv 0$ на $S_0$.
Из $\nabla v \equiv 0$ следует, что $v(t,x) = C(t)$ для каждого $t$. Поскольку $v_t \equiv 0$, то $C'(t)=0$, значит $C(t)$ -- константа.
Так как $v|_{t=0}=0$, то $C=0$.
Следовательно, $v(t,x) \equiv 0$, т.е. $u(t,x) \equiv \tilde{u}(t,x)$. $\blacktriangleleft$
\end{proof}

\begin{theorem}[Th4. Непрерывная зависимость (из N6)]
\label{thm:continuous_dependence}
Классическое решение задачи (1)-(3) непрерывно зависит от $u_0, u_1, F$ при следующих условиях.
Пусть $u$ и $\tilde{u}$ -- два классических решения задачи (1)-(3), отвечающие данным $(F, u_0, u_1)$ и $(\tilde{F}, \tilde{u}_0, \tilde{u}_1)$ соответственно.
Если $\|F - \tilde{F}\|_{C([0,T]; L^2(G))} \le \mathcal{E}$, $\|u_0 - \tilde{u}_0\|_{L^2(G)} \le \mathcal{E}_0$, $\|\nabla u_0 - \nabla \tilde{u}_0\|_{L^2(G)} \le \mathcal{E}_0'$, $\|u_1 - \tilde{u}_1\|_{L^2(G)} \le \mathcal{E}_1$,
то для $t \in [0,T]$ верны оценки:
\begin{align}
\|u(t) - \tilde{u}(t)\|_{L^2(G)} &\le C_1 (\mathcal{E}_0 + T \mathcal{E}_0' + T \mathcal{E}_1 + \frac{1}{2} T^2 \mathcal{E}) \label{eq:cont_dep_u} \\
\|\nabla (u(t) - \tilde{u}(t))\|_{L^2(G)} &\le C_2 (\mathcal{E}_0 + \mathcal{E}_0' + \mathcal{E}_1 + T \mathcal{E}) \label{eq:cont_dep_gradu} \\
\|u_t(t) - \tilde{u}_t(t)\|_{L^2(G)} &\le C_3 (\mathcal{E}_0' + \mathcal{E}_1 + T \mathcal{E}), \quad C_i = \text{const} \label{eq:cont_dep_ut}
\end{align}
(Примечание: в рукописи $C_i$ объединены в одну $C$. Точные коэффициенты могут отличаться от рукописных.)
\end{theorem}
\begin{proof}
$\blacktriangle$ Возьмём $\eta = u - \tilde{u}$. Функция $\eta$ является классическим решением задачи (1)-(3) с заменой $F, u_0, u_1$ на $K = F - \tilde{F}$, $\varphi_0 = u_0 - \tilde{u}_0$, $\varphi_1 = u_1 - \tilde{u}_1$.
Оценим $\mathcal{J}_\eta(0)$.
$$ 2 \mathcal{J}_\eta^2(0) = \int_G (\varphi_1^2 + a^2 |\nabla \varphi_0|^2) \dd x + a^2 \int_{S_0} \frac{\alpha}{\beta} \varphi_0^2 \dd S. $$
$$ \int_G \varphi_1^2 \dd x = \|\varphi_1\|^2 \le \mathcal{E}_1^2. $$
$$ \int_G a^2 |\nabla \varphi_0|^2 \dd x = a^2 \|\nabla \varphi_0\|^2 \le a^2 (\mathcal{E}_0')^2. $$
Для интеграла по $S_0$:
$$ \int_{S_0} \frac{\alpha}{\beta} \varphi_0^2 \dd S \le \max_{x \in S_0} \left(\frac{\alpha(x)}{\beta(x)}\right) \int_{S_0} \varphi_0^2 \dd S. $$
По теореме вложения (или следа), $\|\varphi_0\|_{L^2(S_0)}^2 \le C_S (\|\varphi_0\|_{L^2(G)}^2 + \|\nabla \varphi_0\|_{L^2(G)}^2)$ (эта часть более сложная и в рукописи, похоже, упрощена или используется другая оценка).
Предположим, что $\int_{S_0} \frac{\alpha}{\beta} \varphi_0^2 \dd S \le C' \|\varphi_0\|_{L^2(G)}^2 \le C' \mathcal{E}_0^2$ (так как $\varphi_0$ ограничена по норме $L^2(G)$).
Более точно, в рукописи используется оценка $V_G \cdot (\mathcal{E}_1^2 + a^2 (\mathcal{E}_0')^2) + V_S \cdot a^2 (\mathcal{E}_0)^2 \cdot \max (\alpha/\beta)$, где $V_G=\mes G$, $V_S=\mes S_0$.
Это приводит к
$$ \mathcal{J}_\eta^2(0) \le C_0^2 (\mathcal{E}_0^2 + (\mathcal{E}_0')^2 + \mathcal{E}_1^2) \implies \sqrt{2} \mathcal{J}_\eta(0) \le C_*( \mathcal{E}_0 + \mathcal{E}_0' + \mathcal{E}_1). $$
(Константа $C_*$ зависит от $a, V_G, V_S, \max(\alpha/\beta)$).
Используем априорную оценку \eqref{eq:apriori_u} для $\eta(t)$:
\begin{align*}
\|\eta(t)\| &\le \|\varphi_0\| + \sqrt{2} \mathcal{J}_\eta(0) t + \int_0^t (t-\tau) \|K(\tau)\| \dd\tau \\
&\le \mathcal{E}_0 + C_*(\mathcal{E}_0 + \mathcal{E}_0' + \mathcal{E}_1) t + \int_0^t (t-\tau) \mathcal{E} \dd\tau \\
&= \mathcal{E}_0 + C_*(\mathcal{E}_0 + \mathcal{E}_0' + \mathcal{E}_1) t + \mathcal{E} \left[ \frac{(t-\tau)^2}{-2} \right]_0^t = \mathcal{E}_0 + C_*(\mathcal{E}_0 + \mathcal{E}_0' + \mathcal{E}_1) t + \frac{1}{2} \mathcal{E} t^2.
\end{align*}
На отрезке $[0,T]$, $t \le T$:
$$ \|\eta(t)\| \le \mathcal{E}_0 + C_* T (\mathcal{E}_0 + \mathcal{E}_0' + \mathcal{E}_1) + \frac{1}{2} \mathcal{E} T^2. $$
Это соответствует (1) из рукописи, если перегруппировать с общей константой $C$.
Аналогично, используя (IV) и (V) для $\eta_t$ и $\nabla \eta$:
$$ \|\eta_t(t)\| \le \sqrt{2} \mathcal{J}_\eta(0) + \int_0^t \|K(\tau)\| \dd\tau \le C_*(\mathcal{E}_0 + \mathcal{E}_0' + \mathcal{E}_1) + \mathcal{E} t. $$
$$ \|\nabla \eta(t)\| \le \frac{\sqrt{2}}{a} \mathcal{J}_\eta(0) + \frac{1}{a} \int_0^t \|K(\tau)\| \dd\tau \le \frac{C_*}{a}(\mathcal{E}_0 + \mathcal{E}_0' + \mathcal{E}_1) + \frac{\mathcal{E}}{a} t. $$
На отрезке $[0,T]$ эти оценки дают правые части (2) и (3) из рукописи с соответствующими константами.
Таким образом, малость $\mathcal{E}, \mathcal{E}_0, \mathcal{E}_0', \mathcal{E}_1$ влечет малость разности решений и их производных, что и означает непрерывную зависимость (корректность по Адамару). $\blacktriangleleft$
\end{proof}
\hfill \textit{74}

\setcounter{section}{1} % Начинаем нумерацию теорем с этого раздела (фиктивно)
\addtocounter{theorem}{4} % Продолжаем нумерацию теорем

Аналогично получаются оценки (2) и (3). $\blacktriangleleft$
(N6 -- см. N3 (стр. 8-оборот) и N5 (стр. 14))
\newpage
\section*{Билет №6 -- 2025}\label{sec:ticket6}
\backtotoc
\section*{N7. Задача Коши для волнового уравнения в $\R^3$ и $\R^2$. Принцип Дюамеля. Формула Кирхгофа. Метод Спуска. Формула Пуассона.}

В данном пункте будем иметь дело с интегралами, зависящими от параметров.
\begin{theorem}[Th из М.А. о дифференцировании интеграла по параметру]
\label{thm:diff_integral_param}
Пусть $\Omega_x \subset \R^n$, $\Omega_y \subset \R^m$ -- ограниченные замкнутые множества. Пусть функция $f(x,y)$ непрерывна на множестве $\overline{\Omega_x \times \Omega_y}$. Тогда интеграл $\mathcal{J}(y) = \int_{\Omega_x} f(x,y) \dd y$ является непрерывной функцией на $\overline{\Omega_y}$, т.е. $\mathcal{J}(y) \in C(\overline{\Omega_y})$.
Если, кроме того, $\frac{\partial f}{\partial y_k}(x,y) \in C(\overline{\Omega_x \times \Omega_y})$, то $\mathcal{J}(y)$ имеет непрерывную частную производную $\frac{\partial \mathcal{J}}{\partial y_k}(y) \in C(\overline{\Omega_y})$ и
$$ \frac{\partial \mathcal{J}}{\partial y_k}(y) = \int_{\Omega_x} \frac{\partial f}{\partial y_k}(x,y) \dd x. $$
\end{theorem}

Будем иметь дело с интегралами специального вида (интеграл Кирхгофа по сфере):
\begin{equation} \label{eq:kirchhoff_integral}
u_g(t,x;\tau) = \frac{1}{4\pi a^2 t} \iint_{|\xi-x|=at} g(\xi, \tau) \dd S_\xi, \quad \underline{(1)}
\end{equation}
где $a>0, t>0, x \in \R^3, \tau \ge 0, \xi \in \R^3$.

\begin{lemma}[Лемма 3.1 о свойствах интеграла Кирхгофа]
\label{lem:kirchhoff_properties}
Пусть:
\begin{enumerate}
    \item $g(\xi; \tau) \in C(\{\xi \in \R^3, \tau \ge 0\})$. \quad $\underline{(.2)}$
    \item $g(\xi; \tau)$ обладает по $\xi_1, \xi_2, \xi_3$ всеми частными производными порядка до $p$ включительно, и эти частные производные также непрерывны по $(\xi, \tau)$ на множестве $\{\xi \in \R^3, \tau \ge 0\}$, т.е. $D_\xi^\alpha g(\xi, \tau) \in C(\{\xi \in \R^3, \tau \ge 0\})$ для всех мультииндексов $|\alpha| \le p$.
\end{enumerate}
Тогда:
\begin{enumerate}
    \item Функция $u_g(t,x;\tau)$ из \eqref{eq:kirchhoff_integral} удовлетворяет условию: $D_{t,x}^\alpha u_g(t,x;\tau) \in C(\{t>0, x \in \R^3, \tau \ge 0\})$ для всех $|\alpha| \le p$. \quad $\underline{(.3)}$
    \item Если $p \ge 1$ (т.е. $g \in C^1$ по $\xi$), то $\lim_{t \to +0} u_g(t,x;\tau) = 0$.
          Если $p \ge 2$ (т.е. $g \in C^2$ по $\xi$), то $\lim_{t \to +0} \frac{\partial u_g}{\partial t}(t,x;\tau) = g(x,\tau)$. \quad $\underline{(.4)}$
\end{enumerate}
(В рукописи условие $p \ge 1$ для $\lim u_g = 0$ отмечено как "верно при $p=0$" -- это требует уточнения, обычно для значения интеграла по исчезающей сфере нужна хотя бы непрерывность $g
$, а для производной -- гладкость $g$).

В интеграле \eqref{eq:kirchhoff_integral} от параметров $t$ и $x$ зависит не только подынтегральная функция (косвенно через $\xi$), но и область интегрирования (сфера $|\xi-x|=at$).
\end{lemma}
\begin{proof}
$\blacktriangle$ Сведём интеграл в \eqref{eq:kirchhoff_integral} к интегралу по единичной сфере. Для этого заменим $\xi$ (с индексом) на $\eta$:
$$ \eta = \frac{\xi-x}{at} \quad \iff \quad \xi = x + at\eta. \quad \underline{(5)} $$
Здесь $\eta = (\eta_1, \eta_2, \eta_3)$ и на сфере $|\xi-x|=at$ имеем $|\eta|=1$.
Если т. (точка) $\eta$ пробегает единичную сферу $|\eta|=1$, то т. $\xi$ пробегает сферу $|\xi-x|=at$.
Элемент площади $\dd S_\xi$ связан с элементом площади на единичной сфере $\dd S_\eta$: $\dd S_\xi = (at)^2 \dd S_\eta = a^2 t^2 \dd S_\eta$.
Тогда интеграл \eqref{eq:kirchhoff_integral} принимает вид:
\begin{equation} \label{eq:kirchhoff_unit_sphere}
u_g(t,x;\tau) = \frac{1}{4\pi a^2 t} \iint_{|\eta|=1} g(x+at\eta, \tau) a^2 t^2 \dd S_\eta = \frac{t}{4\pi} \iint_{|\eta|=1} g(x+at\eta, \tau) \dd S_\eta.
\end{equation}
В этом виде интеграл берётся по фиксированной области (единичная сфера), а параметры $t, x, \tau$ входят только в подынтегральную функцию.
Теперь можно применять теорему о дифференцировании интеграла по параметру (Th из М.А., \ref{thm:diff_integral_param}).

\textit{1) Дифференцируемость $u_g$:}
Поскольку $g \in C^p$ по $\xi$, то $g(x+at\eta, \tau)$ будет $C^p$ по $x$ и $t$ (для $t>0$).
Производные $D_{t,x}^\alpha g(x+at\eta, \tau)$ будут непрерывны по $(t,x,\eta,\tau)$ при $t>0, |\eta|=1, \tau \ge 0$.
Множитель $t$ перед интегралом также $C^\infty$ по $t$ при $t>0$.
Следовательно, $u_g(t,x;\tau)$ будет иметь непрерывные производные $D_{t,x}^\alpha$ до порядка $p$ при $t>0$.

\textit{2) Предельные значения при $t \to +0$:}

\textit{Предел $u_g(t,x;\tau)$ при $t \to +0$:}
$$ \lim_{t \to +0} u_g(t,x;\tau) = \lim_{t \to +0} \left( \frac{t}{4\pi} \iint_{|\eta|=1} g(x+at\eta, \tau) \dd S_\eta \right). $$
Поскольку $g$ непрерывна, интеграл $\iint_{|\eta|=1} g(x+at\eta, \tau) \dd S_\eta$ при $t \to +0$ стремится к $\iint_{|\eta|=1} g(x, \tau) \dd S_\eta = g(x,\tau) \iint_{|\eta|=1} \dd S_\eta = g(x,\tau) \cdot 4\pi$.
Тогда предел равен $\lim_{t \to +0} \left( \frac{t}{4\pi} \cdot 4\pi g(x,\tau) \right) = \lim_{t \to +0} t \cdot g(x,\tau) = 0$.
Это верно, если $g$ непрерывна (т.е. $p \ge 0$).

\textit{Предел $\frac{\partial u_g}{\partial t}(t,x;\tau)$ при $t \to +0$. Требуется $p \ge 1$ для дифференцирования $g$.}
$$ \frac{\partial u_g}{\partial t} = \frac{\partial}{\partial t} \left( \frac{t}{4\pi} \iint_{|\eta|=1} g(x+at\eta, \tau) \dd S_\eta \right). $$
Используем правило Лейбница (дифференцирование произведения и интеграла):
\begin{align*}
\frac{\partial u_g}{\partial t} &= \frac{1}{4\pi} \iint_{|\eta|=1} g(x+at\eta, \tau) \dd S_\eta \\
&\quad + \frac{t}{4\pi} \iint_{|\eta|=1} \frac{\partial}{\partial t} g(x+at\eta, \tau) \dd S_\eta.
\end{align*}
Второе слагаемое: $\frac{\partial}{\partial t} g(x+at\eta, \tau) = \nabla_\xi g(x+at\eta, \tau) \cdot (a\eta)$.
\begin{align*}
\frac{\partial u_g}{\partial t} &= \frac{1}{4\pi} \iint_{|\eta|=1} g(x+at\eta, \tau) \dd S_\eta \\
&\quad + \frac{at^2}{4\pi} \iint_{|\eta|=1} \nabla_\xi g(x+at\eta, \tau) \cdot \eta \dd S_\eta.
\end{align*}
При $t \to +0$:
Первое слагаемое стремится к $\frac{1}{4\pi} g(x,\tau) \cdot 4\pi = g(x,\tau)$ (если $g \in C^0$).
Второе слагаемое: множитель $at^2 \to 0$. Интеграл $\iint \nabla_\xi g \cdot \eta \dd S_\eta$. Если $\nabla_\xi g$ непрерывно (т.е. $g \in C^1$, $p \ge 1$), то интеграл ограничен при $t \to +0$. Тогда второе слагаемое стремится к $0$.
Таким образом, $\lim_{t \to +0} \frac{\partial u_g}{\partial t}(t,x;\tau) = g(x,\tau)$, если $p \ge 1$.
(В рукописи указано, что для этого нужен $p \ge 2$. Это может быть связано с необходимостью существования и непрерывности второй производной самой $u_g$ по $t$, чтобы говорить о решении волнового уравнения, но для самого предела $\partial u_g/\partial t$ достаточно $p \ge 1$).
\end{proof}
\newpage

\section*{Билет №5 -- 2025}\label{sec:ticket5}
\backtotoc
\subsection*{Понятие о корректной постановке задачи математической физики (по Адамару). Пример Адама -
ра некорректно поставленной задачи Коши (для уравнения Лапласа). Корректность смешанной
задачи для волнового уравнения в области из априорной оценки решения. [2] – 66,69, [1] – 61,63.}
\documentclass{article}
\usepackage[T2A]{fontenc}
\usepackage[utf8]{inputenc}
\usepackage[russian]{babel}
\usepackage{amsmath, amssymb}
% \usepackage{enumitem} % Not strictly necessary if using manual labels

\begin{document}

\section*{N3}
Корректность постановки задачи мат. физики. Пример Адамара некорректно поставленной задачи. + Задача Коши для ур-ния колебания струны. Представл. решения. Принцип Дюамеля. Формула Даламбера. Теорема существования и единственности классического решения. + корректность З.К. (Задачи Коши).

Пусть дана линейная дифференциальная задача вида (*): $\Omega$ -- область в $\mathbb{R}^2$.
% The drawing in the image shows a domain Omega with a boundary Gamma.
% Text next to drawing: Gamma is a curve in the closure of Omega.
(Геометрически: $\Omega$ -- область, $\Gamma \subset \overline{\Omega}$ -- кривая в $\overline{\Omega}$.)
\begin{equation*} \label{eq:main_problem_star}
(*) \quad
\left\{
\begin{gathered}
L(x,D) u(x) = f(x), \quad x \in \Omega \\
\text{\footnotesize (лин. дифф. опер. порядка р)} \\
B_j(x,D) u(x) = g_j(x), \quad x \in \Gamma \quad (j=\overline{1,m}) \\
\text{\footnotesize (дифф. опер. на поверх. } \Gamma \text{)}
\end{gathered}
\right.
\end{equation*}
% Note from transcriber: The image specifies "$\Omega$ -- область в $\mathbb{R}^2$" at the beginning.
% For the first equation, the image has "$x \in \Omega \subseteq \mathbb{R}^n$".
% The LaTeX version above uses $x \in \Omega$, which implies $x \in \mathbb{R}^2$ given the preamble.
% If a general $\mathbb{R}^n$ was intended for the definition of L(x,D), it can be adjusted.

\underline{Определение.} Пусть $\exists$
\begin{enumerate}
    \item Линейное нормированное пространство (ЛНП) $F(\Omega)$, $\mathcal{U}(\Omega)$ -- функций, определённых на $\Omega$;
    \item ЛНП $G_1(\Gamma), \dots, G_m(\Gamma)$ -- функций, определённых на $\Gamma$;
\end{enumerate}
такие что: $\forall f(x) \in F(\Omega)$ и $\forall g_j(x) \in G_j(\Gamma)$ краевая задача (*) имеет \underline{единственное решение} $u(x) \in \mathcal{U}(\Omega)$ и для этого решения \underline{справедлива оценка}
\begin{equation*}
    \|u\|_{\mathcal{U}(\Omega)} \le C \|f\|_{F(\Omega)} + \sum_{j=1}^{m} C_j \|g_j\|_{G_j(\Gamma)},
\end{equation*}
причём $C, C_1, \dots, C_m$ не зависят от функций $f$ и $g_j$.
% Note from transcriber: оригінальний текст мав "$g_j(x) \in G_j(x)$", що виглядає як можлива помилка. 
% "$g_j(x) \in G_j(\Gamma)$" є більш стандартним, враховуючи визначення $G_j(\Gamma)$ як простору функцій на $\Gamma$ та норму $\|g_j\|_{G_j(\Gamma)}$.

Тогда говорят, что задача \underline{корректна} или \underline{корректно поставлена} в \underline{системе пространств}.

Т.о. корректно поставленная задача должна обладать свойствами:
\begin{itemize}
    \item[а)] решение должно существовать в каком-то классе функций.
    \item[б)] решение должно быть единственным в каком-то классе функций.
    \item[в)] решение должно непрерывно зависеть от данных задачи.
\end{itemize}

\underline{Пример Адамара} некорректно поставленной З.К. (Задачи Коши) (для ур-ния Лапласа):
$$
\begin{cases}
u_{xx} + u_{yy} = 0 \\
u|_{y=0} = u_0(x) \\
u_y|_{y=0} = u_1(x)
\end{cases}
$$
Частный случай:
$$
\begin{cases}
u_{xx} + u_{yy} = 0 \\
u|_{y=0} = 0 \\
u_y|_{y=0} = 0
\end{cases}
$$
--- тривиальное решение: $u(x,y) \equiv 0$, по Th. (Теореме) Коши-Ковалевской оно единственно в некоторой окрестности.

\bigskip % Add some space after the previous content

Теперь рассмотрим при $n \in \mathbb{N}$:
$$
\begin{cases}
u_{xx} + u_{yy} = 0 \\
u|_{y=0} = e^{-\sqrt{n}} \cos nx \xrightarrow{n \to \infty} 0 \\
u_y|_{y=0} = 0
\end{cases}
$$
$f_n(x) = e^{-\sqrt{n}} \cos nx \xrightarrow{n \to \infty} 0$; --- небольшое возмущение начального условия первой задачи.
% (Геометрическое пояснение: рассматривается область $\Omega$ в полуплоскости $y>0$; точка $(x^*=0, y^*>0)$ лежит в этой области.)

Непрерывная зависимость решения $u$ от данных задачи означает следующее: пусть последовательность начальных данных $g_j^k$, $k=1,2,\dots$ в каком-то смысле стремится к $g_j$ при $k \to \infty$, и $u_k$, $k=1,2,\dots$ --- соответствующие решения задачи, тогда должно быть $u_k \xrightarrow{k \to \infty} u$ в смысле выбранной сходимости.

Решение второй задачи: $u_*(x,y) = e^{-\sqrt{n}} \cos nx \cdot \cosh ny$.
Задача была бы корректной, если бы $u_*(x,y) \xrightarrow{n \to \infty} 0$, однако:
$$ u_*(0, y_*) = e^{-\sqrt{n}} \cosh(n y_*) > \frac{1}{2} e^{-\sqrt{n} + n y_*} = \frac{1}{2} e^{n y_* - \sqrt{n}} = \frac{1}{2} e^{n y_*(1 - \frac{1}{y_* \sqrt{n}})} \xrightarrow{n \to \infty} \infty $$
(поскольку $y_*>0$, то $1 - \frac{1}{y_* \sqrt{n}} \to 1$ при $n \to \infty$, и $n y_* \to \infty$).
$\implies$ Вторая задача для ур-ния Лапласа \underline{не является корректной}.
\newpage
\section*{Билет №2 -- 2025}\label{sec:ticket2}
\backtotoc
\subsection*{Задача Коши для уравнения колебания струны. Представление решения. Принцип Дюамеля.
Формула Даламбера. Теорема существования и единственности классического решения. [2] --
39-42, 60-62.}
\subsection*{З.К. для ур-ния колебаний струны}

Найдём в общем виде решение классической задачи Коши для однородного волнового ур-ния в $\mathbb{R}^1$:
$$
\begin{cases}
u_{tt} - a^2 u_{xx} = 0, \quad Q = \{(t,x) : x \in \mathbb{R}^1, t > 0\}, \quad a > 0 \\
u(t,x)|_{t=0} = u_0(x) \\
u_t(t,x)|_{t=0} = u_1(x), \quad x \in \mathbb{R}^1
\end{cases}
$$
Задание начальных условий на поверхности $S = \{(x,t) : x \in \mathbb{R}^1, t=0\}$ и ограничение $t>0$ оправданы содержательной стороной задачи. Для справедливости дальнейших выкладок необходимо, чтобы $u_0(x) \in C^2(\mathbb{R}^1)$, $u_1(x) \in C^1(\mathbb{R})$. Найдём функцию $u(x,t)$ из класса $C^2(Q) \cap C^1(\overline{Q})$, удовлетворяющую поставленной задаче.

\textbf{а) Характеристики:}
Уравнение характеристик для $u_{tt} - a^2 u_{xx} = 0$ имеет вид:
$$ (dx)^2 - a^2 (dt)^2 = 0 \iff (dx - a dt)(dx + a dt) = 0 $$
Отсюда получаем два семейства характеристик:
$$
\begin{cases}
dx + a dt = 0 \implies x + at = C_1 \\
dx - a dt = 0 \implies x - at = C_2
\end{cases}
$$
Введём новые переменные (характеристические координаты):
$$ \begin{cases} \xi = x+at \\ \eta = x-at \end{cases} $$
Тогда $u(t,x) = \hat{u}(\xi(t,x), \eta(t,x))$. Найдём производные:
$$ \frac{\partial u}{\partial t} = \frac{\partial \hat{u}}{\partial \xi} \frac{\partial \xi}{\partial t} + \frac{\partial \hat{u}}{\partial \eta} \frac{\partial \eta}{\partial t} = \frac{\partial \hat{u}}{\partial \xi} \cdot a + \frac{\partial \hat{u}}{\partial \eta} \cdot (-a) = a (\hat{u}_\xi - \hat{u}_\eta) $$
$$ \frac{\partial^2 u}{\partial t^2} = a \frac{\partial}{\partial t} (\hat{u}_\xi - \hat{u}_\eta) = a \left( \frac{\partial (\hat{u}_\xi - \hat{u}_\eta)}{\partial \xi} \frac{\partial \xi}{\partial t} + \frac{\partial (\hat{u}_\xi - \hat{u}_\eta)}{\partial \eta} \frac{\partial \eta}{\partial t} \right) $$
$$ = a \left( (\hat{u}_{\xi\xi} - \hat{u}_{\eta\xi}) \cdot a + (\hat{u}_{\xi\eta} - \hat{u}_{\eta\eta}) \cdot (-a) \right) = a^2 (\hat{u}_{\xi\xi} - 2\hat{u}_{\xi\eta} + \hat{u}_{\eta\eta}) $$
(предполагая $\hat{u}_{\xi\eta} = \hat{u}_{\eta\xi}$).
$$ \frac{\partial u}{\partial x} = \frac{\partial \hat{u}}{\partial \xi} \frac{\partial \xi}{\partial x} + \frac{\partial \hat{u}}{\partial \eta} \frac{\partial \eta}{\partial x} = \frac{\partial \hat{u}}{\partial \xi} \cdot 1 + \frac{\partial \hat{u}}{\partial \eta} \cdot 1 = \hat{u}_\xi + \hat{u}_\eta $$
$$ \frac{\partial^2 u}{\partial x^2} = \frac{\partial}{\partial x} (\hat{u}_\xi + \hat{u}_\eta) = \frac{\partial (\hat{u}_\xi + \hat{u}_\eta)}{\partial \xi} \frac{\partial \xi}{\partial x} + \frac{\partial (\hat{u}_\xi + \hat{u}_\eta)}{\partial \eta} \frac{\partial \eta}{\partial x} $$
$$ = (\hat{u}_{\xi\xi} + \hat{u}_{\eta\xi}) \cdot 1 + (\hat{u}_{\xi\eta} + \hat{u}_{\eta\eta}) \cdot 1 = \hat{u}_{\xi\xi} + 2\hat{u}_{\xi\eta} + \hat{u}_{\eta\eta} $$
Подставляя в уравнение $u_{tt} - a^2 u_{xx} = 0$:
$$ a^2 (\hat{u}_{\xi\xi} - 2\hat{u}_{\xi\eta} + \hat{u}_{\eta\eta}) - a^2 (\hat{u}_{\xi\xi} + 2\hat{u}_{\xi\eta} + \hat{u}_{\eta\eta}) = 0 $$
$$ a^2 (-4 \hat{u}_{\xi\eta}) = 0 \implies \hat{u}_{\xi\eta} = 0 $$
Интегрируя $\frac{\partial^2 \hat{u}}{\partial \xi \partial \eta} = 0$ сначала по $\eta$, получаем $\frac{\partial \hat{u}}{\partial \xi} = \phi(\xi)$, где $\phi(\xi)$ --- произвольная функция.
Интегрируя $\frac{\partial \hat{u}}{\partial \xi} = \phi(\xi)$ по $\xi$, получаем $\hat{u}(\xi, \eta) = \int \phi(\xi) d\xi + g(\eta) = f(\xi) + g(\eta)$,
где $f(\xi)$ и $g(\eta)$ --- произвольные функции (если мы хотим получить классическое решение, то $f, g \in C^2$).
Возвращаясь к исходным переменным:
$$ u(t,x) = f(x+at) + g(x-at) \quad \text{--- \underline{общий вид решения.}} $$
Это решение представляет собой суперпозицию двух волн:
\begin{itemize}
    \item $f(x+at)$ --- волна, бегущая влево (в сторону отрицательных $x$) со скоростью $a$.
    \item $g(x-at)$ --- волна, бегущая вправо (в сторону положительных $x$) со скоростью $a$.
\end{itemize}
% (Изображены графики функций f(x+at_0) и g(x-at_0), а затем g(x-at_1) при t_1 > t_0, демонстрирующие движение волны вправо).

\bigskip

Т.о. решение З.К. является суммой прямой волны $g(x-at)$ и обратной $f(x+at)$.
Чтобы найти $f$ и $g$, используем начальные условия:
$$
\begin{cases}
u(t,x)|_{t=0} = u_0(x) = f(x) + g(x) \\
u_t(t,x)|_{t=0} = u_1(x) = a f'(x) - a g'(x)
\end{cases}
$$
$$
\implies
\begin{cases}
f(x) + g(x) = u_0(x) \\
(f(x) - g(x))' = \frac{1}{a} u_1(x)
\end{cases}
\implies
\begin{cases}
f(x) + g(x) = u_0(x) \\
f(x) - g(x) = \frac{1}{a} \int_{x_0}^{x} u_1(\xi) d\xi + C = \frac{1}{a} U_1(x) + C
\end{cases}
$$
где $U_1(x) = \int_{x_0}^{x} u_1(\xi) d\xi$, $x_0$ -- произвольная фиксированная точка на оси $\mathbb{R}$. Для удобства, можем выбрать $C$ так, чтобы $f(x_0)=g(x_0)$ или другую константу. Часто выбирают $C$ так, чтобы выражение для $f(x)-g(x)$ было проще, например, интегрируя от $-\infty$ или $0$, или просто $f(x) - g(x) = \frac{1}{a} \int u_1(x)dx$.
Если принять $f(x)-g(x) = \frac{1}{a} U_1(x)$ (т.е. $C=0$ или $U_1(x)$ --- некоторая первообразная $\frac{1}{a}u_1(x)$), то:
$$
\begin{cases}
f(x) = \frac{1}{2} u_0(x) + \frac{1}{2a} U_1(x) \\
g(x) = \frac{1}{2} u_0(x) - \frac{1}{2a} U_1(x)
\end{cases}
$$
Тогда решение $u(t,x) = f(x+at) + g(x-at)$ принимает вид:
$$ u(t,x) = \frac{1}{2} u_0(x+at) + \frac{1}{2a} U_1(x+at) + \frac{1}{2} u_0(x-at) - \frac{1}{2a} U_1(x-at) $$
$$ u(t,x) = \frac{u_0(x+at) + u_0(x-at)}{2} + \frac{1}{2a} (U_1(x+at) - U_1(x-at)) $$
Поскольку $U_1(x+at) - U_1(x-at) = \int_{x_0}^{x+at} u_1(\xi) d\xi - \int_{x_0}^{x-at} u_1(\xi) d\xi = \int_{x-at}^{x+at} u_1(\xi) d\xi$, то:
\begin{equation*} \boxed{
u(t,x) = \frac{u_0(x+at) + u_0(x-at)}{2} + \frac{1}{2a} \int_{x-at}^{x+at} u_1(\xi) d\xi
} \quad \text{\underline{Формула Даламбера}}
\end{equation*}

Из представления решения формулой Даламбера следует, что для нахождения решения $u$ в точке $(x_0, t_0) \in Q$ достаточно знать значение $u_1(x)$ на отрезке $[x_0 - at_0, x_0 + at_0]$ и значения $u_0(x)$ в точках $x_0 - at_0$ и $x_0 + at_0$. По этой причине отрезок $[x_0 - at_0, x_0 + at_0]$ называется \underline{областью зависимости} для т. $(x_0, t_0)$.
% (Изображен конус характеристик, исходящий из точки $(x_0, t_0)$ вниз к оси $t=0$, пересекая ее в точках $x_0-at_0$ и $x_0+at_0$. Основание треугольника — отрезок $[x_0-at_0, x_0+at_0]$ на оси $x$ при $t=0$. Вершина — $(x_0, t_0)$. Стороны треугольника — характеристики $(x-x_0) = a(t-t_0)$ и $(x-x_0) = -a(t-t_0)$ или эквивалентно $x-at = x_0-at_0$ и $x+at=x_0+at_0$.)

В случае условий, заданных при $|x| < \ell$:
%_Изображена ось x с отрезком $(-\ell, \ell)$_
Все выкладки для $f(x)$ и $g(x)$ справедливы при $|x| < \ell$, поэтому вне диапазона $|x| < \ell$ $f$ и $g$ неизвестны, т.е. вне $|x| < \ell$ нет однозначного решения. Однозначное решение можем получить, где аргументы $x+at$ и $x-at$ позволяют однозначно восстановить $f$ и $g$, т.е. $|x+at| < \ell$ и $|x-at| < \ell$.

\underline{Подытожим полученные результаты:}

\textbf{Th.2.1} Пусть $u_0(x) \in C^2(-\ell, \ell)$, $u_1(x) \in C^1(-\ell, \ell)$. Тогда задача Коши
$$
\begin{cases}
u_{tt} - a^2 u_{xx} = 0, \quad (t,x) \in Q \\
u|_{t=0} = u_0(x), \quad u_t|_{t=0} = u_1(x), \quad (|x| < \ell)
\end{cases}
$$
в области $Q = \{ (t,x) : |x \pm at| < \ell \}$ имеет \underline{единственное решение} $u(t,x) \in C^2(Q)$, которое называется \underline{классическим}.
Оно представимо формулой Даламбера.


\bigskip

% Image: Characteristic diamond showing the domain Q={ (t,x) : |x \pm at| < l }.
% The base is the interval (-l, l) on the x-axis.
% The region Q is the diamond formed by the lines x+at = l, x+at = -l, x-at = l, x-at = -l.
% The text "интервал данных Коши" points to the interval (-l, l).
% An arrow points from the interval (-l,l) to the region Q.

В описанных в теореме предположениях, решение сформулированной З.К. в точках области $Q$ существует, единственно и принадлежит классу $C^2$.

$\blacktriangle$ \textit{Обратимся к формуле Даламбера}: из неё следует, что $u(t,x) \in C^2$ действительно:
в области $Q$ $|x \pm at| < \ell$ и $u_0(x \pm at) \in C^2$, т.к. $u_0(x) \in C^2(-\ell, \ell)$, и $u_1(x) \in C^1(-\ell, \ell) \implies U_1(x) = \int u_1(x) dx \in C^2$.

Покажем, что формула Даламбера удовлетворяет ур-нию и нач. усл.
Используя данные Коши и ур-ние, получим формулу для $u(t,x) \implies u(t,x)$ -- решение З.К.
Обратное верно: зная $u(t,x)$ можно проверить, что в точках области $Q$ она удовлетворяет ур-нию, а на интервале Коши удовлетворяет данным Коши. Т.о. формула Даламбера даёт \underline{общее решение}.

Пусть $\exists$ 2 решения: $u, v \in C^2(t>0) \cap C^1(\overline{t>0})$.
Рассмотрим $y(t,x) = u(t,x) - v(t,x)$ -- она удовлетворяет
$$ \begin{cases} y_{tt} - a^2 y_{xx} = 0 \\ y|_{t=0} = 0 \\ y_t|_{t=0} = 0 \end{cases} $$
$\implies$ Из формулы Даламбера: $y = u-v \equiv 0 \implies$ \underline{решение единственно}. $\blacktriangle$

$\bigodot$ \textit{Рассмотрим, как зависит решение от изменения входных данных.}
Рассмотрим две задачи Коши:
$$
\text{(З.К.1)} \quad
\begin{cases}
\overset{1}{u}_{tt} - a^2 \overset{1}{u}_{xx} = 0, \quad (t,x) \in Q \\
\overset{1}{u}|_{t=0} = \overset{1}{u}_0(x), \quad \overset{1}{u}_t|_{t=0} = \overset{1}{u}_1(x), \quad |x| < \ell
\end{cases}
$$
$$
\text{(З.К.2)} \quad
\begin{cases}
\overset{2}{u}_{tt} - a^2 \overset{2}{u}_{xx} = 0, \quad (t,x) \in Q \\
\overset{2}{u}|_{t=0} = \overset{2}{u}_0(x), \quad \overset{2}{u}_t|_{t=0} = \overset{2}{u}_1(x), \quad |x| < \ell
\end{cases}
$$
$\overset{1}{u}_0(x), \overset{2}{u}_0(x) \in C^2(-\ell, \ell)$, $\overset{1}{u}_1(x), \overset{2}{u}_1(x) \in C^1(-\ell, \ell) \implies \exists!$ реш. З.К.1, $\exists!$ реш. З.К.2.
Пусть $|\overset{1}{u}_0(x) - \overset{2}{u}_0(x)| \le \delta_0$, $|\overset{1}{u}_1(x) - \overset{2}{u}_1(x)| \le \delta_1$, $\forall x: |x| < \ell$. Покажем, что $\sup_{(t,x) \in Q} |\overset{1}{u}(t,x) - \overset{2}{u}(t,x)|$ -- малая величина.

В силу линейности системы З.К. введём $V_0(x) = \overset{1}{u}_0(x) - \overset{2}{u}_0(x) \in C^2(-\ell, \ell)$, $V_1(x) = \overset{1}{u}_1(x) - \overset{2}{u}_1(x) \in C^1(-\ell, \ell)$, $V(t,x) = \overset{1}{u}(t,x) - \overset{2}{u}(t,x)$.
$\implies$
$$
\begin{cases}
V_{tt} - a^2 V_{xx} = 0, \quad (t,x) \in Q \\
V|_{t=0} = V_0(x), \quad V_t|_{t=0} = V_1(x), \quad |x| < \ell
\end{cases}
$$
причём $|V_0(x)| \le \delta_0$, $|V_1(x)| \le \delta_1$. ($V_0 \in C^2, V_1 \in C^1$)
Получим З.К. на $V(t,x) \implies V(t,x) = \frac{V_0(x+at) + V_0(x-at)}{2} + \frac{1}{2a} \int_{x-at}^{x+at} V_1(\xi) d\xi$.
$\implies |V(t,x)| \le \frac{|V_0(x+at)| + |V_0(x-at)|}{2} + \frac{1}{2a} \int_{x-at}^{x+at} |V_1(\xi)| d\xi \quad [(t,x) \in Q]$
$\le \frac{\delta_0 + \delta_0}{2} + \frac{1}{2a} \cdot \delta_1 \cdot |(x+at) - (x-at)| = \delta_0 + \frac{1}{2a} \delta_1 \cdot 2a|t| = \delta_0 + \delta_1 |t|$.

\underline{Случаи}:


\bigskip

\begin{enumerate}
    \item[1)] $\ell$ -- конечное число.
    $|t| < \ell/a \implies \sup_{(t,x) \in Q} |V(t,x)| = \sup_{(t,x) \in Q} |\overset{1}{u}(t,x) - \overset{2}{u}(t,x)| \le \delta_0 + \delta_1 \frac{\ell}{a} \xrightarrow{\delta_0, \delta_1 \to 0} 0$.

    \item[2)] $\ell = \infty$. Возьмём ограничение по времени: $0 < t < T \implies$
    $\sup_{(t,x) \in Q, 0<t<T} |\overset{1}{u}(t,x) - \overset{2}{u}(t,x)| \le \delta_0 + T \delta_1 \quad \forall T \in \mathbb{R}$
    $\xrightarrow{\delta_0, \delta_1 \to 0} 0$.
\end{enumerate}
Т.о. решение задачи Коши непрерывно зависит от начальных условий в \underline{равномерной метрике}.

\begin{itemize}
    \item В предыдущих теоремах были доказаны единственность и существование решения при дополнительных условиях на гладкость начальных данных. Теперь доказана непрерывная зависимость от начальных данных, что завершает доказательство \underline{корректности} классической З.К. для волнового ур-ния в $\mathbb{R}^1$. \\
    (+ про обобщённое (сильное) решение см. С13-Ф I-сем. lect.)
\end{itemize}

\subsection*{\underline{Принцип Дюамеля для волнового ур-ния.}}
Принцип Дюамеля позволяет, зная решение З.К. для однородного ур-ния, найти решение и для неоднородного.
Рассмотрим З.К. для волн. ур-ния в $\mathbb{R}^1$:
\begin{equation} \label{eq:nonhom_wave_eq}
\left\{
\begin{gathered}
u_{tt}(t,x) - a^2 \Delta_x u(t,x) = f(t,x), \quad x \in \mathbb{R}^1, t>0 \quad (1), \quad a>0 \\
u(t,x)|_{t=0} = u_0(x), \quad u_t(t,x)|_{t=0} = u_1(x) \quad (2)
\end{gathered}
\right.
\end{equation}
В силу линейности задачи будем искать решение в виде $u(t,x) = v(t,x) + w(x,t)$, где
$$
(*) \quad
\begin{cases}
v_{tt}(t,x) - a^2 \Delta_x v(t,x) = 0, \quad x \in \mathbb{R}^1, t>0, \\
v(t,x)|_{t=0} = u_0(x), \quad v_t(t,x)|_{t=0} = u_1(x);
\end{cases}
\quad v(t,x) \text{ - решение З.К. для однородного волнового ур-ния.}
$$
$$
(**) \quad
\begin{cases}
w_{tt}(t,x) - a^2 \Delta_x w(t,x) = f(t,x), \quad x \in \mathbb{R}^1, t>0, \\
w(t,x)|_{t=0} = 0, \quad w_t(t,x)|_{t=0} = 0;
\end{cases}
\quad w(t,x) \text{ - решение З.К. с однор. нач. усл.}
$$
Найдём функцию $w$ из класса $C^2(\mathbb{R}^n \times [0; +\infty))$, для чего рассмотрим З.К. из однопараметрического семейства $\tau \ge 0$:
\begin{equation} \label{eq:aux_problem_p}
\left\{
\begin{gathered}
p_{tt}(x,t;\tau) - a^2 \Delta_x p(x,t;\tau) = 0, \quad x \in \mathbb{R}^1, t>\tau \quad (3) \\
p|_{t=\tau} = 0, \quad p_t|_{t=\tau} = f(x;\tau) \quad (4)
\end{gathered}
\right.
\end{equation}

\textbf{Утв.1 [Принцип Дюамеля]} Если существует решение $p(x,t;\tau)$ вспомогательной задачи (3)-(4), $p(x,t;\tau) \in C^{2,2,0}_{x,t,\tau}(\overline{M})$, где $M = \{(x,t,\tau) : x \in \mathbb{R}^n, \tau>0, t>\tau \}$, то функция
$W(t,x) = \int_0^t p(x,t;\tau) d\tau$ удовлетворяет З.К. для неоднор. волн. ур-ния с однор. нач. усл.



\pagestyle{empty} % To avoid LaTeX's own page numbering if this is part of a larger document.
% Start of content from the image
\bigskip

Т.е. при фикс. $\tau \ge 0$ строится решение $p(x,t;\tau)$ З.К. (3)$\cap$(4) для $t > \tau$.

% Diagram interpretation: The integral for W(x,t_0) is \int_0^{t_0} p(x,t_0; \tau) d\tau.
% The diagram shows the (t, tau) plane. A is (tau=0, t=t_0), B is (tau=t_0, t=t_0).
% The region p=0 is for t < tau.
% The statement implies integration along the line segment AB if t_0 is fixed.
% A more general interpretation for W(x,t) = \int_0^t p(x,t;\tau)d\tau is integrating over the triangle 0 <= \tau <= t.
\[
\begin{tikzpicture}[scale=0.8]
    \draw[->] (0,0) -- (4,0) node[right] {$\tau$};
    \draw[->] (0,0) -- (0,4) node[above] {$t$};
    \node at (2.5, -0.5) {$t_0$};
    \node at (-0.5, 2.5) {$t_0$};
    \draw (2.5,0.1) -- (2.5,-0.1); % Mark t_0 on tau axis
    \draw (0.1,2.5) -- (-0.1,2.5); % Mark t_0 on t axis

    \coordinate (O) at (0,0);
    \coordinate (T0_tau) at (2.5,0);
    \coordinate (T0_t) at (0,2.5);
    \coordinate (P) at (2.5,2.5); % Point (t_0, t_0)

    % Shaded region for integration for a fixed t_0 (rectangle in diagram, actually a triangle)
    % The diagram shows a rectangle 0 <= \tau <= t_0, 0 <= t <= t_0
    % and a line t=t_0, and the line t=\tau
    % Let's draw what is likely intended: domain for \int_0^t p(x,t;\tau') d\tau'
    \fill[gray!30] (O) -- (P) -- (T0_t) -- cycle; % Triangle 0 <= \tau <= t, for t up to t_0
    \node at (1,1.5) {$0 \le \tau \le t$};

    \node[below left] at (O) {$0$};
    \draw (O) -- (P) node[above right, pos=0.8] {$t=\tau$};

    % Points A and B mentioned in text for a fixed t=t_0
    \node[left] at (T0_t) {A}; % A is (tau=0, t=t_0)
    \node[right] at (P) {B};  % B is (tau=t_0, t=t_0)
    \draw[thick, blue] (T0_t) -- (P) node[midway, above, sloped, blue] {integration for fixed $t=t_0$};
    \node at (1.5,0.5) {$p=0$ for $t<\tau$}; % Region where p=0
    \node at (3,3) {}; % To make space
\end{tikzpicture}
\]
Решение $W$ неоднородного ур-ния в момент $t=t_0$ получается интегрированием по $AB$ функции $p(x, t_0, \tau)$.
% Here AB is the segment from (tau=0, t=t_0) to (tau=t_0, t=t_0).

$\blacktriangle$ В силу условий гладкости $p(x,t,\tau)$ выполнено: (т.к. $W(x,t) = \int_0^t p(x,t;\tau)d\tau$, то)
\begin{enumerate}
    \item[1)] $\underline{W(x,0)} = \int_0^0 p(x,0;\tau)d\tau = \underline{0}$;
    \item[2)] $W_t = \frac{\partial}{\partial t} \int_0^t p(x,t;\tau)d\tau = \underbrace{p(x,t;t)}_{\text{=0 из усл.(4) для } p} + \int_0^t p_t(x,t;\tau)d\tau = \int_0^t p_t(x,t;\tau)d\tau, \quad t>0$;
    \item[3)] $\underline{W_t(x,0)} = \left. \int_0^t p_t(x,t;\tau)d\tau \right|_{t=0} = \underline{0}$;
    \item[4)] $W_{tt} = \frac{\partial}{\partial t} \left( \int_0^t p_t(x,t;\tau)d\tau \right) = \underbrace{p_t(x,t;t)}_{\substack{=f(x;t) \\ \text{по усл.(4) для } p}} + \int_0^t \underbrace{p_{tt}(x,t;\tau)}_{\substack{=a^2 \Delta_x p(x,t;\tau) \\ \text{из ур.(3) для } p}} d\tau$
    $= f(x,t) + \int_0^t a^2 \Delta_x p(x,t;\tau)d\tau = f(x,t) + a^2 \Delta_x \underbrace{\int_0^t p(x,t;\tau)d\tau}_{W(x,t)}$
    $\implies W_{tt} - a^2 \Delta_x W = f(x,t), \quad t>0$.
\end{enumerate}
Т.о. получили, что для функции $W(x,t) = \int_0^t p(x,t;\tau)d\tau$ выполнены однородные гран. усл. и неоднородное волновое ур-ние с правой частью $f(x,t)$, т.е. ч.т.д. $\blacktriangle$

\subsection*{\underline{Применение метода Дюамеля совместно с формулой Даламбера.}}
В одномерном случае: $x \in \mathbb{R}^1$, $f(t,x) \in C_{x,t}^{1,0}(\mathbb{R}^1 \times [0;+\infty))$, тогда для решения З.К. (3)$\cap$(4) используем формулу Даламбера:
$$ p(x,t;\tau) = \frac{1}{2a} \int_{x-a(t-\tau)}^{x+a(t-\tau)} f(\xi;\tau) d\xi $$
$=[$В силу гладкости $f] \implies p \in C^{2,2,0}_{x,t,\tau}(\overline{M})$, поэтому:
$$ W(x,t) = \frac{1}{2a} \int_0^t \left( \int_{x-a(t-\tau)}^{x+a(t-\tau)} f(\xi;\tau) d\xi \right) d\tau $$
$\implies$ Решение З.К.
$$
\begin{cases}
u_{tt} - a^2 \Delta_x u(x,t) = f(x,t), \quad x \in \mathbb{R}, t>0 \\
u(x,0)=u_0(x) \in C^2(\mathbb{R}) \\
u_t(x,0)=u_1(x) \in C^1(\mathbb{R}) \\
f \in C^{1,0}_{x,t}(\mathbb{R}^1 \times [0;+\infty))
\end{cases}
$$
выражается суммой двух формул Даламбера для задач (*) и (**):
\begin{multline*}
u(x,t) = \frac{1}{2a} \int_0^t \left( \int_{x-a(t-\tau)}^{x+a(t-\tau)} f(\xi,\tau)d\xi \right) d\tau \\
+ \frac{u_0(x+at)+u_0(x-at)}{2} + \frac{1}{2a} \int_{x-at}^{x+at} u_1(\xi) d\xi.
\end{multline*}

\bigskip
\hfill \texttt{11} %Page number from the image

\newpage

\section*{Билет №7 -- 2025}\label{sec:ticket7}
\backtotoc
\subsection*{Метод Фурье решения начально-краевой задачи для уравнения колебания струны на конечном
отрезке. Условия согласования начальных и граничных данных. [2] – 79-86.}

\textbf{Метод Фурье решения начально-краевой задачи для уравнения колебания струны на конечном отрезке. Условия согласования начальных и граничных данных.}

\subsection*{Суть метода:}
\begin{enumerate}
    \item Сведение к задаче с однородными граничными условиями;
    \item Построение формального решения в виде функционального ряда с разделёнными переменными;
    \item Доказательство сходимости этого ряда к функции, являющейся решением задачи из заданного класса.
\end{enumerate}

$G = (0,l) \times (0;T) \text{ - область.}$
$$
\begin{cases}
    \varphi_{tt}(x,t) - a^2 \varphi_{xx}(x,t) = F(x,t), \quad (x,t) \in G=(0;l)\times(0;T); \\
    \left. \varphi(x,t) \right|_{t=0} = U_0(x), \quad \left. \varphi_t(x,t) \right|_{t=0} = U_1(x); \\
    \left. \varphi(x,t) \right|_{x=0} = \mu(t), \quad \left. \varphi(x,t) \right|_{x=l} = \nu(t);
\end{cases}
$$
Необходимо найти решение в классе $C^2(G) \cap C^1(\bar{G})$.

\bigskip % Add some vertical space

\noindent\textcircled{1} \textbf{Сведение смешанной задачи к задаче с однородными граничными условиями:} $\mu, \nu \in C^2[0,T]$:
\[
u(x,t) = \varphi(x,t) - \nu(t) - \frac{\nu(t) - \mu(t)}{l} (x-l) \implies
\]
% The arrow implies that this transformation leads to the following system for u:
$$
\begin{cases}
    u_{tt} - a^2 u_{xx} = f(x,t) \\
    \left. u(x,t) \right|_{t=0} = u_0(x), \quad \left. u_t(x,t) \right|_{t=0} = u_1(x) \\
    \left. u(x,t) \right|_{x=0} = 0, \quad \left. u(x,t) \right|_{x=l} = 0
\end{cases}
\eqno{(1)}
$$

\bigskip % Add some vertical space

\noindent\textcircled{2} \textbf{Решение частной задачи (1) может быть найдено при помощи метода Фурье, одним из элементов которого является метод разделения переменных:}

Решим задачу о нахождении собственных функций оператора $-\frac{d^2}{dx^2}$, определённого на линейном множестве функций:
\[
M = \{ X(x) : X \in C^2(0,l) \cap C[0,l], X'' \in L_2(0,l), X(0)=X(l)=0 \}
\]




$$
\begin{cases}
    -X''(x) = \Omega X(x), \quad 0 < x < l; \\
    X(0) = X(l) = 0;
\end{cases}
\eqno{(2)}
$$
Заметим, что $X(x)$ удовлетворяют однородным граничным условиям на отрезке $[0,l]$ и являются собственными функциями для оператора $-\frac{d^2}{dx^2}$, входящего в волновое уравнение.
\newline
$\implies$ Имеет смысл решение задачи (1) искать в виде: $u(x,t) = T(t)X(x)$, причём $u(x,t) \not\equiv 0$.
\newline
$\implies T_{tt}(t)X(x) - a^2 T(t)X''(x) = F(x,t)$; Для однородного уравнения:
$$ \frac{T_{tt}(t)}{T(t)} \underbrace{=}_{\text{зависит от } t} a^2 \frac{X''(x)}{X(x)} \underbrace{=}_{\text{зависит от } x} -\lambda a^2 $$
(где $\lambda$ — константа разделения; далее будем считать $\lambda = \Omega$ из задачи (2)).
\newline
Явно возникла рассматриваемая задача на собственные функции оператора $-\frac{d^2}{dx^2}$.

\begin{enumerate}
    \item Пусть $\Omega < 0$. Общее решение уравнения $-X''(x) = \Omega X(x)$, то есть $X''(x) - |\Omega|X(x)=0$, имеет вид:
    $X(x) = c_1 e^{\sqrt{|\Omega|}x} + c_2 e^{-\sqrt{|\Omega|}x}$. Удовлетворяя граничным условиям, получаем:
    $$
    \begin{cases}
        X(0) = c_1 e^{\sqrt{|\Omega|}\cdot 0} + c_2 e^{-\sqrt{|\Omega|}\cdot 0} = c_1 + c_2 = 0 \\
        X(l) = c_1 e^{\sqrt{|\Omega|}l} + c_2 e^{-\sqrt{|\Omega|}l} = 0
    \end{cases}
    \implies
    \begin{cases}
        c_1 + c_2 = 0 \\
        c_1 e^{\sqrt{|\Omega|}l} + c_2 e^{-\sqrt{|\Omega|}l} = 0
    \end{cases}
    $$
    Откуда, при $c_1^2 + c_2^2 > 0$ (в силу требования нетривиальности решения) получаем:
    $$ \begin{vmatrix} 1 & 1 \\ e^{\sqrt{|\Omega|}l} & e^{-\sqrt{|\Omega|}l} \end{vmatrix} = 0 \implies e^{-\sqrt{|\Omega|}l} - e^{\sqrt{|\Omega|}l} = 0 \implies 1 - e^{2\sqrt{|\Omega|}l} = 0 $$
    $$ \implies e^{2\sqrt{|\Omega|}l} = 1 \implies \sqrt{|\Omega|}l = 0 \text{ - невозможно (т.к. } l>0, |\Omega|>0). $$
    Итак, при $\Omega < 0$ задача (2) имеет только тривиальное решение $\implies$ отрицательных С.З. (собственных значений) нет.

    \item Пусть $\Omega = 0$. Общее решение $X''(x)=0$ имеет вид: $X(x) = c_1 x + c_2$. Удовлетворяя граничным условиям, получаем:
    $$
    \begin{cases}
        X(0) = c_1 \cdot 0 + c_2 = c_2 = 0 \\
        X(l) = c_1 l + c_2 = 0
    \end{cases}
    \implies
    \begin{vmatrix} 0 & 1 \\ l & 1 \end{vmatrix} = 0 \implies -l=0 \implies l=0 \text{ - невозможно.}
    $$
    $\implies \Omega=0$ не является С.З. (собственным значением) оператора $-\frac{d^2}{dx^2}$.

    \item Пусть $\Omega > 0$. Общее решение уравнения $-X''(x) = \Omega X(x)$, то есть $X''(x) + \Omega X(x)=0$, имеет вид:
    $X(x) = c_1 \cos(\sqrt{\Omega}x) + c_2 \sin(\sqrt{\Omega}x)$. Удовлетворяя граничным условиям:
    $$
    \begin{cases}
        X(0) = c_1 \cdot 1 + c_2 \cdot 0 = 0 \implies c_1 = 0 \\
        X(l) = c_1 \cos(\sqrt{\Omega}l) + c_2 \sin(\sqrt{\Omega}l) = 0
    \end{cases}
    \implies
    \begin{cases}
        c_1 = 0 \\
        c_2 \sin(\sqrt{\Omega}l) = 0
    \end{cases}
    $$
    С учётом $c_1^2+c_2^2 > 0$ (для нетривиального решения $c_2 \neq 0$):
    $$ \sin(\sqrt{\Omega}l) = 0 \quad \text{ - характеристическое уравнение для определения С.З. } \Omega>0. $$
    $$ \sin(\sqrt{\Omega}l) = 0 \implies \sqrt{\Omega}l = k\pi, \quad k \in \mathbb{N} \implies \Omega = \Omega_k = \left(\frac{k\pi}{l}\right)^2, \quad k \in \mathbb{N}. $$
    Вектор $\vec{c} = \begin{pmatrix} c_1 \\ c_2 \end{pmatrix}$ при этом имеет вид: $\vec{c} = \begin{pmatrix} 0 \\ c_2 \end{pmatrix}$.
    Собственная функция, соответствующая собственному значению $\Omega_k = \left(\frac{k\pi}{l}\right)^2$:
    $X_k(x) = \sin\left(\frac{k\pi x}{l}\right)$. % Original $X_k(x) = c_2 \sin(\frac{k\pi x}{l})$, $c_2$ can be taken as 1 for the eigenfunction basis.
    Какие $k \in \mathbb{Z}$ допустимы?:
    \begin{itemize}
        \item[а)] При $k=0$: $X_0(x) = 0 \implies X_0(x)$ - не является С.Ф. (собственной функцией).
        \item[б)] При $k>0$: $X_k(x) = \sin\left(\frac{k\pi x}{l}\right)$, а $X_{-k}(x) = \sin\left(\frac{\pi(-k)x}{l}\right) = -\sin\left(\frac{\pi k x}{l}\right) = -X_k(x)$.
    \end{itemize}
    $\implies$ Вся система линейно-независимых С.Ф. (собственных функций) оператора $-\frac{d^2}{dx^2}$ такова:
    $$ \Omega_k = \left(\frac{k\pi}{l}\right)^2, \quad X_k(x) = \sin\left(\frac{k\pi x}{l}\right), \quad x \in [0,l], \quad k \in \mathbb{N}. $$
\end{enumerate}





Итак, мы нашли возможный вид функций $X(x)$ для решения вида $u(t,x) = T(t)X(x)$.
Определим теперь вид соответствующих им функций $T_k(t)$.
$u(t,x)$ будем искать в виде функционального ряда:
$$ \sum_{k=1}^{\infty} T_k(t)X_k(x) = \sum_{k=1}^{\infty} T_k(t)\sin\left(\frac{\pi k}{l}x\right) $$
где $T_k(t)$, $k \in \mathbb{N}$ — функции, определённые на $[0,T]$.
\newline
\textbf{Считаем выполненными условия согласования:}
\begin{enumerate}
    \item $u_0(x) \in C^2[0,l]$, $\exists \, u_0'''(x)$ — кусочно-непрерывная на $[0,l]$,
    $u_0(0)=u_0(l)=u_0''(0)=u_0''(l)=0$.
    \item $u_1(x) \in C^1[0,l]$, $\exists \, u_1''(x)$ — кусочно-непрерывная на $[0,l]$,
    $u_1(0)=u_1(l)=0$.
    \item $f(x,t) \in C_{x,t}^{2,0}(\bar{G})$, $f(0,t)=f(l,t)=0$.
\end{enumerate}
\begin{itemize}
    \item Из ограничения 1) следует, что ряд Фурье $u_0(x)$ сходится равномерно к самой $u_0(x)$ (\textsuperscript{ссылка}):
    $$ u_0(x) = \sum_{k=1}^{\infty} a_k \sin\left(\frac{\pi k}{l}x\right), \quad x \in [0,l], \quad \text{где } a_k = \frac{2}{l}\int_0^l u_0(x)\sin\left(\frac{\pi k}{l}x\right)dx \text{ — коэфф. Фурье,} $$
    причём $a_k = \frac{\tilde{a_k}}{k^3}$ и $\sum_{k=1}^{\infty} |\tilde{a_k}|^2 < \infty$, обозначим $C_1 = \sup_{k \in \mathbb{N}} |\tilde{a_k}| < \infty$.
    \item Аналогично для функции $u_1(x)$ из ограничения 2):
    $$ u_1(x) = \sum_{k=1}^{\infty} b_k \sin\left(\frac{\pi k}{l}x\right), \quad x \in [0,l], \quad \text{где } b_k = \frac{2}{l}\int_0^l u_1(x)\sin\left(\frac{\pi k}{l}x\right)dx \text{ — коэфф. Фурье,} $$
    причём $b_k = \frac{\tilde{b_k}}{k^2}$, $\sum_{k=1}^{\infty} |\tilde{b_k}|^2 < \infty$, обозначим $C_2 = \sup_{k \in \mathbb{N}} |\tilde{b_k}| < \infty$.
    \item Ограничение 3) позволяет $\forall t \in [0,T]$ разложить $f(x,t)$ в равномерно сходящийся по $x$ ряд Фурье:
    $$ f(x,t) = \sum_{k=1}^{\infty} B_k(t) \sin\left(\frac{\pi k}{l}x\right), \quad x \in [0,l], \quad \text{где } B_k(t) = \frac{2}{l}\int_0^l f(x,t)\sin\left(\frac{\pi k}{l}x\right)dx \in C[0,T]. $$
    $|B_k(t)| = \frac{|\tilde{B}_k(t)|}{k^2}$, где $\tilde{B}_k(t) = \frac{2}{l}\int_0^l f_{xx}(x,t)\sin\left(\frac{\pi k}{l}x\right)dx \in C[0,T]$.
    \newline
    Неравенство Бесселя для $\tilde{B}_k(t)$ имеет вид: $\sum_{k=1}^{\infty} |\tilde{B}_k(t)|^2 \le \frac{2}{l} \int_0^l |f_{xx}(x,t)|^2 dx$.
    \newline
    Функция $g(t) = \frac{2}{l} \int_0^l |f_{xx}(x,t)|^2 dx$ является непрерывной по $t$, т.к. $f_{xx}(x,t) \in C(\bar{G}) \implies \exists \sup_{t \in [0,T]} g(t) = C_3 < \infty$.
    \item Формально подставляя полученные ряды для $u(t,x)$, $u_0(x)$, $u_1(x)$, $f(x,t)$ в исходную задачу (1):
    $$
    \begin{cases}
        \sum_{k=1}^{\infty} T_k''(t)X_k(x) + a^2 \sum_{k=1}^{\infty} \Omega_k T_k(t)X_k(x) = \sum_{k=1}^{\infty} B_k(t)X_k(x), \quad (x,t) \in G \\
        \sum_{k=1}^{\infty} T_k(0)X_k(x) = \sum_{k=1}^{\infty} a_k X_k(x), \quad x \in [0,l] \\
        \sum_{k=1}^{\infty} T_k'(0)X_k(x) = \sum_{k=1}^{\infty} b_k X_k(x), \quad x \in [0,l]
    \end{cases}
    $$
\end{itemize}



В силу ортогональности системы С.Ф. (собственных функций) $\{X_k(x)\}_{k=1}^{\infty}$, коэффициенты ряда для уравнения должны равняться нулю: для каждой функции $T_k(t)$ получаем З.К. (задачу Коши) для ОДУ (обыкновенного дифференциального уравнения):
$$
\begin{cases}
    T_k''(t) + a^2 \Omega_k T_k(t) = B_k(t) \\
    T_k(0) = a_k \\
    T_k'(0) = b_k
\end{cases}
\quad (3), \quad k \in \mathbb{N}.
$$

\underline{Лемма 1.} Функция $T(t) = \frac{1}{\mu} \int_0^t B(\xi) \sin \mu(t-\xi) d\xi$ является единственным решением из класса $C^2(0,T] \cap C^1[0,T]$ задачи Коши для линейного однородного уравнения:
$$
\begin{cases}
    T''(t) + \mu^2 T(t) = B(t), \quad 0 < t \le T, \\
    T(0) = 0, \\
    T'(0) = 0;
\end{cases}
\quad \mu > 0, \quad B(t) \in C[0,T].
$$
$\blacktriangle$ Единственность решения следует из основной теоремы (Th Cauchy (Теоремы Коши)) курса ДУ (дифференциальных уравнений). Проверим, что функция $T(t)$ является решением З.К. (задачи Коши):
\newline
$T(0) = 0$.
\newline
$T'(t) = \frac{1}{\mu} B(t)\sin(0) + \int_0^t B(\xi) \cos\mu(t-\xi) d\xi = \int_0^t B(\xi) \cos\mu(t-\xi) d\xi$.
\newline
$T''(t) = B(t)\cos(0) - \mu \int_0^t B(\xi) \sin\mu(t-\xi) d\xi = B(t) - \mu^2 T(t)$ — уравнение выполнено. $\blacktriangle$

$\implies T_k(t) = a_k \cos\left(\frac{\pi a k}{l}t\right) + b_k \frac{l}{\pi a k} \sin\left(\frac{\pi a k}{l}t\right) + \frac{l}{\pi a k} \int_0^t B_k(\xi) \sin\left(\frac{\pi a k}{l}(t-\xi)\right) d\xi$
— получили в силу линейности задачи (3):
$$ \begin{cases} T_k'' + a^2 \Omega_k T_k(t) = B_k(t) \\ T_k(0) = 0 \\ T_k'(0) = 0 \end{cases} + \begin{cases} T_k'' + a^2 \Omega_k T_k(t) = 0 \\ T_k(0) = a_k \\ T_k'(0) = 0 \end{cases} + \begin{cases} T_k'' + a^2 \Omega_k T_k(t) = 0 \\ T_k(0) = 0 \\ T_k'(0) = b_k \end{cases} $$
Имеем формальное решение: $\sum_{k=1}^{\infty} T_k(t) \sin\left(\frac{\pi k x}{l}\right)$; (4)

\bigskip
\noindent\textcircled{3} \textbf{Классическое решение.}
\newline
Докажем, что функциональный ряд (4) сходится равномерно на $\bar{G}$. Оценим $k$-ый член:
\begin{align*} \sup_{(x,t) \in \bar{G}} |T_k(t)X_k(x)| &\le \sup_{t \in [0,T]} |T_k(t)| \\ &\le |a_k| + \frac{|b_k|l}{\pi a k} + \frac{l}{\pi a k} \sup_{t \in [0,T]} \left| \int_0^t B_k(\xi) \sin\left(\frac{\pi a k}{l}(t-\xi)\right) d\xi \right| \\ &\le |a_k| + \frac{|b_k|l}{\pi a k} + \frac{lT}{\pi a k} \sup_{t \in [0,T]} |B_k(t)| \end{align*}
используя, что $a_k=\frac{\tilde{a}_k}{k^3}$, $b_k=\frac{\tilde{b}_k}{k^2}$, $B_k(t)=\frac{\tilde{B}_k(t)}{k^2}$, получаем:
$$ \le \frac{|\tilde{a}_k|}{k^3} + \frac{|\tilde{b}_k|l}{\pi a k^3} + \frac{lT}{\pi a k^3} \sup_{t \in [0,T]} |\tilde{B}_k(t)| \le \left[ C_1 + \frac{l}{\pi a}C_2 + \frac{lT}{\pi a}\sqrt{C_3} \right] \frac{1}{k^3} $$
использовали, что $\sup_{k \in \mathbb{N}} \sup_{t \in [0,T]} |\tilde{B}_k(t)| \le \sup_{t \in [0,T]} \sqrt{\sum_{k=1}^\infty |\tilde{B}_k(t)|^2} \le \sup_{t \in [0,T]} \sqrt{g(t)} = \sqrt{C_3}$.
\newline
Т.о. (таким образом) ряд (4) мажорируется сходящимся числовым рядом $\implies$ сходится равномерно на $\bar{G}$ к некоторой функции $u(t,x) = \sum_{k=1}^{\infty} T_k(t)X_k(x) \in C(\bar{G})$, т.к. $T_k(t)X_k(x) \in C(\bar{G})$.
\begin{itemize}
    \item Аналогично можно показать равномерную сходимость рядов $\sum_{k=1}^{\infty} T_k'(t)X_k(x)$, $\sum_{k=1}^{\infty} T_k(t)X_k'(x)$, $\sum_{k=1}^{\infty} T_k''(t)X_k(x)$, $\sum_{k=1}^{\infty} T_k(t)X_k''(x)$ в $\bar{G}$, откуда получаем существование на $\bar{G}$ $u_t(t,x)$, $u_x(t,x)$, $u_{xt}(t,x)$, а также рядов...
\end{itemize}


$u_{tt}(t,x) = \sum_{k=1}^{\infty} T_k''(t)X_k(x)$ и $u_{xx}(t,x) = \sum_{k=1}^{\infty} T_k(t)X_k''(x)$, доказательство для которых мы приведём:
\begin{align*}
\sup_{(x,t) \in \bar{G}} |T_k''(t)X_k(x)| &\le \sup_{t \in [0,T]} |T_k''(t)| \\
&\le \sup_{t \in [0,T]} \left| B_k(t) - \left(\frac{\pi a k}{l}\right)^2 a_k \cos\left(\frac{\pi a k}{l}t\right) - \frac{\pi a k}{l} b_k \sin\left(\frac{\pi a k}{l}t\right) - \frac{\pi a k}{l} \int_0^t B_k(\xi) \sin\left(\frac{\pi a k}{l}(t-\xi)\right) d\xi \right| \\
&\le \frac{1}{k^2} \sup_{t \in [0,T]} |\tilde{B}_k(t)| + \left(\frac{\pi a}{l}\right)^2 \frac{|\tilde{a}_k|}{k} + \frac{\pi a}{l} \frac{|\tilde{b}_k|}{k} + \frac{\pi a}{l} \int_0^T \frac{|\tilde{B}_k(\xi)|}{k^2} d\xi \\
&\le \frac{1}{k^2} \sqrt{C_3} + \left(\frac{\pi a}{l}\right)^2 \frac{C_1}{k} + \frac{\pi a}{l} \frac{C_2}{k} + \frac{\pi a T}{l k^2} \sqrt{C_3} \\
&= \left[ \left(\frac{\pi a}{l}\right)^2 C_1 + \frac{\pi a}{l} C_2 \right] \frac{1}{k} + \left[ \sqrt{C_3} + \frac{\pi a T}{l} \sqrt{C_3} \right] \frac{1}{k^2}
\end{align*}
Не сходится к мажорирующему ряду. Необходимо использовать неравенство Коши-Буняковского (Н.К.-Б.).
Числовой ряд $\sum_{k=1}^\infty \int_0^T |\tilde{B}_k(\xi)|^2 d\xi$ сходится, поскольку:
$$ \sum_{k=1}^n \int_0^T |\tilde{B}_k(\xi)|^2 d\xi \le \int_0^T \left[\sum_{k=1}^\infty |\tilde{B}_k(\xi)|^2\right] d\xi \le \int_0^T g(\xi) d\xi \le C_3 T $$
$\implies$ мажорируемый ряд для $T_k''(t)X_k(x)$ сходится $\implies \sum_{k=1}^{\infty} T_k''(t)X_k(x)$ сходится равномерно в $\bar{G}$.

Далее для $u_{xx}$:
$$ \left| \sum_{k=1}^{\infty} T_k(t)X_k''(x) \right| = \left| \sum_{k=1}^{\infty} T_k(t) \left(-\left(\frac{\pi k}{l}\right)^2 X_k(x)\right) \right| = \left| -\sum_{k=1}^{\infty} \left(\frac{\pi k}{l}\right)^2 T_k(t)X_k(x) \right| $$
\begin{align*} \sup_{(x,t) \in \bar{G}} \left| \left(\frac{\pi k}{l}\right)^2 T_k(t)X_k(x) \right| &\le \sup_{t \in [0,T]} \left| \left(\frac{\pi k}{l}\right)^2 T_k(t) \right| \\ &\le \left(\frac{\pi k}{l}\right)^2 \frac{|\tilde{a}_k|}{k^3} + \frac{\pi k}{l} \frac{1}{a} \frac{|\tilde{b}_k|}{k^2} + \frac{1}{a} \frac{\pi k}{l} \int_0^T \frac{|\tilde{B}_k(\xi)|}{k^2} d\xi \\ &= \left(\frac{\pi}{l}\right)^2 \frac{|\tilde{a}_k|}{k} + \frac{\pi}{la} \frac{|\tilde{b}_k|}{k} + \frac{\pi T}{la} \frac{\sqrt{C_3}}{k} \\ &= \left[ \left(\frac{\pi}{l}\right)^2 C_1 + \frac{\pi}{la} C_2 + \frac{\pi T}{la} \sqrt{C_3} \right] \frac{1}{k} \end{align*}
$\rightarrow$ мажорируется сходящимся числовым рядом $\implies$ сходится равномерно на $\bar{G}$.
\newline
Т.о. (таким образом) доказана теорема:

\newtheorem{theorem}{Th}
\begin{theorem}
Пусть выполнены условия согласования и гладкости:
\begin{enumerate}
    \item $u_0(x) \in C^2[0,l]$, $\exists \, u_0'''(x)$ — кусочно-непрерывная на $[0,l]$; $u_0(0)=u_0(l)=u_0''(0)=u_0''(l)=0$,
    \item $u_1(x) \in C^1[0,l]$, $\exists \, u_1''(x)$ — кусочно-непрерывная на $[0,l]$; $u_1(0)=u_1(l)=0$,
    \item $f(x,t) \in C_{x,t}^{2,0}(\bar{G})$, $f(0,t)=f(l,t)=0$.
\end{enumerate}
Тогда:
Смешанная задача
$$
\begin{cases}
    u_{tt} - a^2 u_{xx} = f(x,t), \quad (x,t) \in G=(0,l)\times(0,T); \\
    \left. u \right|_{t=0} = u_0(x), \quad \left. u_t \right|_{t=0} = u_1(x); \\
    \left. u \right|_{x=0} = 0, \quad \left. u \right|_{x=l} = 0;
\end{cases}
\quad \text{в классе } C^2(G) \cap C^1(\bar{G})
$$
имеет решение, причём ряд
$$ \sum_{k=1}^{\infty} \left[ a_k \cos\left(\frac{\pi a k}{l}t\right) + b_k \frac{l}{\pi a k} \sin\left(\frac{\pi a k}{l}t\right) + \frac{l}{\pi a k} \int_0^t B_k(\xi) \sin\left(\frac{\pi a k}{l}(t-\xi)\right) d\xi \right] \sin\left(\frac{\pi k}{l}x\right) $$
равномерно сходится к этому решению.
\end{theorem}



\noindent\textcircled{4} \textbf{Единственность решения.}

Рассмотрим смешанную (начальную-краевую) задачу для уравнения малых колебаний:
$$
\begin{cases}
    u_{tt} - a^2 u_{xx} = f(t,x), \quad (t,x) \in Q_T = (0,T) \times (0,l) & (\underline{1_1}) \\
    \left. u \right|_{t=0} = u_0(x), \quad \left. u_t \right|_{t=0} = u_1(x), \quad x \in [0,l] & (\underline{1_2}) \\
    k_1 u|_{x=0} - u_x|_{x=0} = \varphi_0(t), \quad k_2 u|_{x=l} + u_x|_{x=l} = \varphi_1(t), \quad t \in [0,T], \quad k_1, k_2 > 0 & (\underline{1_3})
\end{cases}
$$
(Граничные условия в тексте могут быть разными: $\left. u \right|_{x=0} = \varphi_0(t)$, $\left. u \right|_{x=l} = \varphi_1(t)$, $t \in [0,T]$)
(или $u_x|_{x=0} = \varphi_0(t)$, $u_x|_{x=l} = \varphi_1(t)$, $t \in [0,T]$)

\underline{Опр.} Под классическим решением задачи (1) будем понимать функцию $u(t,x): u(t,x) \in C^1(\bar{Q}_T) \cap C^2(Q_T)$ и удовлетворяющую уравнению ($1_1$), начальным условиям ($1_2$), граничным условиям ($1_3$).

\newtheorem{theorem}{Th}
\begin{theorem}[2]
Не может существовать более одного классического решения $u(t,x)$ задачи (1).
\end{theorem}
$\blacktriangle$ Пусть $u_I(t,x)$ и $u_{II}(t,x)$ — два классических решения смешанной задачи (1).
Тогда функция $\omega(t,x) = u_I(t,x) - u_{II}(t,x)$:
\begin{itemize}
    \item[a)] $\omega(t,x) \in C^1(\bar{Q}_T) \cap C^2(Q_T)$
    \item[б)] является решением полностью однородной задачи
    $$
    \begin{cases}
        \omega_{tt} - a^2 \omega_{xx} = 0, \quad (t,x) \in Q_T \\
        \left. \omega \right|_{t=0} = 0, \quad \left. \omega_t \right|_{t=0} = 0, \quad x \in [0,l] \\
        k_1 \omega|_{x=0} - \omega_x|_{x=0} = 0, \quad k_2 \omega|_{x=l} + \omega_x|_{x=l} = 0, \quad t \in [0,T]
    \end{cases}
    $$
    (Для более простых граничных условий: $\omega|_{x=0}=0, \omega|_{x=l}=0, t \in [0,T]$)
\end{itemize}

\begin{center}
\begin{tikzpicture}
\draw[->] (0,0) -- (4,0) node[below right] {$x$};
\draw[->] (0,0) -- (0,3) node[above left] {$t$};
\draw (0,2.5) node[left] {$T$} -- (3.5,2.5) -- (3.5,0) node[below] {$l$};
\draw (0,0) node[below left] {$0$};
\node at (1.75,1.25) {$u_{tt}=a^2u_{xx}+f(t,x)$};
\node at (1.75,0) [below] {$u(0,x)=u_0(x)$};
\node at (1.75,0) [above] {$u_t(0,x)=u_1(x)$};
\node at (0,1.25) [left] {$u(t,0)=\varphi_0(t)$};
\node at (3.5,1.25) [right] {$u(t,l)=\varphi_1(t)$};
\end{tikzpicture}
\end{center}
Покажем, что $\omega(t,x) \equiv 0$ в $Q_T$. Применим метод интеграла энергии. Рассмотрим функцию
$I(t,x) = \omega_t(t,x)[\omega_{tt}(t,x) - a^2 \omega_{xx}(t,x)]$, приведём $I(t,x)$ к дивергентному виду:
\begin{align*}
I &= \omega_t(\omega_{tt} - a^2 \omega_{xx}) = \omega_t \omega_{tt} - a^2 \omega_t \omega_{xx} = \left(\frac{1}{2}\omega_t^2\right)_t - a^2 (\omega_t \omega_x)_x + a^2 (\omega_x \omega_{xt}) \\
&= \left(\frac{1}{2}\omega_t^2\right)_t + \left(\frac{a^2}{2}\omega_x^2\right)_t - (a^2 \omega_t \omega_x)_x = \left[\frac{1}{2}(\omega_t^2 + a^2 \omega_x^2)\right]_t - (a^2 \omega_t \omega_x)_x
\end{align*}
т.к. $\omega_{tt} - a^2 \omega_{xx} = 0$ в $Q_T$, то $I(t,x) \equiv 0$ в $Q_T$.

Наша дальнейшая цель состоит в интегрировании $I(t,x)$ по области $Q_{\tilde{\tau}}$ и в преобразовании получившегося интеграла с помощью формулы Грина. Поскольку мы не требуем от $\omega(t,x)$ того, чтобы $\omega(t,x) \in C^2(\bar{Q}_T)$, то применить формулу Грина в обычной области $Q_T$ для $\omega(t,x)$ нельзя.
Введём $\varepsilon$-урезанную область $Q_{\tilde{\tau}}^\varepsilon$: $Q_{\tilde{\tau}}^\varepsilon = (\varepsilon, \tilde{\tau}) \times (\varepsilon, l-\varepsilon)$, где $0 < \tilde{\tau} < T$ — произвольное, $\varepsilon$ — достаточно малое.
$$ 0 = \iint_{Q_{\tilde{\tau}}^\varepsilon} I(t,x) dxdt = \iint_{Q_{\tilde{\tau}}^\varepsilon} \left[ \frac{\partial}{\partial t} \underbrace{\left(\frac{\omega_t^2+a^2\omega_x^2}{2}\right)}_{Q} - \frac{\partial}{\partial x} \underbrace{(a^2 \omega_t \omega_x)}_{P} \right] dxdt $$

\begin{center}
\begin{tikzpicture}
\draw[->] (0,0) -- (5,0) node[below right] {$x$};
\draw[->] (0,0) -- (0,4) node[above left] {$t$};
\draw (0,3.5) node[left] {$T$} -- (4.5, 3.5);
\draw (4.5,0) node[below] {$l$};
\draw (0.5,0.5) rectangle (4,3);
\node at (0.5,0.5) [below left] {$(\varepsilon,\varepsilon)$};
\node at (4,3) [above right] {$(l-\varepsilon, \tilde{\tau})$};
\node at (2.25,1.75) {$Q_{\tilde{\tau}}^\varepsilon$};
\draw[->] (0.5,1.75) -- (0.2, 1.75); \node at (0.1,1.75) [left] {$\varepsilon$};
\draw[->] (4,1.75) -- (4.3, 1.75); \node at (4.4,1.75) [right] {$\varepsilon$};
\draw[->] (2.25,0.5) -- (2.25, 0.2); \node at (2.25,0.1) [below] {$\varepsilon$};
\node at (0,0) [below left] {$0$};
\node at (0,3) [left] {$\tilde{\tau}$};
\end{tikzpicture}
\end{center}
$$ = \oint_{\partial Q_{\tilde{\tau}}^\varepsilon} \left[ -\frac{1}{2}(\omega_t^2+a^2\omega_x^2)dx - a^2 \omega_t \omega_x dt \right] = \iint_S \left( \frac{\partial P}{\partial x} - \frac{\partial Q}{\partial y} \right) dx dy = \oint_\Gamma (Q dx + P dy) $$
$\partial Q_{\tilde{\tau}}^\varepsilon$ — против обхода контура.
$$ = +\int_\varepsilon^{l-\varepsilon} \left. \frac{1}{2}(\omega_t^2+a^2\omega_x^2) \right|_{t=\tilde{\tau}} dx - \int_\varepsilon^{l-\varepsilon} \left. \frac{1}{2}(\omega_t^2+a^2\omega_x^2) \right|_{t=\varepsilon} dx - \dots $$



\usepackage{dsfont} % Added for \mathds{1} or similar, if that was intended for H

% (Continued from previous part)
$$ \dots - \int_\varepsilon^{\tilde{\tau}} a^2 (\omega_t \omega_x)|_{x=l-\varepsilon} dt + \int_\varepsilon^{\tilde{\tau}} a^2 (\omega_t \omega_x)|_{x=\varepsilon} dt = 0. $$
Перейдём в этом соотношении к пределу при $\varepsilon \to +0$, при этом будем учитывать, что $\omega(t,x) \in C^1(\bar{Q}_T)$ и что $\omega(0,x) \equiv \omega_t(0,x) \equiv 0 \, \forall x \in [0,l]$, $\omega(t,0) \equiv \omega_t(t,0) \equiv 0 \, \forall t \in [0,\tilde{\tau}]$, $\omega(t,l) \equiv \omega_t(t,l) \equiv 0 \, \forall t \in [0,\tilde{\tau}]$ (в силу граничных условий; следствие Н.У. (начальных условий)). Предел производной будет совпадать с производной вдоль границы при $t=0$, а $\omega \equiv 0$ вдоль границы (следствие Н.У.).

\begin{align*}
\lim_{\varepsilon \to +0} \iint_{Q_{\tilde{\tau}}^\varepsilon} I(t,x) dxdt = 0 &= \lim_{\varepsilon \to +0} \left\{ \int_\varepsilon^{l-\varepsilon} \left. \frac{1}{2}(\omega_t^2+a^2\omega_x^2) \right|_{t=\tilde{\tau}} dx - \int_\varepsilon^{l-\varepsilon} \left. \frac{1}{2}(\omega_t^2+a^2\omega_x^2) \right|_{t=\varepsilon} dx \right. \\
&\quad \left. - \int_\varepsilon^{\tilde{\tau}} a^2 (\omega_t \omega_x)|_{x=l-\varepsilon} dt + \int_\varepsilon^{\tilde{\tau}} a^2 (\omega_t \omega_x)|_{x=\varepsilon} dt \right\} \\
&= \int_0^l \left. \frac{1}{2}(\omega_t^2+a^2\omega_x^2) \right|_{t=\tilde{\tau}} dx - \frac{1}{2} \int_0^l [\underbrace{\omega_t^2(0,x)}_{=0 \text{ в силу Н.У.}} + a^2 \underbrace{\omega_x^2(0,x)}_{=0 \text{ в силу Н.У.}} ] dx \\
&\quad - a^2 \int_0^{\tilde{\tau}} \underbrace{\omega_t(t,l)\omega_x(t,l)}_{=0 \text{ в силу Г.У.}} dt + a^2 \int_0^{\tilde{\tau}} \underbrace{\omega_t(t,0)\omega_x(t,0)}_{=0 \text{ в силу Г.У.}} dt \\
&= \frac{1}{2} \int_0^l [\omega_t^2(\tilde{\tau},x) + a^2 \omega_x^2(\tilde{\tau},x)] dx = 0.
\end{align*}
Интеграл по $x$ непрерывен.
Итак, $\forall \tilde{\tau}: 0 < \tilde{\tau} < T \implies \int_0^l [\omega_t^2(\tilde{\tau},x) + a^2 \omega_x^2(\tilde{\tau},x)] dx = 0$. Отсюда заключаем, что
$\omega_t^2(t,x) + a^2 \omega_x^2(t,x) = 0 \, \forall x \in (0,l)$ и $\forall t \in (0,T) \implies \text{grad}_{t,x} \, \omega(t,x) = 0 \, \forall (t,x) \in Q_T$.
$\implies \omega(t,x) = \text{const} \, \forall (t,x) \in Q_T$. В силу непрерывности функции $\omega(t,x)$ в $\bar{Q}_T$ и в силу начальных условий $\omega(0,x) \equiv 0$ получаем, что $\omega(t,x) \equiv 0 \, \forall (t,x) \in \bar{Q}_T \implies u_I(t,x) \equiv u_{II}(t,x) \, \forall (t,x) \in \bar{Q}_T$. $\blacktriangle$

\bigskip
\textbf{ИСПОЛЬЗУЕМЫЕ ТЕОРЕМЫ:}
\newtheorem{utv}{Утв.}
\begin{utv}[1]
Пусть в некотором пространстве H со скалярным произведением оператор $A$ самосопряжённый. Тогда все его С.З. (собственные значения) действительны, а С.Ф. (собственные функции), соответствующие различным С.З., ортогональны.
Свойства скалярного произведения: $(g,g) \ge 0$, $1 \cdot g = (g,f)$, $(\lambda f,g) = \lambda(f,g)$, $(f, \lambda g) = \bar{\lambda}(f,g)$.
\end{utv}
$\blacktriangle$ Пусть $X_k$ — собств. функция оператора $A$, соотв. С.З. $\Omega_k$.
$X_m$ — собств. функция оператора $A$, соотв. С.З. $\Omega_m$, т.е.
$AX_k = \Omega_k X_k$, $(X_k,X_k) \neq 0$.
$AX_m = \Omega_m X_m$, $(X_m,X_m) \neq 0$, $\Omega_k \neq \Omega_m$. Тогда:
\begin{itemize}
    \item[a)] $(AX_k, X_k) = (\Omega_k X_k, X_k) = \Omega_k (X_k,X_k)$.
    Поскольку $(AX,Y)=(X,AY)$:
    $(X_k, AX_k) = (X_k, \Omega_k X_k) = \bar{\Omega}_k (X_k,X_k)$.
    $\implies \Omega_k = \bar{\Omega}_k \implies \Omega_k \in \mathbb{R}$.
    \item[б)] $(AX_k, X_m) = (\Omega_k X_k, X_m) = \Omega_k (X_k,X_m)$.
    $(X_k, AX_m) = (X_k, \Omega_m X_m) = \bar{\Omega}_m (X_k,X_m) = \Omega_m (X_k,X_m)$ (т.к. $\Omega_m \in \mathbb{R}$).
    $\implies (\Omega_k - \Omega_m)(X_k,X_m) = 0 \implies (X_k,X_m) = 0$. $\blacktriangle$
\end{itemize}

\begin{lemma}[1]
Оператор $A = -\frac{d^2}{dx^2}$, определённый на области определения $\mathcal{D}(A)$, является самосопряжённым относительно скалярного произведения пространства $L_2(0,l): (u,v) = \int_0^l u(x)\overline{v(x)}dx$.
\end{lemma}
$\blacktriangle$ $(Au,v) = \int_0^l [-u''(x)]\overline{v(x)}dx = -\left. u'(x)\overline{v(x)} \right|_0^l + \int_0^l u'(x)\overline{v'(x)}dx$. М.к. (Можно как) $v \in \mathcal{D}(A): v(0)=v(l)=0$.
$= \left. u(x)\overline{v'(x)} \right|_0^l + \int_0^l u(x)[-\overline{v''(x)}]dx = (u, Av)$. $\blacktriangle$
$u \in \mathcal{D}(A): u(0)=u(l)=0$.





Отсюда и следует, что $\Omega_k = \left(\frac{\pi k}{l}\right)^2 \in \mathbb{R}$ и $X_k(x) = \sin\left(\frac{\pi k x}{l}\right)$ и $X_m(x) = \sin\left(\frac{\pi m x}{l}\right)$ ортогональны при $m \neq k$ относительно скалярного произведения $L_2(0,l)$.

\begin{lemma}[2]
Пусть $e_1, e_2, \dots, e_n, \dots$ — конечная или счётная ортогональная система элементов в линейном пространстве H со скалярным произведением $(u,v)$, т.е. $(e_k, e_j) = \begin{cases} 0, & k \neq j \\ >0, & k=j \end{cases}, \forall k \in \mathbb{N}$. Тогда $\forall f \in H$:
$$ \sum_{k=1}^\infty |c_k|^2 (e_k, e_k) \le (f,f), \quad \text{где } c_k = \frac{(f,e_k)}{(e_k,e_k)} \text{ — коэфф. Фурье } f \text{ по системе функций } e_1, \dots, e_n, \dots $$
$$ \left\| 0 \le \left(f - \sum_{k=1}^N c_k e_k, f - \sum_{j=1}^N c_j e_j\right) = \dots \right\| \quad \text{+ Th Вейерштрасса (признак)} $$
\end{lemma}

\begin{lemma}[3]
Пусть сходятся два ряда $\sum_{k=1}^\infty |\alpha_k|^2$ и $\sum_{k=1}^\infty |\beta_k|^2$. Тогда сходится, причём абсолютно, ряд $\sum_{k=1}^\infty \alpha_k \beta_k$. При этом имеет место неравенство Коши-Буняковского:
$$ \sum_{k=1}^\infty |\alpha_k \beta_k| \le \sqrt{\sum_{k=1}^\infty |\alpha_k|^2} \cdot \sqrt{\sum_{k=1}^\infty |\beta_k|^2} $$
$$ \left\| \sum_{k=1}^N |\alpha_k \beta_k| \le \sqrt{\sum_{k=1}^N |\alpha_k|^2} \cdot \sqrt{\sum_{k=1}^N |\beta_k|^2} < \sqrt{A} \sqrt{B} \implies \text{сходится} \right\| $$
(Примечание: последнее неравенство относится к частичным суммам, показывая сходимость, если $\sum |\alpha_k|^2 = A < \infty$ и $\sum |\beta_k|^2 = B < \infty$).
\end{lemma}

\begin{lemma}[4]
Пусть функция $\omega(x)$ имеет на отрезке $[0,l]$ непрерывную производную и принимает на концах этого отрезка нулевые значения, т.е.:
1) $\omega(x) \in C^1[0,l]$,
2) $\omega(0) = \omega(l) = 0$.
\begin{itemize}
    \item[$\implies$ а)] $\sum_{k=1}^\infty A_k \sin\left(\frac{\pi k}{l}x\right)$, $x \in [0,l]$, $A_k = \frac{2}{l}\int_0^l \omega(y) \sin\left(\frac{\pi k}{l}y\right)dy$ сходится к функции $\omega(x)$ на отрезке $[0,l]$ абсолютно и равномерно.
    \item[$\implies$ б)] Для $A_k$: $A_k = \frac{l}{\pi} \frac{1}{k} d_k$, $d_k = \frac{2}{l}\int_0^l \omega'(y) \cos\left(\frac{\pi k}{l}y\right)dy$.
\end{itemize}
\end{lemma}

\begin{itemize}
    \item \textbf{Оценки для интегралов:}
    \begin{align*} A_k &= \frac{2}{l}\int_0^l u_0(y) \sin\left(\frac{\pi k}{l}y\right)dy = -\frac{2}{l} \frac{l}{\pi k} \left. \underline{u_0(y)}_{\to 0} \cos\left(\frac{\pi k}{l}y\right) \right|_{y=0}^{y=l} + \frac{l}{\pi k} \frac{2}{l} \int_0^l u_0'(y) \cos\left(\frac{\pi k}{l}y\right)dy \\ &= \left(\frac{l}{\pi k}\right)^2 \frac{2}{l} \left. \underline{u_0'(y)}_{\to 0} \sin\left(\frac{\pi k}{l}y\right) \right|_{y=0}^{y=l} - \frac{2}{l}\left(\frac{l}{\pi k}\right)^2 \int_0^l u_0''(y) \sin\left(\frac{\pi k}{l}y\right)dy \\ &= \frac{2}{l}\left(\frac{l}{\pi k}\right)^3 \left. \underline{u_0''(y)}_{\to 0} \cos\left(\frac{\pi k}{l}y\right) \right|_{y=0}^{y=l} - \left(\frac{l}{\pi k}\right)^3 \frac{2}{l} \int_0^l u_0'''(y) \cos\left(\frac{\pi k}{l}y\right)dy \\ &\implies A_k = -\left(\frac{l}{\pi k}\right)^3 d_k, \quad \text{где } d_k = \frac{2}{l}\int_0^l u_0'''(y) \cos\left(\frac{\pi k}{l}y\right)dy. \end{align*}
    Аналогично $B_k^0 = -\left(\frac{l}{\pi k}\right)^2 \beta_k$, $\beta_k = \frac{2}{l}\int_0^l u_1''(y) \sin\left(\frac{\pi k}{l}y\right)dy$.
    В силу неравенства Бесселя:
    $$ \sum_{k=1}^\infty \beta_k^2 \le \frac{2}{l}\int_0^l |u_1''(y)|^2 dy = m_1^2 < \infty $$
    $$ \sum_{k=1}^\infty d_k^2 \le \frac{2}{l}\int_0^l |u_0'''(y)|^2 dy = m_0^2 < \infty $$
\end{itemize}


\section*{Слабое (обобщённое) решение для волнового уравнения}
Если начальные условия недостаточно гладкие, то:
$$
\begin{cases}
    u_{tt} - a^2 \Delta u = F, \quad \left. u \right|_{t=0} = u_0(x), \quad \left. u'_t \right|_{t=0} = u_1(x) \\
    \left. (\alpha u + \beta \frac{\partial u}{\partial n}) \right|_S = 0 \quad (*)
\end{cases}
$$

\begin{enumerate}
    \item[I)] $H=L_2(G)$.
    $u^0 \in H$, $u^0_{x_j} \in H$, $u^1 \in H$, $F \in \hat{H} = L_2(\overbrace{(0,T) \times G}^{Z_T})$. Тогда $\exists u \in \hat{H}$, $u_{x_j} \in \hat{H}$, $u_t \in \hat{H}$ и $\forall \varphi \in C_{t,x}^{1,2}(\bar{Z}_T) \cap C^0(\bar{Z}_T): \left. \varphi \right|_{\partial G} = 0$ и $\varphi(t=T) \equiv 0$.
    $$ \int_0^T dt \int_G [u_t \varphi_t - a^2 \nabla u \cdot \nabla \varphi - F\varphi] dx = 0. \quad \text{Такое решение } \exists! $$

    \item[II)] Пусть дополнительно к (*) имеется:
    $$
    \begin{cases}
        u_{tt}^k - a^2 \Delta u^k = F^k(t,x), \quad \left. u^k \right|_{t=0} = u_0^k(x), \quad (u^k)'_t = u_1^k(x) \\
        \left. (\alpha u^k + \beta \frac{\partial u^k}{\partial n}) \right|_S = 0
    \end{cases}
    $$
    $u, F^k, u_0^k, u_1^k$ — достаточно гладкие.
    $F^k \xrightarrow{\hat{H}} FE \in \hat{H}$, т.е. гладкими функциями приближаем негладкую.
    $u_0^k \xrightarrow{H} u_0 \in H$.
    $u_0^{kx_j} \xrightarrow{H} u_0 \in H$.
    $u_1^k \xrightarrow{H} u_1 \in H$.

    \item[III)] \textbf{Обобщённые функции:}
    $u_{tt} - a^2 u_{xx} = 0$, $\left. u \right|_{t=0} = \text{sign } x$, $\left. u' \right|_{t=0} = 0$. Формально: $u(t,x) = \frac{1}{2}(\text{sign}(x-at) + \text{sign}(x+at))$.
    \newline
    \framebox{N9} \framebox{N10}
\end{enumerate}








\section*{Билет №8, 9 -- 2025}\label{sec:ticket9}
\backtotoc
\textbf{8. Задача Коши для уравнения теплопроводности. Представление решения формулой Пуассона.
Принцип Дюамеля. [3] 186-189 (для f=0) и [2] – 121.
9. Принцип максимума для уравнения теплопроводности в ограниченной области.
Единственность классического решения первой начально-краевой задачи для уравнения
теплопроводности в ограниченной области. [2] – 94-96, 98-99}

% Первая часть, относящаяся к волновому уравнению
$u_{tt} - a^2 u_{xx} = 0$, \quad $u|_{t=0} = \operatorname{sign} x$, \quad $u_t|_{t=0} = 0$; \\
Формально: $u(t,x) = \frac{1}{2} (\operatorname{sign}(x-at) + \operatorname{sign}(x+at))$
\quad \fbox{N9} \fbox{N10}

\section*{3.К. для ур-ния теплопроводности}
Представление решения формулой Пуассона. Принцип Дюамеля. Принцип максимума для ур-ния теплопроводности в ограниченной области. Единственность классического решения первой начально-краевой задачи для ур-ния теплопроводности в огран. области.

\subsection*{Мультииндекс}
$\alpha = (\alpha_1, \alpha_2, \dots, \alpha_n)$, $\alpha_k \in \mathbb{N} \cup \{0\}$. \\
$|\alpha| = \alpha_1 + \dots + \alpha_n$. Для $x = (x_1, \dots, x_n) \in \mathbb{R}^n$, $x^\alpha = x_1^{\alpha_1} x_2^{\alpha_2} \dots x_n^{\alpha_n}$ --- скалярная функция переменных.

\subsection*{1) З.К. (Задача Коши)}
$G = \mathbb{R}^n \times (0,T)$
$$
\begin{cases}
u_t - a^2 \Delta_x u = f, & f \in C(G) \\
u|_{t=0} = u_0(x), & u_0(x) \in C(\mathbb{R}^n)
\end{cases}
$$
Требуется найти решение $u(t,x) \in C_{x,t}^{2,1}(G) \cap C(\bar{G})$;

\subsection*{2) Смешанная задача}
$\mathcal{D} \subset \mathbb{R}^n$ --- обл. с кусоч.-гладкой границей $\partial\mathcal{D}$, $G = \mathcal{D} \times (0,T)$
$$
\begin{cases}
u_t - a^2 \Delta_x u = f, & f \in C(G) \\
u|_{t=0} = u_0(x), & u_0(x) \in C^1(\bar{\mathcal{D}}) \\
\left(\alpha u + \beta \frac{\partial u}{\partial \vec{n}}\right)\Big|_{\partial\mathcal{D}} = \nu(x,t), & \nu(x,t) \in C(\partial\mathcal{D} \times [0,T])
\end{cases}
$$
Требуется найти реш. $u(t,x) \in C_{x,t}^{2,1}(G) \cap C_{x,t}^{1,0}(\bar{G})$;


\newtheorem{remark}[theorem]{Замечание}


\newcommand{\Dcal}{\mathcal{D}}
\newcommand{\Scal}{\mathcal{S}} % Assuming S_T should be with a calligraphic S if D is.
\newcommand{\Vcal}{\mathcal{V}} % Assuming V_T should be with a calligraphic V if D is.
\newcommand{\Bcal}{\mathcal{B}} % For the set of bounded functions


\begin{remark}
В случае, когда $\beta=0$, достаточно требовать от $u_0(x)$ и $u(x,\dots)$ принадлежности классам $C(\bar{\Dcal})$ и $C_{x,t}^{2,1}(\bar{G}) \cap C(\bar{G})$ соответственно.
\end{remark}

\section*{Принцип максимума в ограниченной области}
$\Dcal \subset \mathbb{R}^n$ --- ограниченная область, $\Dcal_0 = \{(x,t) : x \in \Dcal, t=0\}$, \\
$S_T = \{(x,t) : x \in \partial\Dcal, 0 \le t \le T\}$, $V_T = \{(x,t) : x \in \Dcal, 0 < t \le T\}$, $0 < T < \infty$. \\
Введем класс функций $C_{x,t}^{2,1}(V_T) \cap C(\bar{V}_T)$.

\begin{lemma}
Пусть $W(x,t) \in C_{x,t}^{2,1}(V_T) \cap C(\bar{V}_T)$ удовлетворяет в $V_T$ нер-ву:
\[ L_\Omega(W(x,t)) \equiv W_t(x,t) - a^2 \Delta_x W(x,t) + \Omega W(x,t) \le 0 \quad \text{при некотором } \Omega > 0. \]
Тогда: если $\sup_{\bar{V}_T} W(x,t) > 0$, то $\forall (x,t) \in V_T \implies W(x,t) < \sup_{\Dcal_0 \cup S_T} W$.
\end{lemma}

\begin{proof}
$\triangle$ Т.к. $W(x,t)$ --- непрерывна на компактном мн-ве $\bar{V}_T \implies \exists \sup_{\bar{V}_T} W < \infty$ и достигается в некоторой точке из $\bar{V}_T$. Нер-во $W(x,t) < \sup_{\Dcal_0 \cup S_T} W$ означает, что $\sup_{\bar{V}_T} W$ не может достигаться в т. из $V_T$, т.е. из равенства $W(x,t) = \sup_{\bar{V}_T} W$ следует, что $(x,t) \in \Dcal_0 \cup S_T$. \\
От противного: $\exists$ точка $(x_0, t_0) \in V_T : W(x_0, t_0) = \sup_{\bar{V}_T} W > 0$. Тогда согласно необходимому условию максимума $\Delta_x W(x_0, t_0) \le 0 \stackrel{L_\Omega}{\implies} W_t(x_0, t_0) \le -\Omega W(x_0, t_0) \stackrel{W>0}{\implies} \text{убывает по } t$.
$\implies$ В интервале $(0, t_0)$ сущ. момент времени $\tilde{t}: W(x_0, \tilde{t}) > W(x_0, t_0) \to$ противоречие выбору т. $(x_0, t_0)$ как точки супремума $W(x,t)$ в $V_T$. $\quad \triangle$
\end{proof}

\begin{theorem}[Принцип максимума] \label{Th1}
Если $u(x,t) \in C_{x,t}^{2,1}(V_T) \cap C(\bar{V}_T)$ и $u_t - a^2 \Delta_x u = 0$, то $\forall (x,t) \in V_T \implies$
\[ \inf_{\Dcal_0 \cup S_T} u \le u(x,t) \le \sup_{\Dcal_0 \cup S_T} u. \quad (1) \]
\end{theorem}

\begin{proof}
$\triangle$ Считаем $\sup_{V_T} u(x,t) > 0$ (иначе можно рассмотреть $u(x,t) + \text{const}$, т.к. если $u(x,t)$ удовлетворяет $\inf_{\Dcal_0 \cup S_T} u \le u \le \sup_{\Dcal_0 \cup S_T} u$, то это выполнено и для $u+\text{const}$). Возьмём произвольное $\Omega > 0$ и рассмотрим $W(x,t) = u(x,t) e^{-\Omega t}$. В $V_T$ выполнено:
$W_t - a^2 \Delta_x W + \Omega_0 W = 0 \implies$ По Лемме 1: $e^{-\Omega t} u(x,t) < \sup_{(\xi, \tau) \in \Dcal_0 \cup S_T} e^{-\Omega \tau} u(\xi, \tau)$
$\le \sup_{(\xi, \tau) \in \Dcal_0 \cup S_T} u(\xi, \tau)$.
При $\Omega \to +0: u(x,t) \le \sup_{(\xi, \tau) \in \Dcal_0 \cup S_T} u(\xi, \tau)$, $(x,t) \in V_T \to$ доказали правую часть (1).
Для доказательства левой части (1) заметим, что $-u(x,t)$ удовлетворяет равенству $u_t - a^2 \Delta_x u = 0$, причём $-\sup_{\Dcal_0 \cup S_T} u = \inf_{\Dcal_0 \cup S_T} u$. $\quad \triangle$
\end{proof}

\begin{theorem}[Принцип максимума для неограниченной области] \label{Th2}
$\Dcal = \mathbb{R}^n$, $u(x,t) \in C_{x,t}^{2,1}(\mathbb{R}^n \times (0,T]) \cap C(\mathbb{R}^n \times [0,T]) \cap \Bcal(\mathbb{R}^n \times [0,T])$ и в $\mathbb{R}^n \times (0,T]$ удовлетворяет $u_t - a^2 \Delta_x u = 0$. Тогда:
\[ \inf_{\xi \in \mathbb{R}^n} u(\xi,0) \le u(x,t) \le \sup_{\xi \in \mathbb{R}^n} u(\xi,0), \quad (x,t) \in \mathbb{R}^n \times (0,T] \]
$\Bcal(M)$ --- мн-во ограниченных на $M$ функций.
\end{theorem}

\newpage
\hfill 25

% Continuing from the previous part
% ... code from the previous request ...

% Then add the new part
% ... code from this request ...


% \newtheorem{definition}[theorem]{Определение} % Not used in this fragment

\newcommand{\Fcal}{\mathcal{F}} % For Fourier transform operator

% Roman numerals for multiple solutions
\newcommand{\solI}[1]{\stackrel{\text{I}}{#1}}
\newcommand{\solII}[1]{\stackrel{\text{II}}{#1}}


\section*{Единственность решения З.К. и смешанной задачи для уравнения теплопроводности}

\begin{theorem} \label{Th1_edinstvennost}
$\Dcal \subset \mathbb{R}^n$ --- огран. обл. Если решение смеш. задачи
\begin{equation} \label{smesh_zadacha_star}
\begin{cases}
u_t - a^2 \Delta u = f, & (x,t) \in G = \Dcal \times (0,T) \\
u|_{t=0} = u_0(x), & u_0 \in C(\bar{\Dcal}) \\
u|_{x \in \partial\Dcal} = \nu(x,t), & \nu \in C(\partial\Dcal \times [0,T])
\end{cases}
\quad (*)
\end{equation}
существует в классе $C_{x,t}^{2,1}(G) \cap C(\bar{G})$, то оно единственно и непрерывно зависит от начальных данных.
\end{theorem}

\begin{proof}
$\triangle$ \emph{Единственность}: пусть $\solI{u}$ и $\solII{u}$ --- решения задачи \eqref{smesh_zadacha_star}. Тогда $u = \solII{u} - \solI{u}$ удовлетворяет однородному ур-нию теплопроводности с однородными начальными и граничными условиями
$$
\begin{cases}
u_t = a^2 \Delta_x u, & (x,t) \in G \\
u|_{t=0} = 0 \\
u|_{x \in \partial\Dcal} = 0
\end{cases}
$$
Из принципа максимума для ограниченной области (см. предыдущую Теорему 1 о принципе максимума)
$0 \le u(x,t) \le 0, (x,t) \in \bar{G} \implies \solI{u} \equiv \solII{u}$ в $\bar{G}$.

\emph{Непрерывная зависимость}: пусть $\solI{u}$ и $\solII{u}$ --- решения задачи \eqref{smesh_zadacha_star}, отвечающие различным начально-краевым данным: $\solI{u_0}, \solI{\nu}$ и $\solII{u_0}, \solII{\nu}$ соответственно. Тогда $u = \solII{u} - \solI{u}$ является решением смешанной задачи для однородного ур-ния теплопроводности:
$$
\begin{cases}
u_t = a^2 \Delta_x u, & (x,t) \in G \\
u|_{t=0} = \solII{u_0}(x) - \solI{u_0}(x) \\
u|_{x \in \partial\Dcal} = \solII{\nu}(x,t) - \solI{\nu}(x,t)
\end{cases}
$$
Из принципа максимума для ограниченной области:
\[
\sup_{(x,t) \in \bar{G}} |\solI{u}(x,t) - \solII{u}(x,t)| \le \max \left\{ \sup_{x \in \bar{\Dcal}} |\solI{u_0}(x) - \solII{u_0}(x)|, \sup_{(x,t) \in \partial\Dcal \times [0,T]} |\solI{\nu}(x,t) - \solII{\nu}(x,t)| \right\},
\]
$(x,t) \in \bar{G}$, что и означает непрерывную зависимость от нач. и краевых данных в равномерной метрике. $\quad \triangle$
\end{proof}

\begin{theorem} \label{Th2_edinstvennost_koshi}
Если решение задачи Коши
$$
\begin{cases}
u_t - a^2 \Delta u = f, & G = \mathbb{R}^n \times (0,T) \\
u|_{t=0} = u_0(x), & u_0 \in C(\mathbb{R}^n) \text{ с ограниченной } u_0
\end{cases}
$$
существует в классе $C_{x,t}^{2,1}(G) \cap C(\bar{G}) \cap \Bcal(\bar{G})$, то оно единственно и непрерывно зависит от нач. данных в равномерной метрике.
\end{theorem}

\begin{remark}
Можно показать единственность решения в классе $(G = \mathbb{R}^n \times (0,T))$:
$C_{x,t}^{2,1}(G) \cap C(\bar{G}) \cap F(G)$, где $F(G)$ --- класс функций ограниченного роста
\[ F(G) = \{ u \mid \exists C, \beta > 0 : \forall (x,t) \in G \implies |u(x,t)| \le C e^{\beta|x|^2} \}. \]
\end{remark}

\section*{Представление решения формулой Пуассона}
\begin{equation} \label{eq:poisson_problem}
\begin{cases}
u_t = \Delta u, & t>0, x \in \mathbb{R}^n \\
u|_{t=0} = u_0(x), & u_0(x) \in C(\mathbb{R}^n)
\end{cases}
; \quad u(t,x) \in C_{t,x}^{1,2}((0,T] \times \mathbb{R}^n) \iff
\end{equation}
\begin{equation} \label{eq:poisson_solution}
\iff u(x,t) = \frac{1}{(2\sqrt{\pi t})^n} \int_{\mathbb{R}^n} e^{-\frac{|x-y|^2}{4t}} u_0(y) dy
\end{equation}

$\triangle$ (Обозначения для преобразования Фурье)
\begin{align*}
\Fcal(\varphi(x))(\xi) &= \frac{1}{(2\pi)^{1/2}} \int_{-\infty}^{+\infty} e^{-ix\xi} \varphi(x) dx && \text{в } \mathbb{R}^1 \\
\Fcal(\varphi(\vec{x}))(\vec{\xi}) &= \frac{1}{(2\pi)^{n/2}} \int_{\mathbb{R}^n} e^{-i\langle\vec{x},\vec{\xi}\rangle} \varphi(\vec{x}) d\vec{x} = \hat{\varphi}(\vec{\xi}) && \text{в } \mathbb{R}^n 
\quad \} \text{ преобразование Фурье.}
\end{align*}



\newcommand{\conj}[1]{\overline{#1}} % For complex conjugate if needed, not explicitly used
\newcommand{\abs}[1]{\left|#1\right|}
\newcommand{\norm}[1]{\left\|#1\right\|}
\newcommand{\inp}[2]{\left\langle #1, #2 \right\rangle} % Inner product / duality pairing


% Continuing from the Fourier transform definitions from the previous page.
Обратное преобразование Фурье:
\[ \Fcal^{-1}(w(\vec{\xi}))(\vec{x}) = \frac{1}{(2\pi)^{n/2}} \int_{\mathbb{R}^n} e^{i\inp{\vec{x}}{\vec{\xi}}} w(\vec{\xi}) d\vec{\xi} \]
\[ \hat{\varphi}(x) = \varphi(-x) = \check{\varphi}(x) \] % Notation clarification: $\check{\varphi}$ is common for $\varphi(-x)$
Свойство преобразования Фурье для производных:
\[ \Fcal(D^\alpha u) = \vec{\xi}^\alpha \Fcal(u) \cdot (-i)^{|\alpha|}, \quad \text{где } D^\alpha u = \frac{\partial^{|\alpha|}u}{\partial x_1^{\alpha_1} \dots \partial x_n^{\alpha_n}} \]
(Примечание: если $D^\alpha = (-i\partial_x)^\alpha$, то $\Fcal(D^\alpha u) = \vec{\xi}^\alpha \hat{u}(\vec{\xi})$. Стандартно $\Fcal(\partial_{x_k} u) = i\xi_k \hat{u}(\vec{\xi})$.) \\
Для уравнения $u_t = \Delta u = \sum_{k=1}^n \frac{\partial^2 u}{\partial x_k^2}$, преобразование Фурье дает:
\[ \hat{u}_t(\vec{\xi}, t) = \sum_{k=1}^n (i\xi_k)^2 \hat{u}(\vec{\xi},t) = -|\vec{\xi}|^2 \hat{u}(\vec{\xi},t) \]
Применим к (1) (задаче Коши $u_t = \Delta u, u|_{t=0}=u_0$) преобразование Фурье по $x$, а также к начальному условию:
\[ \begin{cases} \hat{u}_t = -|\vec{\xi}|^2 \hat{u} \\ \hat{u}|_{t=0} = \hat{u}_0(\vec{\xi}) \end{cases}, \quad \text{где } |\vec{\xi}|^2 = \xi_1^2 + \xi_2^2 + \dots + \xi_n^2. \]
Решение этого ОДУ: $\hat{u}(\vec{\xi},t) = \hat{u}_0(\vec{\xi}) e^{-|\vec{\xi}|^2 t}$.
Применим обратное преобразование Фурье:
\begin{align*} u(t,\vec{x}) &= \frac{1}{(2\pi)^{n/2}} \int_{\mathbb{R}^n} e^{i\inp{\vec{x}}{\vec{\xi}}} e^{-|\vec{\xi}|^2 t} \hat{u}_0(\vec{\xi}) d\vec{\xi} \\ &= \frac{1}{(2\pi)^{n/2}} \int_{\mathbb{R}^n} e^{i\inp{\vec{x}}{\vec{\xi}}} e^{-|\vec{\xi}|^2 t} \left( \frac{1}{(2\pi)^{n/2}} \int_{\mathbb{R}^n} e^{-i\inp{\vec{y}}{\vec{\xi}}} u_0(\vec{y}) d\vec{y} \right) d\vec{\xi} \\ &= \int_{\mathbb{R}^n} \left( \frac{1}{(2\pi)^n} \int_{\mathbb{R}^n} e^{i\inp{\vec{x}-\vec{y}}{\vec{\xi}} - |\vec{\xi}|^2 t} d\vec{\xi} \right) u_0(\vec{y}) d\vec{y} \\ &= \int_{\mathbb{R}^n} P_n(t, \vec{x}-\vec{y}) u_0(\vec{y}) d\vec{y}, \quad \text{где } P_n(t, \vec{z}) = \frac{1}{(2\pi)^n} \int_{\mathbb{R}^n} e^{i\inp{\vec{z}}{\vec{\xi}} - |\vec{\xi}|^2 t} d\vec{\xi}. \end{align*}
Ядро $P_n(t, \vec{x}-\vec{y})$ можно представить в виде произведения:
\[ P_n(t, \vec{x}-\vec{y}) = \prod_{k=1}^n P_1(t, x_k-y_k), \quad \text{где } P_1(t, X) = \frac{1}{2\pi} \int_{-\infty}^{+\infty} e^{iX\xi - t\xi^2} d\xi. \]
Обозначим $X = x_k-y_k$, $\xi = \xi_k$ и посчитаем интеграл для $P_1(t,X)$:
\begin{align*} \int_{-\infty}^{+\infty} e^{iX\xi - t\xi^2} d\xi &= \int_{-\infty}^{+\infty} e^{-t(\xi^2 - \frac{iX}{t}\xi)} d\xi = \int_{-\infty}^{+\infty} e^{-t[(\xi - \frac{iX}{2t})^2 - (\frac{iX}{2t})^2]} d\xi \\ &= e^{-t(\frac{iX}{2t})^2} \int_{-\infty}^{+\infty} e^{-t(\xi - \frac{iX}{2t})^2} d\xi = e^{-\frac{X^2}{4t}} \int_{-\infty}^{+\infty} e^{-t(\xi - \frac{iX}{2t})^2} d\xi. \end{align*}
Для вычисления интеграла $\int_{-\infty}^{+\infty} e^{-t(\zeta - \frac{iX}{2t})^2} d\zeta$, рассмотрим контурный интеграл $\oint_\Gamma e^{-tz^2} dz = 0$ (т.к. нет особенностей, $z=\zeta+i\eta$).
Контур $\Gamma$ --- прямоугольник с вершинами $-N, N, N-i\frac{X}{2t}, -N-i\frac{X}{2t}$.
% Contour Diagram visual representation
%    <------- Top (-iX/2t)
%  ^ |             | V
%  L |-------------| R
%    -N     0     N  (Real axis)
%
Интегралы по вертикальным сторонам стремятся к нулю при $N \to \infty$.
\[ \int_{-N}^{N} e^{-t\zeta^2} d\zeta + \int_{N}^{-N} e^{-t(\zeta - \frac{iX}{2t})^2} d\zeta \to 0 \quad \text{при } N \to \infty. \]
Следовательно, $\int_{-\infty}^{+\infty} e^{-t(\zeta - \frac{iX}{2t})^2} d\zeta = \int_{-\infty}^{+\infty} e^{-t\zeta^2} d\zeta = \sqrt{\frac{\pi}{t}}$ (интеграл Пуассона).
Тогда
\[ P_1(t, x_k-y_k) = \frac{1}{2\pi} e^{-\frac{(x_k-y_k)^2}{4t}} \sqrt{\frac{\pi}{t}} = \frac{1}{2\sqrt{\pi t}} e^{-\frac{(x_k-y_k)^2}{4t}}. \]
\[ \implies P_n(t, \vec{x}-\vec{y}) = \prod_{k=1}^n P_1(t, x_k-y_k) = \frac{1}{(2\sqrt{\pi t})^n} e^{-\frac{|\vec{x}-\vec{y}|^2}{4t}}. \]
\[ \implies u(t,\vec{x}) = \frac{1}{(2\sqrt{\pi t})^n} \int_{\mathbb{R}^n} e^{-\frac{|\vec{x}-\vec{y}|^2}{4t}} u_0(\vec{y}) d\vec{y}. \quad \text{Формула (2).} \]

\subsection*{Теория для проверки начальных условий}
Покажем, что $u(t,x)$ удовлетворяет нач. усл.: Вначале теория: свёртка и иже с ней. \\
Для $u, v \in L_1(\mathbb{R}^n)$:
\[ (u * v)(x) = \int_{\mathbb{R}^n} u(x-y)v(y)dy, \quad \int_{\mathbb{R}^n} |u(x)|dx < \infty, \quad \int_{\mathbb{R}^n} |v(x)|dx < \infty. \]
Свойства: $u*v = v*u$; $u*(v*w) = (u*v)*w$; $(\alpha_1 u_1 + \alpha_2 u_2)*v = \alpha_1(u_1*v) + \alpha_2(u_2*v)$.

\begin{definition}
Пусть $f \in \Scal'(\mathbb{R}^n)$, $u \in \Scal(\mathbb{R}^n)$, тогда свёртка $f*u \in \Scal'(\mathbb{R}^n)$ определяется как
\[ \inp{f*u}{\varphi} = \inp{f}{\check{u}*\varphi} \quad \forall \varphi \in \Scal(\mathbb{R}^n), \]
где $\check{u}(x) = u(-x)$. (См. N13)
\end{definition}
Тогда для $\delta$-функции:
\[ \inp{\delta*u}{\varphi} = \inp{\delta}{\check{u}*\varphi} = (\check{u}*\varphi)(0). \]
Поскольку $(\check{u}*\varphi)(x) = \int_{\mathbb{R}^n} \check{u}(x-y)\varphi(y)dy = \int_{\mathbb{R}^n} u(y-x)\varphi(y)dy$, то
\[ (\check{u}*\varphi)(0) = \int_{\mathbb{R}^n} u(y)\varphi(y)dy = \inp{u}{\varphi}. \]
Следовательно, $\delta*u = u$ (в смысле распределений).
Также $\inp{\delta}{\psi} = \psi(0)$.

\newpage
\hfill 26



% Continuing from the previous page (Fourier Transform in R^n)

Свойство преобразования Фурье для свёртки (при нормировке $\Fcal(\varphi)(\xi) = \frac{1}{(2\pi)^{n/2}}\int_{\R^n} e^{-i\langle x, \xi \rangle} \varphi(x) dx$):
\[ \Fcal(u*v) = (2\pi)^{n/2} \Fcal(u) \cdot \Fcal(v) \]

\subsection*{Дельта-образная последовательность}
Пусть $\chi \in L_1(\R^n)$ такая, что $\int_{\R^n} \chi(x)dx = 1$.
Определим \emph{дельта-образную последовательность} (аппроксимативную единицу, или усредняющее ядро) как
\[ \chi_\epsilon(x) = \frac{1}{\epsilon^n} \chi\left(\frac{x}{\epsilon}\right), \quad \epsilon > 0. \]
Тогда $\int_{\R^n} \chi_\epsilon(x)dx = 1$.
Операция $u_\epsilon(x) = (\chi_\epsilon * u)(x) = \int_{\R^n} \chi_\epsilon(x-y)u(y)dy$ называется \emph{регуляризацией} функции $u$.

\begin{lemma}
Пусть $u \in C(\R^n)$. Если $u$ равномерно непрерывна на $\R^n$ (например, если $u \in C_0(\R^n)$ или $u$ периодическая), то $u_\epsilon \to u$ равномерно на $\R^n$ при $\epsilon \to 0$. Если $u \in C(\R^n)$, то $u_\epsilon \to u$ равномерно на компактах.
\end{lemma}
\begin{proof}[Схема доказательства]
$|u_\epsilon(x) - u(x)| = \left| \int_{\R^n} \chi_\epsilon(x-y)u(y)dy - u(x) \int_{\R^n} \chi_\epsilon(x-y)dy \right|$
\[ = \left| \int_{\R^n} \chi_\epsilon(x-y)[u(y)-u(x)]dy \right| \]
Пусть $z = \frac{x-y}{\epsilon}$, тогда $y = x-\epsilon z$, $dy = \epsilon^n dz$.
\[ = \left| \int_{\R^n} \chi(z)[u(x-\epsilon z)-u(x)]dz \right| \le \int_{\R^n} |\chi(z)| |u(x-\epsilon z)-u(x)|dz. \]
Предположим $\chi(z) \ge 0$. Тогда $\int_{\R^n} \chi(z)dz = 1$.
\[ |u_\epsilon(x) - u(x)| \le \int_{\R^n} \chi(z) |u(x-\epsilon z)-u(x)|dz. \]
Для любого $\eta > 0$, так как $u$ равномерно непрерывна, существует $\delta > 0$ такое, что если $|w| < \delta$, то $|u(v+w)-u(v)| < \eta/2$ для всех $v$.
Разобьем интеграл на две части: где $|\epsilon z| < \delta$ и где $|\epsilon z| \ge \delta$.
\begin{align*} \int_{|\epsilon z| < \delta} \chi(z) |u(x-\epsilon z)-u(x)|dz &\le \frac{\eta}{2} \int_{|\epsilon z| < \delta} \chi(z)dz \le \frac{\eta}{2} \int_{\R^n} \chi(z)dz = \frac{\eta}{2}. \\ \int_{|\epsilon z| \ge \delta} \chi(z) |u(x-\epsilon z)-u(x)|dz &\le 2 \sup|u| \int_{|z| \ge \delta/\epsilon} \chi(z)dz. \end{align*}
Поскольку $\chi \in L_1(\R^n)$, то $\int_{|z| \ge R} \chi(z)dz \to 0$ при $R \to \infty$.
Значит, для достаточно малого $\epsilon$, $\int_{|z| \ge \delta/\epsilon} \chi(z)dz$ можно сделать меньше $\frac{\eta}{4 \sup|u|}$.
Тогда $|u_\epsilon(x) - u(x)| < \eta$.
В тексте на рисунке:
$\frac{1}{\epsilon^n} \int_{\R^n} \chi(\frac{x-y}{\epsilon}) [u(y)-u(x)]dy = I_1 + I_2$.
$I_1 = \frac{1}{\epsilon^n} \int_{|y-x|<\delta} \chi(\frac{x-y}{\epsilon}) [u(y)-u(x)]dy \le \eta \frac{1}{\epsilon^n} \int_{\R^n} \chi(\frac{x-y}{\epsilon})dy = \eta$.
$I_2 = \frac{1}{\epsilon^n} \int_{|y-x|\ge\delta} \chi(\frac{x-y}{\epsilon}) [u(y)-u(x)]dy \le 2M \int_{|z|\ge\delta/\epsilon} \chi(z)dz \xrightarrow{\epsilon \to 0} 0$.
$\implies |u_\epsilon(x)-u(x)| \xrightarrow{\epsilon \to 0} 0$ равномерно по $x$, если $u \in C_{unif}(\R^n) \cap L_\infty(\R^n)$.
\end{proof}

Рассмотрим $P_n(t,x) = \frac{1}{(2\sqrt{\pi t})^n} e^{-|x|^2/(4t)}$.
Обозначим $\chi(x) = \frac{1}{(\sqrt{\pi})^n} e^{-|x|^2}$. Тогда $\int_{\R^n} \chi(x)dx = 1$.
Если положить $\epsilon = 2\sqrt{t}$, то $P_n(t,x) = \frac{1}{\epsilon^n} \chi(x/\epsilon) = \chi_\epsilon(x)$.
Тогда решение $u(t,x) = (P_n(t,\cdot)*u_0)(x) = (\chi_\epsilon * u_0)(x)$.
Из леммы следует, что если $u_0 \in C(\R^n)$ (и равномерно непрерывна), то $u(t,x) \to u_0(x)$ при $t \to 0^+$ (т.е. $\epsilon \to 0^+$) равномерно по $x$.

Т.о. доказаны существование (формула Пуассона) и единственность (принцип максимума) решения задачи Коши для ур-ния теплопроводности.

\begin{remark}
Для уравнения $u_t = a^2 \Delta u$ с $u|_{t=0}=u_0(x)$, заменой $\tau = a^2 t$ оно сводится к $u_\tau = \Delta u$.
Решение:
\[ u(t,x) = \frac{1}{(2a\sqrt{\pi t})^n} \int_{\R^n} e^{-\frac{|x-y|^2}{4a^2 t}} u_0(y)dy. \]
Все производные $\partial_t^k \partial_x^\alpha u$ существуют и являются непрерывными для $t>0$ (если $u_0$ достаточно гладкая, например, $u_0 \in C_b(\R^n)$), в силу сходимости интегралов вида
\[ \int_{\R^n} Q_m(t,x-y) \exp\left(-\frac{|x-y|^2}{4t}\right) u_0(y)dy \]
для полинома $Q_m(t,x-y)$ (относительно $x-y$, с коэффициентами, зависящими от $t$) степени $m \in \N$.
\end{remark}

\section*{Принцип Дюамеля для ур-ния теплопроводности}
Рассмотрим задачу
\begin{equation} \label{eq:neh_heat}
\begin{cases}
u_t - a^2 \Delta u = f(x,t) \\
u|_{t=0} = u_0(x)
\end{cases}
\end{equation}
Будем искать решение в виде $u(x,t) = v(x,t) + w(x,t)$, где
\begin{equation} \label{eq:duhamel_v}
(1a) \quad \begin{cases}
v_t - a^2 \Delta v = 0 \\
v|_{t=0} = u_0(x)
\end{cases}
\end{equation}
\begin{equation} \label{eq:duhamel_w}
(2a) \quad \begin{cases}
w_t - a^2 \Delta w = f(x,t) \\
w|_{t=0} = 0
\end{cases}
\end{equation}




\begin{theorem}[Принцип Дюамеля]
Если существует решение $p(x,t,\tau)$ вспомогательной задачи
\begin{equation} \label{eq:aux_duhamel}
\begin{cases}
p_t - a^2 \Delta_x p = 0, & x \in \R^n, t>\tau \\
p(x,\tau,\tau) = f(x,\tau) &
\end{cases}
\end{equation}
из класса $C_{x,t,\tau}^{2,1,0}(\bar{M})$, где $M=\{(x,t,\tau) \mid x \in \R^n, \tau>0, t>\tau \}$, то функция
\[ W(x,t) = \int_0^t p(x,t,\tau)d\tau \]
является решением задачи \eqref{eq:duhamel_w} (т.е. $W_t - a^2 \Delta_x W = f(x,t)$, $W|_{t=0}=0$).
\end{theorem}

\begin{proof}
$\triangle$ В соответствии с предположениями о гладкости $p(x,t,\tau)$:
\begin{enumerate}
    \item $W(x,0) = \int_0^0 p(x,0,\tau)d\tau = 0$; --- выполнено начальное условие.
    \item Дифференцируем $W(x,t)$ по $t$ (правило Лейбница для дифференцирования под знаком интеграла):
    \begin{align*}
        W_t(x,t) &= \frac{\partial}{\partial t} \int_0^t p(x,t,\tau) d\tau \\
        &= p(x,t,t) \cdot \frac{\partial t}{\partial t} - p(x,t,0) \cdot \frac{\partial 0}{\partial t} + \int_0^t \frac{\partial}{\partial t} p(x,t,\tau) d\tau \\
        &= p(x,t,t) + \int_0^t p_t(x,t,\tau) d\tau.
    \end{align*}
    Из \eqref{eq:aux_duhamel} имеем $p(x,t,t) = f(x,t)$ и $p_t(x,t,\tau) = a^2 \Delta_x p(x,t,\tau)$ для $t>\tau$.
    Тогда
    \begin{align*}
        W_t(x,t) &= f(x,t) + \int_0^t a^2 \Delta_x p(x,t,\tau) d\tau \\
        &= f(x,t) + a^2 \Delta_x \int_0^t p(x,t,\tau) d\tau \quad \text{(достаточная гладкость для смены порядка)} \\
        &= f(x,t) + a^2 \Delta_x W(x,t).
    \end{align*}
    Таким образом, $W_t(x,t) - a^2 \Delta_x W(x,t) = f(x,t)$ для $t>0$. --- выполнено уравнение. $\quad \triangle$
\end{enumerate}
\end{proof}

\newpage
\section*{Билет №10 -- 2025}\label{sec:ticket10}
\backtotoc
\textbf{Дельта - образная последовательность. Формула Пуассона для решения уравнения
теплопроводности с непрерывной начальной функцией. [1] – 114-115, [2] – 193, 103-110, 113-114.}



\newtheoremstyle{mydefinition}
  {\topsep}   % ABOVESPACE
  {\topsep}   % BELOWSPACE
  {}  % BODYFONT
  {0pt}       % INDENT (empty value is the same as 0pt)
  {\bfseries} % HEADFONT
  {.}         % HEADPUNCT
  {5pt plus 1pt minus 1pt} % HEADSPACE
  {}          % CUSTOM-HEAD-SPEC
\theoremstyle{mydefinition}
\newtheorem*{definition}{Определение}

\newcommand{\R}{\mathbb{R}}
\newcommand{\N}{\mathbb{N}}
\newcommand{\D}{\mathcal{D}}


\section*{Лекция 11}
Дельта-образная последовательность. Формула Пуассона для решения ур-я теплопроводности с непрерывной начальной ф-ей.
+ Регуляризация обобщенной функции (в конце).

\begin{definition}
Носителем функции $\varphi \in C^m(\R)$ (в рукописи $C^m(\R)$, часто $C^\infty(\R)$ в контексте основных функций) назовем замыкание множества тех точек $x \in \R$, для которых верно $\varphi(x) \neq 0$.
\[ \supp \varphi = \overline{\{x \in \R : \varphi(x) \neq 0\}} \]
\end{definition}

\begin{definition}
Функция $\varphi \in C^m(\R)$ -- финитная, если $\supp \varphi$ -- ограниченное множество.
\end{definition}

Пусть $\mathcal{L}_1, \mathcal{L}_2$ -- линейные пространства, а $N_1 = (\mathcal{L}_1, \| \cdot \|_{1})$ и $N_2 = (\mathcal{L}_2, \| \cdot \|_{2})$ -- нормированные пространства.

\begin{definition}
Оператор $A: N_1 \to N_2$ наз-ся линейным, если для $\forall f, g \in N_1$ и $\forall \alpha, \beta \in \R$ (или $\mathbb{C}$)
\[ A(\alpha f + \beta g) = \alpha A(f) + \beta A(g) \]
\end{definition}

\begin{definition}
Лин. оператор $A: N_1 \to N_2$ -- непрер., если из сходимости произвольной последовательности $\{f_n\}$ элементов из этого пространства к некоторому элементу $f \in N_1$, следует сходимость $\{A(f_n)\}$ элементов в пространстве $N_2$ к элементу $A(f) \in N_2$, т.е.
\[ \|f_n - f\|_{1} \to 0 \implies \|A(f_n) - A(f)\|_{2} \to 0 \]
\end{definition}

\noindent
\textbf{Частный случай} лин. операторов -- лин. функционалы. Пусть $\mathcal{L}$ -- лин. пр-во, $N = (\mathcal{L}, \| \cdot \|)$ -- норм. пр-во. Для функционала $l: N \to \R$ (или $\mathbb{C}$) справедливы определения выше, а $\| \cdot \|_{2}$ понимается как модуль.

\vspace{\baselineskip}
Рассмотрим пространство $\D = \D(\R)$ всех финитных $C^\infty(\R)$ функций (линейное пространство над $C^\infty(\R)$).

\begin{definition}
Будем говорить, что послед. $\{\varphi_n\}_{n=1}^\infty \subset \D$ сходится к $\varphi \in \D$, если:
\begin{enumerate}
    \item $\exists R > 0 : \forall n \in \N \implies \supp \varphi_n \subset (-R, R)$ (или $[-R,R]$)
    \item $\forall p \in \N_0$ (т.е. $p \ge 0$ целое, в рукописи $\forall p \in \N$) функциональная послед. $\left\{\frac{d^p \varphi_n}{dx^p}\right\}_{n=1}^\infty$ сходится равномерно на $\R$ к $\frac{d^p \varphi}{dx^p}$ при $n \to \infty$.
\end{enumerate}
Полученное функц. пр-во $\D = \D(\R)$ назовем пространством основных функций.
\end{definition}

\noindent
Например:
\[ \omega_\varepsilon(x) = \begin{cases} C_\varepsilon \exp\left(-\frac{\varepsilon^2}{\varepsilon^2 - x^2}\right), & |x| < \varepsilon \\ 0, & |x| \ge \varepsilon \end{cases} \]
$\omega_\varepsilon \in \D$. $C_\varepsilon$ выбирается так, чтобы $\int_{-\infty}^{\infty} \omega_\varepsilon(x) dx = 1$.
Такая ф-я называется ядром усреднения (mollifier).

\begin{definition}
Обобщенной функцией наз-ся всякий линейный непрерывный функционал на пространстве основных функций $\D$.
\end{definition}

\noindent
Множество линейных непрерывных функционалов на $\D$ -- лин. пр-во, которое обозначается $\D' = \D'(\R)$.
Действие обобщ. ф-ии $f$ на ф-ию $\varphi \in \D$:
\[ (f, \varphi) \quad \text{или} \quad \langle f, \varphi \rangle = f(\varphi) \]

\begin{definition}
Определим сходимость: говорят, что последовательность функционалов $\{f_n\}$ сходится к $f \in \D'$ при $n \to \infty$, если $\forall \varphi \in \D$
\[ (f_n, \varphi) \xrightarrow{n \to \infty} (f, \varphi) \]
(т.е. числовая последовательность $(f_n, \varphi)$ сходится к числу $(f, \varphi)$).
\end{definition}

\noindent
Рассмотрим функционал на $\D$, $\delta$, действующий как:
\[ (\delta, \varphi) = \varphi(0) \]
Назовем $\delta$-функцией Дирака.





\begin{itemize}
    \item Произвольная локально абсолютно интегрируемая в $\R^n$ ф-я $f(x)$ порождает обобщенную функцию $f$, значение которой на основной функции $\varphi$ задается формулой:
    \[ (f, \varphi) = \int_{\R^n} f(x) \varphi(x) dx \quad (1) \]
\end{itemize}

\begin{definition}[корректно]
\leavevmode % Ensures the definition starts on a new line if it's after a list
\begin{enumerate}
    \item Если $f(x)$ -- локально абсолютно интегрируема в $\R^n$ и $\varphi \in \D(\R^n)$, то интеграл (1) существует.
    \[ \left| \int_{\R^n} f(x)\varphi(x)dx \right| \le \int_{\supp \varphi} |f(x)||\varphi(x)|dx \le \sup_{x \in \supp \varphi} |\varphi(x)| \int_{\supp \varphi} |f(x)|dx < \infty \]
    (т.к. $\supp \varphi$ компактен, а $f$ локально интегрируема).
    \item Линейность $(f, \cdot)$ следует из линейности интеграла.
    \item Непрерывность $f$: пусть $\varphi_k \xrightarrow{k \to \infty} \bar{\varphi}$ в $\D(\R^n)$. Тогда существует компакт $M \subset \R^n$ такой, что $\supp \varphi_k \subset M$ для всех $k \in \N$, и $\sup_{x \in \R^n} |\varphi_k(x) - \bar{\varphi}(x)| \xrightarrow{k \to \infty} 0$.
    Тогда
    \begin{align*} |(f, \varphi_k) - (f, \bar{\varphi})| &= |(f, \varphi_k - \bar{\varphi})| = \left| \int_{\R^n} f(x)[\varphi_k(x) - \bar{\varphi}(x)]dx \right| \\ &\le \int_{M} |f(x)| |\varphi_k(x) - \bar{\varphi}(x)| dx \\ &\le \sup_{x \in M} |\varphi_k(x) - \bar{\varphi}(x)| \int_{M} |f(x)|dx \xrightarrow{k \to \infty} 0 \end{align*}
    Следовательно, $(f, \varphi_k) \xrightarrow{k \to \infty} (f, \bar{\varphi})$.
\end{enumerate}
\end{definition}

\begin{definition}
Обобщенные функции, определяемые локально абсолютно интегрируемыми функциями по формуле (1), называются \textbf{регулярными} обобщенными функциями. Остальные -- \textbf{сингулярными}.
\end{definition}

\begin{itemize}
    \item $\delta$-функция сингулярна. Предположим (от противного), что существует $f \in L^1_{loc}(\R^n)$ такая, что $(\delta, \varphi) = \int_{\R^n} f(x)\varphi(x)dx$ для всех $\varphi \in \D(\R^n)$.
    Тогда $\varphi(0) = \int_{\R^n} f(x)\varphi(x)dx$.
    Возьмем в качестве $\varphi$ ядро усреднения $\omega_\varepsilon(x)$ (см. предыдущую страницу и определение ниже). Для таких ядер $\omega_\varepsilon(0)$ обычно имеет вид $C/\varepsilon^n$ (например, если $\omega_\varepsilon(x) = (1/\varepsilon^n) \omega_1(x/\varepsilon)$) и $\int \omega_\varepsilon(x)dx = 1$.
    Тогда $\omega_\varepsilon(0) = \int_{\R^n} f(x)\omega_\varepsilon(x)dx$.
    С другой стороны,
    \[ \left| \int_{\R^n} f(x)\omega_\varepsilon(x)dx \right| = \left| \int_{B(0,\varepsilon)} f(x)\omega_\varepsilon(x)dx \right| \le \sup_{|x|<\varepsilon}|\omega_\varepsilon(x)| \int_{B(0,\varepsilon)} |f(x)|dx. \]
    Если $\omega_\varepsilon(0) = \sup_{|x|<\varepsilon}|\omega_\varepsilon(x)|$ (что типично), то $1 \le \int_{B(0,\varepsilon)} |f(x)|dx$.
    Но так как $f \in L^1_{loc}(\R^n)$, то $\int_{B(0,\varepsilon)}|f(x)|dx \xrightarrow{\varepsilon \to 0} 0$.
    Получаем противоречие $1 \le 0$. Следовательно, $\delta$-функция не может быть представлена таким интегралом, т.е. она сингулярна.
\end{itemize}

\begin{remark}
Даже для сингулярных функций формально пишут $(f, \varphi) = \int f(x) \varphi(x) dx$.
\end{remark}

\begin{definition}
\textbf{Ядром усреднения} (или \textbf{моллификатором}) называется функция $\omega_\varepsilon(x)$, заданная при $x \in \R^n_x$ и параметре $\varepsilon > 0$, и удовлетворяющая следующим условиям:
\begin{enumerate}
    \item $\omega_\varepsilon(x) \in C^\infty(\R^n_x)$ при всех $\varepsilon > 0$.
    \item $\supp \omega_\varepsilon \subset \overline{B(0, \varepsilon)}$ (т.е. $\omega_\varepsilon(x) = 0$ при $|x| \ge \varepsilon$).
    \item $\omega_\varepsilon(x) \ge 0$ для всех $x \in \R^n_x$.
    \item $\int_{\R^n_x} \omega_\varepsilon(x) dx = 1$ при всех $\varepsilon > 0$.
\end{enumerate}
(Например, см. функцию на предыдущей странице).
\end{definition}

Легко видеть, что функции $\omega_\eta(x)$, образующие ядро усреднения, сходятся к $\delta$-функции при $\eta \to 0$ в смысле сходимости обобщенных функций в $\D'(\R^n_x)$. Действительно, для любой $\varphi \in \D(\R^n_x)$:
\[ (\omega_\eta, \varphi) := \int_{\R^n_x} \omega_\eta(x) \varphi(x) dx \xrightarrow{\eta \to 0} \varphi(0) = (\delta, \varphi) \]
(Это следует из свойств аппроксимации единицы; $\left| \int \omega_\eta(x)\varphi(x)dx - \varphi(0) \right| = \left| \int \omega_\eta(x)(\varphi(x)-\varphi(0))dx \right| \le \sup_{|x|<\eta}|\varphi(x)-\varphi(0)| \int \omega_\eta(x)dx \to 0$).
(В оригинале: $\varphi_\eta(0) \to \varphi(0)$, что может быть ссылкой на значение усредненной функции $\varphi_\eta(y) = (\varphi * \omega_\eta)(y)$ в точке $0$, если $\omega_\eta$ симметрична).
(В оригинале ссылка: "следует из теоремы ниже".)

\begin{definition}
Последовательность функций $\{\omega_\eta\}_{\eta > 0}$, сходящаяся к $\delta$-функции в смысле сходимости в $\D'(\R^n_x)$ (т.е. $(\omega_\eta, \varphi) \to (\delta, \varphi)$ для всех $\varphi \in \D$), называется $\delta$-образной последовательностью.
\end{definition}

\begin{theorem}
Пусть $\Omega$ -- ограниченная область в $\R^n_x$.
\begin{enumerate}
    \item Если $u(x) \in L_p(\Omega)$ ($1 \le p < \infty$) и $u(x)$ равна нулю вне некоторой подобласти $\Omega_1$ такой, что $\bar{\Omega}_1 \subset \Omega$ (т.е. $\Omega_1$ компактно вложена в $\Omega$), то усредненная функция
    \[ u^\varepsilon(x) = (u * \omega_\varepsilon)(x) = \int_{\R^n} \omega_\varepsilon(x-y)u(y)dy = \int_{B(x,\varepsilon)} \omega_\varepsilon(x-y)u(y)dy \]
    принадлежит $C_0^\infty(\Omega)$ (т.е. является бесконечно дифференцируемой функцией с компактным носителем в $\Omega$) при $\varepsilon < \delta_0$, где $\delta_0 = \dist(\bar{\Omega}_1, \partial\Omega)$.
    (В оригинале: "компактн. носитель". Последующий текст в оригинале, видимо, поясняет, почему носитель $u^\varepsilon$ компактен и содержится в $\Omega$).
\end{enumerate}
\end{theorem}

\begin{definition}
Функция $u^\varepsilon(x)$ называется \textbf{средней функцией} от $u(x)$ (или усреднением $u(x)$) с ядром усреднения $\omega_\varepsilon$.
\end{definition}



\newcommand{\G}{\mathcal{G}} % Using calG formathcal{G}




\begin{definition}
Рассмотрим в области $G \subset \R^n_x$ произвольное уравнение
\[ F(x, u(x), D_x^1 u(x), \dots, D_x^m u(x)) = 0, \quad x \in G \quad (2) \]
Пусть $U$ -- взаимно однозначное преобразование независимых переменных
\[ \G \leftrightarrow \G, \quad \bar{x} = U(x) \quad \text{(точнее $U: G \to \bar{G}$)} \quad (3) \]
(нарисована стрелка $G \leftrightarrow \bar{G}$, но в тексте $G \leftrightarrow G$).
Уравнение (2) называется \textbf{инвариантным} относительно пр-ия $U$ (независимых переменных), если оно равносильно
\[ F(U(x), u(U(x)), \dots) = 0, \quad x \in G \]
(т.е. если $u(x)$ решение (2), то $u(U^{-1}(\bar{x}))$ является решением (2) в $\bar{G}$, или, если форма уравнения не меняется при замене $x \to \bar{x}$).
$U_\alpha$ -- преобразование, зависящее от параметра $\alpha$.
\end{definition}

\begin{definition}
Множество преобразований $\{U_\alpha\}_{\alpha \in \mathcal{D}}$ (где $\mathcal{D}$ -- некоторое множество параметров) называется \textbf{однопараметрической группой преобразований}, если:
\begin{enumerate}
    \item $\forall \alpha_1, \alpha_2 \in \mathcal{D} \quad \exists! \alpha_3 \in \mathcal{D} : U_{\alpha_3} = U_{\alpha_1} \circ U_{\alpha_2}$ (т.е. $\mathcal{D} \times \mathcal{D}$ определяет $\gamma: \alpha_3 = \gamma(\alpha_1, \alpha_2)$). (Т.е. композиция двух преобразований из группы также принадлежит группе).
    \item $\exists! \alpha_0 \in \mathcal{D} : \forall \alpha \in \mathcal{D} \implies U_{\alpha_0} \circ U_\alpha = U_\alpha \circ U_{\alpha_0} = U_\alpha$ (т.е. $U_{\alpha_0}$ -- тождественное преобразование).
    \item $\forall \alpha \in \mathcal{D} \quad \exists! \beta \in \mathcal{D} : U_\alpha \circ U_\beta = U_\beta \circ U_\alpha = U_{\alpha_0}$ (т.е. $U_\beta$ -- обратное к $U_\alpha$).
\end{enumerate}
\end{definition}

\begin{definition}
Говорят, что ур-е (2) \textbf{допускает группу} $G = \{U_\alpha\}_{\alpha \in \mathcal{D}}$ однопараметрических преобразований, если оно инвариантно относительно каждого преобразования $U_\alpha$ из этой группы.
\end{definition}

\begin{definition}
Функция $I(x)$ -- \textbf{инвариант} однопараметрической группы преобразований $\{U_\alpha\}_{\alpha \in \mathcal{D}}$, если $\forall \alpha \in \mathcal{D} \implies I(x) = I(U_\alpha(x))$, $x \in G$.
\end{definition}

\noindent \textbf{Однородное уравнение теплопроводности}
\[ u_t(x,t) - a^2 \Delta_x u(x,t) = 0, \quad (x,t) \in G = \R^n_x \times (0, +\infty) \quad (4) \]
Рассмотрим однопараметрическую группу (здесь, скорее, двухпараметрическую или группу растяжений): % Оригинал говорит "однопар. группа"
\[ \begin{cases} \bar{t} = \alpha^{2k} t \\ \bar{x} = \alpha^k x \end{cases}, \quad \alpha \in \mathcal{D} = (0, +\infty), \quad k \in \R \quad (5) \]
(В оригинале $\bar{t} = \alpha t$, $\bar{x} = \alpha^k x$. Исправлено для согласования с последующим выводом $k=1/2$. Если $\bar{t} = \alpha t$, то инвариантность требует $\alpha = (\alpha^k)^2 \implies \alpha = \alpha^{2k} \implies 2k=1 \implies k=1/2$. Если же $\bar{t}=\alpha^{2k} t$ как написано выше, то инвариантность требует $\alpha^{2k} u_{\bar{t}} = a^2 (\alpha^k)^2 \Delta_{\bar{x}} u \implies u_{\bar{t}} = a^2 \Delta_{\bar{x}} u$. Это делает группу (5) группой симметрий уравнения (4) для любого $k$. Но для автомодельных решений обычно используется $k=1/2$).
Сделаем замену в уравнении (4) ($\tilde{t} = \alpha t, \tilde{x} = \alpha^k x$ из оригинала):
$u_t(\alpha^k x, \alpha t)$. Если $\tilde{u}(\tilde{x}, \tilde{t}) = u(x, t) = u(\alpha^{-k}\tilde{x}, \alpha^{-1}\tilde{t})$.
Тогда $\frac{\partial u}{\partial t} = \frac{\partial \tilde{u}}{\partial \tilde{t}} \frac{\partial \tilde{t}}{\partial t} = \alpha \frac{\partial \tilde{u}}{\partial \tilde{t}}$.
$\frac{\partial u}{\partial x_i} = \frac{\partial \tilde{u}}{\partial \tilde{x}_i} \frac{\partial \tilde{x}_i}{\partial x_i} = \alpha^k \frac{\partial \tilde{u}}{\partial \tilde{x}_i}$.
$\frac{\partial^2 u}{\partial x_i^2} = \alpha^{2k} \frac{\partial^2 \tilde{u}}{\partial \tilde{x}_i^2}$.
Уравнение (4) в новых переменных: $\alpha \tilde{u}_{\tilde{t}} - a^2 \alpha^{2k} \Delta_{\tilde{x}} \tilde{u} = 0$.
Чтобы форма уравнения сохранилась (т.е. $\tilde{u}_{\tilde{t}} - a^2 \Delta_{\tilde{x}} \tilde{u} = 0$), нужно $\alpha = \alpha^{2k}$, что означает $2k=1 \implies k=1/2$.
Тогда инвариантной заменой будет $\bar{t} = \alpha t$, $\bar{x} = \sqrt{\alpha} x$.
Инвариант группы (5) при $k=1/2$: $\zeta = \frac{x}{\sqrt{t}}$. (Действительно, $\frac{\bar{x}}{\sqrt{\bar{t}}} = \frac{\sqrt{\alpha}x}{\sqrt{\alpha t}} = \frac{x}{\sqrt{t}}$).

\begin{definition}
Решение ур-я (2), зависящее только от инварианта некоторой допущенной однопараметрической группы преобразований, называется \textbf{автомодельным решением}.
\end{definition}
Ищем $u(x,t) = f\left(\frac{x}{\sqrt{t}}\right) = f(\zeta)$. (Для $n=1$, $\zeta = x/\sqrt{t}$).
Подставляем в (4) (для $n=1$, $\Delta_x u = u_{xx}$):
$u_t = f'(\zeta) \cdot \left(-\frac{1}{2} \frac{x}{t^{3/2}}\right) = -\frac{1}{2t} \zeta f'(\zeta)$.
$u_x = f'(\zeta) \cdot \frac{1}{\sqrt{t}}$.
$u_{xx} = f''(\zeta) \cdot \frac{1}{t}$.
Уравнение (4) становится: $-\frac{1}{2t} \zeta f'(\zeta) - \frac{a^2}{t} f''(\zeta) = 0 \implies \zeta f'(\zeta) + 2a^2 f''(\zeta) = 0$.
Пусть $g(\zeta) = f'(\zeta)$, тогда $\zeta g(\zeta) + 2a^2 g'(\zeta) = 0$.
$\frac{g'}{g} = -\frac{\zeta}{2a^2} \implies \ln|g| = -\frac{\zeta^2}{4a^2} + C_0 \implies g(\zeta) = C_1 e^{-\frac{\zeta^2}{4a^2}}$.
$f(\zeta) = \int C_1 e^{-\frac{\lambda^2}{4a^2}} d\lambda + C_2$.
Недостаток: не удовлетворяет $\int_{\R^n} u(x,t)dx = \text{const}$ при $t>0$ (закону сохранения "тепла" или массы).
Ищем $u(x,t) = c(t) \tilde{f}\left(\frac{x}{\sqrt{t}}\right)$. Если $\tilde{f}(\zeta) = \prod_{i=1}^n h(\zeta_i)$ (разделение переменных для $\tilde{f}$), $\zeta_i = x_i/\sqrt{t}$.
Подставляя в (4), получаем $c'(t)\tilde{f} + c(t) \sum_i \frac{\partial \tilde{f}}{\partial \zeta_i} \left(-\frac{1}{2}\frac{x_i}{t^{3/2}}\right) - a^2 c(t) \sum_i \frac{\partial^2 \tilde{f}}{\partial \zeta_i^2} \frac{1}{t} = 0$.
$\frac{c'(t)}{c(t)}\tilde{f} - \frac{1}{2t} \sum_i \zeta_i \frac{\partial \tilde{f}}{\partial \zeta_i} - \frac{a^2}{t} \Delta_\zeta \tilde{f} = 0$.
Чтобы получить ОДУ для $\tilde{f}$, необходимо, чтобы $\frac{tc'(t)}{c(t)} = \text{const} = -\frac{n}{2}$ (выбор константы для согласования с известным решением).
Тогда $c(t) = C t^{-n/2} = \frac{C}{(\sqrt{t})^n}$.
Уравнение для $\tilde{f}$: $-\frac{n}{2}\tilde{f} - \frac{1}{2} (\zeta \cdot \grad_\zeta \tilde{f}) - a^2 \Delta_\zeta \tilde{f} = 0$.
Если $\tilde{f}(\zeta) = \prod_{i=1}^n h(\zeta_i)$, то для каждой $h(\zeta_i)$:
$-\frac{1}{2}h(\zeta_i) - \frac{1}{2}\zeta_i h'(\zeta_i) - a^2 h''(\zeta_i) = 0$.
$(\frac{1}{2} \zeta_i h(\zeta_i) + a^2 h'(\zeta_i))' = \frac{1}{2}h(\zeta_i) + \frac{1}{2}\zeta_i h'(\zeta_i) + a^2 h''(\zeta_i) = 0$.
Интегрируя: $\frac{1}{2} \zeta_i h(\zeta_i) + a^2 h'(\zeta_i) = C_{1,i}$.
Если $C_{1,i}=0$ (условие убывания на бесконечности для $h$ и $h'$ ), то $a^2 h'(\zeta_i) = -\frac{1}{2} \zeta_i h(\zeta_i)$.
$\frac{h'}{h} = -\frac{\zeta_i}{2a^2} \implies h(\zeta_i) = C_{2,i} \exp\left(-\frac{\zeta_i^2}{4a^2}\right)$.
Тогда $u(x,t) = \frac{C}{(\sqrt{t})^n} \prod_{i=1}^n \exp\left(-\frac{x_i^2}{4a^2t}\right) = \frac{C}{(\sqrt{t})^n} \exp\left(-\frac{|x|^2}{4a^2t}\right)$.
Из нормировки $\int_{\R^n} u(x,t)dx = 1$:
$C \int_{\R^n} \frac{1}{(\sqrt{t})^n} \exp\left(-\frac{|x|^2}{4a^2t}\right) dx = C \int_{\R^n} \exp\left(-\frac{|y|^2}{4a^2}\right) dy = 1$, где $y = x/\sqrt{t}$.
$\int_{-\infty}^\infty e^{-z^2/(4a^2)} dz = 2a\sqrt{\pi}$.
$C (2a\sqrt{\pi})^n = 1 \implies C = \frac{1}{(2a\sqrt{\pi})^n}$.
$u(x,t) = \frac{1}{(4\pi a^2 t)^{n/2}} \exp\left(-\frac{|x|^2}{4a^2t}\right)$.
Это \textbf{фундаментальное решение} (функция источника) $G(x,t)$ для уравнения теплопроводности.






\subsection*{Свойства функции источника $G(x,t)$}
\begin{enumerate}
    \item Функция $G(x,t) \in C^\infty(\R^n_x \times (0, \infty))$ и удовлетворяет уравнению (4) (т.е. $G_t - a^2 \Delta_x G = 0$) в $\R^n_x \times (0, \infty)$.
    \item При $t>0$, $\int_{\R^n} G(x,t) dx = 1$.
    \item $\forall \delta > 0$, $\lim_{t \to 0^+} \int_{|x| \ge \delta} G(x,t) dx = 0$.
    % Рисунок: графики G(x,t) для t1 > 0 и t2 > t1, показывающие концентрацию около x=0 при t -> 0. Ось x - x, ось y - G.
    % Пик выше и уже для меньшего t.
\end{enumerate}
\textbf{Доказательство.}
\begin{enumerate}
    \item[1), 2)] следуют из построения функции и её вида.
    \item[3)] Фиксируем $\delta > 0$. $\int_{|x| \ge \delta} G(x,t) dx = \frac{1}{(4\pi a^2 t)^{n/2}} \int_{|x| \ge \delta} \exp\left(-\frac{|x|^2}{4a^2 t}\right)dx$.
    Сделаем замену $y = x/(2a\sqrt{t})$, $dx = (2a\sqrt{t})^n dy$.
    Тогда интеграл равен $\frac{1}{(\sqrt{\pi})^n} \int_{|y| \ge \delta/(2a\sqrt{t})} e^{-|y|^2} dy$.
    При $t \to 0^+$, $\delta/(2a\sqrt{t}) \to \infty$. Так как $\int_{\R^n} e^{-|y|^2} dy = (\sqrt{\pi})^n < \infty$, хвосты интеграла $\int_{|y| \ge R} e^{-|y|^2} dy \to 0$ при $R \to \infty$.
    Следовательно, $\lim_{t \to 0^+} \int_{|x| \ge \delta} G(x,t) dx = 0$. (На рисунке помечено: "переход следует из предельной теоремы о сходимости").
\end{enumerate}

\begin{remark}[Замечание 1]
Функция $G(x,t)$ (переменная $x$, параметр $t$) при $t \to 0^+$ образует $\delta$-образную последовательность, то есть в $\D'(\R^n_x)$ имеет место $G(x,t) \xrightarrow{t \to 0^+} \delta(x)$.
\end{remark}

\begin{remark}[Замечание 2]
Из свойств функции источника $G(x,t)$ -- распределение тепла в среде, потому что $G(x,t)$ -- решение задачи Коши для уравнения теплопроводности с начальным условием $\delta(x)$ (распределение тепла от точечного источника в $x=0$ в момент $t=0$).
\end{remark}

\textbf{(Интеграл Пуассона) -- Формула Пуассона}
Для задачи Коши:
\[ u_t(x,t) = a^2 \Delta_x u(x,t), \quad (x,t) \in G = \R^n_x \times (0, \infty) \quad (6) \]
\[ u(x,0) = u_0(x) \quad (7) \]
По физическому смыслу (принцип суперпозиции для линейного уравнения):
\[ u(x,t) = \int_{\R^n_\xi} G(x-\xi, t) u_0(\xi) d\xi = (G(\cdot, t) * u_0)(x) \quad (*) \]

\begin{theorem}
Пусть $u_0(x)$ непрерывна и ограничена в $\R^n$. Тогда решение задачи Коши (6),(7) в классе $C^{2,1}_{x,t}(G) \cap C(\bar{G})$ (т.е. дважды непрерывно дифференцируемо по $x$ и один раз по $t$ в $G$, и непрерывно в замыкании $\bar{G}$) единственно и дается формулой $(*)$:
\[ u(x,t) = \frac{1}{(4\pi a^2 t)^{n/2}} \int_{\R^n_\xi} \exp\left(-\frac{|x-\xi|^2}{4a^2 t}\right) u_0(\xi) d\xi \]
Более того, если $u_0(x)$ равномерно непрерывна на $\R^n$, то $u(x,t) \to u_0(x_0)$ при $(x,t) \to (x_0, 0)$ равномерно по $x_0$ в $\R^n$.
(В оригинале $(G(\cdot,\cdot)* u_0)(x) = \dots$ и $B(\bar{G})$ -- класс ограниченных функций в $\bar{G}$).
\end{theorem}

\textbf{Доказательство.}
\begin{enumerate}
    \item $G(x,t) \in C^\infty(\R^n_x \times (0,\infty))$. Для любых мультииндексов $\alpha = (\alpha_1, \dots, \alpha_n)$ и $k \in \N_0$ производная $D_x^\alpha D_t^k G(x-\xi, t)$ существует и непрерывна при $t>0$.
    $D_x^\alpha D_t^k G(x-\xi,t) = P(x-\xi, 1/\sqrt{t}, \dots) \exp\left(-\frac{|x-\xi|^2}{4a^2 t}\right)$, где $P$ -- многочлен от своих аргументов.
    Так как $u_0(\xi)$ ограничена, $|u_0(\xi)| \le M = \sup_{\R^n} |u_0|$.
    Рассмотрим $|D_x^\alpha D_t^k G(x-\xi,t) u_0(\xi)|$.
    Для любого $(x_0, t_0)$ с $t_0 > 0$, в окрестности $B_R(x_0) \times (t_0/2, 2t_0)$, подынтегральное выражение в $(*)$ и его производные по $x,t$ мажорируются интегрируемой функцией от $\xi$ (например, $M \cdot (\text{полином от } \xi) \cdot \exp(-k|\xi|^2)$). Мажорантный признак Вейерштрасса для несобственных интегралов, зависящих от параметров, обеспечивает равномерную сходимость интеграла и возможность дифференцирования под знаком интеграла.
    Следовательно, $u(x,t)$ из $(*)$ удовлетворяет уравнению (6) при $t>0$. (В оригинале: "$\int_{\R^n} D^\alpha_{(x,t)} (G(x-\xi,t) u_0(\xi)) d\xi$ сход. равн. в цилиндре $B_R(x_0) \times (t_0/2, 2t_0)$").
    \item Т.к. $G(x,t)$ удовлетворяет (6) и (8) (вероятно, свойство $\int G dx = 1$), то $u(x,t)$ из формулы $(*)$ удовлетворяет (6) при $t>0$.
    \item Докажем непрерывность $(*)$ в $t=0$. Нужно показать $\lim_{(x,t) \to (x_0, 0^+)} u(x,t) = u_0(x_0)$.
    Из свойства 2) функции источника $\int_{\R^n} G(x-\xi, t) d\xi = 1$.
    \begin{align*} |u(x,t) - u_0(x_0)| &= \left| \int_{\R^n} G(x-\xi, t) u_0(\xi) d\xi - u_0(x_0) \int_{\R^n} G(x-\xi,t)d\xi \right| \\ &= \left| \int_{\R^n} G(x-\xi, t) (u_0(\xi) - u_0(x_0)) d\xi \right| \\ &\le \int_{\R^n} G(x-\xi,t) |u_0(\xi) - u_0(x_0)| d\xi \end{align*}
    Так как $u_0$ непрерывна в $x_0$, $\forall \varepsilon > 0 \quad \exists \delta_0 > 0 : |\xi - x_0| < \delta_0 \implies |u_0(\xi) - u_0(x_0)| < \varepsilon/2$.
    \begin{align*} |u(x,t) - u_0(x_0)| &\le \int_{|\xi-x_0|<\delta_0} G(x-\xi,t) |u_0(\xi) - u_0(x_0)| d\xi \\ &+ \int_{|\xi-x_0|\ge\delta_0} G(x-\xi,t) |u_0(\xi) - u_0(x_0)| d\xi \\ &\le \frac{\varepsilon}{2} \int_{|\xi-x_0|<\delta_0} G(x-\xi,t) d\xi + 2M \int_{|\xi-x_0|\ge\delta_0} G(x-\xi,t) d\xi \\ &\le \frac{\varepsilon}{2} \int_{\R^n} G(x-\xi,t) d\xi + 2M \int_{|\eta - (x-x_0)| \ge \delta_0'} G(\eta,t) d\eta \quad (\text{где } \eta = x-\xi) \end{align*}
    Первое слагаемое $\le \varepsilon/2$.
    Если $|x-x_0| < \delta_0/2$, то $|\xi-x_0| \ge \delta_0 \implies |x-\xi| \ge |\xi-x_0| - |x-x_0| > \delta_0 - \delta_0/2 = \delta_0/2 =: \delta_1$.
    Тогда второе слагаемое: $2M \int_{|x-\xi| \ge \delta_1} G(x-\xi,t) d\xi = 2M \int_{|y| \ge \delta_1} G(y,t) dy$.
    Из свойства 3) функции источника, $\forall \varepsilon > 0 \quad \exists t_0 > 0$ такое, что при $0 < t < t_0$, $\int_{|y| \ge \delta_1} G(y,t)dy < \varepsilon/(4M)$.
    Тогда $|u(x,t)-u_0(x_0)| < \varepsilon/2 + 2M \cdot \varepsilon/(4M) = \varepsilon$, если $|x-x_0|<\delta_0/2$ и $0<t<t_0$.
    Это доказывает, что $u(x,t) \to u_0(x_0)$ при $(x,t) \to (x_0,0)$.
    \item Ограниченность: $|u(x,t)| \le \int_{\R^n} G(x-\xi,t) |u_0(\xi)| d\xi \le M \int_{\R^n} G(x-\xi,t) d\xi = M$.
    \item Единственность и непрерывная зависимость от начальных данных следуют из принципа максимума в неограниченной области (для уравнения теплопроводности).
\end{enumerate}






\subsection*{Регуляризация обобщенных функций}
Если $f$ -- обобщенная функция, $\psi$ -- основная функция (т.е. $\psi \in \D(\R^n)$, имеет ограниченный носитель, финитна), то свертка $f * \psi$ существует (см. 2 билета дальше, т.е. это будет определено позже).

\textbf{Докажем, что} $(f * \psi)(x) = \langle f(y), \psi(x-y) \rangle_y \in C^\infty(\R^n_x)$.
Здесь $\langle f(y), \psi(x-y) \rangle_y$ означает действие функционала $f$ (зависящего от переменной $y$) на основную функцию $\tau_x\check{\psi}(y) = \psi(x-y)$ (которая также является основной функцией от $y$ для каждого фиксированного $x$).
То, что $f * \psi \in C^\infty(\R^n_x)$, является стандартным свойством свертки (если $f \in \D'$, $\psi \in \D$, то $f*\psi \in C^\infty$).

\vspace{\baselineskip}
\noindent
\textbf{Вспомогательное утверждение (определение свертки через действие на пробную функцию):}
Для $f \in \D'(\R^n)$, $\psi \in \D(\R^n)$, $\varphi \in \D(\R^n)$,
\[ \langle f * \psi, \varphi \rangle_x = \langle f(y), \langle \psi(x-y), \varphi(x) \rangle_x \rangle_y \]
(Это определение свертки обобщенной функции с основной функцией).
$\langle \psi(x-y), \varphi(x) \rangle_x = \int_{\R^n_x} \psi(x-y)\varphi(x)dx = \int_{\R^n_\xi} \psi(\xi)\varphi(y+\xi)d\xi = (\check{\psi_0} * \varphi)(y)$, где $\check{\psi_0}(\xi) = \psi(-\xi)$.
Эту внутреннюю часть можно записать как $(\psi * \check{\varphi})(-y)$ или $(\check{\psi} * \varphi)(y)$.
Более стандартное определение: $\langle f * \psi, \varphi \rangle_x = \langle f(y), (\check{\psi} * \varphi)(y) \rangle_y$.

Либо, если $f$ -- регулярная обобщенная функция (т.е. $f \in L^1_{loc}$), то
\begin{align*} \langle f * \psi, \varphi \rangle_x &= \int_{\R^n_x} \left( \int_{\R^n_y} f(y) \psi(x-y) dy \right) \varphi(x) dx \\ &= \int_{\R^n_y} f(y) \left( \int_{\R^n_x} \psi(x-y) \varphi(x) dx \right) dy \\ &= \left\langle f(y), \int_{\R^n_x} \psi(x-y)\varphi(x)dx \right\rangle_y = \langle f(y), (\check{\psi} * \varphi)(y) \rangle_y \end{align*}
Это выражение $\int_{\R^n_x} \psi(x-y)\varphi(x)dx$ является $C^\infty$ функцией от $y$ с компактным носителем (т.к. $\psi, \varphi \in \D$), т.е. принадлежит $\D(\R^n_y)$.
(В оригинале некорректная запись: $(f*\psi, \varphi) = (f(y)\cdot \psi(y), \eta(\xi)\varphi(y+\xi)) \dots$).
Корректная запись для действия свертки:
\[ \langle f * \psi, \varphi \rangle_x = \langle f(y), \int_{\R^n_x} \psi(x-y) \varphi(x) dx \rangle_y \]
Заметим, что $\Phi(y) := \int_{\R^n_x} \varphi(x) \psi(x-y) dx = (\varphi * \check{\psi})(y) \in \D(\R^n_y)$.
Тогда $\langle f * \psi, \varphi \rangle_x = \langle f(y), \Phi(y) \rangle_y$.
Это означает, что $f*\psi$ является обобщенной функцией.
(В оригинале: $( (f*\psi), \varphi) = \int \varphi(x) (f(y), \psi(x-y))_y dx = ((f(y), \psi(x-y))_y, \varphi(x))_x$, $\varphi \in \D$).

\vspace{\baselineskip}
Пусть $\omega_\varepsilon(x)$ -- "шапочка" (ядро усреднения, моллификатор, см. Лекцию 11.1, 11.2). Тогда
\[ f_\varepsilon(x) = (f * \omega_\varepsilon)(x) = \langle f(y), \omega_\varepsilon(x-y) \rangle_y \]
называется \textbf{регуляризацией обобщенной функции} $f$.
Функция $f_\varepsilon(x) \in C^\infty(\R^n_x)$.
Также известно, что $f_\varepsilon \xrightarrow{\varepsilon \to 0} f$ в смысле сходимости в $\D'(\R^n)$.



\section*{Умножение}
$a(x) \in C^\infty(\mathbb{R}^n)$ и $f \in \mathcal{D}'$
$$ (a \cdot f, \varphi) = (f, a\varphi), \quad \forall \varphi \in \mathcal{D} $$

\section*{Замена переменных}
$f \in \mathcal{D}', \quad x = Ay + B, \quad \det A \neq 0$
$$ (f(Ay+B), \varphi) = \left(f, \frac{\varphi(A^{-1}(x-B))}{|\det A|}\right), \quad \forall \varphi \in \mathcal{D} $$

\section*{Дифференцирование}
$$ (\mathcal{D}^\alpha f, \varphi) = (-1)^{|\alpha|} (f, \mathcal{D}^\alpha \varphi) \quad \text{--- множится и непр. всюду, как комп. непр. отн.)} $$
Примеры для регулярных обобщенных функций:
$$ (f', \varphi) = \int_{-\infty}^{+\infty} f'(x) \varphi(x) dx $$
$$ (f^{(k)}, \varphi) = \int_{-\infty}^{+\infty} f^{(k)}(x) \varphi(x) dx $$
$$ (\mathcal{D}^\alpha f, \varphi) = \int_{\mathbb{R}^n} (\mathcal{D}^\alpha f)(x) \varphi(x) dx $$

\section*{Свойства дифференцирования в $\mathcal{D}'$}
\begin{enumerate}
    \item Любая обобщенная функция в $\mathcal{D}'$ бесконечно дифференцируема.
    \item Результат дифференцирования в $\mathcal{D}'$ не зависит от порядка дифференцирования 
    ($\rightarrow$ непосредственно из гладкости основных функций).
    \item Операция дифференцирования $\mathcal{D}^\alpha$ на $\mathcal{D}'$ есть линейное и непрерывное отображение из $\mathcal{D}'$ в $\mathcal{D}'$.
    \item Если $f = \sum_{k=1}^{\infty} f_k$ в $\mathcal{D}'$, то $\forall \alpha$
    $$ \mathcal{D}^\alpha f = \sum_{k=1}^{\infty} \mathcal{D}^\alpha f_k \quad \text{в } \mathcal{D}' $$
\end{enumerate}
$\blacksquare$ Если $f_k \xrightarrow{k \to \infty} f$ в $\mathcal{D}'$, то для $\forall \varphi \in \mathcal{D}$ и $\forall \alpha$:
$$ |(\mathcal{D}^\alpha f_k, \varphi) - (\mathcal{D}^\alpha f, \varphi)| = |(f_k, \mathcal{D}^\alpha \varphi) - (f, \mathcal{D}^\alpha \varphi)| \xrightarrow{k \to \infty} 0 $$
так как $\mathcal{D}^\alpha \varphi \in \mathcal{D} \Rightarrow \mathcal{D}^\alpha f_k \xrightarrow{k \to \infty} \mathcal{D}^\alpha f$ в $\mathcal{D}'$.

\vspace{1cm}
\hrulefill

По-видимому, согласно (определению), $f \in \mathcal{D}'$ равна нулю в области $G \subset \mathbb{R}^n$, если...





\newpage
\section*{Билет №11 -- 2025}\label{sec:ticket11}
\backtotoc
\textbf{Пространства Шварца S и S'. Свертка функций, ее свойства. [2] – 201-203, 205-207.}


\section*{Л2. Пространства Шварца $S$ и $S'$. Свертка функций, её свойства}
Сходимость в $S$ и $S'$, метризуемость пр-ва $S$. Непр. оператора $\mathcal{D}^\alpha$ в $S$ и $S'$.

\subsection*{Определение}
Обозначим через $S$ линейное пр-во, удовл. след. требованиям:
\begin{enumerate}
    \item Элементами $S$ являются функции из класса $C^\infty(\mathbb{R}^n)$:
    $$ (\varphi(x) \in S) \iff \left\{ \begin{array}{l} \varphi \in C^\infty(\mathbb{R}^n) \\ |x|^p \mathcal{D}^\alpha \varphi(x) \xrightarrow{|x| \to \infty} 0, \quad \forall p \in \mathbb{N} \cup \{0\}, \forall \alpha \end{array} \right. $$
    \item послед. $\{\varphi_k\}_{k=1}^\infty \subset S$ сход. к $\varphi \in S$, если $\forall$ мультииндекс. $\alpha$ и $p$:
    $$ \{x^p \mathcal{D}^\alpha \varphi_k(x)\}_{k=1}^\infty \text{ сход. к } x^p \mathcal{D}^\alpha \varphi(x) \quad (\text{равномерно по } x \in \mathbb{R}^n) $$
\end{enumerate}
$\bullet \quad \mathcal{D} \subset S$, но $\mathcal{D} \neq S$. Например $\varphi(x) = \exp(-|x|^2/2) \in S$ и $\notin \mathcal{D}$. (\textit{Примечание: в оригинале $\exp(-|x|/2)$}).

\subsection*{Определение}
Функция $a(x)$ называется ф-ей медленного роста, если $\exists P>0, C>0$ ($P>0$):
$$ |a(x)| \le C|x|^P \quad \text{при } |x| > R $$

\subsection*{Для пространства $S$ верно:}
\begin{enumerate}
    \item Оператор $\mathcal{D}^\beta$ это умножение. $\beta$ --- непр. лин. от. $S \to S$. (\textit{Примечание: "умножение" здесь, вероятно, ошибка в конспекте, обычно $\mathcal{D}^\beta$ --- оператор дифференцирования.})
    
    $\blacksquare$ $\varphi \in S \implies \mathcal{D}^\beta \varphi \in S$, т.к. $\forall p \ge 0$ и $\forall \alpha$, если $\varphi \in S$, то $|x|^p \mathcal{D}^\alpha (\mathcal{D}^\beta \varphi(x)) = |x|^p \mathcal{D}^{\alpha+\beta} \varphi(x) \xrightarrow{|x|\to\infty} 0$.
    Пусть $\{\varphi_k\}_{k=1}^\infty \subset S$ сход. к $\varphi \in S$. Тогда $|x|^p \mathcal{D}^\alpha (\mathcal{D}^\beta \varphi_k(x)) = |x|^p \dot{\mathcal{D}}^{\alpha+\beta} \varphi_k(x)$ сход. равномерно к $|x|^p \mathcal{D}^{\alpha+\beta} \varphi(x)$. УР: $\alpha \implies \mathcal{D}^\beta \varphi_k \xrightarrow{k \to \infty} \mathcal{D}^\beta \varphi$ в $S$. $\blacksquare$

    \item Операция умножения на функцию $a(x) \in C^\infty(\mathbb{R}^n)$, каждая производная которой является функцией медленного роста, есть линейное непр. отображ. $S \to S$.
    
    $\blacksquare$ $\forall$ мультиинд. $\alpha$. $\mathcal{D}^\alpha a(x)$ -- ф-я медл. роста, то $\exists R_\alpha > 0, C_\alpha > 0, p_\alpha > 0$: $|\mathcal{D}^\alpha a(x)| \le C_\alpha |x|^{p_\alpha}$ при $|x| > R_\alpha$. $\implies$ если $\varphi \in S$, то и $a\varphi \in S$, и если $\varphi_k \xrightarrow{k \to \infty} \varphi$ в $S$, то $\forall \alpha, \beta$ $x^\beta \mathcal{D}^\alpha [a(x) (\varphi_k(x) - \varphi(x))] \xrightarrow{k \to \infty} 0$, то есть $a\varphi_k \xrightarrow{k \to \infty} a\varphi$ в $S$. $\blacksquare$

    \item Невырожденная линейная замена переменных $x = Ay+b$ есть линейное непр. отобр. $S \to S$.
    
    $\blacksquare$ $|y|^p \mathcal{D}_y^\alpha \varphi(Ay+b) = |A^{-1}(x-b)|^p \sum_{|\beta|=|\alpha|}' C_\beta \mathcal{D}_x^\beta \varphi(x) \le C(|x|+|B|) \sum_{|\beta|=|\alpha|} C_\beta \mathcal{D}_x^\beta \varphi(x)$, где $C_\beta$ -- пост. коэф. $C \sim \max(\lambda(A^{-1}))$. (\textit{Примечание: $C = \max(\lambda(A))$ собств. знач. в оригинале, что не совсем корректно для оценки $|A^{-1}(x-b)|$. Возможно, речь шла о норме матрицы $A^{-1}$.}) $\blacksquare$

    \item Каждая ф-я из $S$ -- абсол. интегр. в $\mathbb{R}^n$.
    
    $\blacksquare$ $\varphi \in S$, то из опр. $|x|^{n+1} |\varphi(x)| \xrightarrow{|x|\to\infty} 0 \implies \exists R>0: |x|^{n+1} |\varphi(x)| \le 1$ при $|x| \ge R$.
    $$ \int_{\mathbb{R}^n} |\varphi(x)|dx = \int_{|x|<R} |\varphi(x)|dx + \int_{|x|>R} |\varphi(x)|dx \le V_R \sup_{|x|<R} |\varphi(x)| + \int_{|x|>R} \frac{1}{|x|^{n+1}} dx < \infty, $$
    где $V_R$ -- объем $n$-мерн. области $\{x: |x|<R\}$. (\textit{Примечание: в оригинале "с поверх. R", вероятно, имелся в виду радиус $R$.}) $\blacksquare$

    \item Преобразование Фурье $F$ и обратное преобр. $F^{-1}$ непрерывно и взаимно однознач. линейное отобр. $S$ на себя.
    
    $\blacksquare$ Линейность -- свойство ПФ. Пусть $\varphi(x) \in S(\mathbb{R})$. $\forall p, q \ge 0$.
    \begin{align*} |y^p (F\varphi)^{(q)}(y)| &= |y^p F( (ix)^q \varphi(x) )(y)| = |F( (-i D_x)^p ((ix)^q \varphi(x)) )(y)| \\ &= \left|\int_{-\infty}^{\infty} (-i D_x)^p ((ix)^q \varphi(x)) e^{ixy} dx\right| \le \int_{-\infty}^{\infty} |(D_x)^p (x^q \varphi(x))| dx \\ &= \int_{-\infty}^\infty \left|\sum_{l,m} C_{lm} x^l \varphi^{(m)}(x)\right| dx \le \sum_{l,m} |C_{lm}| \int_{-\infty}^\infty |x^l \varphi^{(m)}(x)| dx, \end{align*}
    где $C_{lm}=$const ($l=0, \dots, q$ и $m=0, \dots, p$). Т.к. $\varphi(x) \in S$, то каждая из функций $x^l \varphi^{(m)}(x)$ интегрируема.
    $\implies |y^p (F\varphi)^{(q)}(y)| \le C$, откуда $|y|^p (F\varphi)^{(q)}(y) \xrightarrow{|y|\to\infty} 0$, т.е. $(F\varphi)(y) \in S$.
    Аналогично $(F^{-1}\varphi)(y) \in S$.
    
    $F^{-1}(F\varphi) = \varphi$, $F(F^{-1}\varphi) = \varphi$ (взаимнооднознач). $\psi = F^{-1} \varphi \in S$. $\varphi$ -- образ $\psi$ при ПФ $F$, т.е. $F: S \to S$.
    $F$ -- линейно: $F(\varphi_1 - \varphi_2) = F\varphi_1 - F\varphi_2$. $0 = F^{-1}(\psi_1 - \psi_2) = F^{-1}F(\varphi_1 - \varphi_2) = \varphi_1 - \varphi_2 \implies \psi_1 = \psi_2$.
    Пусть $\{\varphi_k(x)\}_{k=1}^\infty \subset S(\mathbb{R})$ сход. к $0$ в $S(\mathbb{R})$: \textit{(текст неразборчив, возможно "ряд послед. геометрич. числа.")}
    $$ |y^p (F\varphi_k)^{(q)}(y)| \le C_k, \quad C_k = \sum_{l,m} |C_{lm}| \int_{-\infty}^\infty |x^l (\varphi_k)^{(m)}(x)| dx $$
    $$ C_k \le \sum_{l,m} |C_{lm}| \sup_x |(1+x^2) x^l \varphi_k^{(m)}(x)| \cdot \int_{-\infty}^{+\infty} \frac{1}{1+x^2} dx $$
    $\sup_x |(1+x^2) x^l \varphi_k^{(m)}(x)| \xrightarrow{k\to\infty} 0$, то $\sup_y |y^{p+1} (F\varphi_k)^{(q)}(y)| \xrightarrow{k\to\infty} 0$.
    Что означает сходимость послед. $\{(F\varphi_k)(y)\}_{k=1}^\infty$ к нулю в $S$. $\blacksquare$

\end{enumerate}


\section*{Определение}
Пространством обобщенных ф-ий медленного роста $S'$ называется множество непрерывных функционалов на $S_d$ (введенной поточечной сходимостью?).
$$ (f_k \xrightarrow{k \to \infty} f \in S') \iff ((f_k, \varphi) \xrightarrow{k \to \infty} (f, \varphi), \quad \forall \varphi \in S) $$
$S'$ -- линейное.

\section*{Теорема (без доказательства)}
Пусть послед. $\{f_k\}_{k=1}^\infty \subset S'$ явл. функцион. В смысле поточ. сходимости: для $\forall \varphi \in S$ числ. посл. $\{(f_k, \varphi)\}_{k=1}^\infty$ явл. фундаментальной (или сходящейся).
Тогда $(f, \varphi) = \lim_{k \to \infty} (f_k, \varphi)$, $\forall \varphi \in S$ явл. $f$ -- лин. и непр. то есть $f \in S'$.

\section*{Пример}
$f(x)$ -- лок. инт. ф-я медл. роста определяет обобщ. $f \in S'$
$$ (f, \varphi) = \int_{\mathbb{R}^n} f(x) \varphi(x) dx, \quad \forall \varphi \in S $$

\section*{Операции:}
\begin{itemize}
    \item $(\mathcal{D}^\alpha f, \varphi) = (-1)^{|\alpha|} (f, \mathcal{D}^\alpha \varphi), \quad \forall \varphi \in S'$ (\textit{Примечание: Должно быть $\forall \varphi \in S$})
    \item $a(x) \in C^\infty(\mathbb{R}^n)$, каждая производная -- ф-я медл. роста
    $$ (a \cdot f, \varphi) = (f, a\varphi), \quad \forall \varphi \in S $$
    $x = Ay+B, \quad \det A \neq 0$;
    $$ (f(Ay+B), \varphi) = \left(f, \frac{\varphi(A^{-1}(x-B))}{|\det A|}\right), \quad \forall \varphi \in S $$
\end{itemize}

\hrulefill
\section*{Операции в $\mathcal{D}'$ -- ликбез к билету 15}
\subsection*{Умножение} $a(x) \in C^\infty(\mathbb{R}^n)$ и $f \in \mathcal{D}'$
$$ (a \cdot f, \varphi) = (f, a\varphi), \quad \forall \varphi \in \mathcal{D} $$
\subsection*{Замена переменных} $f \in \mathcal{D}', \quad x = Ay+B, \quad \det A \neq 0$
$$ (f(Ay+B), \varphi) = \left(f, \frac{\varphi(A^{-1}(x-B))}{|\det A|}\right), \quad \forall \varphi \in \mathcal{D} $$
\subsection*{Дифференцирование}
$$ (\mathcal{D}^\alpha f, \varphi) = (-1)^{|\alpha|} (f, \mathcal{D}^\alpha \varphi) \quad \text{--- множится и непр. всюду, как комп. непр. отн.)} $$







\newpage
\section*{Билет №12 -- 2025}\label{sec:ticket12}
\backtotoc
\textbf{Преобразование Фурье и его свойства. Фундаментальное решение оператора с постоянными
коэффициентами помогает найти частное решение уравнения. [2] – 200-201, 204-205, 207-208,
222-224.}




\setlength{\parindent}{0pt} % Отключение красной строки
\setlength{\parskip}{6pt plus 2pt minus 1pt} % Абзацный отступ

\newcommand{\Fourier}{\mathcal{F}} % Можно использовать F, если так привычнее по тексту



\hfill \textbf{13.1}

\section*{Преобразование Фурье и его свойства. Свертка функций, её свойства. Фундаментальное решение оператора с постоянными коэффициентами.}

\subsection*{Определение}
Отображение $F$, ставящее функции $f(x)$, $x \in \R^n$ функцию, обозначаемую $(Ff)(y)$, $y \in \R^n$ или $\hat{f}(y)$ и определяемую
\[
(Ff)(y) = \hat{f}(y) = \text{v.p.} \int_{\R^n} f(x) e^{i yx} dx,
\]
называют прямым преобразованием Фурье, а
\[
(F^{-1}f)(y) = \check{f}(y) = \frac{1}{(2\pi)^n} \text{v.p.} \int_{\R^n} f(x) e^{-i yx} dx
\]
--- обратным ПФ (преобразованием Фурье).

Определено в частности для всех абсолютно интегрируемых функций в $\R^n$.

\subsection*{Свойства ПФ}
\begin{enumerate}
    \item Если $f(x) \in C^1(\R^n)$ и абсолютно интегрируема, то
    \[ F^{-1}(Ff) = F(F^{-1}f) = f \]
    (Примечание: в оригинале было $F^{-1}(Ff) = f(F^{-1}f)=f$, что вероятно является опечаткой и исправлено на стандартное свойство.)

    \item Если $f(x)$ абсолютно интегрируема в $\R^n$, то
    \begin{align*}
        Ff &= (2\pi)^n F^{-1}(f(-x)) \\
        F^{-1}f &= \frac{1}{(2\pi)^n} F(f(-x))
    \end{align*}

    \item Умножение. $\alpha$ справедливо: если $f(x) \in C^{|\alpha|}(\R^n)$, и все частные производные вплоть до порядка $|\alpha|$ включительно абсолютно интегрируемы в $\R^n$, то
    \begin{align*}
        (FD^\alpha f)(y) &= (-i)^{|\alpha|} y^\alpha (Ff)(y) \\
        (F^{-1}D^\alpha f)(y) &= i^{|\alpha|} y^\alpha (F^{-1}f)(y), \quad \text{где } y^\alpha = \prod_{k=1}^{n} (y_k)^{\alpha_k}
    \end{align*}
    (Примечание: в оригинале $y^\alpha = \prod_{k=1}^{n} (y^k)^{\alpha^k}$, что, вероятно, означает $(y_k)^{\alpha_k}$ или $(y^k)_{\alpha_k}$, где $y_k$ - $k$-ая компонента $y$, и $\alpha_k$ - $k$-ая компонента мультииндекса $\alpha$. Запись $(y^k)^{\alpha_k}$ также возможна.)

    \item Умножение на $x^\alpha$: если $f(x) \in C(\R^n)$ и функции $f(x), \dots, |x|^{|\alpha|}f(x)$ абсолютно интегрируемы в $\mathbb{R}^n$, то $\exists D^\alpha(Ff)$ и $D^\alpha(F^{-1}f)$, причем
    \begin{align*}
        D^\alpha(Ff) &= (i)^{|\alpha|} F(x^\alpha f(x)) \\
        D^\alpha(F^{-1}f) &= (-i)^{|\alpha|} F^{-1}(x^\alpha f(x))
    \end{align*}

    \item Прямое и обратное ПФ являются линейными отображениями множеств функций, на которых они определены.

    \item Если функция $f(x)$ абсолютно интегрируема на $\R$, то
    \[ (Ff(x-x_0))(y) = e^{ix_0y}(Ff)(y). \]
    (Примечание: $x_0y$ в экспоненте обычно означает скалярное произведение $x_0 \cdot y$ для $x_0, y \in \R^n$.)
\end{enumerate}

\textbf{Для пространства $\mathcal{S}$:}
\begin{itemize}
    \item[\textbullet] ПФ $F$ и $F^{-1}$ непрерывны и взаимно однозначно отображают $\mathcal{S}$ на себя.
    \item[\textbullet] Линейность $\implies$ из п.5.
\end{itemize}

Далее

см. билет 12. стр. 1. св-во 5.

\vspace{1cm}
\hfill \textbf{13.1.}


\setlength{\parindent}{0pt} % Отключение красной строки
\setlength{\parskip}{6pt plus 2pt minus 1pt} % Абзацный отступ

\newcommand{\Schwartz}{\mathcal{S}}
\newcommand{\Distr}{\mathcal{S}'}



\hfill \textbf{13.2}

\subsection*{Преобразование Фурье обобщенных функций}
Для $f \in \Distr(\R^n)$ (пространство обобщенных функций медленного роста, или умеренных распределений) ее преобразование Фурье $\Fourier f$ и обратное преобразование Фурье $\Fourier^{-1}f$ называются функционалами на $\Schwartz(\R^n)$, действие которых на произвольную функцию $\varphi \in \Schwartz(\R^n)$ определяется как:
\begin{align*}
(\Fourier f, \varphi) &= (f, \Fourier\varphi) \\
(\Fourier^{-1}f, \varphi) &= (f, \Fourier^{-1}\varphi)
\end{align*}
Здесь $(g, \psi)$ обозначает действие функционала $g \in \Distr$ на тестовую функцию $\psi \in \Schwartz$.
Известно, что $\Fourier$ и $\Fourier^{-1}$ являются линейными и непрерывными отображениями $\Schwartz \to \Schwartz$. Тогда, если $f \in \Distr$, то $\Fourier f$ и $\Fourier^{-1}f$ корректно определены как элементы $\Distr$.

Пусть $f(x)$ --- локально абсолютно интегрируемая функция медленного роста. Тогда $(f, \varphi)_x = \int_{\R^n} f(x) \varphi(x) dx, \forall \varphi \in \Schwartz$.
Формальное обоснование определения $(\Fourier f, \varphi) = (f, \Fourier\varphi)$:
\begin{align*}
(\Fourier f(x))(y) \Rightarrow \int_{\R^n} ((\Fourier f(x))(y)) \varphi(y) dy &= \int_{\R^n} \varphi(y) \left[ \int_{\R^n} f(x) e^{iyx} dx \right] dy \\
&= \int_{\R^n} f(x) \left[ \int_{\R^n} \varphi(y) e^{iyx} dy \right] dx \quad \text{(по теореме Фубини)} \\
&= \int_{\R^n} f(x) (\Fourier\varphi(y))(x) dx = (f(x), (\Fourier\varphi(y))(x))_x \\
&= (f, \Fourier\varphi) = (\Fourier f, \varphi).
\end{align*}
(Примечание: в оригинале запись $((\Fourier f(x))(y))$ избыточна, достаточно $(\Fourier f)(y)$).

\subsection*{Свойства ПФ на $\Distr$}
\begin{enumerate}
    \item Преобразования Фурье $\Fourier$ и $\Fourier^{-1}$ являются линейными, непрерывными и взаимно однозначными отображениями пространства $\Distr(\R^n)$ на себя.
    \begin{itemize}
        \item \textbf{Линейность:} То, что $\Fourier f \in \Distr$ и $\Fourier^{-1}f \in \Distr$ для $f \in \Distr$, уже обсуждалось (следует из непрерывности $\Fourier, \Fourier^{-1}$ на $\Schwartz$).
        Для $f_1, f_2 \in \Distr$ и $\lambda_1, \lambda_2 \in \mathbb{C}$:
        \begin{align*}
        (\Fourier(\lambda_1 f_1 + \lambda_2 f_2), \varphi) &= ( (\lambda_1 f_1 + \lambda_2 f_2), \Fourier\varphi) \\
        &= \lambda_1 (f_1, \Fourier\varphi) + \lambda_2 (f_2, \Fourier\varphi) \\
        &= \lambda_1 (\Fourier f_1, \varphi) + \lambda_2 (\Fourier f_2, \varphi) = ( (\lambda_1 \Fourier f_1 + \lambda_2 \Fourier f_2), \varphi).
        \end{align*}
        Следовательно, $\Fourier(\lambda_1 f_1 + \lambda_2 f_2) = \lambda_1 \Fourier f_1 + \lambda_2 \Fourier f_2$. Аналогично для $\Fourier^{-1}$.

        \item \textbf{Взаимная однозначность (обратимость):} Для $f \in \Distr$:
        \begin{align*}
        (\Fourier(\Fourier^{-1}f), \varphi) &= (\Fourier^{-1}f, \Fourier\varphi) = (f, \Fourier^{-1}(\Fourier\varphi)) = (f, \varphi), \quad \forall \varphi \in \Schwartz. \\
        \Rightarrow \Fourier \circ \Fourier^{-1}f &= f.
        \end{align*}
        Аналогично, $(\Fourier^{-1}(\Fourier f), \varphi) = (\Fourier f, \Fourier^{-1}\varphi) = (f, \Fourier(\Fourier^{-1}\varphi)) = (f, \varphi)$, $\forall \varphi \in \Schwartz$.
        $\Rightarrow \Fourier^{-1} \circ \Fourier f = f$.
        Таким образом, $\Fourier$ и $\Fourier^{-1}$ являются биекциями $\Distr \to \Distr$.

        \item \textbf{Непрерывность:} Пусть последовательность $\{f_k\}_{k=1}^\infty \subset \Distr$ сходится к $f \in \Distr$ (т.е. $(f_k, \psi) \to (f, \psi)$ при $k \to \infty$ для любого $\psi \in \Schwartz$). Тогда для любого $\varphi \in \Schwartz$:
        \[ (\Fourier f_k, \varphi) = (f_k, \Fourier\varphi) \xrightarrow{k \to \infty} (f, \Fourier\varphi) = (\Fourier f, \varphi). \]
        Поскольку $\Fourier\varphi \in \Schwartz$ (так как $\Fourier: \Schwartz \to \Schwartz$), это означает, что $\Fourier f_k \to \Fourier f$ в $\Distr$.
        Аналогично для $\Fourier^{-1}$.
    \end{itemize}

    \item \textbf{Основные формулы} (аналогичны свойствам ПФ для функций из $\Schwartz$):
    Для $f \in \Distr$:
        \begin{enumerate}[label=\alph*)]
            \item $\Fourier^{-1}(\Fourier f) = \Fourier(\Fourier^{-1}f) = f$ (уже показано в п.1).
            \item $\Fourier f = (2\pi)^n \Fourier^{-1}(f(-x))$, \quad $\Fourier^{-1}f = \frac{1}{(2\pi)^n} \Fourier(f(-x))$.
            (Здесь $f(-x)$ обозначает обобщенную функцию $g$ такую, что $(g, \varphi(x)) = (f(x), \varphi(-x))$).
            \item $D^\alpha(\Fourier f) = \Fourier(i^{|\alpha|} x^\alpha f)$. \\
            $D^\alpha(\Fourier^{-1}f) = \Fourier^{-1}((-i)^{|\alpha|} x^\alpha f)$.\\
            (Дифференцирование ПФ обобщенной функции $f$ эквивалентно ПФ от произведения $f$ на соответствующий полином).
            \item $(\Fourier D^\alpha f)(y) = (-i)^{|\alpha|} y^\alpha (\Fourier f)(y)$. \\
            $(\Fourier^{-1}D^\alpha f)(y) = i^{|\alpha|} y^\alpha (\Fourier^{-1}f)(y)$.\\
            (ПФ производной $D^\alpha f$ обобщенной функции $f$ эквивалентно умножению ПФ $\Fourier f$ на полином. Формально, $y^\alpha (\Fourier f)(y)$ есть обобщенная функция $T$, такая что $(T, \varphi(y)) = (\Fourier f, y^\alpha \varphi(y))$ ).
            \item $\Fourier(f(x-x_0))(y) = e^{ix_0y} (\Fourier f)(y)$.
            (Здесь $f(x-x_0)$ есть сдвинутая обобщенная функция $g_s$ такая, что $(g_s, \varphi(x)) = (f(x), \varphi(x+x_0))$ ).
        \end{enumerate}
\end{enumerate}

\textbf{Например (доказательство свойства 2в для $D^\alpha(\Fourier f)$):}
Для любой $\varphi \in \Schwartz$:
\begin{align*}
(D^\alpha(\Fourier f), \varphi) &= (-1)^{|\alpha|} (\Fourier f, D^\alpha \varphi) \quad &&\text{(по определению производной обобщенной функции)} \\
&= (-1)^{|\alpha|} (f, \Fourier(D^\alpha \varphi)) \quad &&\text{(по определению ПФ обобщенной функции)} \\
&= (-1)^{|\alpha|} (f, (-ix)^\alpha (\Fourier\varphi)(x)) \quad &&\text{(т.к. } \Fourier(D^\alpha\psi)(x) = (-ix)^\alpha (\Fourier\psi)(x) \text{ для } \psi \in \Schwartz) \\
&= (f, (-1)^{|\alpha|} (-i)^{|\alpha|} x^\alpha (\Fourier\varphi)(x)) \\
&= (f, (i)^{|\alpha|} x^\alpha (\Fourier\varphi)(x)) \\
&= (i^{|\alpha|} x^\alpha f, \Fourier\varphi) \quad &&\text{(по определению умножения обобщенной функции на гладкую функцию)} \\
&= (\Fourier(i^{|\alpha|} x^\alpha f), \varphi) \quad &&\text{(по определению ПФ обобщенной функции)}.
\end{align*}
Поскольку это верно для всех $\varphi \in \Schwartz$, то $D^\alpha(\Fourier f) = \Fourier(i^{|\alpha|} x^\alpha f)$.
Здесь $x^\alpha$ является мультипликатором в $\Schwartz$, поэтому $x^\alpha f \in \Distr$, если $f \in \Distr$ (см. билет №12).



\setlength{\parindent}{0pt} % Отключение красной строки
\setlength{\parskip}{6pt plus 2pt minus 1pt} % Абзацный отступ

\newcommand{\Dprime}{\mathcal{D}'} % Пространство обобщенных функций



\subsection*{Носитель обобщенной функции (продолжение)}
% (N13) могло быть ссылкой на пункт/теорему

\paragraph{Определение.}
Говорят, что обобщенная функция $f \in \Dprime(\R^n)$ равна нулю в открытом множестве $B \subset \R^n$, если для любой функции $\varphi \in \Dcal(\R^n)$ такой, что $\supp \varphi \subset B$, выполняется $(f, \varphi) = 0$.

\paragraph{Лемма.}
Пусть $\{G_\delta\}_{\delta \in I}$ — семейство открытых множеств в $\R^n$, и пусть $G = \bigcup_{\delta \in I} G_\delta$.
Обобщенная функция $f \in \Dprime(\R^n)$ равна нулю в $G$ тогда и только тогда, когда $f$ равна нулю в каждом множестве $G_\delta$ (для всех $\delta \in I$).

\begin{proof}
$(\implies)$ Необходимость. Пусть $f=0$ в $G$. Требуется показать, что $f=0$ в каждом $G_\delta$.
Возьмем произвольное $G_\delta$ из семейства. Пусть $\varphi \in \Dcal(\R^n)$ такова, что $\supp \varphi \subset G_\delta$. Поскольку $G_\delta \subset G = \bigcup_{\alpha \in I} G_\alpha$, то $\supp \varphi \subset G$.
Так как $f=0$ в $G$, по определению $(f, \varphi) = 0$.
Поскольку это верно для любой $\varphi$ с $\supp \varphi \subset G_\delta$, то $f=0$ в $G_\delta$. Это справедливо для любого $\delta \in I$.

$(\impliedby)$ Достаточность. Пусть $f=0$ в каждой области $G_\delta$ для всех $\delta \in I$. Требуется показать, что $f=0$ в $G$.
Рассмотрим произвольную функцию $\varphi \in \Dcal(\R^n)$ такую, что $K = \supp \varphi \subset G$.
Поскольку $K$ — компактное множество и $G = \bigcup_{\delta \in I} G_\delta$ является открытым покрытием $K$, существует конечное подпокрытие $\{G_{\delta_l}\}_{l=1}^L$ такое, что $K \subset \bigcup_{l=1}^L G_{\delta_l}$.
Существует разбиение единицы $\{\psi_l(x)\}_{l=1}^L$, подчиненное покрытию $\{G_{\delta_l}\}_{l=1}^L$ компакта $K$. Это означает, что $\psi_l \in \Dcal(\R^n)$, $\supp \psi_l \subset G_{\delta_l}$ для каждого $l=1, \dots, L$, и $\sum_{l=1}^L \psi_l(x) = 1$ для всех $x \in K$.
Тогда $\varphi(x) = \varphi(x) \cdot 1 = \varphi(x) \sum_{l=1}^L \psi_l(x) = \sum_{l=1}^L \varphi(x)\psi_l(x)$.
Обозначим $\varphi_l(x) = \varphi(x)\psi_l(x)$. Каждая функция $\varphi_l \in \Dcal(\R^n)$, и ее носитель удовлетворяет условию:
$\supp \varphi_l = \supp (\varphi \psi_l) \subset \supp \varphi \cap \supp \psi_l \subset K \cap G_{\delta_l} \subset G_{\delta_l}$.
Тогда, используя линейность функционала $f$:
\[
(f, \varphi) = \left(f, \sum_{l=1}^L \varphi_l\right) = \sum_{l=1}^L (f, \varphi_l).
\]
Поскольку $\supp \varphi_l \subset G_{\delta_l}$ и по предположению $f=0$ в $G_{\delta_l}$, то $(f, \varphi_l)=0$ для каждого $l=1, \dots, L$.
Следовательно,
\[
(f, \varphi) = \sum_{l=1}^L 0 = 0.
\]
Так как это верно для любой $\varphi \in \Dcal(\R^n)$ с $\supp \varphi \subset G$, то $f=0$ в $G$.
\end{proof}

\vfill \hfill \text{гл 12.2} % Предполагая, что это номер главы и страницы



\setlength{\parindent}{0pt} % Отключение красной строки
\setlength{\parskip}{6pt plus 2pt minus 1pt} % Абзацный отступ





\subsection*{Дифференцирование свертки обобщенных функций}

% Продолжение свойств свертки, видимо, из предыдущей части.
% Часть с $\varphi(y+\eta)$ симметрична относительно $y, \eta$.
% $(y, \eta) \in (\supp f \times \supp g) \cap \supp \varphi(y+\eta)$.
% Оба ограничены. И как следствие $\supp (f*g)$ компакт, если $\supp f, \supp g$ -- компакты.
% Это, по-видимому, относится к условиям существования и свойствам носителя свертки.

Для свертки $f*g$, где $f, g \in \Dprime(\R^n)$ (и свертка корректно определена):
Для любой $\varphi \in \Dcal(\R^n)$:
\begin{align*}
(D^\alpha(f*g), \varphi) &= (-1)^{|\alpha|} (f*g, D^\alpha\varphi) \\
&= (-1)^{|\alpha|} \langle f(y), \langle g(\eta), (D^\alpha\varphi)(y+\eta) \rangle_\eta \rangle_y \\
&= \langle f(y), \langle g(\eta), (-1)^{|\alpha|} (D_y^\alpha\varphi)(y+\eta) \rangle_\eta \rangle_y \\
&= \langle f(y), \langle (D_y^\alpha g)(\eta), \varphi(y+\eta) \rangle_\eta \rangle_y \\
&= ((D^\alpha f)*g, \varphi) \quad \text{ (если }(D_y^\alpha\varphi)(y+\eta) \text{ интерпретируется как } (D^\alpha_z \varphi)(z)|_{z=y+\eta} \text{)}
\end{align*}
Или, если дифференцировать по другой переменной:
\[
= ((f*(D^\alpha g)), \varphi)
\]
Таким образом,
\begin{align*}
D^\alpha(f*g) &= (D^\alpha f)*g \\
D^\alpha(f*g) &= f*(D^\alpha g)
\end{align*}
(Примечание: в оригинале написано $D^\alpha(f \circ g) = f \circ (D^\alpha g)$ и $D^\alpha(f \circ g) = (D^\alpha f) \circ g$, где $\circ$ может обозначать свертку. Исправлено на стандартное обозначение $*$ для свертки.)

\subsection*{Лемма 2 (О сходимости свертки)}
Пусть последовательность $\{f_k\}_{k=1}^\infty \subset \Dprime(\R^n)$ сходится к $f \in \Dprime(\R^n)$ (т.е. $(f_k, \psi) \to (f, \psi)$ для всех $\psi \in \Dcal(\R^n)$).
Тогда $f_k * g \xrightarrow{k \to \infty} f * g$ в $\Dprime(\R^n)$, если выполнено одно из следующих условий:
\begin{enumerate}[label=\alph*)]
    \item $f_k$ все имеют носители, содержащиеся в одном и том же ограниченном открытом множестве $M$.
    \item носитель функции $g$ ограничен.
\end{enumerate}

\begin{proof}
Рассмотрим $(f_k * g, \varphi) = \langle f_k(y), \langle g(\eta), \varphi(y+\eta) \rangle_\eta \rangle_y$.
Обозначим $\Phi(y) = \langle g(\eta), \varphi(y+\eta) \rangle_\eta$. Эта функция $\Phi(y)$ является основной функцией ($\Phi \in \Dcal(\R^n)$). (Необходимо доказать, что $\Phi$ гладкая и имеет компактный носитель при соответствующих условиях).
\begin{enumerate}[label=\alph*)]
    \item Если носители всех $f_k$ (и, следовательно, $f$) содержатся в компактном (или ограниченном) множестве $M$, и $\supp g$ также компактен, то $\supp \Phi \subset \supp g - \supp \varphi$ (разность Минковского).
    Если $\supp g$ компактен, то $\Phi(y)$ бесконечно дифференцируема.
    Если $\supp \varphi$ компактен, то $\supp \Phi(y)$ будет компактен, так как $\Phi(y)$ будет равна нулю, если $y + \supp g$ не пересекается с $\supp \varphi$.
    Более точно: $\Phi(y) = (g * \check{\varphi})(y)$, где $\check{\varphi}(\eta) = \varphi(-\eta)$. Свертка обобщенной функции с основной функцией есть $C^\infty$-функция. Ее носитель: $\supp \Phi \subset \supp g + \supp \check{\varphi}$.
    Если $\supp g$ ограничен, то $\Phi \in C^\infty(\R^n)$. Если и $\supp \varphi$ ограничен, то $\supp \Phi$ ограничен.
    Если $g \in \mathcal{E}'(\R^n)$ (обобщенная функция с компактным носителем), то $\Phi \in \Dcal(\R^n)$.
    Тогда $(f_k * g, \varphi) = (f_k, \Phi)$. Поскольку $f_k \to f$ в $\Dprime$, то $(f_k, \Phi) \xrightarrow{k \to \infty} (f, \Phi) = (f*g, \varphi)$.
    Значит, $f_k * g \to f * g$ в $\Dprime$.

    \item Если носитель функции $g$ ограничен, то функция $\Phi(y) = \langle g(\eta), \varphi(y-\eta) \rangle_\eta$ (если переопределить для удобства $\Phi(y) = (g * \varphi)(y)$) является основной функцией (гладкая с компактным носителем, т.к. $\supp g$ и $\supp \varphi$ компакты). Тогда аргументация та же, что и в предыдущем пункте.
\end{enumerate}
(Примечание: в оригинале $(f_k*g, \varphi) = (f_k(y), (g(\eta), \varphi(y+\eta)))$. Это можно записать как $(f_k, \Phi_g^\varphi)$, где $\Phi_g^\varphi(y) = (g, \varphi_y)$ и $\varphi_y(\eta) = \varphi(y+\eta)$.  Необходимо, чтобы $\Phi_g^\varphi \in \Dcal(\R^n)$.
Если $g \in \mathcal{E}'(\R^n)$, то $\Phi_g^\varphi \in \Dcal(\R^n)$.
Если $f_k$ имеют носители в общем компакте $M$, то для сходимости нужно, чтобы $\Phi_g^\varphi$ была основной функцией.
Условие (а) "значение функции $\Phi(x)$ и $\varphi(x)$ совпадают в нек. окр. мн. $M$". Это не совсем ясно и, вероятно, относится к более специфическому контексту.  Мы используем стандартные условия для корректности и сходимости свертки.
Условие (б) "$P(y) = (g(y), \varphi(y-\eta))$ - есть осн. ф-ия, для которой справедливо выше". Это, видимо, опечатка и должно быть $\Phi(y) = \langle g(\eta), \varphi(y-\eta) \rangle_\eta$. Если $g \in \Dprime$ и $\varphi \in \Dcal$, то $g * \varphi \in C^\infty$. Если $g \in \mathcal{E}'$, то $g * \varphi \in \Dcal$.)

\subsection*{Определение (Свертка на $\Sprime$)}
Пусть $f, g \in \Sprime(\R^n)$ и обобщенная функция $g \in \Dprime(\R^n)$ имеет ограниченный носитель (т.е. $g \in \mathcal{E}'(\R^n)$).
Тогда свертка $f*g$ определяется для $\varphi \in \mathcal{S}(\R^n)$ (пространство Шварца):
\[ (f*g, \varphi) = \langle (f(y) \times g(\eta)), \varphi(y+\eta) \rangle_{y,\eta} \]
или, более стандартно, если $g \in \mathcal{E}'(\R^n)$, то $(f*g, \varphi) = (f, \check{g} * \varphi)$, где $\check{g}(\eta) = g(-\eta)$, и $\check{g} * \varphi \in \mathcal{S}(\R^n)$.
Или $(f*g, \varphi) = (f(y), (g(\eta), \varphi(y+\eta))_\eta)$.






\subsection*{Дифференцирование свертки обобщенных функций}
% Замечания из начала изображения:
% $\varphi(y+\eta)$ симметрична отн. $y, \eta$.
% $(y, \eta) \in (\supp f \times \supp g) \cap \supp \varphi(y+\eta)$.
% Оба ограничены. И как следствие $\supp(f*g)$ компакт, если $\supp f, \supp g$ компакты.

Для $f, g \in \Dprime(\R^n)$ (таких, что свертка $f*g$ определена):
Для любой $\varphi \in \Dcal(\R^n)$:
\begin{align*}
(D^\alpha(f*g), \varphi) &= (-1)^{|\alpha|} (f*g, D^\alpha\varphi) \\
&= (-1)^{|\alpha|} \langle f(y), \langle g(\eta), (D^\alpha\varphi)(y+\eta) \rangle_\eta \rangle_y \\
% Следуя рукописи (для f*(D^\alpha g)):
&= (-1)^{|\alpha|} \langle f(y), \langle (-1)^{|\alpha|} (D_\eta^\alpha g)(\eta), \varphi(y+\eta) \rangle_\eta \rangle_y \quad \text{(используя } D_y^\alpha = D_\eta^\alpha \text{ и интегрирование по частям для } g) \\
&= \langle f(y), \langle (D^\alpha g)(\eta), \varphi(y+\eta) \rangle_\eta \rangle_y = (f*(D^\alpha g), \varphi).
\end{align*}
% Примечание: в рукописи было $(f \circ (D_x^\alpha D^\alpha g), \varphi)$. Предполагается, что $D_x^\alpha$ относится к переменной интегрирования $g$.

Следовательно,
\begin{align*}
D^\alpha(f*g) &= f*(D^\alpha g) \\
D^\alpha(f*g) &= (D^\alpha f)*g
\end{align*}
(В рукописи используется символ $\circ$ для свертки).

\subsection*{Лемма 2 (О сходимости свертки)}
Пусть последовательность $\{f_k\}_{k=1}^\infty \subset \Dprime(\R^n)$ сходится к $f \in \Dprime(\R^n)$ (т.е. $(f_k, \psi) \to (f, \psi)$ для всех $\psi \in \Dcal(\R^n)$).
Тогда $f_k * g \xrightarrow{k \to \infty} f * g$ в $\Dprime(\R^n)$, если выполнено одно из следующих условий:
\begin{enumerate}[label=\alph*)]
    \item Все $f_k$ имеют носители, содержащиеся в одном и том же ограниченном открытом множестве $M$ (т.е. $\supp f_k \subseteq K_M$ для некоторого компакта $K_M \subset M$ для всех $k$, и $\supp f \subseteq K_M$).
    \item Носитель функции $g$ ограничен (т.е. $g \in \Eprime(\R^n)$).
\end{enumerate}

\begin{proof}[Схема доказательства]
Имеем $(f_k * g, \varphi) = \langle f_k(y), \langle g(\eta), \varphi(y+\eta) \rangle_\eta \rangle_y$.
Обозначим $\Phi_g^\varphi(y) = \langle g(\eta), \varphi(y+\eta) \rangle_\eta = (g * \check{\varphi})(y)$, где $\check{\varphi}(\eta) = \varphi(-\eta)$.
Тогда $(f_k * g, \varphi) = (f_k, \Phi_g^\varphi)$.
Необходимо показать, что $\Phi_g^\varphi \in \Dcal(\R^n)$ или является функцией, на которую $f_k$ и $f$ могут действовать корректно и для которой сходимость сохраняется.

\begin{enumerate}[label=\alph*)]
    \item Если $\supp f_k \subseteq K_M$ (компакт), то $f_k, f \in \Eprime(K_M)$. Функция $\Phi_g^\varphi(y)=(g * \check{\varphi})(y)$ является $C^\infty$-функцией. Действие $(f_k, \Phi_g^\varphi)$ корректно определено. Сходимость $f_k \to f$ в $\Dprime$ (а значит и в $\Eprime(K_M)$) означает $(f_k, \psi) \to (f, \psi)$ для всех $\psi \in C^\infty(\R^n)$ (или, по крайней мере, для $\psi$ гладких в окрестности $K_M$).
    (В рукописи: Функцию $\Phi(y) = \langle g(\eta), \varphi(y-\eta) \rangle_\eta$ можно записать как основную функцию $\Phi$: $\supp \Phi \subset \supp \varphi + \check{\supp g}$ (где $\check{\supp g} = -\supp g$). Значения функции $\Phi(x)$ и $\varphi(x)$ совпадают в некоторой окрестности множества $M$). \\
    Тогда $(f_k, \Phi_g^\varphi) \xrightarrow{k \to \infty} (f, \Phi_g^\varphi) = (f*g, \varphi)$.
    То есть $f_k * g \to f*g$ в $\Dprime(\R^n)$.

    \item Если $g \in \Eprime(\R^n)$ (т.е. $\supp g$ компактен) и $\varphi \in \Dcal(\R^n)$, то $\Phi_g^\varphi(y) = (g * \check{\varphi})(y) \in \Dcal(\R^n)$.
    Поскольку $f_k \to f$ в $\Dprime(\R^n)$, то $(f_k, \Phi_g^\varphi) \to (f, \Phi_g^\varphi)$ для $\Phi_g^\varphi \in \Dcal(\R^n)$.
    Следовательно, $(f_k*g, \varphi) \to (f*g, \varphi)$ для всех $\varphi \in \Dcal(\R^n)$, что означает $f_k*g \to f*g$ в $\Dprime(\R^n)$.
    (В рукописи: $\Phi(y) = \langle g(y), \varphi(y-\eta) \rangle$ -- есть осн. ф-ия, для которой справедливо выше. Это, вероятно, опечатка, имеется в виду $\Phi(y) = \langle g(\eta), \varphi(y-\eta) \rangle_\eta = (g * \varphi)(y)$).
\end{enumerate}
\end{proof}

\subsection*{Определение (Свертка на $\Sprime$)}
Пусть $f \in \Sprime(\R^n)$ и обобщенная функция $g \in \Dprime(\R^n)$ имеет ограниченный носитель (т.е. $g \in \Eprime(\R^n)$).
Тогда свертка $f*g$ определяется для $\varphi \in \mathcal{S}(\R^n)$ (пространство Шварца) как:
\[ (f*g, \varphi) = \langle (f(y) \times g(\eta)), \varphi(y+\eta) \rangle_{y,\eta} \]
Это можно понимать как $(f*g, \varphi) = (f(y), \langle g(\eta), \varphi(y+\eta) \rangle_\eta)$.
Поскольку $g \in \Eprime(\R^n)$ и $\varphi \in \mathcal{S}(\R^n)$, то $\Psi_g^\varphi(y) = \langle g(\eta), \varphi(y+\eta) \rangle_\eta = (g * \check{\varphi})(y) \in \mathcal{S}(\R^n)$.
Действие $(f, \Psi_g^\varphi)$ корректно определено, и $f*g \in \Sprime(\R^n)$.






\subsection*{Фундаментальное решение оператора с постоянными коэффициентами (продолжение)}

\paragraph{Рассмотрение:} Линейное уравнение $k$-го порядка:
\[ L y(x) := \sum_{|\alpha| \le k} a_\alpha(x) D_x^\alpha y(x) = b(x) \]
где $y(x) \in C^k(\R^n)$, $a_\alpha(x) \in C^\infty(\R^n)$. Тогда все операции определены.
Для $f \in \Dprime(\R^n)$, уравнение понимается в смысле обобщенных функций:
$Lf = b$ означает
\[ \sum_{|\alpha| \le k} (-1)^{|\alpha|} (f, D_x^\alpha (a_\alpha(x) \varphi(x))) = (b, \varphi) \quad \forall \varphi \in \Dcal(\R^n) \]
Если $a_\alpha = \text{const}$, то $L f = \sum_{|\alpha| \le k} a_\alpha D^\alpha f = b$ --- линейное дифференциальное уравнение с обобщенной правой частью.

\paragraph{Определение.}
Фундаментальным решением оператора $L$ (или дифференциального уравнения выше) называется обобщенная функция $E \in \Dprime(\R^n)$, удовлетворяющая дифференциальному уравнению
\[ LE = \delta \]
где $\delta$ -- дельта-функция Дирака.

\paragraph{Замечание 1.}
Фундаментальное решение определено неоднозначно, с точностью до решения соответствующего однородного уравнения ($Lw=0$). Если $E_0$ -- какое-либо фундаментальное решение, то любое другое фундаментальное решение имеет вид $E = E_0 + w$, где $Lw=0$.

\paragraph{Теорема.}
Пусть $E \in \Dprime(\R^n)$ --- фундаментальное решение оператора $L$ с постоянными коэффициентами, и $b \in \Dprime(\R^n)$ такова, что свертка $b*E \in \Dprime(\R^n)$ существует. Тогда функция $f = b*E$ является обобщенным решением уравнения $Lf=b$.

\begin{proof}
Из свойств свертки (для операторов с постоянными коэффициентами $L$ коммутирует со сверткой):
\[ Lf = L(b*E) = b*(LE) = b*\delta = b. \]
\end{proof}

\paragraph{Пример.}
Рассмотрим задачу Коши:
\[ y''(x) + \mu^2 y(x) = B(x), \quad x \in [0, +\infty) \]
\[ y(0) = 0, \quad y'(0) = 0 \]
где $B(x) \in C[0, +\infty)$ (в рукописи также упомянуто $g(x) \in C^2[0, +\infty)$, вероятно $y(x)$ обозначалось как $g(x)$), $\mu > 0$ --- константа.

Запишем уравнение для $f \in \Dprime(\R)$ как $Lf = B\Theta$, где $L = \frac{d^2}{dx^2} + \mu^2$, и $\Theta(x)$ --- функция Хевисайда:
\[ \Theta(x) = \begin{cases} 1, & x \ge 0 \\ 0, & x < 0 \end{cases} \]
Ищем фундаментальное решение $E(x)$ вида $E(x) = z(x)\Theta(x)$, где $z(x)$ --- неизвестная функция (будем считать $z \in C^2(\R)$).
Производные $E(x)$:
\begin{align*} E'(x) &= (z(x)\Theta(x))' = z'(x)\Theta(x) + z(x)\Theta'(x) = z'(x)\Theta(x) + z(0)\delta(x) \\ E''(x) &= (z'(x)\Theta(x) + z(0)\delta(x))' = z''(x)\Theta(x) + z'(x)\Theta'(x) + z(0)\delta'(x) \\ &= z''(x)\Theta(x) + z'(0)\delta(x) + z(0)\delta'(x) \end{align*}
(В рукописи немного другой вывод производных, но этот стандартный.)
Подставляем в $LE = \delta$:
\[ (z''(x) + \mu^2 z(x))\Theta(x) + z'(0)\delta(x) + z(0)\delta'(x) = \delta(x) \]
Для выполнения этого равенства необходимо:
\begin{enumerate}
    \item $z''(x) + \mu^2 z(x) = 0$ для $x > 0$.
    \item $z(0) = 0$ (чтобы уничтожить член с $\delta'(x)$).
    \item $z'(0) = 1$ (чтобы коэффициент при $\delta(x)$ был равен 1).
\end{enumerate}
Общее решение уравнения $z''(x) + \mu^2 z(x) = 0$ есть $z(x) = C_1 \cos(\mu x) + C_2 \sin(\mu x)$.
Из $z(0)=0 \implies C_1 \cos(0) + C_2 \sin(0) = C_1 = 0$.
Тогда $z(x) = C_2 \sin(\mu x)$, и $z'(x) = \mu C_2 \cos(\mu x)$.
Из $z'(0)=1 \implies \mu C_2 \cos(0) = \mu C_2 = 1 \implies C_2 = 1/\mu$.
Таким образом, $z(x) = \frac{1}{\mu}\sin(\mu x)$.
Фундаментальное решение:
\[ E(x) = \frac{1}{\mu}\sin(\mu x) \Theta(x) \]
Решение исходной задачи $f(x)$ (которое будет нашим $y(x)$ для $x \ge 0$) дается сверткой $f = (B\Theta)*E$:
\begin{align*} f(x) &= \int_{-\infty}^{\infty} (B\Theta)(s) E(x-s) ds = \int_{-\infty}^{\infty} B(s)\Theta(s) \frac{1}{\mu}\sin(\mu(x-s))\Theta(x-s) ds \\ &= \int_{0}^{\infty} B(s) \frac{1}{\mu}\sin(\mu(x-s))\Theta(x-s) ds \quad (\text{т.к. } \Theta(s)=0 \text{ для } s<0) \end{align*}
Если $x < 0$, то $x-s < 0$ (т.к. $s \ge 0$), поэтому $\Theta(x-s)=0$, и $f(x)=0$.
Если $x \ge 0$, то $\Theta(x-s)=1$ для $s \le x$ и $0$ для $s > x$. Интеграл будет от $0$ до $x$:
\[ f(x) = \frac{1}{\mu} \int_{0}^{x} B(s)\sin(\mu(x-s)) ds \]
Так как носители $E$ и $B\Theta$ ограничены слева ($x \ge 0$), свертка существует.
Для $x \ge 0$, функция $f(x)$ принадлежит $C^2[0, +\infty)$ и удовлетворяет исходному уравнению и начальным условиям. Т.е. для $x \ge 0$ она является решением Задачи Коши.
(В рукописи последняя формула $y(x) = \frac{1}{\mu} \int_0^x \sin(\mu|x-\eta|) B(\eta) d\eta$. Для $\eta \in [0,x]$, $x-\eta \ge 0$, поэтому $|x-\eta|=x-\eta$, что совпадает с полученной формулой.)













\newpage
\section*{Билет №13 -- 2025}\label{sec:ticket13}
\backtotoc
\textbf{Фундаментальное решение оператора Лапласа для n=3. Фундаментальное решение
оператора теплопроводности (без доказательства) [2]– 224-226, 232-235. [1] – 95,153,150.}





\newpage
\section*{Билет №14 -- 2025}\label{sec:ticket14}
\backtotoc
\textbf{Симметрический оператор в гильбертовом пространстве, свойства его собственных чисел и
собственных функций. Формулы Грина и симметричность оператора Лапласа в L2(G) с гранич ны-
ми условиями трех видов. Теорема Гильберта-Шмидта и полнота системы собственных функций
оператора Лапласа с граничными условиями (без доказательства).[1]–3 1 , 244 - 249, 251. [2]—
142-149, 321, 325.}



% Убираем отступ для абзацев и добавляем небольшой отступ между ними
\setlength{\parindent}{0pt}
\setlength{\parskip}{0.5\baselineskip}

% Стили для теорем, определений и замечаний

\newtheorem{definitioninner}{Определение}


\newtheorem{remarkinner}{Замечание}

\newtheorem{theoreminner}{Теорема}


\begin{center}
\textbf{N16}
\end{center}

\textbf{Симметричный (эрмитов) оператор в гильбертовом пространстве, свойства его собственных чисел и собственных функций. Формулы Грина и симметричность оператора Лапласа в $L_2(G)$ с граничными условиями. Теорема о полноте системы собственных функций оператора Лапласа с граничными условиями. (без док-ва)}

\subsection*{1) Симметричный оператор в $H$ [Уров 142]}

Пусть $H$ --- лин. пр-во над полем действит. (комплексных) чисел, $M$ --- линейное многообразие в $H$, $A: M \to H$ --- лин. опер.

\begin{definition}{Опр 1}
Действ. (компл.) число $\lambda$ называется \textit{собственным значением} оператора $A$, если существует ненулевой элемент $\varphi \in M$ такой, что $A\varphi = \lambda\varphi$. (Если $H$ --- функциональное пр-во, то $\varphi$ наз. \textit{собственной ф-ей}).
\end{definition}

\begin{remark}{Замечание 1}
Одному собств. значению могут отвечать несколько ЛНЗ (линейно независимых) собственных векторов (ф-й), их кол-во называется \textit{кратностью}.
\end{remark}

\begin{definition}{Опр 2}
Собственное значение называется \textit{простым}, если $\exists$ единственный ЛНЗ собственный вектор (ф-я), отвечающий ему.
\end{definition}

\textbullet~Если $H$ --- (пред)гильбертово пространство, тогда можно ввести следующие понятия:

\begin{definition}{Опр 3}
Оператор $A$ наз. \textit{симметричным} на многообразии $M$, если для любых двух элементов $\varphi, \psi \in M$ выполняется
\[ (A\varphi, \psi) = (\varphi, A\psi). \]
\end{definition}

\begin{theorem}{Теорема 1}
Пусть $A$ --- линейный симметричный оператор, тогда все его собственные значения (если таковые существуют) являются действительными.
\end{theorem}

\begin{remark}{Замечание 2}
Теорема справедлива только в случае, если $H$ --- (пред)гильбертово пр-во над полем $\mathbb{C}$.
\end{remark}





% Убираем отступ для абзацев и добавляем небольшой отступ между ними
\setlength{\parindent}{0pt}
\setlength{\parskip}{0.5\baselineskip}

% Стили для теорем, определений и замечаний (предполагается, что они определены как в предыдущем файле)

\newtheorem{definitioninner}{Определение}


\newtheorem{remarkinner}{Замечание}

\newtheorem{theoreminner}{Теорема}




\begin{proof}[Д-во.] % Предполагается, что это доказательство Теоремы 1 с предыдущей страницы
Пусть $\lambda$ --- собственное значение, $\varphi$ --- собственный вектор оператора $A$, отвечающий ему. Из свойств скалярного произведения и симметричности оператора $A$:
\[ \lambda(\varphi, \varphi) = (A\varphi, \varphi) = (\varphi, A\varphi) = (\varphi, \lambda\varphi) = \bar{\lambda}(\varphi, \varphi). \]
Поскольку $(\varphi, \varphi) \neq 0$ (т.к. $\varphi \neq 0$ по \underline{Опр. 1}), то $\lambda = \bar{\lambda}$.
\end{proof}

$(\varphi, \psi) = \sum \varphi_i \bar{\psi}_i$; \quad $(\varphi, \psi) = \int_X \varphi \bar{\psi} \, d\mu(x)$.

\begin{theorem}{Теорема 2}
Пусть $A$ --- симметричный оператор, тогда, если $\lambda_1$ и $\lambda_2$ --- различные собственные значения оператора $A$, то собственные векторы $\varphi_1$ и $\varphi_2$, отвечающие им, ортогональны, т.е. $(\varphi_1, \varphi_2)=0$.
\end{theorem}

\begin{proof}[Д-во.]
Пусть $A\varphi_1 = \lambda_1 \varphi_1$; $A\varphi_2 = \lambda_2 \varphi_2$, причем $\lambda_1 \neq \lambda_2$. Тогда
\[ \lambda_1(\varphi_1, \varphi_2) = (A\varphi_1, \varphi_2) = (\varphi_1, A\varphi_2) = (\varphi_1, \lambda_2\varphi_2) = \bar{\lambda}_2(\varphi_1, \varphi_2). \]
Так как $\lambda_2$ --- вещественное (по Теореме 1, как собственное значение симметричного оператора), то $\bar{\lambda}_2 = \lambda_2$.
Следовательно, $\lambda_1(\varphi_1, \varphi_2) = \lambda_2(\varphi_1, \varphi_2)$, что можно переписать как
\[ (\lambda_1 - \lambda_2)(\varphi_1, \varphi_2) = 0. \]
Поскольку $\lambda_1 \neq \lambda_2$, то отсюда следует, что $(\varphi_1, \varphi_2)=0$.
\end{proof}

\begin{definition}{Опр 4}
Оператор $A$ называется \textit{положительно (неотрицательно) определенным} на многообразии $M$, если для любого ненулевого $\varphi \in M$ выполняется $(A\varphi, \varphi)>0$ (соотв. $(A\varphi, \varphi)\ge 0$).
\end{definition}

\begin{theorem}{Теорема 3}
Все собственные значения положительно (неотрицательно) определенного линейного оператора $A$ положительны (неотрицательны).
\end{theorem}

\begin{proof}[Д-во.]
Пусть $\lambda$ --- собственное значение оператора $A$, $\varphi$ --- собственный вектор, отвечающий $\lambda$. Тогда $A\varphi = \lambda\varphi$.
Из этого следует $\lambda(\varphi, \varphi) = (\lambda\varphi, \varphi) = (A\varphi, \varphi)$.
Поскольку оператор $A$ положительно (неотрицательно) определен и $\varphi \neq 0$, то $(A\varphi, \varphi) > 0$ (соответственно, $(A\varphi, \varphi) \ge 0$).
Таким образом, $\lambda(\varphi, \varphi) > 0$ (соответственно, $\lambda(\varphi, \varphi) \ge 0$).
Так как $(\varphi, \varphi) > 0$ по определению скалярного произведения (поскольку $\varphi \neq 0$), то из последнего неравенства вытекает, что $\lambda > 0$ (соответственно, $\lambda \ge 0$).
\end{proof}

\begin{remark}{Замечание 3}
Когда $H$ --- (пред)гильбертово пространство над полем комплексных чисел, из $>0$ ($\ge 0$) определенности оператора вытекает его симметричность.
\end{remark}


\usepackage[T2A]{fontenc}
\usepackage[utf8]{inputenc}
\usepackage[russian]{babel}
\usepackage{amsmath}
\usepackage{amsfonts}
\usepackage{amssymb}
\usepackage{amsthm} % Для теорем и определений
\usepackage{wasysym} % Если вдруг понадобится для каких-то символов, хотя в этом тексте специфических нет

% Убираем отступ для абзацев и добавляем небольшой отступ между ними
\setlength{\parindent}{0pt}
\setlength{\parskip}{0.5\baselineskip}

% Стили для теорем, определений и замечаний (предполагается, что они определены как в предыдущем файле)
\theoremstyle{definition} % Обычный шрифт для текста, заголовок жирный
\newtheorem{definitioninner}{Определение}
\newenvironment{definition}[1]{\begin{definitioninner}[\textbf{#1}]}{\end{definitioninner}}

\newtheorem{remarkinner}{Замечание}
\newenvironment{remark}[1]{\begin{remarkinner}[\textbf{#1}]}{\end{remarkinner}}

\theoremstyle{plain} % Курсив для текста теоремы, заголовок жирный
\newtheorem{theoreminner}{Теорема}
\newenvironment{theorem}[1]{\begin{theoreminner}[\textbf{#1}]}{\end{theoreminner}}

% Настройка отображения "Доказательство"
\renewcommand{\proofname}{Д-во.}



\subsection*{2) Формулы Грина [Уров 146]}

Пусть $D$ --- ограниченная область в $\mathbb{R}^n$ с кусочно гладкой границей $\partial D$, $n$ --- внешняя нормаль к $\partial D$.

\begin{theorem}{Теорема 4 [первая ф-ла Грина]}
Если ф-ии $u(x)$ и $v(x)$ принадлежат классам $C^1(\bar{D})$ и $C^2(D) \cap C^1(\bar{D})$ соответственно, причем $\Delta v \in L_1(D)$, то выполняется
\begin{equation} \label{eq:green1}
\int_D u \Delta v \, dx = - \int_D \sum_{i=1}^n \frac{\partial u}{\partial x_i} \frac{\partial v}{\partial x_i} \, dx + \int_{\partial D} u \frac{\partial v}{\partial n} \, ds
\end{equation}
\end{theorem}

\begin{proof}
Рассмотрим произвольную область $\tilde{D}$ с кус. глад. границей $\partial \tilde{D}$ такую, что $\bar{\tilde{D}} \subset D$. Запишем для векторной ф-ии $P = u \operatorname{grad} v$ и обл. $\tilde{D}$ ф-лу Остроградского--Гаусса:
\[ \int_{\tilde{D}} \operatorname{div} P \, dx = \int_{\partial \tilde{D}} \sum_{i=1}^n P_i n_i \, ds \]
Учтем, что $\operatorname{div} P = u \Delta v + \sum_{i=1}^n \frac{\partial u}{\partial x_i} \frac{\partial v}{\partial x_i}$ и $\sum_{i=1}^n \frac{\partial v}{\partial x_i} n_i = \frac{\partial v}{\partial n}$.
Получаем:
\[ \int_{\tilde{D}} u \Delta v \, dx = - \int_{\tilde{D}} \sum_{i=1}^n \frac{\partial u}{\partial x_i} \frac{\partial v}{\partial x_i} \, dx + \int_{\partial \tilde{D}} u \frac{\partial v}{\partial n} \, ds \]
В последнем равенстве все подынтегральные ф-ии интегрируемы в области $D$, что позволяет сделать предельный переход, устремляя $\tilde{D}$ к $D$.
\end{proof}

\begin{theorem}{Теорема 5 [вторая ф-ла Грина]}
Если ф-ии $u(x)$ и $v(x)$ принадлежат классу $C^2(D) \cap C^1(\bar{D})$, причем $\Delta u \in L_1(D)$, $\Delta v \in L_1(D)$, то справедлива ф-ла:
\begin{equation} \label{eq:green2}
\int_D (u \Delta v - v \Delta u) \, dx = \int_{\partial D} \left(u \frac{\partial v}{\partial n} - v \frac{\partial u}{\partial n}\right) \, ds
\end{equation}
\end{theorem}

\rule{\textwidth}{0.4pt}
$L_1(D)$ --- мн-во ф-ий, модуль которых интегрируем по Лебегу $\leftrightsquigarrow$ Риману на измеримом мн-ве $D$.




% Убираем отступ для абзацев и добавляем небольшой отступ между ними
\setlength{\parindent}{0pt}
\setlength{\parskip}{0.5\baselineskip}

% Стили для теорем, определений и замечаний (предполагается, что они определены как в предыдущем файле)



\newtheorem{remarkinner}{Замечание}

\newtheorem{theoreminner}{Теорема}


% Настройка отображения "Доказательство"
\renewcommand{\proofname}{Д-во.}



\begin{proof}[Д-во \textit{(Теоремы 5).}]
Поменяем в $1^{\text{ой}}$ ф-ле Грина \eqref{eq:green1} местами ф-ии $u$ и $v$:
\[ \int_D v \Delta u \, dx = - \int_D \sum_{i=1}^n \frac{\partial v}{\partial x_i} \frac{\partial u}{\partial x_i} \, dx + \int_{\partial D} v \frac{\partial u}{\partial n} \, ds \]
Вычитая данное равенство из равенства \eqref{eq:green1}, получаем \eqref{eq:green2}.
\end{proof}

\begin{theorem}{Теорема 6 [третья ф-ла Грина]}
Если ф-я $u(x) \in C^2(D) \cap C^1(\bar{D})$, причем $\Delta u \in L_1(D)$, то
\[ \int_D u \Delta u \, dx = - \int_D \sum_{i=1}^n \left(\frac{\partial u}{\partial x_i}\right)^2 \, dx + \int_{\partial D} u \frac{\partial u}{\partial n} \, ds \]
\end{theorem}

\begin{proof}
Достаточно положить $v=u$ в ф-ле \eqref{eq:green1}.
\end{proof}

\subsection*{3) Симметричность оператора Лапласа в $L_2(G)$ [Уров 147]}
Пусть $D$ --- ограниченная область в $\mathbb{R}^n$ с кусочно-гладкой границей $\partial D$. Рассмотрим в гильбертовом пр-ве $L_2(D)$ линейные многообразия:
\begin{align*}
M_1 &= M \cap \{u(x) : u|_{\partial D} = 0 \}; \\
M_2 &= M \cap \left\{u(x) : \left.\frac{\partial u}{\partial n}\right|_{\partial D} = 0 \right\}; \\
M_3 &= M \cap \left\{u(x) : \left.\left(ku + \frac{\partial u}{\partial n}\right)\right|_{\partial D} = 0, k>0 \right\},
\end{align*}
где $M = \{u(x) : u \in C^2(D) \cap C^1(\bar{D}), \Delta u \in L_2(D) \}$, $n$ --- внешняя по отношению к $D$ ед. нормаль к $\partial D$. Линейные многообразия $^{M_1, M_2, M_3}$ \textit{состоят из ф-й, удовлетворяющих граничным условиям соотв. первого рода (Дирихле), второго рода (Неймана) и третьего рода.}

\begin{theorem}{Теорема 7}
Дифференциальный оператор $A = -\Delta$ на линейных многообразиях $M_1, M_2$ и $M_3$ симметричен, причем на линейных многообразиях $M_1$ и $M_3$ он положительно определен, а на $M_2$ --- неотрицательно.
\end{theorem}

\begin{proof}[\textit Д-во: [Симметричность]]
Рассмотрим произвольные $u,v \in M$ и запишем для $u$ и $\bar{v}$ вторую ф-лу Грина \eqref{eq:green2}
\end{proof}

\usepackage[T2A]{fontenc}
\usepackage[utf8]{inputenc}
\usepackage[russian]{babel}
\usepackage{amsmath}
\usepackage{amsfonts}
\usepackage{amssymb}
\usepackage{amsthm} % Для теорем и определений
\usepackage{hyperref} % Для ссылок (хотя здесь не используются явно)

% Убираем отступ для абзацев и добавляем небольшой отступ между ними
\setlength{\parindent}{0pt}
\setlength{\parskip}{0.5\baselineskip}

% Стили для теорем, определений и замечаний (предполагается, что они определены как в предыдущем файле)
\theoremstyle{definition}
\newtheorem{definitioninner}{Определение}
\newenvironment{definition}[1]{\begin{definitioninner}[\textbf{#1}]}{\end{definitioninner}}

\newtheorem{remarkinner}{Замечание}
\newenvironment{remark}[1]{\begin{remarkinner}[\textbf{#1}]}{\end{remarkinner}}

\theoremstyle{plain}
\newtheorem{theoreminner}{Теорема}
\newenvironment{theorem}[1]{\begin{theoreminner}[\textbf{#1}]}{\end{theoreminner}}
\newtheorem{corollaryinner}{Следствие}
\newenvironment{corollary}[1]{\begin{corollaryinner}[\textbf{#1}]}{\end{corollaryinner}}


% Настройка отображения "Доказательство"
\renewcommand{\proofname}{Д-во.}


\begin{proof}[\textit Д-во (Теоремы 7, продолжение): [Симметричность]]
Для $u, v \in M$ (где $M$ было определено на предыдущей странице, а $A=-\Delta$):
\begin{align*}
(Au, v) - (u, Av) &= \int_D (-\Delta u \cdot \bar{v}) dx - \int_D u (-\overline{\Delta v}) dx \\
&= \int_D (u \Delta \bar{v} - \bar{v} \Delta u) dx \\
&= \int_{\partial D} \left(u \frac{\partial \bar{v}}{\partial n} - \bar{v} \frac{\partial u}{\partial n}\right) ds
\end{align*}
Из граничных условий следует, что $\left.\left(u \frac{\partial \bar{v}}{\partial n} - \bar{v} \frac{\partial u}{\partial n}\right)\right|_{\partial D} = 0$.
Например, для $u$ и $v$ из $M_3$:
\[ \left.\left(u \frac{\partial \bar{v}}{\partial n} - \bar{v} \frac{\partial u}{\partial n}\right)\right|_{\partial D} = \left.(u (-k\bar{v}) - \bar{v} (-ku))\right|_{\partial D} = \left.(-ku\bar{v} + ku\bar{v})\right|_{\partial D} = 0. \]
(Аналогично для $M_1$: $u=0, v=0$ на $\partial D \Rightarrow$ член равен 0. Для $M_2$: $\frac{\partial u}{\partial n}=0, \frac{\partial v}{\partial n}=0$ на $\partial D \Rightarrow$ член равен 0).
$\Rightarrow (Au,v) = (u, Av)$.
\end{proof}

\begin{proof}[\textit{[Положит. (неотр.) определенность]}]
Пусть $u(x) \in M, u \neq 0$. Запишем $1^{\text{ую}}$ ф-лу Грина (уравнение (1) с предыдущей страницы, примененное к $u(x)$ и $\bar{u}(x)$):
\[ (Au, u) = \int_D (-\Delta u) \bar{u} dx = \int_D |\operatorname{grad} u|^2 dx - \int_{\partial D} \bar{u} \frac{\partial u}{\partial n} ds. \]
Далее три варианта:
\begin{enumerate}
    \item Пусть $u \in M_1$, тогда $u|_{\partial D}=0 \Rightarrow \bar{u}|_{\partial D}=0$. Следовательно, $\int_{\partial D} \bar{u} \frac{\partial u}{\partial n} ds = 0$.
    То есть $(Au,u) = \int_D |\operatorname{grad} u|^2 dx$.
    Поскольку $u \not\equiv 0$ и $u|_{\partial D}=0$, то $u$ не может быть ненулевой константой. Если $u \not\equiv 0$, то $\int_D |\operatorname{grad} u|^2 dx > 0$.
    (т.к. $u \equiv \text{const} \neq 0 \notin M_1$).
    $\Rightarrow A = -\Delta$ положительно определен на $M_1$.
    \item Пусть $u \in M_2$, тогда $\left.\frac{\partial u}{\partial n}\right|_{\partial D} = 0$. Следовательно, $\int_{\partial D} \bar{u} \frac{\partial u}{\partial n} ds = 0$.
    То есть $(Au,u) = \int_D |\operatorname{grad} u|^2 dx \ge 0$.
    $\Rightarrow$ оператор $A = -\Delta$ неотрицательно определен на многообр. $M_2$, причем $(Au,u)=0$ выполняется только для $u \equiv \text{const}$ (т.к. $\operatorname{grad} u = 0$).
    \item Пусть $u \in M_3$, тогда $\left.\left(ku + \frac{\partial u}{\partial n}\right)\right|_{\partial D} = 0 \Rightarrow \left.\frac{\partial u}{\partial n}\right|_{\partial D} = -ku|_{\partial D}$.
    \[ (Au,u) = \int_D |\operatorname{grad} u|^2 dx - \int_{\partial D} \bar{u} (-ku) ds = \int_D |\operatorname{grad} u|^2 dx + k \int_{\partial D} |u|^2 ds > 0 \]
    (если $u \not\equiv 0$, т.к. $k>0$. Если $(Au,u)=0$, то $\operatorname{grad} u = 0 \Rightarrow u=\text{const}$, и $u|_{\partial D}=0 \Rightarrow u \equiv 0$).
    $\Rightarrow A = -\Delta$ положит. опред. на $M_3$.
\end{enumerate}
\end{proof}


\begin{corollary}{Следствие 1}
Соб. ф-ии дифф-го оп. $A=-\Delta$, заданного на одном из линейных многообразий $M_1, M_2, M_3$, отвечающие различным соб. значениям ортогональны в $L_2(D)$.
\end{corollary}

\begin{corollary}{Следствие 2}
Все соб. значения $\lambda$ для $A = -\Delta$ на $M_1$ и $M_3$ суть $\lambda > 0$.
\end{corollary}

\begin{corollary}{Следствие 3}
Все соб. значения $\lambda$ для $A=-\Delta$ на $M_2$ суть $\lambda \ge 0$, причем единственной (с точностью до множителя) ЛНЗ соб. ф-ей, отвечающей $\lambda=0$, является $u \equiv 1$.
\end{corollary}





% Убираем отступ для абзацев и добавляем небольшой отступ между ними
\setlength{\parindent}{0pt}
\setlength{\parskip}{0.5\baselineskip}

% Стили для теорем, определений и замечаний (предполагается, что они определены как в предыдущем файле)

\newtheorem{definitioninner}{Определение}
\newenvironment{definition}[1]{\begin{definitioninner}[\textbf{#1}]}{\end{definitioninner}}

\newtheorem{remarkinner}{Замечание}
\newenvironment{remark}[1]{\begin{remarkinner}[\textbf{#1}]}{\end{remarkinner}}

\theoremstyle{plain}
\newtheorem{theoreminner}{Теорема}
\newenvironment{theorem}[1]{\begin{theoreminner}[\textbf{#1}]}{\end{theoreminner}}



\subsection*{4) Полнота системы собственных ф-й оператора Лапласа}

\begin{theorem}{Теорема 1 \textnormal{[Владимиров 248]}}
Множество соб. значений оператора $L$ не имеет конечных предельных точек, каждое соб. значение имеет конечную кратность. Всякая ф-я из $M_L$ разлагается в регулярно сходящийся ряд Фурье по соб. зн. оператора $L$.
\[ L = -\operatorname{div}(p \operatorname{grad}) + q; \quad p \in C^1(D), q \in C(D), p(x)>0, q(x) \ge 0, x \in D. \]
т.е. при $p \equiv 1, q \equiv 0 \quad L = -\Delta$ (?).
\[ M_L = C^2(D) \cap C^1(\bar{D}). \]
\end{theorem}

\underline{Без док-ва.}

Все соб. значения оператора $L$ можно пронумеровать в порядке возрастания величины, повторяя кратные:
\[ 0 \le \lambda_1 \le \lambda_2 \le \lambda_3 \le \lambda_4 \le \dots \]
\[ \begin{array}{cccc}
\downarrow & \downarrow & \downarrow & \downarrow \\
X_1 & X_2 & X_3 & X_4
\end{array}
\quad \text{каждому соб. значению } \lambda_k \text{ соотв.} \]
\[ \text{соб. ф-ия } X_k: LX_k = \lambda_k X_k, X_k \in M. \]
При этом $X_k$ можно выбрать взаимно ортонормальными.
\[ (LX_k, X_j) = \lambda_k (X_k, X_j) = \lambda_k \delta_{kj}. \]
$\Rightarrow \forall f \in M_L$ разлагается в ряд Фурье по ортонормир. сист. $\{X_k\}$
\[ f(x) = \sum_{k=1}^{\infty} (f, X_k) X_k \]
и ряд сходится регулярно на $D$. Поскольку $M_L$ плотно в $L_2(D)$.

\begin{definition}{Опр}
Множество ф-ий $M_L \subset L_2(D)$ наз. \underline{плотным} в $L_2(D)$, если $\forall f \in L_2(D)$ сущ. послед. ф-ий из $M_L$, сходящаяся к $f$ в $L_2(D)$.
\end{definition}

\begin{theorem}{Теорема 2}
Система соб. ф-ий оператора $L$ полна в $L_2(D)$.
\end{theorem}














\newpage
\section*{Билет №15 -- 2025}\label{sec:ticket15}
\backtotoc



\section*{Формула представления решения Пуассона. Потенциалы, их физический смысл и свойства (без док-ва)}

\subsection*{1) Формула представления решения Пуассона [Уроев 243]}

\textbf{Гармоническая ф-я} --- вещественная ф-ия $U$, опред. и дважды непр. дифф. на евклидовом пр-ве $D$, удовл. ур-ю: $\Delta U = 0$.

\textbf{Оператор Лапласа}: $\Delta \cdot = \sum_{i=1}^{n} \frac{\partial^2}{(\partial x^i)^2} \cdot$ \quad [Уроев с. 241]

\begin{itemize}
    \item Зафикс. произв. точку $x_0 \in \mathbb{R}^n$ и перейдём в сфер. коорд. $(r, \varphi, \theta, \dots)$, где $r = |x-x_0| = \sqrt{\sum_{i=1}^{n} (x^i-x_0^i)^2}$
    
    $\varphi, \theta, \dots$ --- угловые коорд. вектора $x-x_0$.
    
    $\Delta \cdot = \sum_{i=1}^{n} \frac{\partial}{\partial x^i} \left( \frac{\partial r}{\partial x^i} \frac{\partial}{\partial r} \cdot + \frac{\partial \varphi}{\partial x^i} \frac{\partial}{\partial \varphi} \cdot + \frac{\partial \theta}{\partial x^i} \frac{\partial}{\partial \theta} \cdot + \dots \right) = $
    
    $= \sum_{i=1}^{n} \left( \frac{\partial^2 r}{(\partial x^i)^2} \frac{\partial}{\partial r} \cdot + \left(\frac{\partial r}{\partial x^i}\right)^2 \frac{\partial^2}{\partial r^2} \cdot + \frac{\partial^2 \varphi}{(\partial x^i)^2} \frac{\partial}{\partial \varphi} \cdot + \left(\frac{\partial \varphi}{\partial x^i}\right)^2 \frac{\partial^2}{\partial \varphi^2} \cdot + \dots \right) = $
    
    $= \Delta r \frac{\partial}{\partial r} \cdot + |\nabla r|^2 \frac{\partial^2}{\partial r^2} \cdot + \Delta \varphi \frac{\partial}{\partial \varphi} \cdot + |\nabla \varphi|^2 \frac{\partial^2}{\partial \varphi^2} \cdot + \dots$
    
    \item Вычислим $\Delta r(x) = \sum_{i=1}^{n} \left( \frac{1}{r} - \frac{(x^i-x_0^i)^2}{r^3} \right) = \frac{n}{r} - \frac{r^2}{r^3} = \frac{n-1}{r}$
    
    $\implies \Delta \cdot = \frac{n-1}{r} \frac{\partial}{\partial r} \cdot + \frac{\partial^2}{\partial r^2} \cdot + \Delta \varphi \frac{\partial}{\partial \varphi} \cdot + |\nabla \varphi|^2 \frac{\partial^2}{\partial \varphi^2} \cdot + \dots =$
    
    $= \frac{1}{r^{n-1}} \frac{\partial}{\partial r} \left( r^{n-1} \frac{\partial}{\partial r} \cdot \right) + \frac{1}{r^2} \Delta_{\varphi, \theta, \dots} \cdot$
    
    \item В случае $n=3$: $x^1-x_0^1 = r \sin\theta \cos\varphi$, $x^2-x_0^2 = r \sin\theta \sin\varphi$, $x^3-x_0^3 = r \cos\theta$
    
    $r \ge 0, \quad 0 \le \varphi \le 2\pi, \quad 0 \le \theta \le \pi$
    
    $\Delta \varphi = 0, \quad |\nabla \varphi|^2 = \frac{1}{r^2 \sin^2\theta}, \quad \Delta \theta = \frac{1}{r^2} \frac{\cos\theta}{\sin\theta}, \quad |\nabla \theta|^2 = \frac{1}{r^2}$
    
    $\implies \Delta_{\varphi, \theta} \cdot = \frac{1}{\sin\theta} \frac{\partial}{\partial \theta} \left( \sin\theta \frac{\partial}{\partial \theta} \cdot \right) + \frac{1}{\sin^2\theta} \frac{\partial^2}{\partial \varphi^2} \cdot$
    
\end{itemize}
\hfill \framebox{1}



\underline{Опр.} Оператор "$-\Delta_{\varphi,\theta}$" наз-ся оператором Лапласа-Бельтрами.

\section*{Интегральное представление.}
\begin{itemize}
    \item Ф-я $\dfrac{1}{r} \equiv \dfrac{1}{|x-x_0|}$ - явл. гармонической в $\mathbb{R}^3 \setminus \{x_0\}$
    
    $\left(\Delta \left(\frac{1}{r}\right) = \frac{1}{r^2} \frac{\partial}{\partial r} \left(r^2 \frac{\partial}{\partial r} \left(\frac{1}{r}\right)\right) = 0\right)$
    
    \item Возьмём замкнутый шар $\bar{B}_\varepsilon = \{x: |x-x_0| \le \varepsilon\}$, $\bar{B}_\varepsilon \subset D$,
    для области $D \setminus \bar{B}_\varepsilon$ выпишем 2-ую формулу Грина, считая, что $u(x) \in C^2(D) \cap C^1(\bar{D})$, $\Delta u(x) \in L_2(D):$
    
    $\iiint_{D \setminus \bar{B}_\varepsilon} \left((\Delta u) \frac{1}{r} - \left(\Delta \frac{1}{r}\right) u\right) dx = \iint_{\partial D \cup \partial B_\varepsilon} \left(\left(\frac{\partial u}{\partial n}\right) \frac{1}{r} - \left(\frac{\partial}{\partial n} \left(\frac{1}{r}\right)\right) u\right) dS$
    
    $n$ - внешняя единичная нормаль по отношению к области $D \setminus \bar{B}_\varepsilon$

\begin{center}
\begin{tikzpicture}
    % Outer domain D
    \draw (0,0) ellipse (2.5cm and 2cm);
    \node at (0.5,1.3) {$D$};
    % Inner ball B_epsilon
    \draw (-0.5,0) circle (0.7cm);
    \node at (-0.5,0) [below=0.1cm] {$B_\varepsilon$};
    \filldraw (-0.5,0) circle (1pt) node[above right] {$x_0$};
    \draw[->] (-0.5,0) -- (-0.5,-0.7) node[midway, right] {$\varepsilon$};
    
    % Normal vectors n
    % On outer boundary (pointing outwards from D \setminus B_epsilon, so outwards from D)
    \draw[->, thick] (2.5,0) -- (3,0) node[right] {$n$};
    \draw[->, thick] (0,2) -- (0,2.5) node[above] {$n$};
    % On inner boundary (pointing outwards from D \setminus B_epsilon, so inwards into B_epsilon)
    \draw[->, thick] (-0.5+0.7*0.707, 0.7*0.707) -- (-0.5+0.7*0.707 - 0.3*0.707, 0.7*0.707 - 0.3*0.707) node[above left=0.05cm] {$n$};
    \draw[->, thick] (-0.5-0.7,0) -- (-0.5-0.7+0.3,0) node[left=0.2cm] {$n$};
\end{tikzpicture}
\end{center}

    так как $\Delta \dfrac{1}{r} = 0$ в $D \setminus \bar{B}_\varepsilon$, а $\Delta u$ непр. на $D$ и, следовательно, ограничена на $\bar{B}_\varepsilon \subset D$, то справедлив предельный переход при $\varepsilon \to +0$:
    
    $\iiint_{D \setminus \bar{B}_\varepsilon} \left((\Delta u) \frac{1}{r} - \left(\Delta \frac{1}{r}\right) u\right) dx = \iiint_{D \setminus \bar{B}_\varepsilon} (\Delta u) \frac{1}{r} dx = \iiint_{D \setminus \bar{B}_\varepsilon} (\Delta u) \frac{1}{r} \sin\theta r^2 d\varphi d\theta dr \xrightarrow{\varepsilon \to +0}$
    
    $\rightarrow \iiint_{D} (\Delta u) \frac{1}{r} dx$
    
    \item Оценим интегралы по $\partial D$: (примечание: оцениваются интегралы по $\partial B_\varepsilon$)
    
    $\left| \iint_{\partial B_\varepsilon} \frac{\partial u}{\partial n} \cdot \frac{1}{r} dS \right| = \left| \iint_{\partial B_\varepsilon} -\frac{\partial u}{\partial r} \cdot \frac{1}{r} dS \right| \le \iint_{\partial B_\varepsilon} \left|\frac{\partial u}{\partial r}\right| \left|\frac{1}{r}\right| dS \le \sup_D |\nabla u| \cdot \frac{1}{\varepsilon} \cdot \text{Area}(\partial B_\varepsilon) \sim \sup_D |\nabla u| \cdot \frac{1}{\varepsilon} \cdot C \varepsilon^2 \xrightarrow{\varepsilon \to +0} 0$
    
    (В рукописи: $\dots \le \sup_D |\nabla u| \cdot \frac{1}{\varepsilon} \cdot \varepsilon^2 \xrightarrow{\varepsilon \to +0} 0$)
    
    $\left(\frac{\partial}{\partial n} (\cdot) = -\frac{\partial}{\partial r} (\cdot) \text{ на } \partial B_\varepsilon\right)$
    
\end{itemize}
\hfill \framebox{2}


\newtheorem{theorem}{Теорема}
\newtheorem{corollary}{Следствие}
\theoremstyle{definition}
\newtheorem{definition}{Определение}
\theoremstyle{remark}
\newtheorem{remark}{Замечание}



\begin{itemize}
    \item Второй интеграл по теореме о среднем сходится к:
    $$ \iint_{\partial B_\varepsilon} \left(\frac{\partial}{\partial n} \frac{1}{r}\right) u \, dS = \frac{1}{\varepsilon^2} \iint_{\partial B_\varepsilon} u \, dS \xrightarrow{\varepsilon \to +0} 4\pi u(x_0) $$
    (Примечание: на $\partial B_\varepsilon$, $\frac{\partial}{\partial n} = -\frac{\partial}{\partial r}$, так что $\frac{\partial}{\partial n} \frac{1}{r} = - \left(-\frac{1}{r^2}\right) = \frac{1}{r^2} = \frac{1}{\varepsilon^2}$ на $\partial B_\varepsilon$.)
    
    $$ \implies u(x_0) = \frac{1}{4\pi} \left( -\iint_{\partial D} u \left(\frac{\partial}{\partial n} \frac{1}{r}\right) dS + \iint_{\partial D} \frac{1}{r} \left(\frac{\partial u}{\partial n}\right) dS - \iiint_D (\Delta u) \frac{1}{r} dx \right) $$
    (где $r = |x-x_0|$, $\frac{\partial}{\partial n}$ --- внешняя нормаль к $\partial D$, $u=u(x)$, $\Delta u = \Delta_x u(x)$)
\end{itemize}

\begin{theorem}[1]
Пусть ф-я $u(x) \in C^2(D) \cap C^1(\bar{D})$, $\Delta u(x) \in L_2(D)$, причём 
$$ \left. u(x) \right|_{\partial D} = u_0(x), \quad \left. \frac{\partial u(x)}{\partial n} \right|_{\partial D} = u_1(x), \quad \Delta u(x) = f(x) \quad \text{в } D. $$
Тогда для $\forall x \in D$ справедливо:
\begin{align*}
u(x) &= \frac{1}{4\pi} \left( -\iint_{\partial D} u_0(\xi) \frac{\partial}{\partial n_\xi} \left(\frac{1}{|\xi-x|}\right) dS_\xi + \iint_{\partial D} u_1(\xi) \frac{1}{|\xi-x|} dS_\xi \right. \\
&\quad \left. - \iiint_D f(\xi) \frac{1}{|\xi-x|} d\xi \right) \\
&\equiv \frac{1}{4\pi} (-J_1(x) + J_2(x) - J_3(x))
\end{align*}
\end{theorem}

\begin{definition}
\underline{Опр.} $J_1$ - потенциал двойного слоя, $J_2$ - потенциал простого слоя, $J_3$ - объёмный потенциал с плотностями $u_0, u_1, f$.
\end{definition}

\begin{corollary}[1]
\underline{Следствие 1.} Если $u(x)$ - гармоническая в $D$ ф-я из класса $C^1(\bar{D})$, удовл. на $\partial D$ усл. 
$$ \left. u(x) \right|_{\partial D} = u_0(x), \quad \left. \frac{\partial u(x)}{\partial n} \right|_{\partial D} = u_1(x), $$
то для $\forall x \in D$:
$$ u(x) = \frac{1}{4\pi} \left( \iint_{\partial D} u_1(\xi) \frac{1}{|\xi-x|} dS_\xi - \iint_{\partial D} u_0(\xi) \frac{\partial}{\partial n_\xi} \left(\frac{1}{|\xi-x|}\right) dS_\xi \right) \quad (1) $$
\end{corollary}

\begin{remark}
\underline{Замечание.} Для гармонической ф-ии $u(x)$ значения $u_1$ однозначно определяются ф-ией $u_0$ в силу доказанной единственности решения внутренней задачи Дирихле.
\end{remark}

\hfill \framebox{3}



\section*{Бесконечная дифференцируемость \quad [Уроев с. 249]}

\begin{itemize}
    \item Пусть $u(x)$ - гарм. в $D \subset \mathbb{R}^n$, докажем, что $u(x) \in C^\infty(D)$. \\
    Проведём рассуждения для $\underline{n=3}$: 
    
    Зафиксируем точку $x_0 \in D \subset \mathbb{R}^3$, рассмотрим шар 
    $B_r = \{x : |x-x_0| < r\} \subset D$, достаточно малого радиуса $r$. Тогда согласно ф-ле (1):
    $$ u(x) = \frac{1}{4\pi} \left( \iint_{\partial B_r} \frac{1}{|\xi-x|} \frac{\partial u(\xi)}{\partial n_\xi} dS_\xi - \iint_{\partial B_r} u(\xi) \frac{\partial}{\partial n_\xi} \left( \frac{1}{|\xi-x|} \right) dS_\xi \right) $$
    (Примечание: интеграл берется по $\partial B_r$, а не по $\partial D$, так как $u(x)$ гармоническая в $D$, и мы можем применить формулу Грина к любой области $B_r$ внутри $D$, где $x$ является внутренней точкой.)
    
    \item По теореме из курса мат. анализа интеграл $I(\alpha) = \int_M f(\xi, \alpha) d\xi$ на компакте $M$ можно дифференцировать по $\alpha^i \in \mathbb{R}$, если 
    $f(\xi, \alpha)$, $\frac{\partial}{\partial \alpha^i} f(\xi, \alpha) \in C(M \times U)$, где $U$ - некоторая окрестность точки параметра, где происходит дифференцирование.
    
    \item В нашем случае $x$ - параметр, $B_r$ - окрестность точки дифф. $x_0$, $\partial B_r$ - компакт.
    
    $\dfrac{1}{|\xi-x|}$ и все её производные по $x$ непр. на мн-ве $\{(\xi,x) : \xi \in \partial B_r, x \in B_r\}$ (так как $x \neq \xi$).
    
    Ф-ии $\left. u(\xi) \right|_{\partial B_r}$ и $\left. \frac{\partial u(\xi)}{\partial n_\xi} \right|_{\partial B_r}$ - не зависят от параметра $x$.
    
    $\implies$ Условия теоремы выполнены для обеих ф-ий и их производных в точке $x_0$.
    
    $\implies$ В силу произвольности $x_0 \in D$, получим $u(x) \in C^\infty(D)$.
    
\end{itemize}

\hfill \framebox{4}



\section*{2) Потенциалы, их физический смысл и св-ва (б/д) \quad [Уроев 354]}
\textit{(Сначала читать последнюю страницу.)} % This note was in the original

Пусть $S$ - кусочно гладкая пов-ть, являющаяся границей некоторой ограниченной области $D \subset \mathbb{R}^3$.

\paragraph{\underline{Опр 1}} Пусть ф-я $\mu(x)$ задана и непрерывна на пов-ти $S$.
\underline{Потенциалом простого слоя} с плотностью $\mu(x)$ называется ф-я
$$ V^{(0)}(x) = \iint_S \mu(\xi) \frac{1}{|x-\xi|} dS_\xi \quad (1) $$

\paragraph{\underline{Физический смысл:}} потенциал электрического поля, создаваемого заряженной с плотностью зарядов $\mu$ пов-ю $S$ в точке $x$.

\paragraph{\underline{Опр 2}} Пусть ф-я $D(x)$ задана и непрерывна на пов-ти $S$.
\underline{Потенциалом двойного слоя} с плотностью $D(x)$ наз. ф-я
$$ V^{(1)}(x) = \iint_S D(\xi) \frac{\partial}{\partial n_\xi} \left( \frac{1}{|x-\xi|} \right) dS_\xi \quad (2) $$
где $n_\xi$ - внешняя нормаль по отношению к обл. $D$.

\paragraph{\underline{Физический смысл:}} на пов-ти $S$ с плотностью $D$ расположены электрические диполи с данными дипольными моментами, направленными по нормали к $S$. Рис А.
\begin{center}
ИЛИ
\end{center}
потенциал поля, созданного двумя заряженными пов-ми $S_{+\delta}$ и $S_{-\delta}$ с плотностями $D/\delta$ и $-D/\delta$ и расстоянием между ними $2\delta \to 0$. Рис. Б

\vspace{0.5cm}

\begin{minipage}{0.45\textwidth}
\centering
\begin{tikzpicture}[scale=0.8]
    \draw (0,0) ellipse (2.5cm and 1.5cm);
    \node at (0,-0.5) {$D$};
    \node at (2.6,0) {$S$};
    
    \foreach \angle/\pos in {0/{(2.5,0)}, 45/{(2.5*cos(45),1.5*sin(45))}, 90/{(0,1.5)}, 135/{(-2.5*cos(135),1.5*sin(135))}, 180/{(-2.5,0)}, 225/{(-2.5*cos(225),-1.5*sin(225))}, 270/{(0,-1.5)}, 315/{(2.5*cos(315),-1.5*sin(315))}} {
        \pgfmathsetmacro{\nx}{\posx/2.5}
        \pgfmathsetmacro{\ny}{\posy/1.5}
        \pgfmathsetmacro{\norm}{sqrt(\nx*\nx + \ny*\ny)}
        \pgfmathsetmacro{\ux}{\nx/\norm}
        \pgfmathsetmacro{\uy}{\ny/\norm}
        \draw[->, thick] \pos -- ++(\ux*0.7, \uy*0.7) node[right, scale=0.7] {$D$};
        \node at ($(\pos) + (0.15*\ux, 0.15*\uy)$) [scale=0.7]{$+$};
        \node at ($(\pos) - (0.15*\ux, 0.15*\uy)$) [scale=0.7]{$-$};
    }
\end{tikzpicture}
\\ Рис. А
\end{minipage}
\hfill
\begin{minipage}{0.45\textwidth}
\centering
\begin{tikzpicture}[scale=0.8]
    % Original surface S (dashed)
    \draw[dashed] (0,0) .. controls (1.5,0.5) and (2.5,-0.5) .. (3.5,0.3) node[below right, scale=0.8] {$S$};
    
    % S_(-delta)
    \begin{scope}[yshift=-0.2cm, xshift=-0.1cm, decoration={markings, mark=between positions 0.1 and 0.9 step 0.4cm with {\node[transform shape, scale=0.7]{$-$};}}]
        \draw[postaction={decorate}] (0.1,0.1) .. controls (1.5,0.5) and (2.5,-0.5) .. (3.4,0.4);
        \node at (0.5, -0.3) [scale=0.7] {$S_{-\delta}$};
        \node at (3.0, -0.2) [scale=0.7] {$-D/\delta$};
    \end{scope}
    
    % S_(+delta)
    \begin{scope}[yshift=0.2cm, xshift=0.1cm, decoration={markings, mark=between positions 0.1 and 0.9 step 0.4cm with {\node[transform shape, scale=0.7]{$+$};}}]
        \draw[postaction={decorate}] (-0.1,-0.1) .. controls (1.5,0.5) and (2.5,-0.5) .. (3.6,0.2);
        \node at (0.5, 0.7) [scale=0.7] {$S_{+\delta}$};
        \node at (3.0, 0.6) [scale=0.7] {$+D/\delta$};
    \end{scope}
    
    % Distance 2delta
    \draw[<->] (1.8, -0.2) -- (1.8, 0.2) node[midway, right, scale=0.7] {$2\delta$};
    
    \node at (2.5, -1) {$D$}; % Domain D
\end{tikzpicture}
\\ Рис. Б
\end{minipage}

\vspace{1cm}
\hfill \framebox{5}



\section*{Свойства потенциалов}

\begin{definition}[Поверхность Ляпунова]
\underline{Опр 2.} Пов-ть $S$ в пространстве $\mathbb{R}^3$ наз. \underline{поверхностью Ляпунова}, если выполнены следующие условия:
\begin{enumerate}
    \item В каждой точке $x$ пов-ти $S$ сущ. касательная пл-ть и $\implies$ единичная нормаль $n_x$.
    \item Существует число $k_0 > 0$, что для каждой точки $x$ часть пов-ти $S$, заключенная внутри сферы радиуса $k_0$ с центром в этой т. $x$, является связным множеством и пересекается не более одного раза любой прямой параллельной нормали $n_x$. (Примечание: "является связным множеством, которое можно представить как график функции в локальных координатах, где ось $z$ направлена по $n_x$".)
    \item Существуют такие числа $C > 0$ и $\alpha: 0 < \alpha \le 1$, что для единичных нормалей $n_x$ и $n_y$ в произвольных двух точках пов-ти $S \implies |n_x - n_y| \le C/|x-y|^\alpha$. (Примечание: это условие Гёльдера на нормаль. Ошибка в записи $C/|x-y|^\alpha$, должно быть $C|x-y|^\alpha$).
\end{enumerate}
К пов-м Ляпунова, например, относятся дважды непрерывно дифференцируемые пов-ти.
\end{definition}

\begin{theorem}[1]
\underline{Теорема 1.} Пусть $D$ - ограниченная область в $\mathbb{R}^3$, границей которой является пов-ть Ляпунова $S$. Тогда значение потенциала двойного слоя $V^{(1)}(x)$ с постоянной плотностью $D \equiv \text{const}$ зависит только от расположения точки $x$ по отношению к области (вне, на границе, внутри), а от самой области не зависит.
$$ V^{(1)}(x) = \begin{cases} 0, & x \notin \bar{D} \\ -2\pi D, & x \in S \\ -4\pi D, & x \in D \end{cases} \quad \text{--- имеет разрыв на } S. $$
\end{theorem}

\begin{itemize}
    \item Потенциал же простого слоя - всегда непрерывная ф-я во всем простр-ве $\mathbb{R}^3$.
\end{itemize}

\paragraph{Пример:} равномерно заряженная сфера радиуса $R$.
\begin{center}
\begin{tikzpicture}
    \draw[->] (0,0) -- (5,0) node[right] {$r$};
    \draw[->] (0,0) -- (0,3) node[above] {$V^{(0)}$};
    
    \draw (0,2.5) node[left] {$4\pi \mu R$}; % Assuming mu is surface charge density
    \draw[thick] (0,2.5) -- (2,2.5);
    \draw[thick] (2,2.5) .. controls (2.5,2.5) and (3,1.5) .. (4.5,0.5);
    
    \draw[dashed] (2,0) -- (2,2.5);
    \node at (2,-0.3) {$R$};
\end{tikzpicture}
\end{center}
(Примечание: для сферы радиуса $R$ с равномерной поверхностной плотностью заряда $\mu$, потенциал $V^{(0)}(r) = \frac{Q}{4\pi\epsilon_0 r}$ для $r \ge R$ и $V^{(0)}(r) = \frac{Q}{4\pi\epsilon_0 R}$ для $r \le R$, где $Q=4\pi R^2 \mu$. В системе единиц, где $1/(4\pi\epsilon_0)=1$, $V^{(0)}(r) = \frac{4\pi R^2 \mu}{r}$ для $r \ge R$ и $V^{(0)}(r) = 4\pi R \mu$ для $r \le R$. График соответствует этому, если на оси ординат $4\pi \mu R$.)


\hfill \framebox{6}



\section*{Все еще свойства потенциалов}

Пусть $S$ - поверхность Ляпунова, являющаяся границей ограниченной области $D \subset \mathbb{R}^3$, $\mu(x)$ и $D(x)$ - непрерывные на пов-ти $S$ плотности потенциалов простого и двойного слоя соотв. Тогда справедливо:
\begin{enumerate}
    \item Потенциалы $V^{(0)}(x)$ и $V^{(1)}(x)$ являются гармоническими в областях $D$ и $\mathbb{R}^3 \setminus \bar{D}$.
    \item $V^{(0)}(x)$ и $V^{(1)}(x)$ являются регулярными на $\infty$.
    \item $V^{(0)}(x)$ непрерывен в $\mathbb{R}^3$. $V^{(1)}(x)$ имеет разрыв первого рода на пов-ти $S$.
\end{enumerate}

\begin{remark}
\underline{Замечание.} Потенциал $V^{(1)}(x)$ не просто регулярная на бесконечности ф-я, на самом деле он убывает при $|x| \to \infty$ как $O(|x|^{-2})$.
\end{remark}

\hrulefill
\begin{center}
    Старт потенц.-в
\end{center}

\section*{Потенциал Ньютона (объёмный потенциал) \quad [Владимиров 286]}

\begin{definition}[Опр 1]
Можно определить при помощи обобщенных ф-й как свертку обобщенной ф-и $\rho$ (плотности) с ф-ёй $|x|^{-1}$.
$$ \boxed{V = \frac{1}{|x|} * \rho} $$
\end{definition}

\begin{definition}[Опр 2]
Если $\rho$ - (абсолютно) интегрируемая ф-я на $D$ и $\rho(x)=0, x \in D_1 = \mathbb{R}^3 \setminus \bar{D}$, то ньютонов потенциал $V$ называется \underline{объёмным потенциалом} и выражается ф-лой:
$$ V(x) = \int_D \frac{\rho(\xi)}{|x-\xi|} d\xi $$
и представляет собой локально интегрируемую ф-ю в $\mathbb{R}^3$.
\end{definition}

Потенциал $V$ удовлетворяет ур-ю Пуассона $\Delta V = -4\pi \rho$. \\
\underline{Физический смысл:} $V$ - потенциал, создаваемый зарядами, распределенными в пространстве с плотностью $\rho$.

Отсюда вытекают потенциалы $V^{(0)}, V^{(1)}$ как частн. сл. распред. зар. (на пов-ти).

\hfill \framebox{7}







\newpage
\section*{Билет №16 -- 2025}\label{sec:ticket16}
\backtotoc

\section*{1) Функция Грина задачи Дирихле [Уроев 256]}

Обсуждаются уравнения Лапласа и Пуассона в $\mathbb{R}^3$.

\medskip

\noindent\uline{Опр1.} Пусть $D$ - область в $\mathbb{R}^3$. \uline{Функцией Грина задачи Дирихле} для этой области называется ф-я $G(x, \xi)$, определенная для любых $x \in D$ и $\xi \in \bar{D}$; $\xi \neq x$, принадлежащая при каждом фиксированном $x$ из $D$ классу
$C_{\xi}^2(D \setminus \{x\}) \cap C_{\xi}(\bar{D} \setminus \{x\})$ по переменной $\xi$ и удовлетворяющая следующим требованиям:
\begin{enumerate}
    \item $G(x, \xi)|_{\xi \in \partial D} = 0$
    \item $\Delta_{\xi} G(x, \xi) = 0, \quad \xi \in D \setminus \{x\}$
    \item $G(x, \xi) = \frac{1}{4\pi |x-\xi|} + o\left(\frac{1}{|x-\xi|}\right), \quad \xi \rightarrow x$
\end{enumerate}

\medskip

\noindent\uline{Замечание 1.} \uline{В данном случае $x$ выступает в роли парам-ра.}

\medskip

Введем ф-ю $g(x, \xi) = G(x, \xi) - \frac{1}{4\pi |x-\xi|}$.

Для фикс. $x \in D$ эта ф-я принадл. классу $C_{\xi}^2(D \setminus \{x\}) \cap C_{\xi}(\bar{D} \setminus \{x\})$ и явл. гармонической в области $D \setminus \{x\}$ по переменной $\xi$, как разность двух ф-ий, обладающих этими св-ми.
При $\xi \rightarrow x$ ф-я $g(x, \xi) = o\left(\frac{1}{|x-\xi|}\right)$. Следовательно, по т. о стирании особенности (т. об устранимой особенности) $g(x, \xi)$ имеет предел при $\xi \rightarrow x$ и после доопределения по непрерывности в т. $\xi=x$ становится гармонич. по $\xi$ во всей области $D$.

$\Rightarrow$ Условия 2) и 3) эквивалентны условию 2').



2') $G = \frac{1}{4\pi|x-\xi|} + g(x,\xi)$, где при каждом фикс. $x \in D$ ф-я $g(x,\xi)$ является гармонической в области $D$ ф-ей по переменной $\xi$ из класса $C_{\xi}(\bar{D})$.

\medskip

\noindent\uline{Замечание 2}

Таким образом, построение ф-ии Грина эквивалентно решению задач Дирихле для каждого $x \in D$:
\[
\left.
\begin{array}{l}
g(x,\xi) \in C^2_{\xi}(D) \cap C_{\xi}(\bar{D}) \\
\Delta_{\xi} g(x,\xi) = 0, \quad \xi \in D \\
g(x,\xi)|_{\xi \in \partial D} = -\frac{1}{4\pi|x-\xi|}
\end{array}
\right\} (*)
\]

\section*{\uline{Свойства ф-ии Грина}}

\noindent\textbf{\uline{Теорема 1.}} Пусть $D$ - огр. область в $\mathbb{R}^3$ с такой достаточно гладкой границей $\partial D$, что можно определить правильную нормальную производную на $\partial D$, тогда:
\begin{enumerate}
    \item существует и притом единственная ф-я Грина $G(x,\xi)$ для области $D$;
    \item функция $G(x,\xi)$ имеет на границе $\partial D$ области $D$ правильную нормальную производную $\frac{\partial}{\partial n_{\xi}}G(x,\xi)$, $(x,\xi) \in D \times \partial D$;
    \item ф-я $G(x,\xi)$ явл. симметричной: $G(x,\xi) = G(\xi,x)$ для $x,\xi \in D, \xi \neq x$;
    \item ф-я $G(x,\xi)$ явл. непрерывной на множестве $M = \{(x,\xi): x \in D, \xi \in D, x \neq \xi\}$.
\end{enumerate}

\medskip

\noindent\uwave{Д-во:} 1) Воспользуемся фактом из Замечания 2. Заметим, что ф-я $-\frac{1}{4\pi|x-\xi|}$, входящая в граничное условие, является непрерывной по $\xi$ на $\partial D$ для $\forall x \in D$. $\Rightarrow$ решения этих задач Дирихле сущ-ют, и единственность следует из ед-ти решения задачи Дирихле (билет № 20).





\noindent
2) без док-ва ($\ddot\smile$) % Alternatively: (смайлик)

\noindent
3) Пусть $x_1, x_2$ - внутр. точки $D$, $x_1 \neq x_2$. Рассмотрим две ф-ии переменной $\xi$: $f_1(\xi) = G(x_1, \xi)$; $f_2(\xi) = G(x_2, \xi)$.
Пусть $B_1$ и $B_2$: $\bar{B}_1, \bar{B}_2 \subset D$ - непересекающиеся шары с центрами в точках $x_1$ и $x_2$ соответственно. Ф-ии $f_1$ и $f_2$ - гармонические в области $D \setminus (\bar{B}_1 \cup \bar{B}_2)$. Воспользуемся второй ф-лой Грина для этой области, разбив интеграл по пов-ти на три:
\begin{align*}
0 &= \iiint_{D \setminus (\bar{B}_1 \cup \bar{B}_2)} (f_2(\xi) \Delta_{\xi} f_1(\xi) - f_1(\xi) \Delta_{\xi} f_2(\xi)) d\xi \tag{**} \\
&= \iint_{\partial D} \left(f_2 \frac{\partial f_1}{\partial n} - f_1 \frac{\partial f_2}{\partial n}\right) ds + \iint_{\partial B_1} \left(f_2 \frac{\partial f_1}{\partial n} - f_1 \frac{\partial f_2}{\partial n}\right) ds \\
&\quad + \iint_{\partial B_2} \left(f_2 \frac{\partial f_1}{\partial n} - f_1 \frac{\partial f_2}{\partial n}\right) ds,
\end{align*}
где $n$ - единичная нормаль по отношению к области $D \setminus (\bar{B}_1 \cup \bar{B}_2)$. $\iint_{\partial D} \dots ds = 0$, т.к. $f_1(\xi)|_{\xi \in \partial D} = G(x_1, \xi)|_{\xi \in \partial D} = 0$ и $f_2(\xi)|_{\xi \in \partial D} = G(x_2, \xi)|_{\xi \in \partial D} = 0$.

Воспользуемся теоремой 0 (см. дальше):
\begin{align*}
f_1(x_2) &= -\iint_{\partial B_2} f_1(\xi) \frac{\partial G(x_2, \xi)}{\partial n_{\xi}} ds_{\xi} + \iint_{\partial B_2} G(x_2, \xi) \frac{\partial f_1(\xi)}{\partial n_{\xi}} ds_{\xi} \\
&= \iint_{\partial B_2} f_1 \frac{\partial f_2}{\partial n} ds - \iint_{\partial B_2} f_2 \frac{\partial f_1}{\partial n} ds
\end{align*}
Аналогично: $f_2(x_1) = \iint_{\partial B_1} f_2 \frac{\partial f_1}{\partial n} ds - \iint_{\partial B_1} f_1 \frac{\partial f_2}{\partial n} ds$.

Здесь $n_{\xi}$ - внешняя нормаль к шарам $B_1$ и $B_2$, $n$ - внутренняя.
Объединяя с выражением (**), получаем:
\[ 0 = f_2(x_1) - f_1(x_2) = G(x_2, x_1) - G(x_1, x_2) \]
4) Непрерывность является следствием симметричности.
Пусть $(x_0, \xi_0)$ и $(x, \xi)$ принадлежат $M$, тогда
\[ |G(x, \xi) - G(x_0, \xi_0)| \le |G(x, \xi) - G(x, \xi_0)| + |G(x, \xi_0) - G(x_0, \xi_0)| = \dots \]
% The equation is cut off in the image


\noindent
$= |G(x,\xi) - G(x,\xi_0)| + |G(\xi_0, x) - G(\xi_0, x_0)|$. \\
Т.к. $G(x,\xi)$ непрерывна по $\xi$, $|G(x,\xi) - G(x_0, \xi_0)| \to 0$, при $(x,\xi) \to (x_0, \xi_0)$.

\vspace{\baselineskip}

\noindent\uwave{Замечание 2} Для изучения св-в ф-ий Грина для неогранич. областей применяется более тонкий анализ.
\hfill [Уроев 245]

\vspace{\baselineskip}

\noindent\textbf{\uline{Теорема 0.}} Для ф-ии $u(x)$, удовлетворяющей условиям
$u(x) \in C^2(D) \cap C^1(\bar{D})$, $\Delta u(x) = L_2(D)$, причем $u(x)|_{\partial D} = u_0(x)$,
$\frac{\partial u(x)}{\partial n}|_{\partial D} = u_1(x)$, $\Delta u(x)|_D = f(x)$, справедливо следующее интегральное представление:
\begin{align*}
u(x) = &-\iint_{\partial D} u_0(\xi) \frac{\partial}{\partial n_{\xi}} G(x,\xi) ds_{\xi} + \iint_{\partial D} u_1(\xi) G(x,\xi) ds_{\xi} \\
&- \iiint_D f(\xi) G(x,\xi) d\xi, \quad x \in D
\end{align*}
где $G(x,\xi) = \frac{1}{4\pi|x-\xi|} + g(x,\xi)$, $g(x,\xi)$ - ф-я на $D \times \bar{D}$, которая при каждом фикс. $x \in D$ явл. гармонической в $D$ по переменной $\xi$ и принадлежит классу $C^1_{\xi}(\bar{D})$.

\vspace{\baselineskip}

\noindent\uwave{Д-во:} Запишем вторую ф-лу Грина для ф-ий $u(\xi)$ и $g(x,\xi)$:
\[
\iiint_D (g(x,\xi) \Delta_{\xi} u(\xi) - u(\xi) \Delta_{\xi} g(x,\xi)) d\xi = \iint_{\partial D} g(x,\xi) \frac{\partial u(\xi)}{\partial n_{\xi}} ds_{\xi} - \iint_{\partial D} u(\xi) \frac{\partial g(x,\xi)}{\partial n_{\xi}} ds_{\xi}
\]
т.к. $\Delta_{\xi} g(x,\xi) = 0 \quad \Rightarrow$
\[
\iiint_D g(x,\xi) \Delta u(\xi) d\xi = \iint_{\partial D} g(x,\xi) \frac{\partial u(\xi)}{\partial n_{\xi}} ds_{\xi} - \iint_{\partial D} u(\xi) \frac{\partial g(x,\xi)}{\partial n_{\xi}} ds_{\xi}
\]
Остается сложить с интегральным представл. $u(x)$:
\begin{align*}
u(x) = &-\iint_{\partial D} u_0(\xi) \frac{\partial}{\partial n_{\xi}} \left( \frac{1}{4\pi|x-\xi|} + g(x,\xi) \right) ds_{\xi} \\
&+ \iint_{\partial D} u_1(\xi) \left( \frac{1}{4\pi|x-\xi|} + g(x,\xi) \right) ds_{\xi} \\
&- \iiint_D f(\xi) \left( \frac{1}{4\pi|x-\xi|} + g(x,\xi) \right) d\xi
\end{align*}




\section*{\uline{Использование ф-ии Грина для решения задачи Дирихле}}
\hfill [Уроев 261]

\noindent\textbf{\uline{Теорема 1.}} Пусть $D \subset \mathbb{R}^3$ - огр. обл. с достаточно гладкой границей $\partial D$, $G(x,\xi)$ - ф-я Грина для $D$, $u_0(x) \in C(\partial D)$ и $f(x) \in C(D) \cap L_2(D)$ - заданные ф-ии. Если решение $u(x)$ внутренней задачи Дирихле для ур-я Пуассона
\[ \Delta u = f(x), \quad x \in D \]
с граничными условиями
\[ u(x)|_{\partial D} = u_0(x) \]
существует в классе $C^2(D) \cap C(\bar{D})$, причем $u(x)$ имеет на границе $\partial D$ правильную нормальную производную $\frac{\partial u}{\partial n_{\xi}}$, то справедлива следующая ф-ла:
\begin{equation}
u(x) = -\iint_{\partial D} u_0(\xi) \frac{\partial G(x,\xi)}{\partial n_{\xi}} ds_{\xi} - \iiint_D f(\xi) G(x,\xi) d\xi, \quad x \in D \label{eq:green_representation_poisson} \tag{¥}
\end{equation}

\vspace{\baselineskip}

\noindent\uwave{Д-во:} При условии сущ-я решения для него можно воспользоваться ф-лой из теоремы 0:
\[
u(x) = \iint_{\partial D} G(x,\xi) \frac{\partial u(\xi)}{\partial n_{\xi}} ds_{\xi} - \iint_{\partial D} u_0(\xi) \frac{\partial G(x,\xi)}{\partial n_{\xi}} ds_{\xi} - \iiint_D f(\xi) G(x,\xi) d\xi, \quad x \in D
\]
т.к. $G(x,\xi)|_{\xi \in \partial D} = 0$, то первый интеграл $\iint_{\partial D} G(x,\xi) \frac{\partial u(\xi)}{\partial n_{\xi}} ds_{\xi} = 0$.
Получаем \eqref{eq:green_representation_poisson}.

\vspace{\baselineskip}

\noindent\uline{Замечание 3.} Сформулированные требования на $f(x)$ не обеспечивают сущ-е решения.

\section*{\uline{2) Интеграл Пуассона}}

Построим ф-ю Грина для шара $B_R \subset \mathbb{R}^3$ радиуса $R$ с центром в т. $O$, в которую перенесено начало координат.
Пусть $x^*$ - образ точки $x \neq 0$ при инверсии относительно сферы $\partial B_R$, то есть
\[
x^* = \frac{R^2}{|x|^2} x, \quad x \neq 0
\]





Покажем, что для фикс. точки $x \neq 0$ отношение $\frac{|\xi-x^*|}{|\xi-x|}$ при $\xi \in \partial B_R$ равно \text{const}:
\begin{align*}
|\xi-x^*|^2 &= |\xi|^2 + |x^*|^2 - 2(\xi, x^*) = R^2 + \frac{R^4}{|x|^2} - 2(\xi, x) \frac{R^2}{|x|^2} \\
&= \frac{R^2}{|x|^2} (|x|^2 + R^2 - 2(\xi, x)) = \frac{R^2}{|x|^2} |\xi-x|^2
\end{align*}
$\Rightarrow \frac{|\xi-x^*|}{|\xi-x|} = \frac{R}{|x|}$, \quad $x \neq 0, \xi \in \partial B_R$.

\medskip

Поместим в точку $x^*$ воображаемый заряд $Q$, в каждой точке границы шара $B_R$ "уравновешивающий" поле единичного заряда, расположенного в т. $x$.
\[
\left( \frac{1}{|\xi-x|} + \frac{Q}{|\xi-x^*|} \right) \Bigg|_{\xi \in \partial B_R} = 0
\]
$\Rightarrow Q = -\frac{|\xi-x^*|}{|\xi-x|} \Bigg|_{\xi \in \partial B_R} = -\frac{R}{|x|}$ \quad \framebox{$\Delta$}

\medskip

$\Rightarrow$ Ф-я Грина для шара в т. $x \neq 0$ имеет вид:
\[
G(x,\xi) = \frac{1}{4\pi} \left( \frac{1}{|\xi-x|} - \frac{R}{|x||\xi-x^*|} \right)
\]
Доопределим ее для $x=0$ по непрерывности:
\begin{equation}
G(x,\xi) =
\begin{cases}
\frac{1}{4\pi} \left( \frac{1}{|\xi-x|} - \frac{R}{|x||\xi-x^*|} \right), & x \neq 0 \\
\frac{1}{4\pi} \left( \frac{1}{|\xi|} - \frac{1}{R} \right), & x=0
\end{cases}
\label{eq:green_sphere} \tag{¥¥}
\end{equation}

\medskip

Построенная ф-я удовл-ет всем св-м ф-ии Грина:
\begin{enumerate}
    \item $G(x,\xi) \equiv 0, \xi \in \partial B_R$
    \item $\Delta_{\xi} G = 0$
    \item При фикс $x \in B_R$, для $\xi \to x \Rightarrow G(x,\xi) = \frac{1}{4\pi|\xi-x|} + o\left( \frac{1}{|\xi-x|} \right)$
\end{enumerate}




Найдем производную ф-ии $G(x,\xi)$ из (\ref{eq:green_sphere}) по внешней единичной нормали $n_{\xi} = \frac{\xi}{R}$ к пов-ти $\partial B_R$. Зафиксируем произвольную точку $y \in \mathbb{R}^3$ и проведем вычисления:
\[
\left. \frac{\partial}{\partial n_{\xi}} \frac{1}{|\xi-y|} \right|_{\xi \in \partial B_R} = \left( \nabla_{\xi} \frac{1}{|\xi-y|}, \frac{\xi}{R} \right) = \left( -\frac{\xi-y}{|\xi-y|^3}, \frac{\xi}{R} \right) = \frac{(y,\xi)-R^2}{R|\xi-y|^3}
\]
$\Rightarrow$ для $x \neq 0$ справедливо:
\begin{align*}
\left. \frac{\partial}{\partial n_{\xi}} G(x,\xi) \right|_{\xi \in \partial B_R} &= \frac{1}{4\pi} \left. \frac{\partial}{\partial n_{\xi}} \frac{1}{|\xi-x|} \right|_{\xi \in \partial B_R} - \frac{R}{4\pi|x|} \left. \frac{\partial}{\partial n_{\xi}} \frac{1}{|\xi-x^*|} \right|_{\xi \in \partial B_R} \\
&= \frac{1}{4\pi} \left( \frac{(x,\xi)-R^2}{R|\xi-x|^3} - \frac{R}{|x|} \frac{(x^*,\xi)-R^2}{R|\xi-x^*|^3} \right) \stackrel{(\Delta)}{=} \\
&\stackrel{(\Delta)}{=} \frac{1}{4\pi R |\xi-x|^3} \left( (x,\xi)-R^2 - \frac{|x|^2}{R^2} \left( (x,\xi)\frac{R^2}{|x|^2} - R^2 \right) \right) \\
&= \frac{|x|^2-R^2}{4\pi R |\xi-x|^3}
\end{align*}
а для $x=0$:
\[
\left. \frac{\partial}{\partial n_{\xi}} G(x,\xi) \right|_{\xi \in \partial B_R} = \frac{1}{4\pi} \left. \frac{\partial}{\partial n_{\xi}} \frac{1}{|\xi|} \right|_{\xi \in \partial B_R} = -\frac{R}{4\pi|\xi|^3} = -\frac{1}{4\pi R^2}
\]
$\Rightarrow$
\[
\left. \frac{\partial}{\partial n_{\xi}} G(x,\xi) \right|_{\xi \in \partial B_R} = \frac{|x|^2-R^2}{4\pi R |\xi-x|^3}
\]
Применение этой ф-лы вместе с теоремой 1 дает интегральное представление для ф-ии гармонической в шаре.

\medskip

\noindent\textbf{\uwave{Теорема 3 [ ф-ла Пуассона ]}}

Пусть ф-я $u_0(x)$ непрерывна на сфере $\partial B_R = \{x: |x|=R\}$.
Тогда решение $u(x)$ задачи Дирихле для ур-я Лапласа в шаре $B_R = \{x: |x|<R\}$
\[ \Delta u(x) = 0, \quad x \in B_R \]
с гр. условиями: $u(x)|_{\partial B_R} = u_0(x)$ представляется интегралом Пуассона:





\begin{equation*}
\boxed{
u(x) = \frac{R^2-|x|^2}{2\pi R} \iint_{|\xi|=R} \frac{u_0(\xi)}{|x-\xi|^3} ds_{\xi}
}
\end{equation*}

\medskip

\noindent\uline{Замечание.} Формула получена в предположении, что решение сущ-ет. Однако гармоничность ф-ии задаваемой данной ф-лой, внутри шара $B_R$ очевидна, а выполнение краевых условий можно было бы проверить предельным переходом при $|x| \to R-0$.








\newpage
\section*{Билет №17 -- 2025}\label{sec:ticket17}
\backtotoc
\textbf{Теорема о среднем для гармонических функций. Принцип максимума для гармонических
функций в ограниченной области. Единственность классического решения задачи Дирихле для
уравнения Пуассона в ограниченной области. [2] – 251-254.}




Теорема о среднем для гармонических ф-ий. Принцип максимума для гармонических ф-ий в ограниченной области. Единственность классического решения задачи Дирихле для уравнения Пуассона в ограниченной области.

\hrulefill

\noindent \textbf{Теорема о среднем} \quad [\text{Уравн. с. 251}] \hfill \boxed{W 19} \\

\noindent \textbf{Лемма 1} Пусть $D$ - огранич. обл. в $\mathbb{R}^n$ с кус. гл. границей, $u(x)$ - гарм. в $D$, $u(x) \in C^1(\overline{D})$, тогда выполнено:
$$ \oint_{\partial D} \frac{\partial u(\xi)}{\partial n_{\xi}} dS_{\xi} = 0 $$

\noindent \textit{\textbf{Док-во:}} вторая формула Грина для обл. $D$ и $u(x)$, $v(x) \equiv 1$.
$$ \int_D (v \Delta u - u \Delta v) dx = 0 = \int_{\partial D} \left(v \frac{\partial u}{\partial n} - u \frac{\partial v}{\partial n}\right) dS = \int_{\partial D} \frac{\partial u}{\partial n} dS $$
% Note: The original has a dot at the end of this line, which might be a period or a smudge.
% If u is harmonic, \Delta u = 0. If v=1, \Delta v = 0. So \int_D 0 dx = 0.
% (v \frac{\partial u}{\partial n} - u \frac{\partial v}{\partial n}) = (1 \cdot \frac{\partial u}{\partial n} - u \cdot 0) = \frac{\partial u}{\partial n}.
% So the equality holds.

\hrulefill

\noindent \textbf{Теорема 1 (о среднем)} \\
Пусть $u(x)$ - гарм. в шаре $B_R = \{x: |x-x_0| < R\} \subset \mathbb{R}^n$, $u(x) \in C^1(\overline{B}_R)$, тогда справедливо:
$$ u(x_0) = \frac{1}{C_{\partial B_R}} \int_{\partial B_R} u(\xi) d\xi \quad , \text{где } C_{\partial B_R} = \frac{2\pi^{n/2}}{\Gamma(n/2)} R^{n-1} \text{ - площадь сферы } \partial B_R $$
В частности, для $n=3$: $u(x_0) = \frac{1}{4\pi R^2} \iint_{|\xi-x_0|=R} u(\xi) dS_{\xi}$

\noindent \textit{\textbf{Док-во:}} для \underline{n=3}. \\
Для обл. $B_R$ и ф-ии $u(x)$ воспользуемся инт. представл.:
$$ u(x_0) = \frac{1}{4\pi} \left( \iint_{\partial B_R} \frac{1}{|\xi-x_0|} \frac{\partial u(\xi)}{\partial n_{\xi}} dS_{\xi} - \iint_{\partial B_R} u(\xi) \frac{\partial}{\partial n_{\xi}} \frac{1}{|\xi-x_0|} dS_{\xi} \right) $$
найдем производную по нормали: $\frac{\partial}{\partial n_{\xi}} \frac{1}{|\xi-x_0|} = \left. \frac{\partial}{\partial r} \frac{1}{r} \right|_{r=R} = -\frac{1}{R^2}$
$$ \Rightarrow u(x_0) = \frac{1}{4\pi} \left( \frac{1}{R} \iint_{\partial B_R} \frac{\partial u(\xi)}{\partial n_{\xi}} dS_{\xi} + \frac{1}{R^2} \iint_{\partial B_R} u(\xi) dS_{\xi} \right) = \frac{1}{4\pi R^2} \iint_{\partial B_R} u(\xi) dS_{\xi} $$
% The first integral term vanishes because of Lemma 1: \iint_{\partial B_R} \frac{\partial u(\xi)}{\partial n_{\xi}} dS_{\xi} = 0
% as u is harmonic in B_R.
% So u(x_0) = \frac{1}{4\pi} ( \frac{1}{R} \cdot 0 + \frac{1}{R^2} \iint_{\partial B_R} u(\xi) dS_{\xi} ) = \frac{1}{4\pi R^2} \iint_{\partial B_R} u(\xi) dS_{\xi}
% This is consistent.



\underline{Строгий принцип максимума}

\noindent \textbf{Теорема 2 (Строгий принцип максимума)} \\
Пусть $D$ - огранич. обл. в $\mathbb{R}^n$, $u(x)$ - гарм. ф-я в $D$, $u \in C(\overline{D})$, тогда, если сущ. т. $x_0 \in D$, в которой достигается максимум (минимум) ф-ии $u(x)$ на $\overline{D}$, то $u(x) \equiv \text{const} = u(x_0)$ в $\overline{D}$.

\noindent \textit{\textbf{Док-во:}} \\
докажем сначала, что если $C = \max_{x \in \overline{B}_R} u(x)$ достигается в центре $x_0$ шара $B_R \subset D$, то $u(x) \equiv u(x_0)$ в $\overline{B}_R$.
\begin{itemize}
    \item Рассмотрим произвольную сферу $S_r$ с центром в $x_0$, радиуса $r < R$. \\
    \underline{От противного:} если $u(x) \not\equiv u(x_0)$ на $S_r$, то в силу непр. $u(x)$ с учётом \underline{теоремы о среднем} справедливо:
    $$ u(x_0) = \frac{1}{C_{S_r}} \int_{S_r} u(\xi) dS_{\xi} < \frac{1}{C_{S_r}} C \cdot C_{S_r} = u(x_0) $$
    % Note: C_{S_r} is the surface area of the sphere S_r. C = u(x_0) is the maximum value.
    % If u(x) is not identically u(x_0) on S_r, and u(x_0) is the maximum, then u(\xi) <= u(x_0) on S_r
    % and u(\xi) < u(x_0) on some subset of S_r of non-zero measure (due to continuity).
    % Thus \int_{S_r} u(\xi) dS_{\xi} < \int_{S_r} u(x_0) dS_{\xi} = u(x_0) C_{S_r}.
    % So \frac{1}{C_{S_r}} \int_{S_r} u(\xi) dS_{\xi} < u(x_0).
    % The inequality written in the text is correct because C = u(x_0).
    $\Rightarrow$ противоречие $\Rightarrow$ утв. верное.
    \item Пусть в т. $x_0 \in D$ достигается макс. $C = \max_{x \in \overline{D}} u(x)$. \\
    Рассмотрим произвольную $X \in D$, т.к. $D$ - область (связное открытое множество), то $x_0$ и $X$ можно соединить непр. кривой $\Gamma$, лежащей в $D$.
    \begin{itemize}
        \item[-] Построим для каждой точки кривой шар с центром в этой точке, лежащий вместе со своим замыканием в $D$.
        \item[-] Выделим конечное покрытие кривой $\Gamma$ (\underline{лемма Гейне-Бореля})
        \item[-] Измельчим это покрытие так, чтобы центр каждого следующего шара лежал внутри предыдущего, разместив центр первого шара в т. $x_0$. (Это можно сделать, т.к. в конечном покрытии найдётся шар минимального радиуса)
    \end{itemize}
\end{itemize}



\begin{figure}[ht]
\centering
\begin{tikzpicture}[scale=1.5]
    % Domain D
    \draw[thick] plot [smooth cycle, tension=0.8] coordinates {(0,1) (1.5,1.8) (4,1.5) (5,0.5) (3.5,-0.5) (0.5,0)};
    \node at (0.3,1.3) {$D$};

    % Points and Path Gamma
    \coordinate (x0) at (1,0.8);
    \coordinate (x1) at (1.2,0.6);
    \coordinate (x2prime) at (1.5,0.9); % x_2'
    \coordinate (X) at (4,0.9);
    \coordinate (XN) at (3.8,0.7); % x_N

    \fill (x0) circle (1pt) node[below left] {$x_0$};
    \fill (x1) circle (1pt) node[below right] {$x_1$};
    \fill (x2prime) circle (1pt) node[above right] {$x_2'$}; % x_2'
    \fill (X) circle (1pt) node[above right] {$X$};
    \fill (XN) circle (1pt) node[below left] {$x_N$};

    \draw[thick, ->] (x0) to[out=0, in=180, looseness=0.8] (X);
    \node at (2.5,0.6) {$\Gamma$};

    % Ball around x0
    \draw (x0) circle (0.6cm);
    \draw[dashed] (x0) circle (0.8cm); % Dotted circle for the ball boundary
    \draw[fill=black, opacity=0.1] (x0) circle (0.6cm); % shaded ball

    % Ball around X
    \draw (X) circle (0.5cm);
    \draw[dashed] (X) circle (0.7cm); % Dotted circle for the ball boundary
    \draw[fill=black, opacity=0.1] (X) circle (0.5cm); % shaded ball
\end{tikzpicture}
\end{figure}

По доказанному утв. получаем, что в первом шаре ф-я $u(x)$ - константа, значит в центре следующего шара достигается макс. $\Rightarrow$ в след. шаре $u(x)$ - константа, и так далее.

За конечное число шагов достигаем последнего шара, в котором содержится т. $X$.

Получим, что в произвольной точке $x \in D \quad u(x) = u(x_0) = C$.
$\Rightarrow$ С учётом непр. $U$ на $\overline{D}$ ф-я $u(x) \equiv \text{const} = u(x_0)$ в $\overline{D}$.

Для минимума рассмотрим ф-ю $-u(x)$, $\min_{x \in \overline{D}} u(x) = -\max_{x \in \overline{D}} (-u(x))$.

\hrulefill

\noindent \textbf{Следствие 1} \\
Если в усл. теоремы гарм. ф-я $u(x)$ отлична от тожд. константы в $\overline{D}$, то
$$ \min_{\xi \in \partial D} u(\xi) < u(x) < \max_{\xi \in \partial D} u(\xi), \quad \forall x \in D $$




\underline{Единственность классического решения задачи Дирихле для уравнения Пуассона в ограниченной области.}
\hfill [\text{Уравн. с. 176}]

\begin{itemize}
    \item \textbf{Внутренняя задача Дирихле.} \\
    Пусть $D$ - огранич. обл. в $\mathbb{R}^n$, $f(x)$ и $U_0(x)$ - заданные ф-ии из классов $C(\overline{D})$, $C(\partial D)$ (т.е. $f \in C(\overline{D})$ и $U_0 \in C(\partial D)$). \\
    Требуется найти ф-ю $u(x)$ из класса $C^2(D) \cap C(\overline{D})$, удовл. в $D$ ур-ю Пуассона:
    $$ \Delta u(x) = f(x), \quad x \in D $$
    и граничному условию первого рода:
    $$ u(x) = U_0(x), \quad x \in \partial D $$
\end{itemize}

\noindent \textbf{Теорема 3.} \\
Если сущ. реш. внутр. задачи Дирихле в классе ф-ий $C^2(D) \cap C(\overline{D})$, то оно единственно в этом классе и непрерывно зависит от граничных данных.

\noindent \textit{\textbf{Док-во:}}
\begin{enumerate}
    \item \textit{Единственность} \\
    Пусть $\tilde{u}(x)$, $\tilde{\tilde{u}}(x)$ - реш. внутр. зад. Дирихле, удовл. одним и тем же гранич. усл. $U_0(x)$. \\
    Рассмотрим $u(x) = \tilde{u}(x) - \tilde{\tilde{u}}(x)$. Тогда $u(x) \in C^2(D) \cap C(\overline{D})$.
    $$ \Delta u(x) = \Delta \tilde{u}(x) - \Delta \tilde{\tilde{u}}(x) = f(x) - f(x) = 0, \quad x \in D $$
    $$ u(x)|_{\partial D} = \tilde{u}(x)|_{\partial D} - \tilde{\tilde{u}}(x)|_{\partial D} = U_0(x) - U_0(x) = 0, \quad x \in \partial D $$
    $\Rightarrow u(x)$ - гарм. ф-я. \\
    Из принципа максимума (для гармонической функции $u(x)$, равной нулю на границе) $\Rightarrow u(x) \equiv 0$ в $\overline{D}$ $\Rightarrow \tilde{u}(x) \equiv \tilde{\tilde{u}}(x)$ в $\overline{D}$.

    \item \textit{Непрерывная зависимость} \\
    Пусть $\tilde{u}(x)$, $\tilde{\tilde{u}}(x)$ - реш., удовл. гр. усл. $\tilde{U}_0(x)$, $\tilde{\tilde{U}}_0(x)$ соответственно. Тогда $u(x) = \tilde{u}(x) - \tilde{\tilde{u}}(x)$ - реш. задачи Дирихле:
    $$ \Delta u(x) = \Delta \tilde{u}(x) - \Delta \tilde{\tilde{u}}(x) = f(x) - f(x) = 0, \quad x \in D $$
    $$ u(x)|_{\partial D} = \tilde{U}_0(x) - \tilde{\tilde{U}}_0(x), \quad x \in \partial D $$
    Обозначим $U_D(x) = \tilde{U}_0(x) - \tilde{\tilde{U}}_0(x)$. Тогда $u(x)=U_D(x)$ на $\partial D$. \\
    Так как $u(x)$ гармоническая, по принципу максимума:
    $$ \max_{x \in \overline{D}} |u(x)| = \max_{x \in \partial D} |u(x)| $$
    Поскольку $u(x) = U_D(x)$ на $\partial D$, то
    $$ \max_{x \in \partial D} |u(x)| = \max_{x \in \partial D} |U_D(x)| $$
    Следовательно,
    $$ \max_{x \in \overline{D}} |u(x)| = \max_{x \in \partial D} |U_D(x)| $$
    То есть,
    $$ \max_{x \in \overline{D}} |\tilde{u}(x) - \tilde{\tilde{u}}(x)| = \max_{x \in \partial D} |\tilde{U}_0(x) - \tilde{\tilde{U}}_0(x)| $$
    Это и означает непрерывную зависимость решения от граничных данных в равномерной метрике.
\end{enumerate}





\newpage
\section*{Билет №18 -- 2025}\label{sec:ticket18}
\backtotoc
\textbf{Определение слабого решения задачи Дирихле для уравнения Пуассона. Теорема сущест-
вования и единственности слабого решения краевой задачи для уравнения Пуассона в
ограниченной области. [3] – 131-133, 136-138.}

 

\noindent\fbox{\textbf{\Large №20}}

\vspace{0.5em}

\noindent Определение обобщённого решения задачи Дирихле для ур-я Пуассона. Теорема существования и единственности обобщенного решения краевой задачи для ур-я Пуассона в ограниченной области. Понятие о повышении гладкости решения.

\vspace{0.5em}

\noindent Решения задач Дирихле ищутся в пространствах Соболева $H^1(\Omega)$, $\mathring{H}^1(\Omega)$.

\begin{itemize}
    \item $L_2(\Omega)$ - пространство ф-ий, интегр. в квадрате в обл. $\Omega$
    \[ (u,v) = \int_{\Omega} uv \, dx \quad \text{- скал. произв., } u,v \in L_2(\Omega) \]
    \[ \|u\|_0^2 = (u,u) = \int_{\Omega} u^2 dx \quad \text{- норма} \]

    \item $H^1(\Omega)$ включает в себя те ф-ии $u(x) \in L_2(\Omega)$, для которых все их производные 1-го порядка $\frac{\partial u}{\partial x_i} \in L_2(\Omega)$ (производные в смысле теории обобщённых ф-ий). \\
    $H^1(\Omega)$ также явл. гильбертовым простр. отн. скал. произв.:
    \[ [u,v]_{H^1(\Omega)} = \int_{\Omega} uv \, dx + \int_{\Omega} \sum_{k=1}^{n} \frac{\partial u}{\partial x_k} \frac{\partial v}{\partial x_k} \, dx, \quad u,v \in H^1(\Omega) \]
    \[ \|u\|_{H^1(\Omega)}^2 = [u,u]_{H^1(\Omega)} = \int_{\Omega} u^2 dx + \int_{\Omega} \sum_{k=1}^{n} \left(\frac{\partial u}{\partial x_k}\right)^2 dx = \|u\|_0^2 + \|\nabla u\|_0^2 \]
    \vspace{-1.5ex} % Adjust vertical space to bring annotation closer
    \begin{flushright}
    \footnotesize
    \begin{tabular}{@{}c@{}}
    $\uparrow$ (к $\|\nabla u\|_0^2$) \\
    норма вектор-функции \\
    в $(L_2(\Omega))^n$
    \end{tabular}
    \end{flushright}
    \vspace{1ex} % Add some space after annotation

    \item $\mathring{H}^1(\Omega)$-пополнение $C_0^\infty(\Omega)$ по норме пр-ва $H^1(\Omega)$
    \[ \Rightarrow C_0^\infty(\Omega) \subset \mathring{H}^1(\Omega) \subset H^1(\Omega) \subset L_2(\Omega) \]
    \[ \|u\|_1^2 = [u,u] = \int_{\Omega} \sum_{k=1}^{n} \left(\frac{\partial u}{\partial x_k}\right)^2 dx = \|\nabla u\|_0^2, \quad u \in \mathring{H}^1(\Omega) \]
    \[ [u,v] = \int_{\Omega} \sum_{k=1}^{n} \frac{\partial u}{\partial x_k} \frac{\partial v}{\partial x_k} \, dx = (\nabla u, \nabla v)_{(L_2(\Omega))^n}, \quad u,v \in \mathring{H}^1(\Omega) \]
\end{itemize}

\hfill (1/4)



\noindent $\|u\|_{H^1(\Omega)}^2 = \|u\|_0^2 + \|u\|_1^2, \quad [u,v]_{H^1(\Omega)} = (u,v) + [u,v]$

\begin{itemize}
    \item Нер-во Фридрихса: $\|u\|_0^2 \le C(\Omega) \|u\|_1^2, \quad \forall u \in \mathring{H}^1(\Omega)$
\end{itemize}

\noindent\underline{Теорема 1} \\
В пр-ве $\mathring{H}^1(\Omega)$ нормы $\|\cdot\|_{H^1(\Omega)}$ и $\|\cdot\|_1$ эквивалентны.

\vspace{0.5em}
\noindent\underline{Док-во:} \\
Оценка $\|u\|_1$ сверху через $\|u\|_{H^1(\Omega)}$ очевидна:
\[ \|u\|_1^2 \le \|u\|_0^2 + \|u\|_1^2 = \|u\|_{H^1(\Omega)}^2 \]
Обратная оценка прямо следует из нер-ва Фридрихса:
\[ \|u\|_{H^1(\Omega)}^2 = \|u\|_0^2 + \|u\|_1^2 \le (C(\Omega) + 1) \|u\|_1^2 \]

\vspace{0.5em}
\textbullet % For the large dot

\vspace{0.5em}
\noindent\underline{Задача Дирихле.}
\[
\begin{cases}
    \Delta u = f, & x \in \Omega \\
    u|_{\partial\Omega} = 0
\end{cases}
\quad (*)
\]
классическое реш. задачи (*) $u(x) \in C^2(\Omega) \cap C(\overline{\Omega})$ \\
$f(x) \in C(\Omega)$

\vspace{0.5em}
\noindent\underline{Предложение 1} \\
Пусть ф-я $u \in C^2(\Omega)$ удовл. ур-ю (*) с $f \in C(\Omega)$, тогда:
\[ [u, \varphi] + (f, \varphi) = 0, \quad \forall \varphi \in \mathring{H}^1(\Omega) \]

\vspace{0.5em}
\noindent\underline{Док-во:} для $\forall \varphi \in C_0^\infty(\Omega)$ из (*) следует:
\[ \int_{\Omega} \varphi \Delta u \, dx = \int_{\Omega} f \varphi \, dx \]
интегрируя выр-е слева по частям:
\[ - \int_{\Omega} \sum_{i=1}^{n} \frac{\partial u}{\partial x_i} \frac{\partial \varphi}{\partial x_i} \, dx = \int_{\Omega} f \varphi \, dx, \quad \forall \varphi \in C_0^\infty(\Omega) \]

\hfill (2/4)



\newtheorem{definition}{Определение}



\noindent $\Rightarrow -[u, \varphi] = (f, \varphi), \quad \forall \varphi \in C_0^\infty(\Omega)$ \\
Скал. произв. $[u,\varphi]$ в смысле $\mathring{H}^1(\Omega)$ и $(f,\varphi)$ в смысле $L_2(\Omega)$ непр. зависят от $\varphi$ как элем. пр-ва $H^1(\Omega)$ \\
$\Rightarrow$ тождество продолжается на всё пр-во $\mathring{H}^1(\Omega)$
\textbullet % Large dot

\vspace{0.5em}
\noindent\underline{Определение 1}: Ф-я $u(x) \in \mathring{H}^1(\Omega)$ наз-ся обобщенным решением задачи Дирихле (*), где $f \in L_2(\Omega)$, если
\[ [u, \varphi] + (f, \varphi) = 0, \quad \forall \varphi \in \mathring{H}^1(\Omega) \]

\vspace{0.5em}
\noindent\underline{Предложение 2.} (возможно не нужно) \\
Пусть ф-я $u \in C^2(\Omega) \cap C(\overline{\Omega})$ явл. обобщ. реш. задачи Дирихле, $f \in C(\Omega)$, тогда $u$ - классическое реш. этой задачи.

\vspace{0.5em}
\noindent\underline{Док-во:} \\
при $u \in C^2(\Omega)$, $\varphi \in C_0^\infty(\Omega)$ все выкладки \underline{предложения 1} проводятся и в обратную сторону, а из тождества:
\[ \int_{\Omega} (\Delta u - f) \varphi \, dx = 0, \quad \forall \varphi \in C_0^\infty(\Omega) \]
вытекает тождественное равенство нулю непр. в обл. $\Omega$ ф-ии $(\Delta u - f) \Rightarrow$ ур-е $\Delta u = f$, $x \in \Omega$ выполнено в классическом смысле. \\
про граничные усл. [с. 133-136]
\textbullet % Large dot

\hfill (3/4)






\newpage
\section*{Билет №19 -- 2025}\label{sec:ticket19}
\backtotoc
\textbf{Понятие слабого решения краевой задачи для дивергентного равномерно эллиптического
уравнения. Теорема сущест вования и единственности слабого решения краевой задачи для
дивергентного равномерно эллиптического уравнения в ограниченной области.
[3] – 131-133, 136-138.}




\newpage
\section*{Билет №20 -- 2025}
\textbf{Метод Фурье решения задачи Дирихле для уравнения Лапласа в круге и во внешности круга.
Внешние краевые задачи, условия на бесконечности. Метод Фурье решения краевой задачи для
уравнения Лапласа в шаре и вне шара. Сферические функции. [2] – 178-181, 270-282, [4].}

\end{document}




























